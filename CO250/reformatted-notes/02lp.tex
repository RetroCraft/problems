\section{Preparation}
\lecture{May 23}

\begin{example}
  $\max{} (0,-2,-3,0)x+7$ subject to $\mqty(1&3&-5&0\\0&-1&2&1)x=\mqty(6\\9)$ and $x \geq \mathbb 0$
\end{example}

\begin{sol}
  Trivial solution: $\bar x = (6,0,0,9)\trans$ gives objective value 7

  Claim: $\bar x$ is optimal

  Proof: the term $(0,-2,-3,0)x \leq 0$ since $x \geq \mathbb 0$.
  Its highest value is then 0, so the highest objective value is 7.
  Since the objective value of $\bar x$ is 7, it is optimal.
\end{sol}

\begin{theorem}[Fundamental Theorem of Linear Programming]
  For a linear program $P$, exactly one of the following holds:
  \begin{itemize}[nosep]
    \item $P$ is infeasible
    \item $P$ is unbounded
    \item $P$ has an optimal solution
  \end{itemize}
\end{theorem}

This does not apply to non-linear programs: e.g., $\max x$ subject
to $x < 1$. This NLP is feasible (consider $\bar x = 0$) and
bounded ($x < 1$), but has no optimal solution (given $\bar x$ a
solution, $\frac{\bar x+1}{2}$ is a better solution)

\begin{defn}[standard equality form]
  A linear program of the form
  $\max \{c\trans x + \bar z : Ax = b, x \geq \mathbb 0\}$

  Requires maximization, equality constraint, and non-negative
  variables
\end{defn}

Simplex requires SEF, so must convert LPs into equivalent SEF LP.

\begin{defn}[equivalence]
  $P$ and $P'$ are equivalent if
  (1) $P$ infeasible iff $P'$ infeasible,
  (2) $P$ unbounded iff $P'$ unbounded, and
  (3) optimal solutions of $P$ can be constructed from $P'$ and vice versa
\end{defn}

Find an equivalent SEF by:

\begin{itemize}[nosep]
  \item Given a minimization LP $\min f(x)$, just take $\max -f(x)$
  \item Given an inequality $x \leq k$, define a slack variable
        $x + x' = k$ with $x' \geq 0$
  \item Given a free variable $x$, define two non-negative variables
        $x^+$ and $x^-$ so that we can replace $x = x^+ - x^-$
\end{itemize}


\begin{example}
  Find an equivalent LP in SEF
  \begin{align*}
    \min         & (-1,2,-3)x             \\
    \text{s.t. } & \mqty(1        & 5 & 3 \\2&-1&2\\1&2&-1)\mqty(x_1\\x_2\\x_3)\mqty{\leq\\\geq\\=}\mqty(5\\4\\2) \\
                 & x_1,x_2 \geq 0
  \end{align*}
\end{example}
\begin{sol}
  Switch $\min$ to $\max$ with new objective function $(1,-2,3)x$

  Divide $x_3 = x_3^+ - x_3^-$ giving $\mqty(1&5&3&-3\\2&-1&2&-2\\1&2&-1&1)\mqty(x_1\\x_2\\x_3^+\\x_3^-)$

  Add slack variables $x_4$, $x_5$ giving new rows $(1,5,3,1,0)x=5$
  and $(2,-1,2,0,1)x=4$

  Let $x = (x_1,x_2,x_3^+,x_3^-,x_4,x_5)\trans$
  and combine to get the SEF
  \[\max \qty{ (1,-2,3,-3,0,0)x : \mqty(1&5&3&-3&1&0\\2&-1&2&-2&0&1\\1&2&-1&1&0&0)x=\mqty(5\\4\\2), x \geq \mathbb 0}\]
  Suppose simplex solves this and gives optimal solution
  $(\frac{11}{4},0,\frac34,0,0,3)\trans$ with optimal value $5$. Then, the
  original LP is solved by
  $(\frac{11}{4},0,\frac34-0)\trans =(\frac{11}{4},0,\frac34)$ with optimal
  value $-5$.
\end{sol}

\lecture{May 25}
\begin{example}
  $\max (3,-2,0,0,0)x$ such that $x \geq \mathbb 0$
  and $\mqty(4&-1&1&0&0\\3&-3&0&1&0\\-2&2&0&0&1)x=\mqty(8\\9\\1)$
\end{example}
\begin{sol}
  Feasible solution: $\bar x = (0,0,8,9,1)\trans$ with objective value
  $(3,-2,0,0,0)(0,0,8,9,1)\trans = 0$

  To increase objective value, must increase $x_1$ (the only one
  with a positive coefficient in the objective function)

  Change $x_1 \mapsto t$, keep $x_2 = 0$, and set
  $x_3 \mapsto 8-4t$, $x_4 \mapsto 9-3t$, and $x_5 \mapsto 1+2t$
  to maintain feasibility

  But we still have non-negativity, giving us $t \leq 2$, i.e.,
  $\bar{\bar x} = (2,0,0,3,5)\trans$
\end{sol}

This example worked because we had (1) the identity matrix embedded
in columns, (2) those columns have zero coefficients in the
objective function.
Equivalent strategies will exist if the matrix's rows are
independent.

\begin{notation}
  Given a matrix $A$, notate the column $j = 1,\dotsc,n$ of $A$ by
  $A_j$ and then the submatrix formed by columns
  $J \subseteq \{1,\dotsc,n\}$ by $A_J$
\end{notation}

\begin{prop}
  \Tfae: $B \subseteq \{1,\dotsc,n\}$ is a basis;
  $A_B$ is invertible; and
  $\abs{B} = m$ and the columns $A_B$ are linearly independent
\end{prop}

\begin{defn}
  Given a basis $B$, a \term{non-basis}
  $N = \{1,\dotsc,n\}\setminus B$.

  The variables $x_B$ where $B$ a basis are \term{basic variables};
  conversely, $x_N$ are \term{non-basic variables}.
  This lets us write $Ax = A_Bx_B + A_Nx_N$.

  A \term{basic solution} (with respect to $B$) is a solution $x$ where $x_N = 0$.
\end{defn}

To find a basic solution $x_B$, multiply both sides of the
constraint $Ax = b$ on the left by $A_B^{-1}$. Then,
$A_B^{-1}Ax = A_B^{-1}b$ and we can read the basic solution
$x_B = A_B^{-1}b$.

\begin{defn}[basic feasible solution]
  A value $\bar x$ such that $\bar x_B \geq \zero$.
\end{defn}

\begin{defn}[canonical form]
  SEF $\max\{c\trans x+\bar z:Ax=b,x\geq\mathbb0\}$ if $A_B=I$ and
  $c_B=\mathbb 0$
\end{defn}

\lecture{May 30}

\begin{example}
  Write the canonical form for $B=\{2,3\}$ of
  $\max (3,-1,4,0,-1)x + 2$ s.t. $x \geq \mathbb 0$ and
  $\mqty(2&1&-2&5&-4\\-1&0&-1&-3&2)x = \mqty(3\\-2)$
\end{example}

\begin{sol}
  Multiply the constraint on the left by
  $A_B^{-1} = \mqty(1&-2\\0&-1)$ to get new constraint
  $\mqty(4&1&0&11&-8\\1&0&1&3&-2)x=\mqty(7\\2)$ which is equivalent

  Then, we need to set the entries in the objective function vector
  $c_B = \mathbb 0$

  To find an expression for $-c_B\trans x_B$ which can cancel, take
  $-c_B\trans Ax = -c_B b$ to get
  $(0,1,-4,-1,0)x = -1 \implies (0,1,-4,-1,0)x + 1 = 0$

  Add this to $c\trans x + \bar z$ to get the new objective function
  $\max (3,0,0,-1,-1)x + 3$
\end{sol}

In general, to convert from SEF to canonical form given a basis $B$:

\begin{enumerate}[1.,nosep]
  \item Amend the constraint: $Ax = b \mapsto A_B^{-1} A x = A_B^{-1} b$
  \item Calculate $y = (A_B^{-1})\trans c_B$
        (this will be used later as a certificate)
  \item Amend the objective function:
        $c\trans x + \bar z \mapsto (c\trans - y\trans A)x + (\bar z - y\trans b)$
\end{enumerate}

\section{Simplex}

Main idea: go between feasible bases, attempting to raise the objective value
\begin{enumerate}
  \item Start with a feasible basis $B$.

        Use ``2-phase simplex'' to find one if there is not a trivial feasible basis
  \item Convert the LP to canonical form with respect to $B$.

        If $c_N \leq \mathbb 0$, then the optimal solution is the basic
        feasible solution. \textbf{Stop}.
  \item Pick a non-basic entering variable $x_k$ where $c_k > 0$.

        (Bland's rule) Pick the variable with lowest index $k$.

        If $A_k \leq \mathbb 0$, then the linear program is
        unbounded. \textbf{Stop}.

        Increase $x_k = t$ while reducing $x_B$
        to increase the objective value, i.e., maximize $t$ subject to
        $x_B = b - A_k t$ and $x \geq \mathbb 0$.
        That is,
        $t = \min\{\frac{b_i}{A_{ik}} : A_{ik} > 0\}$
  \item Pick a leaving variable $x_\ell$.

        Let $x_\ell$ be the index used for the minimum
        above. Then, $x_\ell = 0$ after setting $x_k = t$ making $x_\ell$
        non-basic.

        Repeat step 2 with the new basis
        $B = B \cup \{k\} \setminus\{\ell\}$
\end{enumerate}

\begin{example}
  Maximize $(0,2,1,0,0,0)x$ such that
  $x \geq \mathbb 0$ and
  \[\mqty(\mqty{1&-1&1\\-2&1&0\\0&1&-2}&\mqty{\imat{3}})x = \mqty(5\\3\\5)\]
\end{example}
\begin{sol}
  Perform simplex:
  \begin{enumerate}
    \item Entering variable $x_2$ since $c_2 = 2 > 0$

          Set $x_2 = t = \min\{-,\frac{3}{1},\frac{5}{1}\} = 3$

          Leaving variable is $x_5$, new basis $B = \{2,4,6\}$

          New canonical form $\max(4,0,1,0,-2,0)x + 6$ with
          $x \geq \mathbb 0$ and
          \[\mqty(-2&1&0&0&1&0\\-1&0&1&1&1&0\\2&0&2&0&-1&1)x = \mqty(3\\8\\2)\]

    \item Entering variable $x_1$.
          Set $x_1 = t = \min\{-,-,\frac{2}{2}\}=1$.

          Leaving variable $x_6$. New basis $B = \{1,2,4\}$

          New canonical form $\max(0,0,5,0,0,-2)x + 10$ with
          $x \geq \mathbb 0$ and
          \[\mqty(1&0&-1&0&-\frac12&\frac12\\0&1&-2&0&0&1\\0&0&0&1&\frac12&\frac12)x = \mqty(1\\5\\9)\]

    \item Entering variable $x_3$.
          Set $x_3 = t = \min\{-,-,-\} = \infty$.
          No bound on $t$, so this LP is unbounded.
  \end{enumerate}
  \lecture{June 1}
  Iteration 3 entered $x_3$, and $A_3 \leq 0$.
  Then, $t$ had no bounds, so the LP is unbounded.
  Find a certificate of unboundedness:

  Current basic feasible solution: $\bar x = (1,5,0,9,0,0)\trans$

  Set $x_3 = t$.
  New solution is $x(t) = (1+t,5+2t,t,9,0,0)\trans = \bar x + td$
  where $d = (1,2,1,0,0,0)\trans$.

  Then, $x(t)$ is feasible for all $t \geq 0$ and
  $c\trans x(t) = 5t - 10 \to \infty$.
  This is all we need to show that $(\bar x, t)$
  form a certificate of unboundedness.
\end{sol}


Recall: simplex iterates over feasible bases to optimize one variable
at a time and set coefficients to zero, eventually finding an optimal
solution or showing unboundedness

Does simplex always terminate? Intuitively it should, since each
iteration increases the objective value. However, this is not true in
general

\begin{defn}[degenerate iteration]
  Iteration where the objective value is held constant.
  That is, only the basis is changed.
\end{defn}

\begin{example}
  Maximize $(1,0,0,0)x$ subject to $x \geq \mathbb0$
  and $\mqty(3&2&1&0\\1&1&0&1)x = (0,2)\trans$.

  The start basis is clearly $\{1,4\}$
  The only entering variable is $x_1$. Let $t = x_1$ and
  $t = \min\{\frac21,\frac03\} = 0$
  We pick $x_4$ for a leaving variable, giving $B' = \{1,4\}$,
  which is the same.
\end{example}

\begin{defn}[cycling]
  Repeating bases with the same basic feasible solution.
\end{defn}

This happens when there is a series of degenerate iterations that repeat bases already used.
Bland's Rule, when applied, prevents cycling and guarantees simplex terminates.
That is, when there is a choice for the entering or leaving variable,
pick the one with the lowest index.

\paragraph{\term{Tableau method} \rm (easier to perform by hand)}
Use a matrix to encode the entire state of the algorithm, allowing
row reductions to replace taking inverses

\begin{example}
  Maximize $z = (3,-2,0,0,0)x+0$ subject to
  $x \geq \mathbb0$ and
  \[\mqty(\mqty{4&-1\\3&-3\\-2&2}&\mqty{\imat{3}})x = \mqty(8\\9\\1)\]
\end{example}
\begin{sol}
  Create a tableau
  $\qty(\begin{array}{c|c|c}1 & -c& \bar z \\ \hline \mathbb0 & A & B\end{array}) = \qty(\begin{array}{c|ccccc|c}1&-3&2&0&0&0&0 \\\hline 0&4&-1&1&0&0&8 \\ 0&3&-3&0&1&0&9 \\ 0&-2&2&0&0&1&1 \end{array})$

  Entering variable: look for negative coefficient in objective row
  $x_k$ (in this case $x_1$)

  Leaving variable: look at ratio between $b$ and $A_k$ to find
  row with $x_\ell$ (in this case $A^1$)

  Take the entry in column of $x_k$ and the row of $x_\ell$
  (in this case $A_{11}$).
  Use row operations to make it 1 and the rest of the column 0.

  In this case $\frac14R_2$, $R_1 + 3R_2$, $R_3 - 3R_2$,
  $R_4 + 2R_2$, to yield
  $\qty(\begin{array}{c|ccccc|c}1&0&\frac54&\frac34&0&0&6 \\\hline 0&1&-\frac14&\frac14&0&0&2 \\ 0&0&-\frac94&-\frac34&1&0&3 \\ 0&0&\frac32&\frac12&0&1&5 \end{array})$

  Since top row is all positive, this means that $c$ is all
  negative, so we have an optimal solution with optimal
  objective value 6
\end{sol}

Since tableau method involves iterative dividing of decimals,
computers will lose precision after each iteration. With canonical
form, calculations are always done from the original LP, so no
compounded error.

If no obvious basic feasible solution to start simplex, do two-stage
simplex.

\section{Two-phase simplex}

Main idea: Form an auxiliary LP with an obvious BFS, then solve with simplex to
help find a BFS or prove that no BFS exists
\begin{itemize}[nosep]
  \item Negate constraints with negative right-hand sides
  \item Arbitrarily add auxiliary variables to append an identity matrix
        to $A$
  \item Apply simplex to the new LP: there is a solution to the original
        LP if and only if this LP has solutions with the auxiliary
        variables set to zero
\end{itemize}

\begin{example}
  Maximize $(-3,2,0,0,0)x$ subject to
  $x \geq \mathbb0$ and
  \[\mqty(\mqty{-1&1\\-1&-2\\0&1}&\mqty{\imat{3}})x = \mqty(-1\\-4\\4)\]
\end{example}
\begin{sol}
  Notice that $B = \{3,4,5\}$ is a basis.
  However, the trivial basic solution $(0,0,-1,-4,4)\trans$ is not feasible.

  Negate constraints to get
  $\mqty(\mqty{1&-1&-1&0&0\\1&2&0&-1&0\\0&1&0&0&1})x = \mqty(1\\4\\4)$.

  Add auxiliary variables to get
  $\mqty(\mqty{1&-1&-1&0&0\\1&2&0&-1&0\\0&1&0&0&1}&\mqty{\imat{3}})x = \mqty(1\\4\\4)$.

  This LP has a BFS $(0,0,0,0,0,1,4,4)\trans$.

  \lecture{June 6}
  Start with $B = \{6,7,8\}$ with BFS
  $\bar x = (0,0,0,0,0,1,4,4)\trans$.

  Minimize the sum of auxiliary variables:
  $\min{(0,0,0,0,0,1,1,1)x}$ subject to same constraints.

  Same as saying $\max{(0,0,0,0,0,-1,-1,-1)x}$, which we can solve
  with simplex.

  If optimal value is 0, then auxiliary variables must be 0.
  If optimal value is negative, no feasible solution exists.

  Based on canonical form, multiply constraint on left by $(1,1,1)$
  to get $(2,1,-1,-1,1,1,1,1)x-9=0$.

  Add to objective function to get new objective function
  $(2,1,-1,-1,1,0,0,0)x-9$.

  Run simplex to get basis $\{1,3,5\}$ and BFS
  $(4,0,3,0,4,0,0,0)\trans$.

  Now that we have a BFS $(4,0,3,0,4)\trans$, run simplex.
\end{sol}

If auxiliary LP does not have a solution, then $y = A_B^{-T}x_B$ is
a certificate of infeasibility for the original LP.

Optimal solution must exist: it is feasible by construction and not
unbounded because the objective value cannot exceed 0.

Therefore, two-phase simplex will always work.

\section{Geometry of simplex}
Consider $\max{(1,1)x}$ subject to
$x \geq \mathbb0$ and $\mqty(1&0\\0&1\\2&1)x \leq \mqty(4\\3\\4)$.

Equivalently, $\max{(1,1,0,0,0)x}$ subject to $x \geq \mathbb0$ and
$\mqty(\mqty{\imat2\\2&1}&\mqty{\imat3})x = \mqty(4\\3\\4)$.

Equalities define lines, inequalities define halfspaces:
\lecture{June 8}
\begin{itemize}[noitemsep]
  \item Non-negativity bounds $(x_1, x_2) \in \R^2$ to the first quadrant.
  \item Constraints bound $x_1 \leq 4$, $x_2 \leq 3$, and $x_1 + 2x_2 \leq 8$.
\end{itemize}
All feasible solutions are thus contained in the convex polygon bounded by
$(0,0)$, $(0,3)$, $(2,3)$, $(4,2)$, $(4,0)$.

Consider contours of the objective function $x_1 + x_2 = k$ starting at $k=0$.
Imagine dragging the objective function line ``up'' along the normal
vector $(1,1)\trans$ maintaining the slope.
Last point $(4,2)\trans$ which touches the highest contour $k=6$ is
the optimal solution.

Notice that $(4,2)\trans$ is generated by $x_1=4$ and $x_1 + 2x_2 = 8$,
i.e., setting constraints (1) and (3) to equality.
Then, in SEF, $(4,2)\trans \mapsto (4,2,0,1,0)\trans$ is a BFS.

Likewise, consider the extreme point $(4,0)\trans$
generated by $x_1 = 4$ and $x_2 = 0$, i.e.,
constraints (1) and (5) to equality.
In SEF, $(4,0)\trans \mapsto (4,0,0,3,4)\trans$ which is a BFS.

We can make some observations.

\begin{defn}[boundary]
  A point $p \in S$ is on the \term*{boundary of $S$}
  if every open neighbourhood of $p$
  contains a point in $S$ and a point not in $S$.
\end{defn}

\begin{remark}
  Optimal solutions must always be on the boundary.
\end{remark}

\begin{defn}[tight constraint]
  An inequality satisfied with equality by a feasible solution $\bar x$.
\end{defn}

\begin{theorem}
  If an optimal solution exists and the rows of $A$ are linearly independent,
  then there is at least one optimal ``corner''.
\end{theorem}
\begin{prf}
  Setting inequality to equality means setting a slack variable in SEF to 0.
  Every corner point is defined by $n$ linear equalities in $\R^n$,
  i.e., $n$ (or more) tight constraints.
  Therefore, a corner point's canonical form solution will have
  $n$ slack variables set to zero, creating a BFS.
\end{prf}

Geometrically, simplex walks along extreme points (BFSs) until it finds an optimal solution.
Two-phase simplex is required when the origin is not available as a corner to start from.
Each iteration picks an adjacent extreme point,
because it swaps one tight constraint for another one and maintains the other $n-1$ constraints.

\begin{defn}[degenerate]
  $\bar x$ with more than $n$ tight constraints, i.e., some basic variables are 0.
\end{defn}

This means an iteration may bounce endlessly between degenerate BFSs.
Degenerate iterations are avoided by Bland's rule.

Finally, simplex requires that the feasible region be convex.
It always is because it is the convex hull generated by the extreme points.

\lecture{June 13}

Formalize some basic geometric ideas from last lecture.

\begin{defn*}[geometry]
  A \term{hyperplane} is an equation of the form $a\trans x = \beta$ with
  \term{normal vector} $a$.

  A \term{halfspace} is an inequality of the form $a\trans x \leq \beta$
  with \term*{normal vector} $a$. Note: the halfspace is always on the
  opposite side of the normal vector.

  A \term{polyhedron} is a set of points
  $\{x \in \R^n : Ax \leq b\}$, i.e., the intersection of $m$
  halfspaces for a polytope with $m$ facets.
\end{defn*}

\begin{defn}[convexity]
  Subset $S \subseteq \R^n$ such that for all
  $x^1, x^2 \in S$ and $\lambda \in [0,1]$,
  $\lambda x^1 + (1-\lambda)x^2 \in S$.
  This expression is a \term{convex combination}.
\end{defn}

\begin{prop}
  A halfspce is convex.
\end{prop}
\begin{prf}
  Let $H = \{x \in \R^n : a\trans x \leq \beta\}$ be a halfspace and $x^1, x^2 \in H$.
  Then, with arbitrary $\lambda \in [0,1]$, let $x = \lambda x^1 + (1-\lambda)x^2$.

  We have $a\trans x = \lambda a\trans x^1 + (1-\lambda) a\trans x^2$.
  But since $x^1, x^2 \in H$, we get
  $a\trans x \leq \lambda \beta + (1-\lambda)\beta = \beta$ as desired.
\end{prf}

\begin{prop}
  The intersection of convex sets is convex.
\end{prop}
\begin{prf}
  If two points are in the intersection, they are in each convex set.
  The line segment joining them is then also in each convex set,
  meaning it is in the intersection.
\end{prf}

\begin{corollary}
  A polyhedron is convex.
\end{corollary}

\begin{defn}[extreme point]
  Point $\bar x \in C$ a convex set such that $x$ is not
  a convex combination of two distinct points in $C$ distinct from $\bar x$.
\end{defn}

\lecture{June 15}

\begin{theorem}[characterization of extreme points in polyhedra]\label{thm:xp-r}
  For $\bar x \in P = \{x \in \R^n : Ax \leq b\}$ a polyhedron
  where $A^=x = b^=$ is the set of tight constraints for $\bar x$,
  $\bar x$ is an extreme point of $P$ if and only if $\rank(A^=) = n$.
\end{theorem}
\begin{prf}
  Proceed by the contrapositive.

  ($\Larr$) Suppose $\bar x$ is not an extreme point.
  Then, there exist distinct $x^1, x^2 \in P$ such that
  $\bar x = \lambda x^2 + (1-\lambda)x^2$ for some $0 < \lambda < 1$.

  We have
  \[ A^= \bar x = A^=(\lambda x^1 + (1-\lambda)x^2) = \lambda(A^= x^1) + (1-\lambda)(A^= x^2) \leq b^=. \]
  That is, there is equality throughout the line $A^= x^1 = A^= x^2 = b^=$.

  Then, since $A^= x = b^=$ has at least three solutions $(x^1, x^2, \bar x)$,
  we cannot have $\rank(A^=) = n$.

  ($\Rarr$) Suppose $\rank(A^=) < n$.
  Then, there exists non-zero $d$ such that $A^= d = \mathbb 0$.

  Let $x^1 = \bar x + \varepsilon d$ and $x^2 = \bar x - \varepsilon d$
  for some small positive $\varepsilon$.
  By construction, $\bar x$ is properly contained in the line segment from $x^1$ to $x^2$.

  Then, we can write
  \[ A^= x^1 = A^=(\bar x + \varepsilon d) = A^= \bar x + \varepsilon A^= d = A^= \bar x = b^= \]
  which means $x^1$ (and likewise $x^2$) satisfy the tight constraints.

  For other non-tight constraints $a\trans x \leq \beta$, we have
  \[ a\trans \bar x + \varepsilon(a\trans d) < \beta + \varepsilon(a\trans d) \leq \beta \]
  for sufficiently small $\varepsilon$.
  Therefore, $x^1$ and $x^2$ are in $P$ and $\bar x$ is not an extreme point.
\end{prf}

\begin{theorem}[characterization of extreme points in SEF]
  Let $P = \{x \in \R^n : Ax = b, x \geq \mathbb0\}$
  where $A$ has full row rank and let $\bar x \in P$.

  Then, $\bar x$ is an extreme point of $P$ if and only if
  $\bar x$ is a BFS of $Ax = b$.
\end{theorem}
\begin{prf}
  Construct a polyhedron
  $\qty(\begin{array}{c}A \\\hline -A \\\hline -I\end{array})x
    \leq \qty(\begin{array}{c}b \\\hline -b \\\hline \zero\end{array})x$.

  Suppose $A\bar x = b$.
  By \cref{thm:xp-r}, $\bar x$ and extreme point if and only if
  the tight constraints have rank $n$.
  The top two submatrices are always tight,
  so we automatically get rank at least $m$.
  We need at least $n-m$ tight constraints in $-I\bar x \leq \mathbb0$.
  That is, $n-m$ entries of $\bar x$ are zero.
  
  Then, $\bar x$ is a basic feasible solution
  with $m$ basic variables and $n-m$ non-basic variables.
\end{prf}
