\section{Strong and weak duality}

\begin{example}
  Find $\max\qty{(4,-9,2,4)x : \mqty(1&-4&3&0\\-1&7&-5&1)x = \mqty(7\\3), x \geq \zero}$.
\end{example}
\begin{sol}
  Notice that $(6,2,3,10)\trans$ is feasible. Objective value 52.

  $(7,0,0,10)\trans$ is feasible.
  Objective value 68. Claim this is optimal.

  Notice that:
  \begin{align*}
    68 & = 7\cdot8 + 3\cdot4                                  \\
       & = 8(x_1 - 4x_2 + 3x_3) + 4(-x_1 + 7x_2 - 5x_3 + x_4) \\
       & = 4x_1 - 4x_2 + 4x_3 + 4x_4                          \\
       & \geq 4x_1 - 9x_2 + 2x_3 + 4x_4 = c\trans x
  \end{align*}
  Since $c\trans x \leq 68$,
  ny feasible solution has objective value at most 68.
  As $(7,0,0,10)\trans$ is one such solution, it is optimal.
\end{sol}

In general, consider $y$. Then, $y\trans b = y\trans A x \stackrel{?}{\geq} c\trans x$.
This works when $y\trans A \geq c\trans$ or equivalently $A\trans y \geq c\trans$.
Then, $y\trans b$ is an upper bound on the objective value.
If $y\trans b$ is the lowest upper bound, it is the optimal objective value.

To find $y$, we can solve the dual LP.

\begin{defn}[dual LP]
  Given an LP (P) in SEF
  $\max\{c\trans x : Ax = b, x \geq \zero\}$, construct the dual (D)
  $\min\{b\trans y : A\trans y \geq c\}$
\end{defn}

Each constraint $a_i x_i = b_i$ in (P) corresponds to a variable $y_i$ in (D).
Each variable $x_j$ in (P) corresponds to a constraint $(a^j)\trans y_j \leq c_j$ in (D).

\lecture{June 20}

Notice that the feasible objective values of the dual are at most the
feasible objective values of the primal.

Then, the highest (optimal) value of the primal is bounded above by
the lowest (optimal) value of the dual.
In fact, they meet only at optimality.

\begin{theorem}[weak duality]
  For an LP in SEF (P) $\max\{c\trans x : Ax = b, x \geq \zero\}$
  and its dual (D) $\min\{b\trans y : A\trans y \geq c\}$,
  if $\bar x$ and $\bar y$ are feasible,
  then $c\trans\bar x \leq b\trans\bar y$.

  Moreover, if $c\trans\bar x = b\trans\bar y$,
  then $\bar x$ and $\bar y$ are optimal.
\end{theorem}
\begin{prf}
  We have $b\trans \bar y = (A\bar x)\trans \bar y =\bar x\trans A\trans \bar y \geq \bar x\trans c = c\trans \bar x$,
  as desired.

  Suppose $c\trans \bar x = b\trans \bar y$.
  Let $x$ be arbitrary.
  Then, $c\trans x \leq b\trans \bar y = c\trans \bar x$.
  This is what it means for $\bar x$ to be optimal.
  Argue symmetrically for $y$.
\end{prf}

\begin{corollary}
  To certify $\bar x$ is optimal for (P), it suffices to show that:
  $\bar x$ is feasible, $\bar y$ is feasible, and $c\trans \bar x = b\trans \bar y$.
\end{corollary}

\begin{corollary}
  If (P) is unbounded, then (D) must be infeasible.
  Similarly, if (D) is unbounded, then (P) is infeasible.
  Finally, if both (P) and (D) are feasible,
  then they must have optimal solutions.
\end{corollary}
\begin{prf}
  Any feasible solution to (D) is an upper bound to the
  objective values of (P). Likewise in the other direction.
\end{prf}

\begin{theorem}[strong duality]
  If (P) has an optimal solution,
  then (D) has an optimal solution with the same optimal values.
\end{theorem}
\begin{prf}
  Perform simplex on (P) to get a basic optimal solution $\bar x$ with basis $B$.
  The objective function in canonical form for $B$
  is $(c\trans - \bar y\trans A)x + b\trans \bar y$
  where $\bar y = A_B^{-\intercal}c_B$.

  Since $B$ is optimal, $c\trans - \bar y\trans A \leq \zero\trans$
  which gives $A\trans y \geq c$, i.e., $\bar y$ is feasible.
  Also, the objective value of $\bar x$ is $c\trans x = b\trans y$.
\end{prf}

\lecture{June 22}

Summarize the weak duality theorem results:
\begin{center}
  \begin{tabular}{rccc}
    Dual \textbackslash{} Primal & Optimal      & Unbounded    & Infeasible               \\ \hline
    Optimal                      & $\checkmark$ & $\times$     & $\times$                 \\
    Unbounded                    & $\times$     & $\times$     & $\checkmark$             \\
    Infeasible                   & $\times$     & $\checkmark$ & $\checkmark$ (edge case) \\
  \end{tabular}
\end{center}

\begin{prop}[dual of inequality]
  Dual of $\max\{c\trans x : Ax \leq b, x \geq \zero\}$
  is $\min\{b\trans y : A\trans y \geq c, y \geq \zero\}$.
\end{prop}
\begin{prf}
  Turn into SEF by adding slack variables:
  $\max\{c\trans x + \zero\trans x' : Ax + Ix' = b, x \geq \zero\}$.

  Get the dual of that which is
  $\min\{b\trans y : A\trans y \geq c, y \geq \zero\}$.

  Likewise, the dual of $\max\{c\trans x : Ax \geq b, x \geq \zero\}$
  is $\min\{b\trans y : A\trans y \geq c, y \leq \zero\}$
\end{prf}

\begin{prop}[dual with free variables]
  Dual of $\max\{c\trans x : Ax = b\}$ is $\min\{b\trans y : A\trans y = c\}$.
\end{prop}
\begin{prf}
  Turn into SEF:
  $\max \{c\trans x^+ - c\trans x^- : Ax^+ - Ax^- = b, x^+ \geq \zero, x^- \geq \zero\}$.

  Dual is
  $\min\{b\trans y : A\trans y \geq c, -A\trans y \geq -c\}
    = \min\{b\trans y : A\trans y = c\}$.

  In fact, having $x \geq \zero$ will give $A\trans y \geq c$ and
  $x \leq \zero$ will give $A\trans y \leq c$.
\end{prf}

\begin{prop}[dual of min]
  Dual of the minimization LP (P*) is the maximization LP (D*) whose dual is (P*).
\end{prop}
\begin{prf}
  From max to min: constraints (swapped) $\to$ variables, variables $\to$ constraints.

  From min to max: constraints $\to$ variables, variables (swapped) $\to$ constraints.
\end{prf}

\begin{example}
  Find the dual: Maximize $(3,1,4,1)x$ such that
  $\mqty(1&2&3&4\\4&3&2&1\\2&2&2&2)x\mathrel{\mqty{\geq\\\leq\\=}}\mqty(3\\4\\5)$
  with $x_1, x_2 \geq 0$, $x_3 \leq 0$, and $x_4$ free.
\end{example}
\begin{sol}
  Minimize $(3,4,5)y$ such that:
  \begin{itemize}[nosep]
    \item Dual constraints max $\to$ min kept:
          $\mqty(1&4&2\\2&3&2\\3&2&2\\4&1&2)y\mathrel{\mqty{\geq\\\geq\\\leq\\=}}\mqty(3\\1\\4\\1)$.
    \item Dual variables max $\to$ min flipped: $y_1 \leq 0$, $y_2 \geq 0$, $y_3$ free.
  \end{itemize}
\end{sol}

\begin{theorem}[general weak duality]
  Let (P) be a maximization LP with dual (D).
  If $\bar x$ and $\bar y$ are feasible for (P) and (D),
  respectively, then the objective value of $\bar x$ in (P)
  is at most the objective value of $\bar y$ in (D).

  If they are the same value, then they are both optimal.
\end{theorem}

\begin{theorem}[general strong duality]
  Let (P) be an LP with dual (D).
  If (P) has an optimal solution,
  then (D) has an optimal solution with the same optimal value.
\end{theorem}

\section{Complementary slackness}

\begin{defn}[complementary slackness conditions]
  For $(A\trans y)\trans x \geq c\trans x$ to hold with equality,
  we have $A\trans_i y x_i = c_i x_i$.
  Equivalently, either $A\trans_i y = c_i$ or $x_i = 0$.
\end{defn}

CS conditions are satisfied by feasible $\bar x$ and $\bar y$ if and
only if $\bar x$ and $\bar y$ are optimal.

\lecture{June 27}

For an LP in SEF, the CS conditions are
``either $x_i = 0$ or the $i$-th dual constraint is tight''.
Equivalently, ``if $x_i > 0$, then the $i$-th dual constraint is tight''
and ``if the $i$-th dual constraint is not tight, $x_i = 0$''.

In general, given primal (P) and dual (D), their CS conditions are:
(1) either $x_i = 0$ or the $i$-th constraint of (D) is tight and
(2) either $y_j = 0$ or the $j$-th constraint of (P) is tight.
If there are equality constraints,
we can ignore that set of CS conditions since all constraints are tight.

\begin{theorem}[complementary slackness]
  Let (P) and (D) be a primal-dual pair with feasible solutions $\bar x$ and $\bar y$, respectively.
  Then, $\bar x$ and $\bar y$ are optimal if and only if all CS conditions hold.
\end{theorem}

To show $\bar x$ is optimal for (P), we can provide $\bar y$ and check that
(1) $\bar x$ is feasible, (2) $\bar y$ is feasible, (3) CS conditions.

\begin{example}
  Find an optimal solution for
  $\max\qty{(1,-2)x : \mqty(1&-1\\2&1\\0&1)x\mathrel{\mqty{=\\\geq\\\leq}}\mqty(2\\1\\4), x_1 \geq 0}$
  with dual
  $\min\qty{(2,1,4)y : \mqty(1&2&0\\-1&1&1)y\mathrel{\mqty{\geq\\=}}\mqty(1\\-2) : y_2 \leq 0, y_3 \geq 0}$
\end{example}
\begin{sol}
  CS conditions:
  \begin{itemize}[nosep]
    \item $x_1 = 0$ or $y_1 + 2y_2 = 1$
    \item (always true) $x_2 = 0$ or $\boxed{-y_1 + y_2 + y_3 = -2}$
    \item (always true) $y_1 = 0$ or $\boxed{x_1 - x_2 = 2}$
    \item $y_2 = 0$ or $2x_1 + x_2 = 1$
    \item $y_3 = 0$ or $x_2 = 4$
  \end{itemize}
  Suppose we think that $\bar x = (1,-1)\trans$ is optimal.
  By inspection, $\bar x$ feasible.
  Since $\bar x_1 \neq 0$, we require $\bar y_1 + 2\bar y_2 = 1$.
  The second condition is satisfied.
  Also, $\bar x_2 \neq 4$, so $\bar y_3 = 0$.

  Adding on the requirement that $\bar y$ feasible, we can solve for
  $\bar y = (\frac53, -\frac13,0)\trans$.

  Then, $\bar y$ is feasible and satisfies CS conditions, so
  $\bar x$ and $\bar y$ are optimal.
\end{sol}

If we try the above example with $\bar x = (4,2)\trans$, we get CS
conditions implying $\bar y = (1,0,0)\trans$, but this is not feasible.
Therefore, $\bar x$ is not optimal.

The above example with simplex gives first
\[ \max\qty{(3,-2,0,0,0)x : \mqty(\mqty{4&-1\\3&-3\\-2&2}&\mqty{\imat{3}})x=\mqty(8\\9\\1), x \geq \zero} \]
and ends with
\[ \max\qty{(0,-\frac54,-\frac14,0,0)x + 6 : \mqty(1&-\frac14&\frac14&0&0\\0&-\frac94&-\frac34&1&0\\0&\frac32&\frac12&0&1)x=\mqty(2\\3\\5), x \geq \zero} \]

For each variable $i$, either $x_i = 0$ (it is non-basic) or $c_i = 0$ (it is basic).
Recall: objective coefficients are $c' = c - y\trans A$ where $y = A_B^{-\intercal}c_B$ is the dual solution.

Then, $c'_i = 0$ implies $c_i - y\trans A_i = 0$ or $y\trans A_i = c_i$ or $A_i\trans y = c_i$.
That is, the $i$\xth constraint in the dual is tight.

That is, either $x_i = 0$ or the $i$-th constraint is tight,
so all simplex BFS's satisfy the CS conditions.

Also, when picking $c_e > 0$ for the next iteration,
$A_e\trans y < c_e$, i.e., we picked an infeasible $y$ to the dual
(since the dual constraint is $Ay \leq c$)

That is, whenever positive coefficients exist, $y$ is infeasible
for the dual. Once the coefficients are non-positive, $y$ is
feasible and we have an optimal solution.

\section{Geometry of the dual}

\begin{example}
  For arbitrary $c$,
  $\max\qty{c\trans x : \mqty(\mqty{\imat{2}}\\\mqty{1&2\\-1&0\\0&-1})x \leq \mqty(4\\3\\8\\0\\0)}$.
\end{example}
\begin{sol}
  (imagine polyhedron bounded by $(0,0)$, $(0,3)$, $(2,3)$, $(4,2)$, and $(4,0)$)

  $(4,2)\trans$ is the optimal solution whenever the slope is between
  the two constraints defining $(4,2)\trans$, i.e.,
  $-\infty \leq m \leq -\frac12$.

  Formally, any $c$ in the \term{cone} of $(1,0)\trans$ and $(1,2)\trans$ (the tight constraints).
\end{sol}

\begin{defn*}[cone (generated by $x$)]
  The set $\{\lambda\trans x : \lambda \geq \zero\}$
\end{defn*}

\begin{lemma}[cone of tight constraints]
  $c$ is in the cone of tight constraints for $\bar x$
  if and only if $\bar x$ is optimal for $c$.
\end{lemma}

\lecture{June 29}

\begin{example}
  Consider the primal-dual pair
  $\max\qty{(c_1,c_2)x : \mqty(\imat{2}\\1&2\\-1&0\\0&-1)x \leq \mqty(4\\3\\8\\0\\0)}$
  and
  $\min\qty{(4,3,8,0,0)y : \mqty(1&0&1&-1&0\\0&1&2&0&-1)y = \mqty(c_1\\c_2), y \geq \zero}$
\end{example}
\begin{sol}
  Suppose $c$ makes $\bar x = (4,2)\trans$ optimal.
  Let $y$ be the correponding optimal dual solution.
  The dual constraints are
  \[ \mqty(1\\0)y_1 + \mqty(0\\1)y_2 + \mqty(1\\2)y_3 + \mqty(-1\\0)y_4 + \mqty(0\\-1)y_5 = \mqty(c_1\\c_2) \]

  CS conditions: either $y_j = 0$ or the $j$-th primal constraint
  is tight. Here, constraints 1 and 3 are tight.
  Then, $y_2 = y_4 = y_5 = 0$ and we have
  \[ \mqty(1\\0)y_1 + \mqty(1\\2)y_3 = \mqty(c_1\\c_2) \]
  with $y_1, y_3 \geq 0$.

  That is, $c$ is in the cone formed by the tight constraints for $\bar x$.
\end{sol}

\begin{lemma}[Farkas' lemma]
  Given $A$ and $b$, either there exists $x$
  such that $Ax = b$ and $x \geq \zero$,
  or there exists $y$ such that $y\trans A \geq \zero\trans$ and $y\trans b < 0$.

  Equivalently, either a solution or certificate of infeasibility exists.
\end{lemma}
\begin{prf}
  First, notice that having a certificate of
  infeasibility proves that there cannot be a solution to $Ax = b$
  with $x \geq \zero$. Thus, both cannot hold.

  Now, suppose the first condition is false.
  Consider $\max\{\zero\trans x : Ax = b, x \geq \zero\}$.

  This is infeasible by supposition.
  The dual is $\min\{b\trans y : A\trans y \geq \zero\}$
  which is trivially feasible since it admits $y = \zero$, so it
  must be unbounded.

  Then, there must exist a solution $\bar y$ with negative objective value,
  i.e., $b\trans \bar y < 0$ and $A\trans y \geq \zero$, as desired.
\end{prf}

Geometrically, consider the cone of column vectors.
That is, all $b$ such that $Ax = b$, $x \geq \zero$.

If $b$ is in the cone, we are done.
If $b$ is not in the cone, then there must exist a hyperplane with normal $y$
between the cone and $b$.
Such a $y$ will give by definition $y\trans b < 0$ and $y\trans A > \zero\trans$.

\section{Shortest paths, revisited}
\lecture{July 4}

Recall the formulation for shortest path problems:
$\min\{c\trans x : \sum_{e \in \delta(S)}x_e \geq 1, x_e \geq 0, x_e \in \Z\}$

Since there are $O(2^n)$ constraints and it is an IP, this is hard to solve.

Instead, ignore integrality and take the dual:
$\max\{\mathbb1\trans y : \sum_{e \in \delta(S)}y_S \leq c, y \geq \zero\}$.

There are dual variables $y_S$ corresponding to each $s,t$-cut
and dual constraints for each edge.
The objective function is the sum of all $y$ variables.

Dual constraints: One constraint for each edge $e$, coefficient
of $y_S$ is 1 if $e$ is in the cut $\delta(S)$. That is, sum
of $y$ values is at most the length of $e$

Interpret a dual solution $y$ as assigning a value $y_S$ to each
cut so that the ``sum of cuts'' across an edge is at most the edge's weight.

Since an $s,t$-path must cross all $s,t$-cuts, this means that
the ``sum of cuts'' is at most the sum of weights of the edges in
the path, i.e., the sum of cuts is at most the length of the path.

Equivalently, by weak duality, the objective value of the primal
(length of the path) is at least the objective value of the dual
(sum of all cuts)

Imagine $y_S$ as the width of a barrier that a path must cross, so
that a complete path must pass all barriers. This is the width
assignment.

Since we are maximizing, the barriers will expand to fill the path.

Take the dual and consider CS conditions:
\begin{itemize}[noitemsep]
  \item If $e$ is in the path, then the sum of width of all cuts
        containing $e$ is equal to its length and $e$ is tight
  \item If $y_S$ has positive width, then exactly one edge in
        $\delta(S)$ is in the path
\end{itemize}
Therefore, we can construct an algorithm to raise the $y$-value of $\delta(U)$
to create a tight edge.

\lecture{July 6}
\begin{example}
  $E = \{sa, sb, ab, at, bt\}$ with weights $w = \{50,30,30,40,70\}$.
\end{example}
\begin{sol}
  Walk through our ``algorithm'':
  \begin{itemize}
    \item Initialize $y = \zero$, $U = \{s\}$, $T = \varnothing$
    \item Iteration 1: $U = \{s\}$, $\delta(U) = \{sa,ab\}$.
          Can increase by 30 until $sb$ becomes tight, so set
          $y_{\{s\}} = 30$. Append $b$ to $U$.
    \item Iteration 2: $U = \{s, b\}$, $\delta(U) = \{sa,ab,bt\}$.
          Can increase by 20 until $sa$ becomes tight, so set
          $y_{\{s,b\}} = 20$. Append $a$ to $U$.
    \item Iteration 3: $U = \{s,b,a\}$, $\delta(U) = \{at, bt\}$.
          Can increase by 40 until $at$ becomes tight, so set
          $y_{\{s,b,a\}} = 40$. Append $t$ to $U$.
  \end{itemize}
  Tight edges $T = \{sb,sa,at\}$ so select path $x_{sa}=x_{at}=1$
  with $y_{\{s\}}=30$, $y_{\{s_b\}}=20$, $y_{\{s,a,b\}}=40$.
\end{sol}

Now, formalize the algorithm.

\begin{defn}[slack]
  The length of $e$ minus width of all cuts through $e$, i.e.,
  $\operatorname{slack}(e) = c_e - \sum_{\{S : e \in \delta(S)\}}y_S$
\end{defn}

\begin{algorithm}
  \caption{Shortest path algorithm}
  \begin{algorithmic}[1]
    \State Set $y = \zero$, $U = \{s\}$, $T = \varnothing$
    \While{$t \not\in U$}
    \State Calculate slack of all edges in $\delta(U)$
    \State Pick edge $uv$ with minimum slack where $u \in U$, $v \not\in U$
    \State Set $y_U := \operatorname{slack}(uv)$
    \State $U := U \cup \{v\}$, $T := T \cup \{uv\}$
    \EndWhile
    \State Output an $s,t$-path in $T$. Use $y$ as an optimal dual solution.
  \end{algorithmic}
\end{algorithm}

\begin{example}
  $V = \{s,a,b,c,t\}$,
  $E = \{sa,sb,sc,ab,bc,at,bt,ct\}$ with
  $w = \{20,50,30,20,10,50,40,40\}$.
\end{example}
\begin{sol}
  Perform the algorithm:
  \begin{enumerate}
    \item $U = \{s\}$, $T = \varnothing$, $\delta(U) = \{sa,sb,sc\}$.

          Slacks are 20, 50, and 30, so set $y_{\{s\}} = 20$ to get tight $sa$.

    \item $U = \{s,a\}$, $T = \{sa\}$, $\delta(U) = \{at,ab,sb,sc\}$.

          Slacks are 50, 20, 50, and 10, so set $y_{\{s,a\}} = 10$ to get tight $sc$.
    \item $U = \{s,a,c\}$, $T = \{sa, sc\}$, $\delta(U) = \{at,ab,sb,bc,ct\}$.

          Slacks are 40, 10, 20, 10, 40, so set $y_{\{s,a,c\}}=10$ to get
          tight $ab$ and $bc$. Arbitrarily pick $ab$.
    \item $U = \{s,a,b,c\}$, $T = \{sa,sc,ab\}$, $\delta(U) = \{at,bt,ct\}$.

          Slacks are 30, 40, and 30, so set $y_{\{s,a,b,c\}}=30$ to get
          tight $at$ and $ct$. Arbitrarily pick $ct$.
    \item $U = \{s,a,b,c,t\}$, $T = \{sa,sc,ab,ct\}$.

          Conclude with path $\{sc,ct\}$.
  \end{enumerate}
\end{sol}

Claim (proof omitted) that the algorithm produces a path that
satisfies the second CS condition (i.e., every cut with positive width
crosses only one edge in the path)

Sketch: To cross a cut more than once, must go back and forth. To go
back, must select an edge going ``backwards'', i.e., selecting a
node already in $U$ to append to $U$. Contradiction.
