\chapter{Breadth First Search}

\section{Graph Theory Review}
Recall graph theory from MATH 239, specifically:

\begin{defn}[graph]
  A graph $G$ is a pair $(V,E)$
  where $V$ is a finite set of \term{vertices}
  and $E$ is a set of unordered pairs of distinct vertices, called \term{edges}.
  By convention, we write $n = \abs{V}$ and $m = \abs{E}$.
\end{defn}

Now, we can define some structures on a graph:

\begin{defn}[adjacency list]
  An array $A[1..n]$ such that $A[v]$ is a linked list
  containing all edges connected to $v$.
  This contains $2m$ list cells with total size $\Theta(n+m)$
  but takes more than $O(1)$ time to test if an edge exists.
\end{defn}

\begin{defn}[adjacency matrix]
  A matrix $M \in M_{n\times n}(\{0,1\})$
  such that $M[v,w] = 1$ if and only if $\{v,w\} \in E$.
  Size is $\Theta(n^2)$ but testing if an edge exists is $O(1)$.
\end{defn}

\begin{example}
  Write the adjacency list and matrix for
  \tikz[baseline=-17pt]\graph{{1,5}--{2,4}--3[y=-0.5];1--5,2--4,2--5};
\end{example}
\begin{sol}
  The adjacency list is:
  \begin{enumerate}[1,nosep]
    \item $\to 2 \to 5$
    \item $\to 1 \to 3 \to 4 \to 5$
    \item $\to 2 \to 4$
    \item $\to 2 \to 3 \to 5$
    \item $\to 1 \to 2 \to 4$
  \end{enumerate}
  and the matrix is $M = \begin{bmatrix}
      0 & 1 & 0 & 0 & 1 \\
      1 & 0 & 1 & 1 & 1 \\
      0 & 1 & 0 & 1 & 0 \\
      0 & 1 & 1 & 0 & 1 \\
      1 & 1 & 0 & 1 & 0
    \end{bmatrix}$
\end{sol}

\begin{defn*}[graph terminology]
  We also recall some terms from MATH 239:
  \begin{itemize}[nosep]
    \item A \term{path} is a sequence of vertices $v_1,\dotsc,v_k$
          such that $\{v_i,v_{i+1}\} \in E$ for all $i$.
          If a path from $v$ to $w$ exists, we write $v \leadsto w$.
    \item A \term{connected graph} has $v \leadsto w$ for all $v,w \in V$.
    \item A \term{cycle} is a path $v \leadsto v$ of length at least 3
          with all elements pairwise distinct.
    \item A \term{tree} is a graph with no cycles.
    \item A \term{rooted tree} is a tree with a vertex chosen to be the \term{root}.
    \item A \term{subgraph} of $G = (V,E)$ is a graph $G' = (V',E')$
          where $V' \subseteq V$, $E' \subseteq E$, and $u,v \in V'$ for all $uv \in E'$.
    \item A \term{connected component} of $G$ is a connected subgraph of $G$
          that is not a subset of any other connected subgraph.
  \end{itemize}
\end{defn*}

\section{Breadth-First Exploration of a Graph}

\begin{problem}
  Search a graph $G$ starting from a vertex $s$
  in order of distance from $s$.
\end{problem}

\begin{algorithm}
  \caption{\Call{BFS}{$G$, $s$}}
  \begin{algorithmic}[1]
    \State let $Q$ be an empty queue
    \State let $\vv{visited}$ be a boolean array of size $n$ with all entries set to $\bot$
    \State \Call{enqueue}{$s$, $Q$}
    \State $\vv{visited}[s] \gets \top$
    \While{$Q$ is not empty}
    \State $v \gets \Call{dequeue}{Q}$
    \For{$w$ neighbours of $v$} \label{line:bfs:forloop}
    \If{$\vv{visited}[w] = \bot$}
    \State \Call{enqueue}{$w$, $Q$}
    \State $\vv{visited}[w] \gets \top$
    \EndIf
    \EndFor
    \EndWhile
  \end{algorithmic}
\end{algorithm}

Each vertex is enqueued at most once and dequeued at most once,
which has cost $O(n)$.
Therefore, each adjacency list is read at most once.
The cost for the for loop is $O(\sum \deg v) = O(m)$ by the Handshaking Lemma.

Therefore, the total cost of \Call{BFS}{} is $O(n+m)$.

\begin{lemma}
  $\vv{visited}[v]$ is true for some vertex $v$
  if and only if $s \leadsto v$ in $G$.
\end{lemma}
\begin{prf}
  Let $s = v_0, \dotsc, v_K$ be the vertices with $\vv{visited}{v_i} = \top$,
  in order of discovery.
  By induction, we show that $s \leadsto v_i$.
  
  For $i = 0$, $v_0 = s$, so trivially $s \leadsto s$.

  Otherwise, suppose $s \leadsto v_j$ for all $j < i$.
  We are currently in the for loop for some vertex $w$ already visited.
  Therefore, by assumption, $s \leadsto w$.
  But since $v_i$ is a neighbour of $w$, $s \leadsto v_i$.
\end{prf}