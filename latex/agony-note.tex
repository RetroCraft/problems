\newcounter{chapter}
\newcommand{\chapter}[1]{
  \pagebreak
  \addtocounter{chapter}{1}
  {\begin{center}\Large\underline{\textbf{Part \Roman{chapter}:~#1}}\end{center}}
  \addcontentsline{toc}{section}{\hspace*{-1em}Part~\Roman{chapter}:~#1}
}
\titleformat*{\section}{\normalfont\bfseries\large}
\setlength\cftsecindent{1em}

\setlength\topsep{6pt}
\declaretheorem[style=definition,name=Definition,
  thmbox={M,leftmargin=1.5em,rightmargin=1.5em,nocut,headstyle=\textbf{Definition},bodystyle=\noindent},
  refname={definition,definitions},Refname={Definition,Definitions}]{defn}
\declaretheorem[name=Theorem,
  thmbox={L,leftmargin=1.5em,rightmargin=1.5em,nocut,headstyle=\textbf{Theorem},bodystyle=\noindent},
  refname={theorem,theorems},Refname={Theorem,Theorems}]{theorem}
\declaretheorem[name=Lemma,
  thmbox={M,leftmargin=1.5em,rightmargin=1.5em,nocut,headstyle=\textbf{Lemma},bodystyle=\noindent},
  refname={lemma,lemmas},Refname={Lemma,Lemmas}]{lemma}
\declaretheorem[name=Proposition,
  thmbox={S,leftmargin=1.5em,rightmargin=1.5em,nocut,headstyle=\textbf{Proposition},bodystyle=\noindent},
  refname={proposition,propositions},Refname={Proposition,Propositions}]{prop}
\declaretheorem[style=remark,unnumbered]{remark}
\declaretheorem[style=remark,unnumbered]{corollary}
\declaretheorem[style=definition,name=Example,numberwithin=section,
  refname={example,examples},Refname={Example,Examples}]{example}
