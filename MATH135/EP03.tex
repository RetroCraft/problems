\usepackage{physics}
\usepackage{amsfonts,amsmath,amssymb,amsthm}
\usepackage{enumerate}
\usepackage{titlesec}
\usepackage{fancyhdr}
\usepackage{multicol}

\headheight 13.6pt
\setlength{\headsep}{10pt}
\textwidth 15cm 
\textheight 24.3cm
\evensidemargin 6mm
\oddsidemargin 6mm
\topmargin -1.1cm
\setlength{\parskip}{1.5ex}

\author{James Ah Yong}

\pagestyle{fancy}
\fancyhf{}
\fancyfoot[c]{\thepage}
\makeatletter
\lhead{\@title}
\rhead{\@author}

\fancypagestyle{firstpage}{
    \fancyhf{}
    \rhead{\@author}
    \fancyfoot[c]{\thepage}
}

% Sets
\newcommand{\N}{\mathbb{N}}
\newcommand{\Z}{\mathbb{Z}}
\newcommand{\Q}{\mathbb{Q}}
\newcommand{\R}{\mathbb{R}}
\newcommand{\C}{\mathbb{C}}

% Functions
\DeclareMathOperator{\sgn}{sgn}
\DeclareMathOperator{\im}{im}

% Operators
\newcommand{\Rarr}{\Rightarrow}
\newcommand{\Larr}{\Leftarrow}

% Typesetting
\usepackage{array}   % for \newcolumntype macro
\newcolumntype{C}{>{$}c<{$}} % math version of "C" column type
\newcommand{\dlim}{\displaystyle\lim} % totally not \dfrac ripoff
\newcommand{\dilim}[1]{\dlim_{#1\to\infty}} % infinite limits
\newcommand{\ilim}[1]{\lim_{#1\to\infty}}
\usepackage{cancel}

% Auto-number questions
\newcommand{\QType}{Q}
\renewcommand{\theparagraph}{\textbf{\QType\ifnum\value{paragraph}<10 0\fi\arabic{paragraph}.}}
\setcounter{secnumdepth}{4}
\newcommand{\question}{\par\refstepcounter{paragraph}\theparagraph\space}

% Question sections
\titleformat{\section}{\normalsize\bfseries}{\thesection}{1em}{}
\newcommand{\qsection}[2]{%
    \renewcommand{\QType}{#2}
    \section*{#1}
    \refstepcounter{section}
}

\title{MATH 135 Fall 2020: Extra Practice 3}

\begin{document}
\thispagestyle{firstpage}

\textbf{\@title}

\qsection{Warm-Up Exercises}{WE}

\question Prove the following two quantified statements.
\begin{enumerate}[(a)]
  \item $\forall n \in \N, n+1 \geq 2$
        \begin{proof}
          Let $n \in \N$.
          Recall that 1 is the smallest natural.
          $n \geq 1 \iff n+1 \geq 2$.
        \end{proof}

  \item $\exists n \in \Z, \frac{5n-6}{3} \in \Z$
        \begin{proof}
          Select $n=3$. Then, $\frac{5n-6}{3}=\frac{15-6}{3}=\frac{9}{3}=3 \in \Z$.
        \end{proof}
\end{enumerate}


\question Prove that for all $k \in \Z$, if $k$ is odd, then $4k + 7$ is odd.
\begin{proof}
  Let $k$ be an odd integer.
  Then, it can be written as $2n+1$ for some integer $n$.

  Substituting, $4k+7 = 4(2n+1) + 7 = 8n+11 =2(4n+5)+1$.
  By definition, since $4k+7$ can be written as $2m+1$ where $m=4n+5$ is an integer, it is odd.
\end{proof}


\question Consider the following proposition
\begin{center}
  \emph{For all $a,b\in\Z$, if $a^3 \mid b^3$, then $a \mid b$.}
\end{center}
We now give three erroneous proofs of this proposition. Identify the major error in each proof, and explain why it is an error.
\begin{enumerate}[(a)]
  \item \emph{Consider $a = 2$, $b = 4$. Then $a^3 = 8$ and $b^3 = 64$. We see that $a^3 \mid b^3$ since $8 \mid 64$. Since $2 \mid 4$, we have $a \mid b$.}

        This proof is erroneous as it only considers one specific case of $a$ and $b$ and not the general case of integer $a$ and $b$.

  \item \emph{Since $a \mid b$, there exists $k\in\Z$ such that $b=ka$. By cubing both sides, we get $b^3 = k^3a^3$. Since $k^3\in\Z$, then $a^3\mid b^3$.}

        This proof supposes the conclusion instead of the hypothesis.

  \item \emph{Since $a^3 \mid b^3$, there exists $k\in\Z$ such that $b^3=ka^3$. Then $b=(ka^2/b^2)a$, hence $a \mid b$.}

        The proof does not guarantee that $\frac{ka^2}{b^2}$ is an integer.
\end{enumerate}


\question Let $x$ be a real number. Prove that if $x^3-5x^2+3x \neq 15$, then $x \neq 5$.
\begin{proof}
  Suppose for the contrapositive that $x=5$.
  Then, $x^3-5x^2+3x=(5)^3-5(5)^2+3(5)=15$, as required.
  Since the contrapositive is true, the original implication must be true.
\end{proof}


\question Prove that there do not exist integers $x$ and $y$ such that $2x + 4y = 3$.
\begin{proof}
  For the sake of contradiction, suppose the negation is true.

  Consider the negation of the statement: there exist integers $x$ and $y$ such that $2x + 4y = 3$.
  Let $x$ and $y$ be such integers.
  Then, $x+2y$ is an integer.
  Therefore, $2x+4y=2(x+2y)$ is even.
  However, 3 is odd.
  An integer cannot be both even and odd, therefore, the negation is false, and the original statement is true.
\end{proof}


\question Prove that an integer is even if and only if its square is an even integer.
\begin{proof}
  ($\Rarr$) Let $n$ be an even integer.
  Then, $n=2k$ for some integer $k$.
  $n^2=(2k)^2=4k^2=2(2k^2)$.
  Since $2k^2$ is an integer, $n^2$ is even.

  ($\Larr$) Let $n$ be an even square integer.
  Then, $n=2k$ for some integer $k$ and $n=x\cdot x$ for some integer $x$.
  Since $2k=x\cdot x$, and 2 is prime, 2 must divide $x$.
  Therefore, $x=2y$ for some integer $y$, which is the definition of being even.

  Since the implication is true in both directions, the biconditional is true.
\end{proof}



\qsection{Recommended Problems}{RP}

\question Prove that $x^2 + 9 \geq 6x$ for all real numbers $x$.
\begin{proof}
  Let $x$ be a real number.
  $x^2+9 \geq 6x \iff x^2-6x+9 \geq 0 \iff (x-3)^2 \geq 0$.
  Since the square of a real is always non-negative, the statements are true.
\end{proof}


\question Prove that for all $r\in\R$ where $r\neq -1$ and $r\neq -2$,
\begin{equation*}
  \frac{2^{r+1}}{r+2}-\frac{2^r}{r+1}=\frac{r(2^r)}{(r+1)(r+2)}
\end{equation*}
\begin{proof}
  Let $r$ be a real number that is neither $-1$ nor $-2$.
  Then,
  \begin{align*}
    LHS & = \frac{2^{r+1}}{r+2}-\frac{2^r}{r+1}                                      \\
        & = \frac{2^{r+1}(r+1) - 2^r(r+2)}{(r+1)(r+2)}                               \\
        & = \frac{r2^{r+1} + 2^{r+1} - r2^r - 2\cdot 2^r}{(r+1)(r+2)}                \\
        & = \frac{r2^{r+1} + \cancel{2^{r+1}} - r2^r - \cancel{2^{r+1}}}{(r+1)(r+2)} \\
        & = \frac{r(2^{r+1} - 2^r)}{(r+1)(r+2)}                                      \\
        & = \frac{r(2^r\cdot 2 - 2^r)}{(r+1)(r+2)}                                   \\
        & = \frac{r(2^r + \cancel{2^r} - \cancel{2^r})}{(r+1)(r+2)}                  \\
        & = \frac{r(2^r)}{(r+1)(r+2)}                                                \\
        & = RHS
  \end{align*}
  Since the left side equals the right side, the equality is true.
\end{proof}


\question Prove that there exists a real number $x$ such that $x^2 - 6x + 11 \leq 2$.
\begin{proof}
  Let $x=3$.
  $x^2-6x+11=(3)^2-6(3)+11=9-18+11=2 \leq 2$, as required.
  Since 3 is a real number, the statement is true.
\end{proof}


\question Prove or disprove each of the following statements.
\begin{enumerate}[(a)]
  \item $\forall n\in\Z$, $\frac{5n-6}{3}$ is an integer.
        \begin{proof}
          Let $n=1$ as a counter-example.
          Then, $\frac{5n-6}{3}=\frac{5-6}{3}=-\frac{1}{3}$, which is not an integer.
          Therefore, the statement is false.
        \end{proof}

  \item $\forall a\in\Z$, $a^3+a+2$ is even.
        \begin{proof}
          Let $a$ be an integer. Then, $a$ is either even or odd.
          Suppose that $a$ is even and can be written as $a=2k$ for an integer $k$.
          Then, $a^3+a+2=(2k)^3+2k+2=8k^3+2k+2=2(4k^3+k+1)$, an even number.

          Suppose $a$ is odd and can be written as $a=2k+1$ for an integer $k$.
          Then, $a^3+a+2 = (2k+1)^3 + (2k+1) + 2 = 8k^3 + 12k^2 + 8k + 4 = 2(4k^3 + 6k^2 + 4k + 2)$, an even number.

          Therefore, the statement is true.
        \end{proof}

  \item For every prime number $p$, $p + 7$ is composite.
        \begin{proof}
          Let $p$ be a prime number.

          If $p$ is even, then $p = 2$, and $p + 7 = 9$ which is composite.

          If $p$ is odd, $p = 2k + 1$ for some integer $k \geq 0$ (as there are no negative primes).
          Then, $p + 7 = 2k + 8 = 2(k + 4)$, which is even.
          The only even prime is 2, but $2k + 8 \geq 8$, so $p + 7$ is composite.

          Therefore, since all primes are either even or odd, $p + 7$ is composite for all primes.
        \end{proof}

  \item For all $x \in \R$, $|x-3| + |x-7| \geq 10$.
        \begin{proof}
          Let $x=3$ as a counter-example.
          Then, $|x-3|+|x-7|=|(3)-3|+|(3)-7|=0+4=4 \not\geq 10$.
          Therefore, the statement is false.
        \end{proof}

  \item There exists a natural number $m < 123456$ such that $123456m$ is a perfect square.
        \begin{proof}
          Let $m=1929$, which is a natural number less than $123456$.
          Then, $123456m=238146624=15432^2$.
          Since $123456m$ can be written as $n^2$ where $n=15432\in\Z$, it is a perfect square, and the statement is true.
        \end{proof}

  \item $\exists k\in\Z, 8\nmid(4k^2+12k+8)$.
        \begin{proof}
          Consider the negation, $\forall k\in\Z, 8\mid(4k^2+12k+8)$.
          Notice that the open sentence is logically equivalent to $8 \mid (4k^2+12k)$.
          Let $k$ be a natural number. Then, $k$ is either even or odd.

          Suppose that $k$ is even and can be written as $k=2n$.
          Then, $4k^2=16n^2=8(2n^2)$, so $8 \mid 4k^2$.
          Likewise, $12k=24n=8(3n)$, so $8 \mid 12k$.
          By DIC, $8 \mid (4k^2+12k)$.

          Now, suppose that $k$ is odd and can be written as $k=2n+1$.
          Then, $4k^2+12k = 4(4n^2+2n+1)+12(2n+1) = 16n^2 + 40n + 16 = 8(2n^2+5n+1)$, so $8 \mid (4k^2+12k)$.

          Therefore, the negation is true, so the original statement is false.
        \end{proof}
\end{enumerate}


\question Prove or disprove each of the following statements involving nested quantifiers.
\begin{enumerate}[(a)]
  \item For all $n \in\Z$, there exists an integer $k > 2$ such that $k \mid (n^3 - n)$.
        \begin{proof}
          Let $n$ be an integer. If $n=0$ or $n=\pm1$, $n^3-n=0$ and all integers (including any $k$) divide zero.

          If $n > 1$, we select $k=n+1 > 2$.
          Factor: $n^3-n = n(n-1)(n+1)$.
          Then, $n^3 - n = [n(n-1)](n+1)$, so $k \mid (n^3-n)$.

          If $n < 1$, first let $m = -n$ so $n^3-n = (-m)^3+m = -(m^3-m)$.
          Now, select $k = m+1 > 2$.
          Then, $n^3 - n = -m(m-1)(m+1)$, so $k \mid (n^3-n)$.

          Therefore, the statement is true.
        \end{proof}

  \item For every positive integer $a$, there exists an integer $b$ with $|b| < a$ such that $b$ divides $a$.
        \begin{proof}
          We disprove by counter-example.
          Let $a = 1$.
          Then, $|b| < 1$, and the only such integer is 0.
          However, $0 \nmid 1$ since there is no integer $k$ where $k \cdot 0 = 1$.
          Therefore, the statement is false.
        \end{proof}

  \item There exists an integer $n$ such that $m(n - 3) < 1$ for every integer $m$.
        \begin{proof}
          Choose $n = 3$ and let $m$ be an integer.
          Then, $m(n-3) = m(3-3) = 0 < 1$, as desired.
          Therefore, the statement is true.
        \end{proof}

  \item $\exists n\in\N, \forall m\in\Z, -nm < 0$
        \begin{proof}
          Consider the negation $\forall n\in\N, \exists m\in\Z, -nm \geq 0$.
          Let $n$ be a natural number.

          We can choose an integer $m$, namely $m=-1$.
          Notice that because $n$ is a natural number, $n > 0 \iff n(-1)(-1) > 0 \iff -nm > 0 \iff -nm \geq 0$.

          Because the negation is true, the original statement is false.
        \end{proof}
\end{enumerate}


\question Prove that for all integers $a$ and $b$, if $a \mid (2b + 3)$ and $a \mid (3b + 5)$, then $a \mid 13$.
\begin{proof}
  Let $a$ and $b$ be arbitrary integers, and assume that $a \mid (2b+3)$ and $a \mid (3b+5)$.

  Recall the divisibility of integer combinations: since $2b+3$ and $3b+5$ are integers,
  $a$ must divide $n(2b+3) + m(3b+5)$ for all integers $n$ and $m$.
  Specifically, let $n = -39$ and $m = 26$.
  Then, $n(2b+3) + m(3b+5) = -78b - 117 + 78b + 130 = 13$.
  Therefore, $a \mid 13$.
\end{proof}


\question Let $a$, $b$, $c$ and $d$ be positive integers.
Prove that if $\frac a b < \frac c d$, then $\frac a b < \frac{a+c}{b+d} < \frac c d$.
\begin{proof}
  Let $a$, $b$, $c$ and $d$ all be positive integers.
  Suppose $\frac a b < \frac c d$, which means $ad < bc$, because $b$ and $d$ are positive.
  Now, adding $ab$ and $cd$ to both sides, respectively:
  \begin{alignat*}{3}
    ad        & < bc              & \hspace{4em} ad              & < bc        \\
    ad + ab   & < bc + ab         & \hspace{4em} ad + cd         & < bc + cd   \\
    a(b+d)    & < b(c+a)          & \hspace{4em} d(a+c)          & < c(b+d)    \\
    \frac a b & < \frac{a+c}{b+d} & \hspace{4em} \frac{a+c}{b+d} & < \frac c d
  \end{alignat*}
  Therefore, $\frac a b < \frac{a+c}{b+d} < \frac c d$.
\end{proof}


\question Prove that for all integers $n$, if $1-n^2 > 0$, then $3n - 2$ is an even integer.
\begin{proof}
  Let $n$ be an integer where $1-n^2 > 0$.
  Since squares of integers are positive, $1 > n^2$.
  This is only true when $|n| < 1$, but the only such integer is 0.
  $3(0)-2=-2$, which is even.
\end{proof}


\question Let $a$ and $b$ be integers. Prove each of the following implications.
\begin{enumerate}[(a)]
  \item If $ab = 4$, then $(a-b)^3 - 9(a-b) = 0$
        \begin{proof}
          Let $a$ and $b$ be integers with product 4.

          Consider the possible values for $a$ and $b$.
          4's divisor pairs are $(\pm 1, \pm 4)$ and $(\pm 2, \pm 2)$.
          For all of these pairs, either $a=b$ or $a=b\pm3$.
          Specifically:
          \begin{itemize}
            \item If $b=\pm2$, then $a=b$
            \item If $b=1$, then $a=4=b+3$ (for $b=-1$, $a=-4=b-3$)
            \item If $b=4$, then $a=1=b-3$ (for $b=-4$, $a=-1=b+3$)
          \end{itemize}

          Notice that the conclusion factors to $(a-b)(a-b-3)(a-b+3)=0$.
          This is true when $a=b$ or $a=b\pm3$, which we just showed.
        \end{proof}
  \item If $a$ and $b$ are positive, then $a^2(b+1) + b^2(a+1) \geq 4ab$
        \begin{proof}
          Let $a$ and $b$ be positive integers, i.e., at least 1.

          If $a$ and $b$ are both at least 1, then $a+b \geq 2$, or $a+b-2 \geq 0$.
          Likewise, $ab$ is a positive integer, so $ab(a+b-2) \geq 0$.
          \begin{align*}
            ab(a+b-2)         & \geq 0 \\
            a^2b + b^2a - 2ab & \geq 0
          \end{align*}
          Recall that squares are non-negative:
          \begin{align*}
            (a-b)^2 + a^2b + b^2a - 2ab         & \geq 0            \\
            a^2 - 2ab + b^2 + a^2b + b^2a - 2ab & \geq 0            \\
            a^2 + a^2b + b^2 + b^2a             & \geq 4ab          \\
            a^2(b+1) + b^2(a+1)                 & \geq 4ab \qedhere
          \end{align*}
        \end{proof}
\end{enumerate}


\question Let $a$, $b$, $c$ and $d$ be integers.
Prove that if $a \mid b$ and $b \mid c$ and $c \mid d$, then $a \mid d$.
\begin{proof}
  Let $a$, $b$, $c$, and $d$ be integers where $a \mid b$, $b \mid c$, and $c \mid d$.

  Recall the transitivity of divisibility:
  for integers $x$, $y$, and $z$, if $x \mid y$ and $y \mid z$, then $x \mid z$.

  Then, $a \mid b$ and $b \mid c$ implies $a \mid c$.
  Likewise, $a \mid c$ and $c \mid d$ implies $a \mid d$.
\end{proof}


\question Prove that the product of any four consecutive integers is one less than a perfect square.
\begin{proof}
  The statement is equivalently expressed that for any integer $k$,
  $k(k+1)(k+2)(k+3)=r^2-1$ for some positive integer $r$.

  Let $k$ be an integer.
  The product $k(k+1)(k+2)(k+3)$ expands to $k^4 + 6k^3 + 11k^2 + 6k$.
  As a fourth-degree polynomial, its square root would be a quadratic.

  Expanding algebraically, the square of a quadratic in $x$, $ax^2+bx+c$, is $a^2x^4 + 2abx^3 + (2ac+b^2)x^2 + 2bcx + c^2$.

  Notice that when $a=c=1$ and $b=3$, this formula becomes $x^4 + 6x^3 + 11x^2 + 6x + 1$.
  The coefficients on $x$ are precisely our original product (with a constant +1).
  Therefore, $x^4 + 6x^3 + 11x^2 + 6x = (x^2+3x+1)^2 - 1$ for all real $x$.

  We can now let $r = k^2 + 3k + 1$, which is a positive integer such that
  \begin{align*}
    r^2-1
     & = (k^2 + 3k + 1)^2 - 1            \\
     & = k^4 + 6k^3 + 11k^2 + 6k + 1 - 1 \\
     & = k(k+1)(k+2)(k+3)
  \end{align*}
  and conclude that the statement is true.
\end{proof}


\question Let $x$, $y\in\R$. Prove that if $xy + 2x - 3y - 6 < 0$, then $x < 3$ or $y < -2$.
\begin{proof}
  Let $x$ and $y$ be real solutions to $xy + 2x - 3y - 6 < 0$.

  Notice that the inequality factors to $(x-3)(y+2) < 0$.
  This is true when $x$ and $y$ are non-zero and have opposite signs: either $x < 3$ and $y > -2$, or $x > 3$ and $y < -2$.
  Therefore, either $x < 3$ or $y < -2$.
\end{proof}


\question Is the following implication true for all integers $a$, $b$ and $c$? Prove that your answer is correct.
\begin{center}
  \emph{$a \mid b$ if and only if $ac \mid bc$}
\end{center}
\begin{proof}[Solution]
  The statement is false.
  Consider the counterexample $a=2$, $b=3$, and $c=0$.
  Then, the backwards implication's hypothesis is true ($0 \mid 0$) but the conclusion is false ($2 \nmid 3$).
\end{proof}


\question Let $n$ be an integer. Prove that $2 \mid (n^4 - 3)$ if and only if $4 \mid (n^2 + 3)$.
\begin{proof}
  Consider the two implications of the biconditional statement:

  ($\Rarr$) Let $n$ be an integer where 2 divides $n^4 - 3$.
  This means there is an integer $k$ where $n^4-3 = 2k$.
  Notice that this means $n^4-3$ is even, so $n^4=2(k+1)+1$ is odd.
  Even numbers raised to the fourth power remain even, so $n$ must be odd.
  Therefore, $n = 2m+1$ for some integer $m$.

  Now, expand $n^2+3$:
  \begin{align*}
    n^2 + 3 & = (2m+1)^2 + 3      \\
            & = 4m^2 + 4m + 1 + 3 \\
            & = 4(m^2 + m + 1)
  \end{align*}
  Because $m^2 + m + 1$ is an integer, $4 \mid (n^2 + 3)$.

  ($\Larr$) Let $n$ be an integer where 4 divides $n^2 + 3$.
  This means there is an integer $k$ where $n^2 + 3 = 4k$ or $n^2 = 4k-3$, and
  \begin{align*}
    n^2     & = 4k-3              \\
    n^4     & = (4k-3)^2          \\
    n^4     & = 16k^2 - 24k + 9   \\
    n^4 - 3 & = 16k^2 - 24k + 6   \\
            & = 2(8k^2 - 12k + 3)
  \end{align*}
  Because $8k^2 - 12k + 3$ is an integer, $2 \mid (n^4 - 3)$.

  Therefore, since both expressions imply the other, $2 \mid (n^4 - 3)$ if and only if $4 \mid (n^2 + 3)$.
\end{proof}


\question Let $x$ and $y$ be integers. Prove that if $xy = 0$ then $x = 0$ or $y = 0$.
\begin{proof}
  Consider the contrapositive, $x \neq 0$ and $y \neq 0$ implies $xy \neq 0$.

  Let $x$ and $y$ be non-zero integers.
  WLOG, take $x \leq y$.

  Now, take cases of the signs of $x$ and $y$:
  \begin{itemize}
    \item If $0 < x \leq y$, then $xy > 0$, since two positive numbers' product is a positive number.
    \item $xy$ is also positive when $x \leq y < 0$, with two negative numbers.
    \item When $x < 0 < y$, i.e.\ the signs are opposite, $xy < 0$.
  \end{itemize}

  Since $xy$ can never be 0 for any combination of non-zero integers, the contrapositive, and by extension, the original implication, is true.
\end{proof}


\question Prove that $\forall a,b\in\Z,[(a \mid b \land b \mid a) \iff a = \pm b]$.
\begin{proof}
  Let $a$ and $b$ be integers.
  Suppose $a$ divides $b$ and vice versa.
  Equivalently, integers $p$ and $q$ exist such that $a = pb$ and $b = qa$.
  Substituting, $a = pb = p(qa) \iff 1 = pq \iff p = \frac{1}{q}$.

  The only integers of the form $\frac{1}{k}$ with integer $k$ are 1 and -1.
  Therefore, $p = \frac{1}{q}$ if and only if $p = \pm1$, i.e., $a = \pm b$.
\end{proof}


\question Let $a$ be an integer. Prove that $a^2 + 2a - 3$ is even if and only if $a$ is odd.
\begin{proof}
  Consider the two implications of the biconditional statement:

  ($\Rarr$) Let $a$ be an odd integer, or, $a = 2k+1$ for some integer $k$. Then,
  \begin{align*}{2}
    a^2 + 2a - 3                    \\
     & = (2k + 1)^2 + 2(2k + 1) - 3 \\
     & = 4k^2 + 4k + 1 + 4k + 2 - 3 \\
     & = 4k^2 + 8k - 2              \\
     & = 2(2k^2 + 4k - 1)
  \end{align*}
  which is even, because $2k^2+4k-1$ is an integer.

  ($\Larr$) Consider the contrapositive, where even $a$ implies odd $a^2 + 2a - 3$.
  Let $a$ be an even integer, i.e., $a = 2k$ for some integer $k$. Then,
  \begin{align*}
    a^2 + 2a - 3
     & = (2k)^2 + 2(2k) - 3   \\
     & = 4k^2 - 4k - 3        \\
     & = 2(2k^2 - 2k - 2) + 1
  \end{align*}
  which is odd, because $2k^2 - 2k - 2$ is an integer.
  Since the contrapositive is true, the original implication is also true.

  Therefore, since both implications hold, the statement is true.
\end{proof}


\question Prove or disprove each of the following for any integers $x$ and $y$.
\begin{enumerate}[(a)]
  \item If $2 \nmid xy$ then $2 \nmid x$ and $2 \nmid y$.
  \item If $2 \nmid y$ and $2 \nmid x$ then $2 \nmid xy$.
        \begin{proof}
          First, notice that if $2 \mid n$ for an integer $n$, then $n = 2k$ for some integer $k$.
          This is the definition of saying $n$ is even.
          Therefore, $2 \nmid n$ is the same as saying $n$ is not even, i.e., $n$ is odd.

          Let $x$ and $y$ be odd integers.
          Equivalently, $x = 2p + 1$ and $y = 2q + 1$ for some integers $p$ and $q$.
          Substituting into $xy$, $(2p + 1)(2q + 1) = 2pq + 2p + 2q + 1 = 2(pq + p + q) + 1$.
          By definition, since $pq + p + q$ is an integer, $xy$ is odd.

          Therefore, $x$ and $y$ are odd if and only if $xy$ is odd, so (a) and (b) are both true.
        \end{proof}
  \item If $10 \nmid xy$ then $10 \nmid x$ and $10 \nmid y$.
        \begin{proof}
          Consider the contrapositive, ``if $10 \mid x$ and $10 \mid y$ then $10 \mid xy$''.
          Let $x$ and $y$ be integers where 10 divides both.

          This means $x = 10n$ and $y = 10m$ for some integers $n$ and $m$.
          Then, $xy = (10n)(10m) = 10(10nm)$, and since $10nm$ is an integer, $10 \mid xy$.

          Since the contrapositive, the original implication is true.
        \end{proof}
  \item If $10 \nmid x$ and $10 \nmid y$ then $10 \nmid xy$.
        \begin{proof}
          For a counterexample, let $x = 5$ and $y = 2$.
          $10 \nmid x$ and $10 \nmid y$ since $2 < 5 < 10$.

          However, $xy = 10$ and $10 \mid 10$, so the statement is false.
        \end{proof}
\end{enumerate}


\question For every odd integer $n$, prove that there exists a unique integer $m$ such that $n^2 = 8m + 1$.
\begin{proof}
  Let $n$ be an odd integer, i.e., $n = 2k+1$ for some other integer $k$.
  Then, $n^2 = (2k+1)^2 = 4k^2 + 4k + 1$.
  We must show that $8m = 4k^2 + 4k \iff 2m = k^2 + k$, or, $k^2 + k$ is even.
  Now, consider $k$'s parity:

  Suppose $k$ is even.
  Then, $k = 2p$ for an integer $p$ and $k^2 + k = 4p^2 + 2p = 2(2p^2 + p)$,
  which means that $k^2+k$ is even.

  Now, suppose $k$ is odd.
  Then, $k = 2p+1$ for an integer $p$ and $k^2 + k = 4p^2 + 6p + 2 = 2(2p^2 + 3p + 1)$,
  which means that $k^2+k$ is even.

  Since $k$ is either even or odd, $k^2 + k$ is even for all $k$.

  Therefore, $m = \frac{k^2+k}{2}$ is an integer, but recall $k = \frac{n-1}{2}$, so:
  \begin{equation*}
    m = \frac{k^2+k}{2}
    = \frac{\left(\frac{n-1}{2}\right)^2+\frac{n-1}{2}}{2}
    = \frac{\frac{(n-1)^2}{4}+\frac{n-1}{2}}{2}
    = \frac{(n-1)^2 + 2(n-1)}{8}
    = \frac{n^2-1}{8}
  \end{equation*}
  is the same integer, but $m = \frac{n^2-1}{8}$ if and only if $n^2 = 8m + 1$, so the statement is true.
\end{proof}


\question Prove the following statements.
\begin{enumerate}[(a)]
  \item There is no smallest positive real number.
        \begin{proof}
          Suppose, for a contradiction, that there is a smallest positive real number $n$.
          Now, consider $\frac{n}{2}$.

          This number is still real ($\R$ is closed under division).
          $\frac{n}{2}$ is positive because $n$ and 2 are positive.
          Therefore, $\frac{n}{2}$ is a positive real number.

          Clearly $\frac{n}{2} < n$, so the supposition must be false.
          Therefore, there is no smallest positive real number.
        \end{proof}

  \item For every even integer $n$, $n$ cannot be expressed as the sum of three odd integers.
        \begin{proof}
          We will prove by the contrapositive.
          Let $r$, $s$, and $t$ be arbitrary integers so $2r+1$, $2s+1$, and $2t+1$ are odd.

          Then, $r + s + t = 2r + 2s + 2t + 3 = 2(r + s + t + 1) + 1$, so this sum is odd.

          Therefore, the sum of three odd integers is always odd, and no even integer may be expressed as such a sum.
        \end{proof}

  \item Let $a,b\in\Z$.
        If $a$ is an even integer and $b$ is an odd integer, then $4 \nmid (a^2 + 2b^2)$.
        \begin{proof}
          Let $a$ and $b$ be integers and suppose, for a contradiction, that the negation is true.
          Then, $a$ is even, $b$ is odd, and $4 \mid (a^2 + 2b^2)$.

          Rewrite $a=2n$ and $b=2m+1$ with some integers $n$ and $m$.
          Now, expand $a^2 + 2b^2 = (2n)^2 + 2(2m+1)^2 = 4n^2 + 8m^2 + 8m + 2$.

          We can extract a factor of four, and get $4 \mid (4(n^2 + 2m^2 + 2m) + 2)$.
          Then, 4 must divide 2, which is a contradiction.

          Therefore, the negation is false, so the original statement is true.
        \end{proof}

  \item For every integer $m$ with $2 \mid m$ and $4 \nmid m$, there are no integers $x$ and $y$ that satisfy $x^2 + 3y^2 = m$.
        \begin{proof}
          Those negations are ugly so we can consider the contrapositive:
          \begin{center}
            \emph{If $x^2 + 3y^2 = m$ has integer solutions in $x$ and $y$, $m$ is odd or $4 \mid m$.}
          \end{center}
          Notice that $2 \nmid m \lor 4 \mid m \equiv 4 \mid m$.

          Suppose integers $x$ and $y$ so $x^2 + 3y^2 = m$ exist.
          Break into cases for $x$ and $y$'s parities.

          \begin{itemize}
            \item If $x$ and $y$ are odd, they can respectively be expressed as $2p+1$ and $2q+1$ for integers $p$ and $q$.
                  Then, $m = (2p+1)^2 + 3(2q+1)^2 = 4p^2 + 4p + 1 + 3(4q^2 + 4q + 1)$.
                  This simplifies to $4(p^2+3q^2+p+3q+1)$, so $m \mid 4$.
            \item If $x$ and $y$ are even, let $x = 2p$ and $y = 2q$.
                  Then, $m = (2p)^2 + 3(2q)^2 = 4p^2 + 12q^2 = 4(p^2 + 3q^2)$, so $m \mid 4$.
            \item If $x$ is odd and $y$ is even, let $x = 2p+1$ and $y = 2q$.
                  Then, $m = (2p+1)^2 + 3(2q)^2 = 4p^2 + 4p + 1 + 3(4q^2) = 2(2p^2 + 2p + 6q^2) + 1$, so $m$ is odd.
            \item If $x$ is even and $y$ is odd, let $x = 2p$ and $y = 2q+1$.
                  Then, $m = (2p)^2 + 3(2q+q)^2 = 4p^2 + 3(4q^2+4q^2+1) = 2(2p^2 + 6q^2 + 6q + 1) + 1$, so $m$ is odd.
          \end{itemize}

          Therefore, either 4 divides $m$ or $m$ is odd, so the contrapositive, and by extension the original statement, is true.
        \end{proof}

  \item The sum of a rational number and an irrational number is irrational.
        \begin{proof}
          First, recall that rational numbers are those which can be expressed by $\frac{p}{q}$ for integers $p$ and $q$.

          Let $x$ be a rational number and suppose for a contradiction that $y$ is irrational such that $x+y$ is rational.

          Then, $x = \frac{p}{q}$ for integers $p$ and $q$.
          Also, $x+y = \frac{n}{m}$ for integers $n$ and $m$.
          Substituting, $\frac{p}{q} + y = \frac{n}{m}$.
          Rearranging,
          \begin{equation*}
            \frac{p}{q} + y = \frac{n}{m}        \iff
            p+yq            = \frac{qn}{m}       \iff
            y               = \frac{qn - mp}{qm}
          \end{equation*}
          but if $y$ equals the ratio of two integers ($qn-mp$ and $qm$), by definition, $y$ is rational.

          Therefore, by contradiction, the sum of a rational number and an irrational number is irrational.
        \end{proof}

  \item Let $x$ be a non-zero real number. If $x + \frac{1}{x} < 2$, then $x < 0$.
        \begin{proof}
          Let $x$ be a non-zero real such that $x + \frac{1}{x} < 2$. Then,
          \begin{align*}
            x + \frac{1}{x}        & < 2 \\
            x + \frac{1}{x} - 2    & < 0 \\
            \frac{x^2 + 1 - 2x}{x} & < 0 \\
            \frac{(x-1)^2}{x}      & < 0
          \end{align*}
          Because $(x-1)^2$ is a square, so is always non-negative, $\frac{1}{x} < 0$, which is true if and only if $x < 0$.
        \end{proof}
\end{enumerate}


\qsection{Challenges}{C}

\question Let $n$ be an integer.
Prove that if $2 \mid n$ and $3 \mid n$, then $6 \mid n$.

\question Let $a,b,c\in\R$.
Prove that if $a^2 + b^2 + c^2 = 1$, then $-1/2 \leq ab+bc+ca \leq 1$.

\question Show that if $p$ and $p^2+2$ are prime, then $p^3+2$ is also prime.

\question Express the following statement in symbolic form and prove that it is true.
\begin{quote}
  There exists a real number $L$ such that for every positive real number $\epsilon$,
  there exists a positive real number $\delta$ such that for all real numbers $x$,
  if $\abs{x} < \delta$, then $\abs{3x-L} < \epsilon$.
\end{quote}

\question Prove that there are no positive integers $a$ and $b$ such that $b^4+b+1 = a^4$.

\question Prove that the length of at least one side of a right-angled triangle with integer side lengths must be divisible by 3.

\end{document}