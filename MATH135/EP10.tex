\documentclass[11pt]{article}

\usepackage{physics}
\usepackage{amsfonts,amsmath,amssymb,amsthm}
\usepackage{enumerate}
\usepackage{titlesec}
\usepackage{fancyhdr}
\usepackage{multicol}

\headheight 13.6pt
\setlength{\headsep}{10pt}
\textwidth 15cm
\textheight 24.3cm
\evensidemargin 6mm
\oddsidemargin 6mm
\topmargin -1.1cm
\setlength{\parskip}{1.5ex}
\parindent=0pt

\author{James Ah Yong}

\pagestyle{fancy}
\fancyhf{}
\fancyfoot[c]{\thepage}
\makeatletter
\lhead{\@title}
\rhead{\@author}

\fancypagestyle{firstpage}{
  \fancyhf{}
  \rhead{\@author}
  \fancyfoot[c]{\thepage}
}

% Sets
\newcommand{\N}{\mathbb{N}}
\newcommand{\Z}{\mathbb{Z}}
\newcommand{\Q}{\mathbb{Q}}
\newcommand{\R}{\mathbb{R}}
\newcommand{\C}{\mathbb{C}}
\newcommand{\U}{\mathcal{U}}
\newcommand{\sym}{\mathbin{\triangle}}

% Functions
\DeclareMathOperator{\sgn}{sgn}
\DeclareMathOperator{\im}{im}
\DeclareMathOperator{\lcm}{lcm}

% Operators
\newcommand{\Rarr}{\Rightarrow}
\newcommand{\Larr}{\Leftarrow}
\newcommand{\Harr}{\Leftrightarrow}
\usepackage{mathtools} % for \DeclarePairedDelimiter macro
\DeclarePairedDelimiter\ceil{\lceil}{\rceil}
\DeclarePairedDelimiter\floor{\lfloor}{\rfloor}
\newcommand{\dyx}{\dv{y}{x}}

% Macros
% properly typeset ε-δ (epsilon en dash delta)
\newcommand{\epsdel}[1][\delta]{\ensuremath{\epsilon\mathit{\textnormal{--}}#1}}
\newcommand{\by}[1]{& \text{by #1}}
\newcommand{\IH}{\by{inductive hypothesis}}
\newcommand{\pf}[2]{%
\let\tmp\relax\newcommand\tmp[1]{#1}
\ensuremath{p_1^{\tmp{1}}p_2^{\tmp{2}}\cdots p_#2^{\tmp{#2}}}}
% multiple choice (remove spacing between items)
\newenvironment{choices}
{\begin{enumerate}[(a)]
    \setlength{\parskip}{0ex}
    }{
  \end{enumerate}}

% Typesetting
\usepackage{array}   % for \newcolumntype macro
\newcolumntype{C}{>{$}c<{$}} % math version of "C" column type
\newcommand{\dlim}[2]{\displaystyle\lim_{#1\to#2}} % totally not \dfrac ripoff
\newcommand{\dilim}[1]{\dlim{#1}{\infty}} % infinite limits
\newcommand{\ilim}[1]{\lim_{#1\to\infty}}
\usepackage{cancel}

% Auto-number questions
\newcommand{\QType}{Q}
\renewcommand{\theparagraph}{\QType\ifnum\value{paragraph}<10 0\fi\arabic{paragraph}}
\setcounter{secnumdepth}{6}
\newcommand{\question}{\par\refstepcounter{paragraph}\textbf{\theparagraph}.\space}

% Question sections
\titleformat{\section}{\normalsize\bfseries}{\thesection}{1em}{}
\newcommand{\qsection}[2]{%
  \renewcommand{\QType}{#2}
  \section*{#1}
  \refstepcounter{section}
}

\usepackage{minted}

\title{MATH 135 Fall 2020: Extra Practice 10}

\begin{document}
\thispagestyle{firstpage}

\textbf{\@title}

\qsection{Warm-Up Exercises}{WE}

\question Express $\dfrac{2-i}{3+4i}$ in standard form.
\begin{proof}[Solution]
  Multiply numerator and denominator by the conjugate of the denominator:
  \[ \frac{2-i}{3+4i} = \frac{(2-i)(3-4i)}{9+16} = \frac{2-11i}{25} = \frac{2}{25}-\frac{11}{25}i \qedhere \]
\end{proof}


\question Write $x=\dfrac{9+i}{5-4i}$ in polar form, $r(\cos\theta+i\sin\theta)$, with $0\leq\theta<2\pi$.
\begin{proof}[Solution]
  We express first in standard form by multiplying through the conjugate:
  \[ \frac{9+i}{5-4i} = \frac{(9+i)(5+4i)}{41} = \frac{41+41i}{41} = 1+i \]
  We can geometrically interpret this as $\sqrt2\cis\frac\pi4$.
\end{proof}


\question Write $(\sqrt{3}+i)^4$ in standard form.
\begin{proof}[Solution]
  We first place the quantity within the brackets in polar form.
  By inspection, this is $2\cis\frac\pi6$.
  Now, applying DMT, we have $(2\cis\frac\pi6)^4 = 2^4\cis^4\frac\pi6 = 16\cis\frac{2\pi}{3}$.

  Expressing in standard form,
  $16(\cos\frac{2\pi}{3} + i\sin\frac{2\pi}{3}) = 16(-\frac12+i\frac{\sqrt3}{2}) = -8 + 8\sqrt3 i$
\end{proof}


\question Find all $z\in\C$ such that $z^5=1$ and plot the solutions in the complex plane.
(You may state values in polar form.)
\begin{proof}[Solution]
  Note that $1 = 1\cis0$.
  Applying the CRNT, we have that the five roots are given by
  $\sqrt[5]{1}\cis\left(\frac{2k\pi}{n}\right)$ for $k=0,1,2,3,4$.
  These values are $\{ 1, \cis\frac\pi5, \cis\frac{4\pi}{5}, \cis\frac{6\pi}{5}, \cis\frac{8\pi}{5} \}$.
  I am too lazy to learn \verb|tikz| to draw the diagram.
\end{proof}


\question Find all $z\in\C$ such that $z^2=\dfrac{1+i}{1-i}$.
\begin{proof}[Solution]
  Simplifying the fraction on the right-hand side, $\frac{(1+i)(1+i)}{2}=\frac{1+2i-1}{2}=i$.
  On the complex plane, $i=1\cis\frac\pi2$.
  Then, by CRNT, the solutions are $\cis\frac\pi4$ and $\cis\frac{5\pi}{4}$.
  Evaluating to get standard form, we have $z = \pm(\frac{\sqrt{2}}{2}+\frac{\sqrt{2}}{2}i)$.
\end{proof}


\qsection{Recommended Problems}{RP}

\question Express the following complex numbers in standard form.
\begin{enumerate}[(a)]
  \item $\dfrac{(\sqrt2-i)^2}{(\sqrt2+i)(1-\sqrt2i)}$
  \item $(\sqrt5-i\sqrt3)^4$
\end{enumerate}

\question Prove all of the Properties of Complex Arithmetic that were not proved in the notes or in class.

\question Let $n \in \N$. Prove that if $n \equiv 1 \pmod 4$, then $i^n = i$.

\question Find all $z \in \C$ which satisfy
\begin{enumerate}[(a)]
  \item $z^2+2z+1=0$
  \item $z^2+2\bar{z}+1=0$
  \item $z^2 = \dfrac{1+i}{1-i}$
\end{enumerate}

\question \begin{enumerate}[(a)]
  \item Find all $w\in\C$ satisfying $w^2 = -15 + 8i$.
  \item Find all $z\in\C$ satisfying $z^2-(3+2i)z+5+i=0$.
\end{enumerate}

\question Let $z, w \in \C$. Prove that if $zw = 0$ then $z = 0$ or $w = 0$.

\question Let $a,b,c \in \C$. Prove: if $|a|=|b|=|c|=1$, then $\overline{a+b+c}=\frac1a+\frac1b+\frac1c$.

\question Find all $z\in\C$ satisfying $z^2=|z|^2$.

\question Find all $z\in\C$ satisfying $|z+1|^2 \leq 3$ and shade the corresponding region in the complex plane.

\question Let $z,w\in\C$ such that $\bar z w \neq 1$.
Prove that if $|z|=1$ or $|w|=1$, then $\abs{\dfrac{z-w}{1 -\bar z w}}=1$.

\question Show that for all complex numbesr $z$, $|\Re(z)|+|\Im(z)|\leq\sqrt{2}|z|$.

\question Use \emph{De Moivre's Theorem} (DMT) to prove that
$\sin 4\theta = 4\sin\theta\cos^3\theta - 4\sin^3\theta\cos\theta$ for all $\theta\in\R$.

\question Let $n\in\N$ and $a,b\in\R$. Show that $z=(a+bi)^n+(a-bi)^n$ is real.

\question An \emph{$n$\textsuperscript{th} root of unity} is any complex solution to $z^n=1$.
Prove that if $w$ is an $n$\textsuperscript{th} root of unity,
$\frac1w$ is also an $n$\textsuperscript{th} root of unity.

\question A complex number $z$ is called a \emph{primitive} $n$-th root of unity
if $z^n=1$ and $z^k \neq 1$ for all $1 \leq k \leq n-1$.
\begin{enumerate}[(a)]
  \item For each $n=1,3,5,6$ list all the primitive $n$-th roots of unity.
  \item Let $z$ be a primitive $n$-th root of unity. Prove the following statements:
        \begin{enumerate}[i.]
          \item For any $j\in\Z$, $z^j=1$ if and only if $n \mid j$.
          \item For any $m\in\Z$, if $\gcd(m,n)=1$, then $z^m$ is a primitive $n$-th root of unity.
        \end{enumerate}
\end{enumerate}

\question Let $u$ and $v$ be fixed complex numbers.
Let $\omega$ be a non-real cube root of unity.
For each $k\in\Z$, define $y_k\in\C$ by the formula \[ y_k = \omega^k u + \omega^{-k}v \]
\begin{enumerate}[(a)]
  \item Compute $y_1$, $y_2$, and $y_3$ in terms of $u$, $v$, and $\omega$.
  \item Show that $y_k = y_{k+3}$ for any $k\in\Z$.
  \item Show that for any $k\in\Z$, \[ y_k-y_{k+1} = \omega^k(1-\omega)(u-\omega^{k-1}v) \]
\end{enumerate}


\qsection{Challenges}{C}

\question Let $z,w\in\C$.
\begin{enumerate}[(a)]
  \item Prove that $|z+w| \leq |z| + |w|$.
  \item Prove that $\abs{|z| - |w|} \leq |z-w| \leq |z|+|w|$.
\end{enumerate}

\question Let $a,b,c\in\C$.
Show that if $\dfrac{b-a}{a-c} = \dfrac{a-c}{c-b}$ then $|b-a| = |a-c| = |c-b|$.

\question Let $n \geq 2$ be an integer. Prove that
\[ \sum_{k=0}^{n-1}\cos\left(\frac{2k\pi}{n}\right)=0=\sum_{k=0}^{n-1}\sin\left(t\frac{2k\pi}{n}\right) \]

\end{document}