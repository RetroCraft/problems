\documentclass{agony}
\title{MATH 138 Winter 2021: Practice Assignment 2}

\begin{document}

\begin{prob}
  \begin{enumerate}[(a)]
    \item Find the average value of $f(x) = x^2 + 2x + 1$ on $[1,5]$.
    \item For $f(x) = \frac1x$, find $c\in\R$ such that
          $f(c)$ equals the average value of $f$ on $[1,5]$.
    \item Find all values of $a,b\in\R$ where $b>0$ so that
          the average value of $f(x) = 3x^2 + ax$ on $[0,b]$ is zero.
  \end{enumerate}
\end{prob}
\begin{sol}
  Recall the definition of the average value and calculate (a):
  \begin{align*}
    f_{avg}([1,5])
     & = \frac{1}{5-1}\int_1^5 x^2 + 2x + 1 \dd{x}                                          \\
     & = \frac{1}{4}\left(\int_1^5 x^2 \dd{x} + \int_1^5 2x \dd{x} + \int_1^5 \dd x \right) \\
     & = \left(\frac{1}{12}x^3 + \frac{1}{4}x^2 + \frac{1}{4}x\middle)\right|_1^5           \\
     & = \left(\frac{5^3}{12}+\frac{5^2}{4}+\frac{5}{4}\right)
    - \left(\frac{1^3}{12}+\frac{1^2}{4}+\frac{1}{4}\right)                                 \\
     & = \frac{52}{3}
  \end{align*}

  For (b), we find the average value then solve for $c$:
  \begin{align*}
    f_{avg}([1,3])
    = \frac{1}{3-1}\int_1^3 \frac{\dd{x}}{x}
    = \left.\frac{1}{2}\ln(|x|)\right|_1^3
    = \frac{\ln3-\ln1}{2}
    = \frac{\ln3}{2}
  \end{align*}
  And since $f(x) = \frac{1}{x}$, $f(\frac{2}{\ln3}) = \frac{\ln3}{2}$.

  In (c), we just solve:
  \begin{align*}
    f_{avg}([0,b]) & = \frac{1}{b-0} \int_0^b 3x^2 + ax \dd{x}                  \\
    0              & = \frac{1}{b}\left(x^3 + \frac{a}{2}x^2\middle)\right|_0^b \\
    0              & = b\left(b + \frac{a}{2}\right)
  \end{align*}
  Therefore, either $b = 0$ or $a = -2b$.
  But $b > 0$, so we have that $a = -2b$.
\end{sol}

\begin{prob}
  Suppose $h$ is a function such that $h(1) = -4$, $h'(1) = 7$, $h''(1) = 3$,
  $h(2) = 0$, $h'(2) = 15$, $h''(2) = 16$, and $h''$ is continuous everywhere.
  Evaluate $\int_1^2 h''(x) \dd{x}$.
\end{prob}
\begin{sol}
  Apply the Fundamental Theorem of Calculus:
  \begin{align*}
    \int_1^2 h''(x) \dd{x} = h'(2) - h'(1) = 15 - 7 = 8
  \end{align*}
  since $h'$ is an antiderivative to $h''$.
\end{sol}

\begin{prob}
  Use the Fundamental Theorem of Calculus
  to find the \textbf{derivative} of each function:
  \begin{enumerate}[(a)]
    \item $f(x) = {\displaystyle\int_1^{x^5}}\sin(t)\ln(t^2) \dd{t}$
    \item $g(x) = {\displaystyle\int_{\sin(x)}^{\cos(x)}}e^{t^2} \dd{t}$
  \end{enumerate}
\end{prob}
\begin{sol}
  We apply the Extended Version of the Fundamental Theorem of Calculus.
  \begin{align*}
    f'(x) & = \sin(x^5)\ln(x^{10})(5x^4) - \sin(1)\ln(1)(0)             \\
          & = 50x^4\sin(x^5)\ln(x)                                      \\
    g'(x) & = e^{\cos[2](x)}(-\sin x) - e^{\sin[2](x)}(\cos x) \qedhere
  \end{align*}
\end{sol}

\begin{prob}
  Find the following definite integrals using any method:
  \begin{enumerate}[(a)]
    \item $\int_0^2 x^3 + 3^x \dd{x}$
    \item $\int_0^4 (4-x)\sqrt{x} \dd{x}$
  \end{enumerate}
\end{prob}
\begin{sol}
  Evaluate (a) by FTC:\@
  \[
    \int_0^2 x^3 + 3^x \dd{x}
    = \left. \frac{1}{4}x^3 + \frac{1}{\ln 3}3^x \right|_0^2
    = \frac{16}{4} + \frac{9}{\ln 3} - \frac{1}{\ln 3}
    = 4+\frac{8}{\ln 3}
  \]
  Expand the integrand of (b), $(4-x)\sqrt{x} = 4x^{1/2} - x^{3/2}$, then apply FTC:\@
  \[
    \int_0^4 4\sqrt{x} - x^{3/2} \dd{x}
    = \left. \frac{8}{3}x^{3/2} - \frac{2}{5}x^{5/2} \right|_0^4
    = \frac{128}{15} \qedhere
  \]
\end{sol}

\begin{prob}
  Find a function $f$ and a number $a\in\R$ such that
  \[ 6+\int_a^x \frac{f(t)}{t^2}\dd{t} = 2\sqrt{x} \]
  for all $x > 0$
\end{prob}
\begin{sol}
  We differentiate both sides using FTC1, supposing $f$ is continuous:
  \begin{align*}
    \dv{x}(6+\int_a^x \frac{f(t)}{t^2}\dd{t}) & = \dv{x}(2\sqrt{x})  \\
    0+\dv{x}(\int_a^x \frac{f(t)}{t^2}\dd{t}) & = \frac{1}{\sqrt{x}} \\
    \frac{f(x)}{x^2}                          & = \frac{1}{\sqrt{x}} \\
    f(x)                                      & = x^{3/2}
  \end{align*}
  We now must select $a$.
  Since we know $f$, we can now find the definite integral with FTC2:
  \begin{align*}
    6+\int_a^x \frac{t^{3/2}}{t^2}\dd{t} & = 2\sqrt{x}     \\
    \int_a^x \frac{1}{\sqrt{t}} \dd{t}   & = 2\sqrt{x} - 6 \\
    2\sqrt{x} - 2\sqrt{a}                & = 2\sqrt{x} - 6 \\
    a                                    & = 9
  \end{align*}
  Therefore, the solution is $f(x) = x^{3/2}$ and $a=9$.
\end{sol}

\begin{prob}
  Prove that $\displaystyle\int_0^{\pi/6}\cos(x^2)\dd{x} \geq \frac{1}{2}$
  by comparing $\cos(x^2)$ to $\cos(x)$
\end{prob}
\begin{sol}
  Notice first that on $[0,\frac\pi6]$, $x^2 \geq x$ since $\frac\pi6 < 1$.
  As $\cos(x)$ is decreasing, $\cos(x^2) \geq \cos(x)$.
  Then, $\int_0^{\pi/6} \cos(x^2) \dd{x} \geq \int_0^{\pi/6} \cos(x) \dd{x}$.
  But $\at{\sin(x)}{0}^{\pi/6} = \sin(\frac\pi6) = \frac12$.
\end{sol}

\begin{prob}
  The Fundamental Theorem of Calculus is a very powerful theorem,
  but relies on us being able to find an antiderivative,
  which is not always possible.

  For example, $f(x) = e^{-x^2}$ does not have an elementary antiderivative
  (that is, an antiderivative made up of elementary functions).
  However, it does have an antiderivative since it is continuous,
  and it is very important in probability, statistics, and engineering.
  Define the \textbf{error function} to be
  \[ \erf(x) = \frac{2}{\sqrt{\pi}}\int_0^x e^{-t^2} \dd{t} \]
  \begin{enumerate}[(a)]
    \item Show that
          $\int_a^b e^{-t^2} \dd{t} = \frac{\sqrt{\pi}}{2}(\erf(b)-\erf(a))$
    \item Show that the function $y = e^{x^2}\erf(x)$ satisfies
          the differential equation $y' = 2xy + \frac{2}{\sqrt{\pi}}$
  \end{enumerate}
\end{prob}
\begin{sol}
  Notice that the error function is an antiderivative of $e^{-x^2}$ up to a multiple,
  so we can say that $\int e^{-x^2} \dd{x} = \frac{\sqrt{\pi}}{2}\erf(x)$.

  Then, part (a) is a restatement of FTC2.

  For part (b), we simply differentiate both sides.
  We can use FTC1 since $e^{-x^2}$ is continuous at $x=0$.
  \begin{align*}
    y' & = \dv{x}(e^{x^2}\erf(x))                                                               \\
       & = \dv{x}(e^{x^2})\erf(x) + e^{x^2}\dv{x}(\frac{2}{\sqrt{\pi}}\int_0^x e^{-t^2} \dd{t}) \\
       & = 2xe^{x^2}\erf(x) + \frac{2}{\sqrt{\pi}}e^{x^2}e^{-x^2}                               \\
       & = 2xe^{x^2}\erf(x) + \frac{2}{\sqrt{\pi}}e^0                                           \\
       & = 2xy + \frac{2}{\sqrt{\pi}}
  \end{align*}
  completing the proof.
\end{sol}

\begin{prob}
  Evaluate the following integrals.
\end{prob}
\begin{enumerate}[(a)]
  \item $\displaystyle\int \frac{3}{\sqrt{2x+1}} \dd{x}$
        \begin{sol}
          Let $u = 2x+1$, so $\dd{u} = 2\dd{x}$:
          \begin{align*}
            \int \frac{3}{\sqrt{2x+1}} \dd{x}
             & = \int \frac{3}{2\sqrt{u}} \dd{u} \\
             & = 3\sqrt{u} + C                   \\
             & = 3\sqrt{2x+1} + C \qedhere
          \end{align*}
        \end{sol}
  \item $\displaystyle\int_{11}^{16} \frac{x}{\sqrt{x-7}} \dd{x}$
        \begin{sol}
          Let $u = x-7$, so $\dd{u} = \dd{x}$:
          \begin{align*}
            \int_{11}^{16} \frac{x}{\sqrt{x-7}} \dd{x}
             & = \int_{4}^{9} \frac{u+7}{\sqrt{u}} \dd{u}          \\
             & = \int_{4}^{9} \sqrt{u} + \frac{7}{\sqrt{u}} \dd{u} \\
             & = \at{\frac{2}{3}u^{3/2} + 14\sqrt{u}}{4}^9         \\
             & = \frac{80}{3} \qedhere
          \end{align*}
        \end{sol}
  \item $\displaystyle\int (x^2-1)e^{x^3-3x} \dd{x}$
        \begin{sol}
          Let $u = x^3 - 3x$ so $\dd{u} = 3(x^2 - 1) \dd{x}$:
          \begin{equation*}
            \int (x^2-1)e^{x^3-3x} \dd{x}
            = \int \frac{1}{3} e^u \dd{u}
            = \frac{1}{3} e^u + C
            = \frac{1}{3} e^{x^3+3x} + C \qedhere
          \end{equation*}
        \end{sol}
  \item $\displaystyle\int \frac{\sqrt{2+\ln(x)}}{x} \dd{x}$
        \begin{sol}
          Let $u = 2+\ln(x)$ so $\dd{u} = \frac{\dd{x}}{x}$:
          \begin{equation*}
            \int \frac{\sqrt{2+\ln(x)}}{x} \dd{x}
            = \int \sqrt{u} \dd{u}
            = \frac{2}{3}u^{3/2} + C
            = \frac{2}{3}(2+\ln(x))^{3/2} + C \qedhere
          \end{equation*}
        \end{sol}
  \item $\displaystyle\int \frac{x^2+2}{x^3+6x+3} \dd{x}$
        \begin{sol}
          Let $u = x^3 + 6x + 3$ so $\dd{u} = 3(x^2+2)\dd{x}$:
          \begin{equation*}
            \int \frac{x^2+2}{x^3+6x+3} \dd{x}
            = \int \frac{1}{3u} \dd{u}
            = \frac{1}{3} \ln\abs{u} + C
            = \frac{1}{3} \ln\abs{x^3 + 6x + 3} + C \qedhere
          \end{equation*}
        \end{sol}
  \item $\displaystyle\int_0^{\frac\pi4} \sqrt{1-\cos(4x)} \dd{x}$
  \begin{sol}
    Recall that $\cos 2\theta = 1-2\sin^2\theta$.
    Then, we can rewrite the integral:
    \begin{equation*}
      \int_0^{\frac\pi4} \sqrt{1-\cos(4x)} \dd{x}
      = \int_0^{\frac\pi4} \sqrt{2\sin^2(2x)} \dd{x}
      = \int_0^{\frac\pi4} \sin(2x)\sqrt{2} \dd{x}
    \end{equation*}
    since $\sin 2x$ is non-negative on $[0,\frac\pi4]$.
    Let $u = 2x$ so $\dd{u} = 2\dd{x}$ and
    \begin{equation*}
      \int_0^{\frac\pi4} \sin(2x)\sqrt{2} \dd{x}
      = \int_0^{\frac\pi2} \frac{\sqrt{2}}{2} \sin(u) \dd{u}
      = \at{-\frac{\sqrt{2}}{2}\cos(u)}{0}^{\frac\pi2}
      = \frac{\sqrt{2}}{2} \qedhere
    \end{equation*}
  \end{sol}
  \item $\displaystyle\int \frac{x^5+x-1}{x^3+1} \dd{x}$ by first performing polynomial long division.
        \begin{sol}
          First, do polynomial long division:
          \[ \polylongdiv{x^5+x-1}{x^3+1} \]
          Then, we can rewrite the integral and factor:
          \begin{align*}
            \int \frac{x^5+x-1}{x^3+1} \dd{x}
             & = \int x^2 - \frac{x^2 - x + 1}{(x+1)(x^2 - x + 1)} \dd{x} \\
             & = \int x^2 - \frac{1}{x+1} \dd{x}                          \\
             & = \frac{1}{3}x^3 - \ln\abs{x+1} + C \qedhere
          \end{align*}
        \end{sol}
\end{enumerate}


\end{document}