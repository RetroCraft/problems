\documentclass[notes,tikz]{agony}

\DeclareMathOperator*{\E}{\mathbb{E}}

\title{CO 432 Spring 2025: Lecture Notes}

\begin{document}
\renewcommand{\contentsname}{CO 432 Spring 2025:\\{\huge Lecture Notes}}
\thispagestyle{firstpage}
\tableofcontents

Lecture notes taken, unless otherwise specified,
by myself during the Spring 2025 offering of CO 432,
taught by Vijay Bhattiprolu.

\begin{multicols}{2}
  \listoflecture
\end{multicols}

\chapter{Introduction}

\section{Entropy}
\lecture{May 6}
TODO

\begin{defn}[entropy]
  For a random variable $\rv X$ which is equal to $i$ with probability $p_i$,
  the \term{entropy} $H(\rv X) := \sum_i p_i \log \frac{1}{p_i}$.
\end{defn}

\subsection{Axiomatic view of entropy}
\lecture{May 8}

We want $S : [0,1] \to [0,\infty)$ to capture how ``surprised''
we are $S(p)$ that an event with probability $p$ happens.
We want to show that under some natural assumptions,
this is the only function we could have defined as entropy.
In particular:

\begin{enumerate}
  \item $S(1) = 0$, a certainty should not be surprising
  \item $S(q) > S(p)$ if $p > q$, less probable should be more surprising
  \item $S(p)$ is continuous in $p$
  \item $S(pq) =  S(p) + S(q)$, surprise should add for independent events.
        That is, if I see something twice, I should be twice as surprised.
\end{enumerate}

\begin{prop}
  If $S(p)$ satisfies these 4 axioms, then $S(p)=c\cdot \log_2(1/p)$ for some $c > 0$.
\end{prop}
\begin{prf}
  Suppose a function $S : [0,1] \to [0,\infty)$ exists satisfying the axioms.
  Let $c := S(\frac12) > 0$.

  By axiom 4 (addition), $S(\frac{1}{2^k}) = kS(\frac12)$.
  Likewise, $S(\frac{1}{2^{1/k}}\cdots\frac{1}{2^{1/k}})
    = S(\frac{1}{2^{1/k}}) + \dotsb + S(\frac{1}{2^{1/k}}) = kS(\frac{1}{2^{1/k}})$.

  Then, $S(\frac{1}{2^{m/n}}) = \frac{m}{n}S(\frac12) = \frac{m}{n}\cdot c$
  for any rational $m/n$.

  By axiom 3 (continuity), $S(\frac{1}{2^z}) = c \cdot z$ for all $z \in [0,\infty)$
  because the rationals are dense in the reals.
  In particular, for any $p \in [0,1]$,
  we can write $p = \frac{1}{2^z}$ for $z = \log_2(1/p)$
  and we get \[ S\qty(p) = S\qty(\frac{1}{2^z}) = c \cdot z = c \cdot \log_2(1/p) \]
  as desired.
\end{prf}

We can now view entropy as expected surprise. In particular,
\[ \sum_i p_i \log_2\frac{1}{p_i} = \E_{x \sim \rv X}\qty[S(p_x)] \]
for a random variable $\rv X = i$ with probability $p_i$.

\subsection{Entropy as optimal lossless data compression}

Suppose we are trying to compress a string consisting of $n$
symbols drawn from some distribution.

\begin{problem}
  What is the expected number of bits you need to store the results of $n$ independent samples
  of a random variable $\rv X$?
\end{problem}

We will show this is $nH(\rv X)$.

Notice that we assume that the symbols we are drawn \uline{independently},
which is violated by almost all data we actually care about.

\begin{defn}
  Let $C : \Sigma \to (\Sigma')^*$ be a code.
  We say $C$ is \term{uniquely decodable} if there does not exist
  a collision $x, y \in \Sigma^*$,
  with identical encoding $C(x_1)C(x_2)\cdots C(x_k) = C(y_1)C(y_2)\cdots C(y_{k'})$.

  Also, $C$ is \term{prefix-free} (sometimes called \term*{instantaneous})
  if for any distinct $x,y \in \Sigma$, $C(x)$ is not a prefix of $C(y)$.
\end{defn}

\begin{prop}
  Prefix-freeness is sufficient for unique decodability.
\end{prop}

\begin{example}
  Let $C : \{A,B,C,D\} \to \{0,1\}^*$ where
  $C(A) = 11$, $C(B) = 101$, $C(C) = 100$, and $C(D) = 00$.
  Then, $C$ is prefix-free and uniquely decodable.

  We can easily parse $1011100001100$ unambiguously as $101.11.00.00.11.00$
  ($BADDAD$).
\end{example}

Recall from CS 240 that a prefix-free code is equivalent to a trie,
and we can decode it by traversing the trie in linear time.

\begin{theorem}[Kraft's inequality]
  A prefix-free binary code $C : \{1,\dotsc,n\} \to \{0,1\}^*$
  with codeword lengths $\ell_i = \abs{C(i)}$ exists if and only if
  \[ \sum_{i=1}^n \frac{1}{2^{\ell_i}} \leq 1. \]
\end{theorem}
\begin{prf}
  Suppose $C : \{1,\dotsc,n\} \to \{0,1\}^*$ is prefix-free
  with codeword lengths $\ell_i$.
  Let $T$ be its associated binary tree
  and let $W$ be a random walk on $T$ where 0 and 1 have equal weight
  (stopping at either a leaf or undefined branch).

  Define $E_i$ as the event where $W$ reaches $i$ and
  $E_\varnothing$ where $W$ falls off. Then,
  \begin{align*}
    1 & = \Pr(E_\varnothing) + \sum_i \Pr(E_i)                                   \\
      & = \Pr(E_\varnothing) + \sum_i \frac{1}{2^{\ell_i}} \tag{by independence} \\
      & \geq \sum_i \frac{1}{2^{\ell_i}} \tag{probabilities are non-negative}
  \end{align*}

  Conversely, suppose the inequality holds for some $\ell_i$.
  \WLOG, suppose $\ell_1 < \ell_2 < \dotsb < \ell_n$.

  Start with a complete binary tree $T$ of depth $\ell_n$.
  For each $i = 1,\dotsc,n$, find any unassigned node in $T$ of depth $\ell_i$,
  delete its children, and assign it a symbol.

  Now, it remains to show that this process will not fail.
  That is, for any loop step $i$, there is still some unassigned node at depth $\ell_i$.

  Let $P \gets 2^{\ell_n}$ be the number of leaves
  of the complete binary tree of depth $\ell_n$.
  After the $i$\xth step, we decrease $P$ by $2^{\ell_n - \ell_i}$.
  That is, after $n$ steps,
  \begin{align*}
    P & = 2^{\ell_n} - \sum_{i=1}^n \frac{2^{\ell_n}}{2^{\ell_i}}   \\
      & = 2^{\ell_n} - 2^{\ell_n} \sum_{i=1}^n \frac{1}{2^{\ell_i}} \\
      & \geq 0
  \end{align*}
  by the inequality.
\end{prf}

\end{document}
