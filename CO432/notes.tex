\documentclass[class=co432,notes,tikz]{agony}

\DeclareMathOperator*{\E}{\mathbb{E}}

\title{CO 432 Spring 2025: Lecture Notes}

\begin{document}
\renewcommand{\contentsname}{CO 432 Spring 2025:\\{\huge Lecture Notes}}
\thispagestyle{firstpage}
\tableofcontents

Lecture notes taken, unless otherwise specified,
by myself during the Spring 2025 offering of CO 432,
taught by Vijay Bhattiprolu.

\begin{multicols}{2}
  \listoflecture
\end{multicols}

\chapter{Introduction}

\begin{notation}
  I will be using my usual \LaTeX{} typesetting conventions:
  \begin{itemize}[nosep]
    \item $[n]$ means the set $\{1,2,\dotsc,n\}$
    \item $\bits*$ means the set of bitstrings of arbitrary length (i.e., the Kleene star)
    \item $\rv A, \rv B, \dotsc, \rv Z$ are random variables (in sans-serif)
    \item $\rv X = (p_1,p_2,\dotsc,p_k)$ means $\rv X$ is a discrete random variable
          such that $\Pr[\rv X = 1] = p_1$, $\Pr[\rv X=2] = p_2$, etc.
          (abbreviate further as $\rv X = (p_i)$)
  \end{itemize}
\end{notation}

\section{Entropy}
\lecture{May 6}

\textrule{$\downarrow$ Lecture 1 adapted from Arthur $\downarrow$}

\begin{defn}[entropy]
  For a random variable $\rv X = (p_i)$,
  the \term*{entropy} $H(\rv X)$ is
  \[ H(\rv X) = -\sum_i p_i \log p_i = \sum_i p_i \log \frac{1}{p_i}. \]
\end{defn}

\begin{convention}
  By convention, we usually use $\log_2$.
  Also, we define entropy such that $\log_2(0) = 0$ so that
  impossible values do not break the formula.
\end{convention}

\begin{example}
  If $\rv X$ takes on the values $a$, $b$, $c$, $d$
  with probabilities 1, 0, 0, 0, respectively, then $H(\rv X) = 1 \log 1 = 0$.

  If $\rv X$ takes on those values instead with probabilities
  $\frac12$, $\frac14$, $\frac18$, $\frac18$, respectively,
  then $H(\rv X) = \frac74$.
\end{example}

\begin{fact}
  $H(\rv X) = 0$ if and only if $\rv X$ is a constant.
\end{fact}
\begin{prf}
  Suppose $\rv X$ is constant. Then, $H(\rv X) = 1 \log 1 = 0$.

  Suppose $H(\rv X) = 0$.
  Probabilities are in $[0,1]$, so $p_i \log \frac{1}{p_i} \geq 0$.
  Since $H(\rv X) = \sum_i p_i \log \frac{1}{p_i} = 0$
  and each term is non-negative, each term must be zero.
  Thus, each $p_i$ is either 0 or 1.
  We cannot have $\sum p_i > 1$, so exactly one $p_i = 1$ and the rest are zero.
  That is, $\rv X$ is constant.
\end{prf}

\begin{theorem}[Jensen's inequality]\label{thm:jensen}
  Let $f : \R \to \R$ be concave. That is,
  for any $a$ and $b$ in the domain of $f$ and $\lambda \in [0,1)$,
  $f(\lambda a + (1-\lambda)b) \geq \lambda f(a) + (1-\lambda)f(b)$.
  For any discrete random variable $\rv X$,
  \[ \E[f(\rv X)] \leq f(\E[\rv X]) \]
\end{theorem}
\begin{prf}
  Consider a random variable $\rv X$ with two values $a$ and $b$,
  each with probabilities $\lambda$ and $1-\lambda$.
  Then, notice that
  \[ \E[f(\rv X)] = \lambda f(a) + (1-\lambda) f(b) \leq f(\lambda a + (1-\lambda)b) = f(\E[\rv X]) \]
  by convexity of $f$.

  TODO: This can be generalized by induction.
\end{prf}

\begin{fact}
  Assume $\rv X$ is supported on $[n]$. Then, $0 \leq H(\rv X) \leq \log n$.
\end{fact}
\begin{prf}
  Start by claiming without proof that $\log n$ is concave, so we can apply
  \nameref{thm:jensen}.

  Let $\rv X' = \frac{1}{p_i}$ with probability $p_i$. Then,
  \begin{align*}
    H(\rv X) & = \sum_i p_i \log \frac{1}{p_i}    \\
             & = \E\qty[\log(\rv X')]             \\
             & \leq \log(\E[\rv X'])              \\
             & = \log\qty(\sum p_i \frac{1}{p_i}) \\
             & = \log n \qedhere
  \end{align*}
\end{prf}

It is not a coincidence that $\log_2 n$ is the minimum number of bits to encode $[n]$.

\section{Entropy as expected surprise}

We want $S : [0,1] \to [0,\infty)$ to capture how ``surprised''
we are $S(p)$ that an event with probability $p$ happens.
We want to show that under some natural assumptions,
this is the only function we could have defined as entropy.
In particular:

\begin{enumerate}
  \item $S(1) = 0$, a certainty should not be surprising
  \item $S(q) > S(p)$ if $p > q$, less probable should be more surprising
  \item $S(p)$ is continuous in $p$
  \item $S(pq) =  S(p) + S(q)$, surprise should add for independent events.
        That is, if I see something twice, I should be twice as surprised.
\end{enumerate}

\textrule{$\uparrow$ Lecture 1 adapted from Arthur $\uparrow$}
\lecture{May 8}

\begin{prop}
  If $S(p)$ satisfies these 4 axioms, then $S(p)=c\cdot \log_2(1/p)$ for some $c > 0$.
\end{prop}
\begin{prf}
  Suppose a function $S : [0,1] \to [0,\infty)$ exists satisfying the axioms.
  Let $c := S(\frac12) > 0$.

  By axiom 4 (addition), $S(\frac{1}{2^k}) = kS(\frac12)$.
  Likewise, $S(\frac{1}{2^{1/k}}\cdots\frac{1}{2^{1/k}})
    = S(\frac{1}{2^{1/k}}) + \dotsb + S(\frac{1}{2^{1/k}}) = kS(\frac{1}{2^{1/k}})$.

  Then, $S(\frac{1}{2^{m/n}}) = \frac{m}{n}S(\frac12) = \frac{m}{n}\cdot c$
  for any rational $m/n$.

  By axiom 3 (continuity), $S(\frac{1}{2^z}) = c \cdot z$ for all $z \in [0,\infty)$
  because the rationals are dense in the reals.
  In particular, for any $p \in [0,1]$,
  we can write $p = \frac{1}{2^z}$ for $z = \log_2(1/p)$
  and we get \[ S\qty(p) = S\qty(\frac{1}{2^z}) = c \cdot z = c \cdot \log_2(1/p) \]
  as desired.
\end{prf}

We can now view entropy as expected surprise. In particular,
\[ \sum_i p_i \log_2\frac{1}{p_i} = \E_{x \sim \rv X}\qty[S(p_x)] \]
for a random variable $\rv X = i$ with probability $p_i$.

\section{Entropy as optimal lossless data compression}

Suppose we are trying to compress a string consisting of $n$
symbols drawn from some distribution.

\begin{restatable}{problem}{bitproblem}
  What is the expected number of bits you need to store the results of $n$ independent samples
  of a random variable $\rv X$?
\end{restatable}

We will show this is $nH(\rv X)$.

Notice that we assume that the symbols we are drawn \uline{independently},
which is violated by almost all data we actually care about.

\begin{defn}
  Let $C : \Sigma \to (\Sigma')^*$ be a code.
  We say $C$ is \term[code!uniquely decodable]{uniquely decodable} if there does not exist
  a collision $x, y \in \Sigma^*$,
  with identical encoding $C(x_1)C(x_2)\cdots C(x_k) = C(y_1)C(y_2)\cdots C(y_{k'})$.

  Also, $C$ is \term[code!prefix-free]{prefix-free} (sometimes called \term*{instantaneous})
  if for any distinct $x,y \in \Sigma$, $C(x)$ is not a prefix of $C(y)$.
\end{defn}

\begin{prop}
  Prefix-freeness is sufficient for unique decodability.
\end{prop}

\begin{example}
  Let $C : \{A,B,C,D\} \to \{0,1\}^*$ where
  $C(A) = 11$, $C(B) = 101$, $C(C) = 100$, and $C(D) = 00$.
  Then, $C$ is prefix-free and uniquely decodable.

  We can easily parse $1011100001100$ unambiguously as $101.11.00.00.11.00$
  ($BADDAD$).
\end{example}

Recall from CS 240 that a prefix-free code is equivalent to a trie,
and we can decode it by traversing the trie in linear time.

\begin{theorem}[Kraft's inequality]\label{thm:kraft}
  A prefix-free binary code $C : \{1,\dotsc,n\} \to \{0,1\}^*$
  with codeword lengths $\ell_i = \abs{C(i)}$ exists if and only if
  \[ \sum_{i=1}^n \frac{1}{2^{\ell_i}} \leq 1. \]
\end{theorem}
\begin{prf}
  Suppose $C : \{1,\dotsc,n\} \to \{0,1\}^*$ is prefix-free
  with codeword lengths $\ell_i$.
  Let $T$ be its associated binary tree
  and let $W$ be a random walk on $T$ where 0 and 1 have equal weight
  (stopping at either a leaf or undefined branch).

  Define $E_i$ as the event where $W$ reaches $i$ and
  $E_\varnothing$ where $W$ falls off. Then,
  \begin{align*}
    1 & = \Pr(E_\varnothing) + \sum_i \Pr(E_i)                                   \\
      & = \Pr(E_\varnothing) + \sum_i \frac{1}{2^{\ell_i}} \tag{by independence} \\
      & \geq \sum_i \frac{1}{2^{\ell_i}} \tag{probabilities are non-negative}
  \end{align*}

  Conversely, suppose the inequality holds for some $\ell_i$.
  \WLOG, suppose $\ell_1 < \ell_2 < \dotsb < \ell_n$.

  Start with a complete binary tree $T$ of depth $\ell_n$.
  For each $i = 1,\dotsc,n$, find any unassigned node in $T$ of depth $\ell_i$,
  delete its children, and assign it a symbol.

  Now, it remains to show that this process will not fail.
  That is, for any loop step $i$, there is still some unassigned node at depth $\ell_i$.

  Let $P \gets 2^{\ell_n}$ be the number of leaves
  of the complete binary tree of depth $\ell_n$.
  After the $i$\xth step, we decrease $P$ by $2^{\ell_n - \ell_i}$.
  That is, after $n$ steps,
  \begin{align*}
    P & = 2^{\ell_n} - \sum_{i=1}^n \frac{2^{\ell_n}}{2^{\ell_i}}   \\
      & = 2^{\ell_n} - 2^{\ell_n} \sum_{i=1}^n \frac{1}{2^{\ell_i}} \\
      & \geq 0
  \end{align*}
  by the inequality.
\end{prf}

\lecture{May 13}
Recall the problem we are trying to solve:
\bitproblem*
\begin{sol}[Shannon \& Faro]
  Consider the case where $\rv X$ is symbol $i$ with probability $p_i$.
  We want to encode independent samples $x_i \sim \rv X$
  as $C(x_i)$ for some code $C : [n] \to \bits*$.

  Suppose for simplification that $p_i = \frac{1}{2^{\ell_i}}$
  for some integers $\ell_i$.
  Since $\sum p_i = 1$, we must have $\sum \frac{1}{2^{\ell_i}} = 1$.
  Then, by \nameref{thm:kraft}, there exists a prefix-free binary code
  $C : [n] \to \bits*$ with codeword lengths $\abs{C(i)} = \ell_i$.
  Now,
  \[
    \E_{x_i \sim \rv X}\qty[\sum_i\abs{C(x_i)}] = \sum_i p_i\ell_i = \sum_i p_i\log_2\frac{1}{p_i} = H(\rv X)
  \]
  Proceed to the general case.
  Suppose $\log_2\frac{1}{p_i}$ are non-integral.
  Instead, use $\ell'_i = \ceil*{\log_2\frac{1}{p_i}}$.
  We still satisfy Kraft since $\sum_i \frac{1}{2^{\ell'_i}} \leq \sum_i p_i = 1$.
  Then,
  \[
    \E_{x_i \sim \rv X}\qty[\sum_i\abs{C(x_i)}] = \sum_i p_i\ell'_i = \sum_i p_i\ceil*{\log_2\frac{1}{p_i}}
  \]
  which is bounded by
  \[ H(\rv X) = \sum_i p_i\log_2\frac{1}{p_i} \leq \sum_i p_i\ceil*{\log_2\frac{1}{p_i}} < \sum_i p_i\qty(1+\log_2\frac{1}{p_i}) = H(\rv X) + 1 \]
  We call the code $C$ generated by this process the \term[code!Shannon--Faro]{Shannon--Faro code}.
\end{sol}

We can improve on this bound $[H(\rv X), H(\rv X) + 1)$
by amortizing over longer batches of the string.

\begin{sol}[batching]
  For $\rv Y$ defined on $[n]$ equal to $i$ with probability $q_i$,
  define the random variable $\rv Y^{(k)}$ on $[n]^k$
  equal to the string $i_1\cdots i_k$ with probability $q_{i_1}\cdots q_{i_k}$.
  That is, $\rv Y^{(k)}$ models $k$ independent samples of $\rv Y$.

  Apply the Shannon--Fano code to $\rv Y^{(k)}$
  to get an encoding of $[n]^k$ as bitstrings of expected length $\ell$
  satisfying $H(\rv Y^{(k)}) \leq \ell \leq H(\rv Y^{(k)}) + 1$.
  \begin{align*}
    H(\rv Y^{(k)}) & = \E_{i_1\cdots i_k \sim \rv Y^{(k)}}\qty[\log_2 \frac{1}{q_{i_1}\cdots q_{i_k}}] \tag{by def'n}                       \\
                   & = \E_{i_1\cdots i_k \sim \rv Y^{(k)}}\qty[\log_2 \frac{1}{q_{i_1}} + \dotsb + \log_2\frac{1}{q_{i_k}}] \tag{log rules} \\
                   & = \sum_{j=1}^k \E_{i_1\cdots i_k \sim \rv Y^{(k)}}\qty[\log_2 \frac{1}{q_{i_j}}] \tag{linearity of expectation}        \\
                   & = \sum_{j=1}^k \E_{i \sim \rv Y}\qty[\log_2 \frac{1}{q_{i}}] \tag{$q_{i_j}$ only depends on one character}             \\
                   & = kH(\rv Y) \tag{by def'n, no $j$-dependence in sum}
  \end{align*}
  For every $k$ symbols, we use $\ell$ bits, i.e., $\frac{\ell}{k}$ bits per symbol.
  From the Shannon--Faro bound, we have
  \begin{align*}
    \frac{H(\rv Y^{(k)})}{k} & \leq \frac{\ell}{k} < \frac{H(\rv Y^{(k)})}{k} + \frac{1}{k} \\
    H(\rv Y)                 & \leq \frac{\ell}{k} < H(\rv Y) + \frac{1}{k}
  \end{align*}
  Then, we have a code for $\rv Y$ bounded by
  $[H(\rv Y), H(\rv Y) + \frac{1}{k})$.

  Taking a limit of some sort, we can say that we need $H(\rv Y) + o(1)$ bits.
\end{sol}

\begin{defn*}[relative entropy]
  Given two discrete distributions $p = (p_i)$ and $q = (q_i)$,
  the \term[entropy!relative]{relative entropy}
  \[ D(p \parallel q) :=
    \sum p_i \log_2\frac{1}{q_i} - \sum_i p_i \log_2 \frac{1}{p_i}
    = \sum p_i \log_2 \frac{p_i}{q_i} \]
  This is also known as the \term{KL divergence}.
\end{defn*}

\begin{fact}
  $D(p \parallel q) \geq 0$ with equality exactly when $p = q$.
\end{fact}
\begin{prf}
  Define $\rv X' = \frac{p_i}{q_i}$ with probability $p_i$.
  Then,
  \[ D(p \parallel q) = \E[-\log_2 \rv X'] \geq -\log_2 E[\rv X'] \]
  by Jensen's inequality (as $f(x) = -\log_2 x$ is convex), and then
  \[ D(p \parallel q) \geq  -\log_2 \sum p_i \frac{q_i}{p_i} = -\log_2 1 = 0 \qedhere \]
\end{prf}

\begin{prop}
  Any prefix-free code has an expected length at least $H(\rv X)$.
\end{prop}
\begin{prf}
  We can show this by interpreting the expected length $H(\rv X)$
  as $D(p \parallel q)$ for some $q$.

  We will take $q$ to be the random walk distribution corresponding to the binary tree
  associated to the candidate prefix-free code.
\end{prf}

\pagebreak
\phantomsection\addcontentsline{toc}{chapter}{Back Matter}
\renewcommand{\listtheoremname}{List of Named Results}
\phantomsection\addcontentsline{toc}{section}{\listtheoremname}
\listoftheorems[ignoreall,numwidth=4em,onlynamed={theorem,lemma,corollary,prop}]
\printindex

\end{document}
