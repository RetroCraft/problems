\documentclass[class=cs480,notes]{agony}

\title{CS 480/680 Winter 2024: Lecture Notes}
\begin{document}
\renewcommand{\contentsname}{CS 480/680 Winter 2024:\\{\huge Lecture Notes}}
\thispagestyle{firstpage}
\tableofcontents

Lecture notes taken, unless otherwise specified,
by myself during section 002 of the Winter 2024 offering of CS 480/680,
taught by Hongyang Zheng.

\begin{multicols}{2}
  \listoflecture
\end{multicols}

\chapter{Classic Machine Learning}

\section{Introduction}
\lecture{Jan 9}

There have been three historical AI booms:
\begin{enumerate}[1.,nosep]
  \item 1950s--1970s: search-based algorithms (e.g., chess),
        failed when they realized AI is actually a hard problem
  \item 1980s--1990s: expert systems
  \item 2012 -- present: deep learning
\end{enumerate}

Machine learning is the subset of AI where a program can learn from experience.

Major learning paradigms of machine learning:
\begin{itemize}[nosep]
  \item Supervised learning: teacher/human labels answers (e.g., classification, ranking, etc.)
  \item Unsupervised learning: without labels (e.g., clustering, representation, generation, etc.)
  \item Reinforcement learning: rewards given for actions (e.g., gaming, pricing, etc.)
  \item Others: semi-supervised, active learning, etc.
\end{itemize}

Active focuses in machine learning research:
\begin{itemize}[nosep]
  \item Representation: improving the encoding of data into a space
  \item Generalization: improving the use of the model on new distributions
  \item Interpretation: understanding how deep learning actually works
  \item Complexity: improving time/space requirements
  \item Efficiency: reducing the amount of samples required
  \item Privacy: respecting legal/ethical concerns of data sourcing
  \item Robustness: gracefully failing under errors or malicious attack
  \item Applications
\end{itemize}


\end{document}
