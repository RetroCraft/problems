\documentclass[11pt]{article}

\usepackage{physics}
\usepackage{amsfonts,amsmath,amssymb,amsthm}
\usepackage{enumerate}
\usepackage{titlesec}
\usepackage{fancyhdr}
\usepackage{multicol}

\headheight 13.6pt
\setlength{\headsep}{10pt}
\textwidth 15cm
\textheight 24.3cm
\evensidemargin 6mm
\oddsidemargin 6mm
\topmargin -1.1cm
\setlength{\parskip}{1.5ex}
\parindent=0pt

\author{James Ah Yong}

\pagestyle{fancy}
\fancyhf{}
\fancyfoot[c]{\thepage}
\makeatletter
\lhead{\@title}
\rhead{\@author}

\fancypagestyle{firstpage}{
  \fancyhf{}
  \rhead{\@author}
  \fancyfoot[c]{\thepage}
}

% Sets
\newcommand{\N}{\mathbb{N}}
\newcommand{\Z}{\mathbb{Z}}
\newcommand{\Q}{\mathbb{Q}}
\newcommand{\R}{\mathbb{R}}
\newcommand{\C}{\mathbb{C}}
\newcommand{\U}{\mathcal{U}}
\newcommand{\sym}{\mathbin{\triangle}}

% Functions
\DeclareMathOperator{\sgn}{sgn}
\DeclareMathOperator{\im}{im}
\DeclareMathOperator{\lcm}{lcm}

% Operators
\newcommand{\Rarr}{\Rightarrow}
\newcommand{\Larr}{\Leftarrow}
\newcommand{\Harr}{\Leftrightarrow}
\usepackage{mathtools} % for \DeclarePairedDelimiter macro
\DeclarePairedDelimiter\ceil{\lceil}{\rceil}
\DeclarePairedDelimiter\floor{\lfloor}{\rfloor}
\newcommand{\dyx}{\dv{y}{x}}

% Macros
% properly typeset ε-δ (epsilon en dash delta)
\newcommand{\epsdel}[1][\delta]{\ensuremath{\epsilon\mathit{\textnormal{--}}#1}}
\newcommand{\by}[1]{& \text{by #1}}
\newcommand{\IH}{\by{inductive hypothesis}}
\newcommand{\pf}[2]{%
\let\tmp\relax\newcommand\tmp[1]{#1}
\ensuremath{p_1^{\tmp{1}}p_2^{\tmp{2}}\cdots p_#2^{\tmp{#2}}}}
% multiple choice (remove spacing between items)
\newenvironment{choices}
{\begin{enumerate}[(a)]
    \setlength{\parskip}{0ex}
    }{
  \end{enumerate}}

% Typesetting
\usepackage{array}   % for \newcolumntype macro
\newcolumntype{C}{>{$}c<{$}} % math version of "C" column type
\newcommand{\dlim}[2]{\displaystyle\lim_{#1\to#2}} % totally not \dfrac ripoff
\newcommand{\dilim}[1]{\dlim{#1}{\infty}} % infinite limits
\newcommand{\ilim}[1]{\lim_{#1\to\infty}}
\usepackage{cancel}

% Auto-number questions
\newcommand{\QType}{Q}
\renewcommand{\theparagraph}{\QType\ifnum\value{paragraph}<10 0\fi\arabic{paragraph}}
\setcounter{secnumdepth}{6}
\newcommand{\question}{\par\refstepcounter{paragraph}\textbf{\theparagraph}.\space}

% Question sections
\titleformat{\section}{\normalsize\bfseries}{\thesection}{1em}{}
\newcommand{\qsection}[2]{%
  \renewcommand{\QType}{#2}
  \section*{#1}
  \refstepcounter{section}
}

\title{MATH 137 Fall 2020: Practice Assignment 8}

\begin{document}
\thispagestyle{firstpage}

\textbf{\@title}

\question Find $\dyx$ for $\arcsin(x^2y)+xy=1$.
\begin{proof}[Solution]
  We implicitly differentiate with respect to $x$:
  \begin{align*}
    \dv{x}(\arcsin(x^2y) + xy)                            & = \dv{x}(1)                                                   \\
    \frac{1}{\sqrt{1-(x^2 y)^2}}\dv{x}(x^2y) + \dv{x}(xy) & = 0                                                           \\
    \frac{2xy + x^2\dyx}{\sqrt{1-x^4y^2}} + x\dyx + y     & = 0                                                           \\
    \dyx{}\left(\frac{x^2}{\sqrt{1-x^4y^2}}+x\right)      & = - y - \frac{2xy}{\sqrt{1-x^4y^2}}                           \\
    \dyx                                                  & = -\frac{2xy+y\sqrt{1-x^4y^2}}{x^2+x\sqrt{1-x^4y^2}} \qedhere
  \end{align*}
\end{proof}


\question Find the equation of the tangent line at the point $(0, -1)$ to the curve defined by
\[ xy+y^3 = \arctan(x)-1. \]
\begin{proof}[Solution]
  We apply the point-slope form of a line: $y = y_0+m(x-x_0)$.
  To find the slope, implicitly differentiate with respect to $x$:
  \begin{align*}
    \dv{x}(xy+y^3)         & = \dv{x}(\arctan x - 1)            \\
    x\dyx + y + (3y^2)\dyx & = \frac{1}{1+x^2}                  \\
    \dyx{}(x+3y^2)         & = \frac{1-y(1+x^2)}{1+x^2}         \\
    \dyx                   & = \frac{1-y-x^2y}{(1+x^2)(x+3y^2)}
  \end{align*}
  We can now find $m$ by substituting $x=0$ and $y=-1$:
  \begin{align*}
    m & = \frac{1-(-1)-(0)^2(-1)}{(1+(0)^2)((0)+3(-1)^2)} \\
      & = \frac23
  \end{align*}
  Therefore, the equation of the tangent line is $y = -1 + \frac23(x-0) = \frac23x - 1$.
\end{proof}


\question Use the logarithmic differentiation to find $\dv{y}{x}$
\begin{enumerate}[(a)]
  \item $y=x^{\sin x}$ with $x > 0$.
        \begin{proof}[Solution]
          We may take the logarithm of both sides,
          then implicitly differentiate with respect to $x$:
          \begin{align*}
            \ln y           & = \sin x \ln x                                \\
            \dv{x}(\ln y)   & = \dv{x}(\sin x \ln x)                        \\
            \frac{1}{y}\dyx & = \cos x\ln x + \frac{\sin x}{x}              \\
            \dyx            & = \frac{y}{x}(x\cos x\ln x + \sin x) \qedhere
          \end{align*}

        \end{proof}
  \item $y=(2x)^{x^{1/3}}$.
        \begin{proof}[Solution]
          Likewise,
          \begin{align*}
            \ln y       & = x^{\frac13}\ln 2x                                                 \\
            \frac1y\dyx & = \frac{1}{3x^{\frac23}}\cdot \ln 2x + x^{\frac13}\cdot \frac{1}{x} \\
            \dyx        & = \frac{y(\ln 2x + 3)}{3x^{\frac23}} \qedhere
          \end{align*}
        \end{proof}
\end{enumerate}


\question Find all critical points for the following functions.
\begin{enumerate}[(a)]
  \item $f(x)=x+\dfrac{1}{x}$.
        \begin{proof}[Solution]
          Calculate that $f'(x) = 1 - \frac{1}{x^2} = \frac{x^2-1}{x^2}$.
          Recall that critical points are defined as those where the derivative is zero or undefined.
          Since this is a rational function, we may say that all zeroes are zeroes of the denominator,
          and all undefined points are zeroes of the numerator.
          The numerator is zero at $x=\pm1$, and the denominator is zero at $x=0$.

          Therefore, critical points occur at $x=-1,0,1$.
        \end{proof}
  \item $f(x)=\dfrac{(x-1)^3}{(x+1)^4}$.
        \begin{proof}[Solution]
          Likewise, the numerator is zero at $x=1$, and the denominator at $x=-1$.

          Therefore, critical points occur at $x=-1,1$.
        \end{proof}
\end{enumerate}


\question Find the global maximum and minimum of $f(x)=2\cos x + \sin 2x$ for $x \in [0,\frac\pi2]$.
\begin{proof}[Solution]
  Global extrema can either occur on the endpoints or on the open interval.
  We can calculate $f(0) = 2(1)+0 = 2$ and $f(\frac\pi2) = 2(0) + 0 = 0$.

  Local extrema on an open interval are given by points where $f'(x)=0$.
  Taking the derivative, $f'(x) = -2\sin x + 2\cos 2x$.
  We can solve the trigonometric equation $0 = \cos 2x - \sin x$ using the double angle formula:
  \begin{align*}
    0 & = \cos 2x - \sin x          \\
      & = 1 - 2\sin^2 x - \sin x    \\
      & = 2\sin^2 x + \sin x - 1    \\
      & = (2\sin x - 1)(\sin x + 1)
  \end{align*}
  So we must have either $\sin x = \frac12$ or $\sin x = -1$.
  Limited to the domain $x\in(0, \frac\pi2)$, our only solution is $x = \frac\pi6$.
  We check $f'(\frac\pi6)$ and notice it is
  $2(\frac{\sqrt{3}}{2}) + \frac{\sqrt{3}}{2} = \frac{3\sqrt{3}}{2} \cong 2.6$.

  Of these three, we have the global minimum $0$ at $x=\frac\pi2$
  and maximum $\frac{3\sqrt{3}}{2}$ at $x=\frac\pi6$.
\end{proof}


\question Let us use the Mean Value Theorem to compare the geometric and arithmetic means.
\begin{enumerate}[(a)]
  \item Use the MVT to show that \[ \sqrt{b} - \sqrt{a} < \frac{b-a}{2\sqrt{a}} \] for $0 < a < b$.
        \begin{proof}
          Consider the function $f(x) = \sqrt{x}$.
          Since $\sqrt{x}$ is both continuous and differentiable on $[a,b]$, we may apply the MVT\@.
          There exists a $c\in(a,b)$ such that \begin{align*}
            f'(c)               & = \frac{f(b)-f(a)}{b-a}         \\
            \frac{1}{2\sqrt{c}} & = \frac{\sqrt{b}-\sqrt{a}}{b-a}
          \end{align*}
          Now, $c > a$ so $\frac{1}{2\sqrt{a}} > \frac{1}{2\sqrt{c}}$.
          We multiply through $b-a$, which is positive since $b > a$:
          \begin{align*}
            \frac{1}{2\sqrt{a}}   & > \frac{1}{2\sqrt{c}}           \\
                                  & > \frac{\sqrt{b}-\sqrt{a}}{b-a} \\
            \frac{b-a}{2\sqrt{a}} & > \sqrt{b}-\sqrt{a}
          \end{align*}
          completing the proof.
        \end{proof}
  \item Use part (a) to show that, for $0 < a < b$,
        the geometric mean $\sqrt{ab}$ is always smaller than the arithmetic mean $\frac12(a+b)$,
        that is, show that \[ \sqrt{ab} < \frac{a+b}{2}. \]
        \begin{proof}
          Manipulate the result from above, knowing that $\sqrt{a}$ is a positive number:
          \begin{align*}
            \sqrt{b} - \sqrt{a}    & < \frac{b-a}{2\sqrt{a}} \\
            2\sqrt{a}\sqrt{b} - 2a & < b-a                   \\
            2\sqrt{ab}             & < a + b                 \\
            \sqrt{ab}              & < \frac{a+b}{2}
          \end{align*}
          as desired.
        \end{proof}
\end{enumerate}


\question Show that the function $f(x)=2x^5+2x+1$ has exactly one root without sketching the graph of the function.

\textbf{Hint:} Assume there is more than 1 root and use the MVT to build a contradiction.
\begin{proof}
  Notice that $f(0) = 1$ and $f(-1) = -3$.
  Since polynomials are continuous, we may apply the IVT\@.
  Therefore, a root exists on $x\in(-1,0)$.

  Suppose for a contradiction that $f(x) = 2x^5 + 2x + 1$ has more than one root.
  Then, by Rolle's Theorem, there exists some point where $f'(x) = 0$.
  However, $f'(x) = 10x^4 + 2$ which is positive for all real $x$.
  Therefore, by contradiction, $f(x)$ has at most one root.
\end{proof}


\question Let $f(x)$ be differentiable on $(a, b)$ and $f'(x)$ be continuous on $(a, b)$.
Assume there are three points $x_1$, $x_2$, and $x_3$ with each $x_i\in(a,b)$
and with $x_1<x_2<x_3$ such that \[ f(x_1) < f(x_2) \quad \textrm{and} \quad f(x_2) > f(x_3). \]
Use the MVT and the IVT to show that there must be a point $c\in(x_1, x_3)$ such that $f'(c) = 0$.
\begin{proof}
  Since $f$ is differentiable on both $[x_1,x_2]$ and $[x_2, x_3]$, we may apply the MVT\@.
  As $f(x_1) < f(x_2)$, the mean slope is a positive value.
  Therefore, there exists a $c_1\in(x_1,x_2)$ where $f'(c_1)$ is that positive value.
  As $f(x_2) > f(x_3)$, the mean slope is a negative value.
  Therefore, there also exists a $c_2\in(x_2,x_3)$ where $f'(c_2)$ is that negative value.
  
  We are also given that $f'$ is continuous on $[x_1,x_3]$, so we may apply the IVT\@.
  From above, $f'(c_2) < 0 < f'(c_1)$, so there must exist a $c\in(c_1,c_2)$ where $f'(c)=0$.
  Since $(c_1,c_2) \subsetneq (x_1,x_3)$, we may say that $c\in(x_1,x_3)$, completing the proof.
\end{proof}

\end{document}