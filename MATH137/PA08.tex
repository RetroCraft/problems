\documentclass[11pt]{article}

\usepackage{physics}
\usepackage{amsfonts,amsmath,amssymb,amsthm}
\usepackage{enumerate}
\usepackage{titlesec}
\usepackage{fancyhdr}
\usepackage{multicol}

\headheight 13.6pt
\setlength{\headsep}{10pt}
\textwidth 15cm
\textheight 24.3cm
\evensidemargin 6mm
\oddsidemargin 6mm
\topmargin -1.1cm
\setlength{\parskip}{1.5ex}
\parindent=0pt

\author{James Ah Yong}

\pagestyle{fancy}
\fancyhf{}
\fancyfoot[c]{\thepage}
\makeatletter
\lhead{\@title}
\rhead{\@author}

\fancypagestyle{firstpage}{
  \fancyhf{}
  \rhead{\@author}
  \fancyfoot[c]{\thepage}
}

% Sets
\newcommand{\N}{\mathbb{N}}
\newcommand{\Z}{\mathbb{Z}}
\newcommand{\Q}{\mathbb{Q}}
\newcommand{\R}{\mathbb{R}}
\newcommand{\C}{\mathbb{C}}
\newcommand{\U}{\mathcal{U}}
\newcommand{\sym}{\mathbin{\triangle}}

% Functions
\DeclareMathOperator{\sgn}{sgn}
\DeclareMathOperator{\im}{im}
\DeclareMathOperator{\lcm}{lcm}

% Operators
\newcommand{\Rarr}{\Rightarrow}
\newcommand{\Larr}{\Leftarrow}
\newcommand{\Harr}{\Leftrightarrow}
\usepackage{mathtools} % for \DeclarePairedDelimiter macro
\DeclarePairedDelimiter\ceil{\lceil}{\rceil}
\DeclarePairedDelimiter\floor{\lfloor}{\rfloor}
\newcommand{\dyx}{\dv{y}{x}}

% Macros
% properly typeset ε-δ (epsilon en dash delta)
\newcommand{\epsdel}[1][\delta]{\ensuremath{\epsilon\mathit{\textnormal{--}}#1}}
\newcommand{\by}[1]{& \text{by #1}}
\newcommand{\IH}{\by{inductive hypothesis}}
\newcommand{\pf}[2]{%
\let\tmp\relax\newcommand\tmp[1]{#1}
\ensuremath{p_1^{\tmp{1}}p_2^{\tmp{2}}\cdots p_#2^{\tmp{#2}}}}
% multiple choice (remove spacing between items)
\newenvironment{choices}
{\begin{enumerate}[(a)]
    \setlength{\parskip}{0ex}
    }{
  \end{enumerate}}

% Typesetting
\usepackage{array}   % for \newcolumntype macro
\newcolumntype{C}{>{$}c<{$}} % math version of "C" column type
\newcommand{\dlim}[2]{\displaystyle\lim_{#1\to#2}} % totally not \dfrac ripoff
\newcommand{\dilim}[1]{\dlim{#1}{\infty}} % infinite limits
\newcommand{\ilim}[1]{\lim_{#1\to\infty}}
\usepackage{cancel}

% Auto-number questions
\newcommand{\QType}{Q}
\renewcommand{\theparagraph}{\QType\ifnum\value{paragraph}<10 0\fi\arabic{paragraph}}
\setcounter{secnumdepth}{6}
\newcommand{\question}{\par\refstepcounter{paragraph}\textbf{\theparagraph}.\space}

% Question sections
\titleformat{\section}{\normalsize\bfseries}{\thesection}{1em}{}
\newcommand{\qsection}[2]{%
  \renewcommand{\QType}{#2}
  \section*{#1}
  \refstepcounter{section}
}

\title{MATH 137 Fall 2020: Practice Assignment 8}

\begin{document}
\thispagestyle{firstpage}

\textbf{\@title}

\question Find $\dv{y}{x}$ for $\arcsin(x^2y)+xy=1$.
\begin{proof}[Solution]
\end{proof}


\question Find the equation of the tangent line at the point $(0, -1)$ to the curve defined by
\[ xy+y^3 = \arctan(x)-1. \]
\begin{proof}[Solution]
\end{proof}


\question Use the logarithmic differentiation to find $\dv{y}{x}$
\begin{enumerate}[(a)]
  \item $y=x^{\sin x}$ with $x > 0$.
        \begin{proof}[Solution]
        \end{proof}
  \item $y=(2x)^{x^{1/3}}$.
        \begin{proof}[Solution]
        \end{proof}
\end{enumerate}

\question Find all critical points for the following functions.
\begin{enumerate}
  \item $f(x)=x+\dfrac{1}{x}$.
        \begin{proof}[Solution]
        \end{proof}
  \item $f(x)=\dfrac{(x-1)^3}{(x+1)^4}$.
        \begin{proof}[Solution]
        \end{proof}
\end{enumerate}


\question Find the global maximum and minimum of $f(x)=2\cos x + \sin 2x$ for $x \in [0,\frac\pi2]$.
\begin{proof}[Solution]
\end{proof}


\question Let us use the Mean Value Theorem to compare the geometric and arithmetic means.
\begin{enumerate}
  \item Use the MVT to show that \[ \sqrt{b} - \sqrt{a} < \frac{b-a}{2\sqrt{a}} \] for $0 < a < b$.
        \begin{proof}
        \end{proof}
  \item Use part (a) to show that, for $0 < a < b$,
        the geometric mean $\sqrt{ab}$ is always smaller than the arithmetic mean $\frac12(a+b)$,
        that is, show that \[ \sqrt{ab} < \frac{a+b}{2}. \]
        \begin{proof}
        \end{proof}
\end{enumerate}


\question Show that the function $f(x)=2x^5+2x+1$ has exactly one root without sketching the graph of the function.

\textbf{Hint:} Assume there is more than 1 root and use the MVT to build a contradiction.
\begin{proof}
\end{proof}


\question Let $f(x)$ be differentiable on $(a, b)$ and $f'(x)$ be continuous on $(a, b)$.
Assume there are three points $x_1$, $x_2$, and $x_3$ with each $x_i\in(a,b)$
and with $x_1<x_2<x_3$ such that \[ f(x_1) < f(x_2) \quad \textrm{and} \quad f(x_2) > f(x_3). \]
Use the MVT and the IVT to show that there must be a point $c\in(x_1, x_3)$ such that $f'(c) = 0$.
\begin{proof}
\end{proof}

\end{document}