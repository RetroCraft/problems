\documentclass{agony}
\title{MATH 137 Fall 2020: Practice Assignment 1}

\begin{document}
\thispagestyle{firstpage}

\textbf{\thetitle}

\question Solve the following equations/inequalities:
\begin{enumerate}
  \item $|x-2|=5\iff x-2=\pm5\iff x=2\pm5\iff x=-3,7$
  \item $|x-4|=|3x+2|$: Take cases: $x-4=3x+2\iff x=-1$ and $x-4=-3x-2\iff x=\frac12$.
        Check: $|-1-4|\neq|-3+2|$ but $|\frac12-4|=\frac72=|\frac32+2|$, so $x=\frac12$.
  \item $|x+1|>1$ = ``distance between $x$ and $-1$ greater than $1$'' = $x\in\R\setminus[-2,0]=(-\infty,-2)\cup(0,\infty)$
  \item $|x+3|+|1-2x|\leq5$. Take cases $x\in(-\infty,-3],[-3,\frac12],[\frac12,\infty)$:
        \begin{enumerate}[1.]
          \item $-(x+3)+(1-2x)\leq5\iff x\geq-\frac37$ but $x\in(-\infty,-3]$ so no solution
          \item $(x+3)+(1-2x)\leq5\iff x\geq-1$ but $x\in[-3,\frac12]$ so $x\in[-1,\frac12]$
          \item $(x+3)-(1-2x)\leq5\iff x\leq1$ but $x\in[\frac12,\infty)$ so $x\in[\frac12,1]$
        \end{enumerate}
        Therefore $x\in[-1,\frac12]\cup[\frac12,1]=[-1,1]$.
  \item $|x-4||x+2|=|(x-4)(x+2)|=|x^2-2x-8|>7$. Take cases:
        \begin{enumerate}[1.]
          \item $x^2-2x-8>7\iff x^2-2x-15>0\iff(x-5)(x+3)>0\iff x\in(-\infty,-3)\cup(5,\infty)$
          \item $x^2-2x-8<-7\iff x^2-2x-1<0\iff (x-1)^2<0$ which is false for all  real $x$.
        \end{enumerate}
        Therefore, $x\in(-\infty,-3)\cup(5-,\infty)$.
\end{enumerate}

\question Prove that, for any real numbers $a$ and $b$:
\begin{enumerate}
  \item $|a|-|b| \leq |a+b|$
        \begin{proof}
          Moving the $|b|$ term to the other side, $|a|\leq|a+b|+|b|$.
          Let $b'=-b$, so $|a|\leq|a+(-b')|+|-b'|$.
          Add some zeroes, $|a-0|\leq|a-b'|+|b'-0|$, which holds by the triangle inequality.
        \end{proof}
  \item $|a|+|b| \leq |a+b|$
        \begin{proof}
          Again move the $|b|$ term to the other side, $|a|\leq|a-b|+|b|$, and add some zeroes, $|a-0|\leq|a-b|+|b-0|$, which holds by the triangle inequality.
        \end{proof}
  \item $|a-b| \leq |a+b|$ or $|a+b| \leq |a-b|$
        \begin{proof}
          Consider for $a=b\neq0$: the first is true as $0\leq2a$ and the second is false as $2a\not\leq0$. Since the second is false for some case, it is false generally.

          For $a=-b\neq0$: the first is now false as $2a\not\leq0$.

          Therefore, both statements are false for some $x\in\mathbb R$, so they are not true.
        \end{proof}
\end{enumerate}

\question Write the expression
\[ \frac{1}{2}(a+b+|a-b|) \]
as a piecewise function. You should consider 2 cases. What is this expression calculating for any two real numbers $a$ and $b$? What about the expression
\[ \frac{1}{2}(a+b-|a-b|)? \]
\begin{proof}[Solution]
  Consider the cases:
  \begin{align*}
    \frac12(a+b+|a-b|)
     & = \begin{cases}\frac12(a+b+a-b)&a\geq b \\ \frac12(a+b-a+b)&a<b\end{cases} \\
     & =\begin{cases} a&a\geq b \\ b&a<b \end{cases}  \\
  \end{align*}
  Which is just $\max \{a,b\}$.

  Again, consider the cases:
  \begin{align*}
    \frac12(a+b-|a-b|)
     & =\begin{cases}\frac12(a+b-a+b)&a\geq b \\ \frac12(a+b+a-b)&a<b\end{cases} \\
     & =\begin{cases} b&a\geq b \\ a&a<b \end{cases} \\
  \end{align*}
  Which is just $\min \{a,b\}$.
\end{proof}

\end{document}