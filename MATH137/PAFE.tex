\documentclass[11pt]{article}

\usepackage{physics}
\usepackage{amsfonts,amsmath,amssymb,amsthm}
\usepackage{enumerate}
\usepackage{titlesec}
\usepackage{fancyhdr}
\usepackage{multicol}

\headheight 13.6pt
\setlength{\headsep}{10pt}
\textwidth 15cm
\textheight 24.3cm
\evensidemargin 6mm
\oddsidemargin 6mm
\topmargin -1.1cm
\setlength{\parskip}{1.5ex}
\parindent=0pt

\author{James Ah Yong}

\pagestyle{fancy}
\fancyhf{}
\fancyfoot[c]{\thepage}
\makeatletter
\lhead{\@title}
\rhead{\@author}

\fancypagestyle{firstpage}{
  \fancyhf{}
  \rhead{\@author}
  \fancyfoot[c]{\thepage}
}

% Sets
\newcommand{\N}{\mathbb{N}}
\newcommand{\Z}{\mathbb{Z}}
\newcommand{\Q}{\mathbb{Q}}
\newcommand{\R}{\mathbb{R}}
\newcommand{\C}{\mathbb{C}}
\newcommand{\U}{\mathcal{U}}
\newcommand{\sym}{\mathbin{\triangle}}

% Functions
\DeclareMathOperator{\sgn}{sgn}
\DeclareMathOperator{\im}{im}
\DeclareMathOperator{\lcm}{lcm}

% Operators
\newcommand{\Rarr}{\Rightarrow}
\newcommand{\Larr}{\Leftarrow}
\newcommand{\Harr}{\Leftrightarrow}
\usepackage{mathtools} % for \DeclarePairedDelimiter macro
\DeclarePairedDelimiter\ceil{\lceil}{\rceil}
\DeclarePairedDelimiter\floor{\lfloor}{\rfloor}
\newcommand{\dyx}{\dv{y}{x}}

% Macros
% properly typeset ε-δ (epsilon en dash delta)
\newcommand{\epsdel}[1][\delta]{\ensuremath{\epsilon\mathit{\textnormal{--}}#1}}
\newcommand{\by}[1]{& \text{by #1}}
\newcommand{\IH}{\by{inductive hypothesis}}
\newcommand{\pf}[2]{%
\let\tmp\relax\newcommand\tmp[1]{#1}
\ensuremath{p_1^{\tmp{1}}p_2^{\tmp{2}}\cdots p_#2^{\tmp{#2}}}}
% multiple choice (remove spacing between items)
\newenvironment{choices}
{\begin{enumerate}[(a)]
    \setlength{\parskip}{0ex}
    }{
  \end{enumerate}}

% Typesetting
\usepackage{array}   % for \newcolumntype macro
\newcolumntype{C}{>{$}c<{$}} % math version of "C" column type
\newcommand{\dlim}[2]{\displaystyle\lim_{#1\to#2}} % totally not \dfrac ripoff
\newcommand{\dilim}[1]{\dlim{#1}{\infty}} % infinite limits
\newcommand{\ilim}[1]{\lim_{#1\to\infty}}
\usepackage{cancel}

% Auto-number questions
\newcommand{\QType}{Q}
\renewcommand{\theparagraph}{\QType\ifnum\value{paragraph}<10 0\fi\arabic{paragraph}}
\setcounter{secnumdepth}{6}
\newcommand{\question}{\par\refstepcounter{paragraph}\textbf{\theparagraph}.\space}

% Question sections
\titleformat{\section}{\normalsize\bfseries}{\thesection}{1em}{}
\newcommand{\qsection}[2]{%
  \renewcommand{\QType}{#2}
  \section*{#1}
  \refstepcounter{section}
}

\title{MATH 137 Fall 2020: Practice Assignment on Chapter 5}

\begin{document}
\thispagestyle{firstpage}
\textbf{\@title}

\question Approximate $f(x)=x^{-1/2}$ with a Taylor polynomial of degree 2 centered at $x=4$.
Use Taylor's Theorem to get an upper bound on the error if $3.5 \leq x \leq 4.5$.
\begin{proof}[Solution]
  First calculate $f'(x) = -\frac12 x^{-3/2}$ and $f''(x) = \frac34 x^{-5/2}$.
  Then, $f(4) = 4^{-1/2} = \frac12$, $f'(4) = -\frac12 4^{-3/2} = \frac{1}{16}$,
  and $f''(4) = \frac34 4^{-5/2} = \frac{3}{128}$. Then,
  \begin{align*}
    T_{2,4}(x) & = f(4) + f'(4)(x-4) + \frac12 f''(4)(x-4)^2       \\
               & = \frac12 + \frac{x-4}{16} + \frac{3(x-4)^2}{256}
  \end{align*}

  Taylor's Theorem gives the error $R_{2,4}(x)=\frac{f^{(3)}(c)}{3!}(x-4)^3$ for some $c\in[3.5,4.5]$.
  The third derivative of $f$ is $-\frac{15}{8} x^{-7/2}$.
  This is strictly increasing (i.e. $f^{(4)} > 0$) and negative (i.e. $f^{(3)} < 0$),
  so the maximum $|f^{(3)}(c)|$ is at $|f^{(3)}(3.5)| \approx 0.02338$.

  Thus, $|R_{2,4}(x)| \leq \abs{\frac{f^{(3)}(3.5)}{6}(0.5)^3} \approx 4.87 \times 10^{-4}$.
\end{proof}


\question Approximate $f(x)=\ln(1+2x)$ with a Taylor polynomial of degree 3 centered at $x=1$.
Use Taylor's Theorem to get an upper bound on the error if $0.5 \leq x \leq 1.5$.
\begin{proof}[Solution]
  We have $f'(x)=\frac{2}{1+2x}$, $f''(x)=-\frac{4}{(1+2x)^2}$, and $f^{(3)}(x)=\frac{16}{(1+2x)^3}$.
  Calculate $f(1)=\ln 3$, $f'(1)=\frac{2}{3}$, $f''(1)=-\frac{4}{9}$, and $f^{(3)}(1)=\frac{16}{27}$.
  The Taylor polynomial is
  \begin{align*}
    T_{3,1}(x) & = f(1) + f'(1)(x-1) + \frac12 f''(1)(x-1)^2 + \frac16 f^{(3)}(1)(x-1)^3 \\
               & = \ln 3 + \frac{2}{3}(x-1) - \frac{2}{9}(x-1)^2 + \frac{8}{81}(x-1)^3
  \end{align*}

  Taylor's Theorem gives the error $R_{3,1}(x)=\frac{f^{(4)}(c)}{4!}(x-4)^3$ for some $c\in[0.5,1.5]$.
  The fourth derivative of $f$ is $-\frac{48}{(1+2x)^4}$.
  We can conclude that $|f^{(4)}|$ reaches its max at $c=0.5$ through geometric argument,
  knowing the function is rational with one asymptote at $x=-\frac12$ and no roots.
  We have $|f^{(4)}(0.5)| = 6$.

  Thus, $|R_{3,1}(x)| \leq \abs{\frac{f^{(4)}(0.5)}{24}(0.5)^4} = 0.015625$.
\end{proof}


\question Here we approximate the value of $\ln 2$ in two ways.
\begin{enumerate}[(a)]
  \item Find the degree 3 Taylor polynomial for $\ln(1+x)$ centred at $x = 0$.
        \begin{proof}[Solution]
          Let $f(x) = \ln(1+x)$.
          Then we have $f'(x) = \frac{1}{1+x}$, $f''(x) = -\frac{1}{(1+x)^2}$,
          and $f^{(3)}(x) = \frac{2}{(1+x)^3}$.
          Evaluating at $x=0$, we have $f(0) = 0$, $f'(0) = 1$, $f''(0) = -1$, and $f^{(3)}(0)=2$.
          Therefore, the Taylor polynomial $T_{3,0}(x)$ is
          \begin{align*}
            T_{3,0}(x) & = \frac{f^{(3)}(x)}{3!}x^3 + \frac{f''(x)}{2!}x^2 + f'(x) x + f(x) \\
                       & = \frac{1}{3}x^3 - \frac{1}{2}x^2 + x \qedhere
          \end{align*}
        \end{proof}
  \item Use $x=1$ in your polynomial from part (a) to approximate the value of $\ln 2$.
        \begin{proof}[Solution]
          Plug and chug: $T_{3,0}(1) = \frac{1}{3} - \frac{1}{2} + 1 = \frac{5}{6}$.
        \end{proof}
  \item Use $x=-\frac12$ in your polynomial from part (a) to approximate the value of $\ln 2$.
        You will need to relate your answer to $\ln 2$ with log rules.
        Show that the upper bound on the error given by Taylor's Theorem is the same
        for your approximations from parts (b) and (c).
        \begin{proof}[Solution]
          At $x=-\frac12$, we have $f(x) = \ln(\frac12) = -\ln 2$. Plugging and chugging,
          $T_{3,0}(-\frac12) = \frac13(-\frac18) - \frac12(\frac14) - \frac12 = -\frac23$.
          Therefore, our estimate is $\ln 2 \approx \frac23$.

          The error $|R_{3,0}|$ depends on the maximum value of $|f^{(4)}(x)| = \frac{6}{(1+x)^4}$.
          This value is decreasing everywhere, the maximum value is at $x=0$ for $[0,1]$
          and $x=-\frac12$ for $[-\frac12,0]$: $|f^{(4)}(0)| = 6$ and $|f^{(4)}(-\frac12)| = 96$.

          Therefore, the error for part (b) is at least
          \[ |R_{3,0}(1)| \leq \frac{6}{4!}(1)^4 = 0.25 \]
          and the error above is at least
          \[ |R_{3,0}(-0.5)| \leq \frac{96}{4!}(0.5)^4 = 0.25 \qedhere \]
        \end{proof}
  \item Use a calculator to compare your approximations in part (b) and (c)
        with the actual value of $\ln 2$. Which is actually closer, and why does this make sense?
        \begin{proof}[Solution]
          Calculator gives $\ln 2 \approx 0.693147$.

          Part (b) estimated $\frac56 \approx 0.833333$ which an error of about $-0.140186$
          and part (c) estimated $\frac23 \approx 0.666667$ which is off by $0.026480$.

          Part (c) was actually closer, and this makes sense because we are working closer
          to the center of the Taylor polynomial.
        \end{proof}
\end{enumerate}


\question Use Taylor's Theorem to find $n\in\N$ so that using $T_{n,0}(x)$
to approximate $e^x$ at $x=0.1$ has an error of at most $0.00001$
\begin{proof}[Solution]
  Let $f(x) = e^x$. Recall that $f^{(n)}(x) = e^x$ for any $n \in \N$.
  Since $e^x$ is increasing everywhere, the maximum on $[0,0.1]$ will be at $x=0.1$
  Then, Taylor's Theorem gives
  \[ |R_{n,0}(x)| \leq \frac{e^{0.1}}{(n+1)!}(0.1)^{n+1} \]
  but we have $e^{0.1}$ stuck in there.
  We can give an upper bound by doing some shenanigans.
  $e^0.1$ is the tenth root of $e$.
  This is clearly less than the tenth root of 3.
  Now, $1.1^{10} \approx 2.6$ and $1.2^{10} \approx 6.2$, so we give $\sqrt[10]{e} \leq 1.2$. Then,
  \begin{align*}
    |R_{n,0}(x)| & \leq \frac{e^x}{(n+1)!}(0.1)^{n+1} \\
    0.00001      & \leq \frac{1.2}{(n+1)!}(0.1)^{n+1}
  \end{align*}
  and we find by Pain and Agony\textsuperscript{TM} that we need $n \geq 3$.
\end{proof}


\question Let us revisit Newton's Method one more time using Taylor's Theorem.
Suppose we are approximating the root $r$ of the function $f$.
Recall that from an initial approximation $x_1$,
we obtained the successive approximations using the recursive formula
\[ x_{n+1} = x_n - \frac{f(x_n)}{f'(x_n)}. \]
Use Taylor's Theorem (or inequality) with $n=1$, $a=x_n$, and $x=r$
to show that if $f''(x)$ exists on an interval $I$ containing $r$, $x_n$, and $x_{n+1}$,
and $|f''(x)| \leq M$, $|f'(x)| \geq K$ for all $x \in I$, then
\[ \abs{x_{n+1}-r} \leq \frac{M}{2K}\abs{x_n-r}^2. \]
(Note that this says that if the error at stage $n$ is at most $10^{-m}$,
then the error at stage $n+1$ is at most $\frac{M}{2K}10^{-2m}$, or in other words,
that successive iterations are accurate to approximately twice as many decimal places!)
\begin{proof}
  We follow the instructions and do as we're told.
  Then, we have $f(r) = 0$, $f(r) = T_{1,x_n}(r) + R_{1,x_n}(r)$, and $T_{1,x_n}(r) = f'(r)(r-x_n) + f(r)$.

  Substituting, $0 = f'(r)(r-x_n) + f(r) + R_{1,x_n}(r)$.

  Then, we have $R_{1,x_n}(r) = -f(r) + f'(r)(x_n-r)$.
  We want $\frac{f(x_n)}{f'(x_n)}$ so we divide through by $f'(x_n)$
  to get $\frac{R_{1,x_n}(r)}{f'(x_n)} = x_n - \frac{f(x_n)}{f'(x_n)} - r = x_{n+1} - r$.
  Therefore, $|\frac{R_{1,x_n}(r)}{f'(x_n)}| = |x_{n+1} - r|$.
  Since $f'(x_n) \geq K$, we have $|x_{n+1} - r| \leq \frac{1}{K}|R_{1,x_n}(r)|$.

  We can finally apply Taylor's Theorem and get that $|R_{1,x_n}(r)| = \frac{f''(c)}{2}|x_n-r|^2$.
  We know that $f''(c) \leq M$ so $|R_{1,x_n}(r)| \leq \frac{M}{2}|x_n-r|^2$.

  Combining these, we have $|x_{n+1} - r| \leq \frac{M}{2K}|x_n-r|^2$.
\end{proof}


\end{document}