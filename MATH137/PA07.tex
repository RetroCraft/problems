\documentclass[11pt]{article}

\usepackage{physics}
\usepackage{amsfonts,amsmath,amssymb,amsthm}
\usepackage{enumerate}
\usepackage{titlesec}
\usepackage{fancyhdr}
\usepackage{multicol}

\headheight 13.6pt
\setlength{\headsep}{10pt}
\textwidth 15cm
\textheight 24.3cm
\evensidemargin 6mm
\oddsidemargin 6mm
\topmargin -1.1cm
\setlength{\parskip}{1.5ex}
\parindent=0pt

\author{James Ah Yong}

\pagestyle{fancy}
\fancyhf{}
\fancyfoot[c]{\thepage}
\makeatletter
\lhead{\@title}
\rhead{\@author}

\fancypagestyle{firstpage}{
  \fancyhf{}
  \rhead{\@author}
  \fancyfoot[c]{\thepage}
}

% Sets
\newcommand{\N}{\mathbb{N}}
\newcommand{\Z}{\mathbb{Z}}
\newcommand{\Q}{\mathbb{Q}}
\newcommand{\R}{\mathbb{R}}
\newcommand{\C}{\mathbb{C}}
\newcommand{\U}{\mathcal{U}}
\newcommand{\sym}{\mathbin{\triangle}}

% Functions
\DeclareMathOperator{\sgn}{sgn}
\DeclareMathOperator{\im}{im}
\DeclareMathOperator{\lcm}{lcm}

% Operators
\newcommand{\Rarr}{\Rightarrow}
\newcommand{\Larr}{\Leftarrow}
\newcommand{\Harr}{\Leftrightarrow}
\usepackage{mathtools} % for \DeclarePairedDelimiter macro
\DeclarePairedDelimiter\ceil{\lceil}{\rceil}
\DeclarePairedDelimiter\floor{\lfloor}{\rfloor}
\newcommand{\dyx}{\dv{y}{x}}

% Macros
% properly typeset ε-δ (epsilon en dash delta)
\newcommand{\epsdel}[1][\delta]{\ensuremath{\epsilon\mathit{\textnormal{--}}#1}}
\newcommand{\by}[1]{& \text{by #1}}
\newcommand{\IH}{\by{inductive hypothesis}}
\newcommand{\pf}[2]{%
\let\tmp\relax\newcommand\tmp[1]{#1}
\ensuremath{p_1^{\tmp{1}}p_2^{\tmp{2}}\cdots p_#2^{\tmp{#2}}}}
% multiple choice (remove spacing between items)
\newenvironment{choices}
{\begin{enumerate}[(a)]
    \setlength{\parskip}{0ex}
    }{
  \end{enumerate}}

% Typesetting
\usepackage{array}   % for \newcolumntype macro
\newcolumntype{C}{>{$}c<{$}} % math version of "C" column type
\newcommand{\dlim}[2]{\displaystyle\lim_{#1\to#2}} % totally not \dfrac ripoff
\newcommand{\dilim}[1]{\dlim{#1}{\infty}} % infinite limits
\newcommand{\ilim}[1]{\lim_{#1\to\infty}}
\usepackage{cancel}

% Auto-number questions
\newcommand{\QType}{Q}
\renewcommand{\theparagraph}{\QType\ifnum\value{paragraph}<10 0\fi\arabic{paragraph}}
\setcounter{secnumdepth}{6}
\newcommand{\question}{\par\refstepcounter{paragraph}\textbf{\theparagraph}.\space}

% Question sections
\titleformat{\section}{\normalsize\bfseries}{\thesection}{1em}{}
\newcommand{\qsection}[2]{%
  \renewcommand{\QType}{#2}
  \section*{#1}
  \refstepcounter{section}
}

\title{MATH 137 Fall 2020: Practice Assignment 7}

\begin{document}
\thispagestyle{firstpage}

\textbf{\@title}

\question Let $f(x)=\sin x$.
\begin{enumerate}[(a)]
  \item Determine the equation for the linear approximation to $f(x)$ at $x=a$, $L^f_a$, for any $a\in\R$.
        \begin{proof}[Solution]
          Apply the formula, knowing $(\sin x)' = \cos x$:
          \begin{align*}
            L^f_a & = f(a)+f'(a)(x-a)                     \\
                  & = x\cos a + \sin a - a\cos a \qedhere
          \end{align*}
        \end{proof}
  \item Determine the equation for the linear approximation to $f(x)$ at $x=\frac\pi3$, $L^f_{\pi/3}$.
        \begin{proof}[Solution]
          Substituting for $x=\frac\pi3$ into part (a):
          \begin{align*}
            L^f_{\pi/3} & = x\cos\frac\pi3 + \frac{\sqrt3}{2} - \frac\pi3\cos\frac\pi3 \\
                        & = \frac12x + \frac{\sqrt3}{2} - \frac\pi6                    \\
                        & = \frac12x + \frac{3\sqrt3 - \pi}{6} \qedhere
          \end{align*}
        \end{proof}
  \item Use your answer to (b) to approximate $\sin(1)$.
        \begin{proof}[Solution]
          Substituting for $x=1$ into part (b),
          \[ L^f_{\pi/3}=\frac12(1) + \frac{3\sqrt3 - \pi}{6} =\dfrac{3+3\sqrt3-\pi}{6} \approx 0.842. \qedhere \]
        \end{proof}
  \item Since $f''(x)=-\sin x$, we can safely say that $f''(x) \leq 1$.
        Given this fact, what is the maximum error (worst-case scenario) for your answer in (c)?
        \begin{proof}[Solution]
          Recall that if $|f''(x)| \leq M$ for all $x$ in an interval $I$ containing $a$,
          then the worst case error is given by $|f(x)-L_a(x)| \leq \frac{M}{2}(x-a)^2$.

          Substitute $x=1$, $M=1$, and $a=\frac\pi3$:
          \begin{align*}
            |f(1)-L_{\pi/3}| & \leq \frac12(1-\frac\pi3)^2 \\
                             & = \frac12(\frac{2\pi}{3})^2 \\
                             & = \frac{2\pi}{9}
          \end{align*}
          Therefore, the maximum error is $\frac{2\pi}{9}$.
        \end{proof}
\end{enumerate}


\question Rich bought a yoga ball that is made of material which,
when the ball is properly inflated, is a sphere with outer radius $R$.
The manufacturer has determined that the material can tolerate a 4\% ``stretch'' beyond specifications,
meaning that if the ball is inflated in such a way that the surface area increases by more than 4\%
of the actual size, the material will rupture and the ball will deflate rather suddenly.

In the instructions for inflation, consumers are told to inflate the ball
so that one side touches a wall and the other side touches a box placed $2R$ units away from the wall.
Rich uses a ruler to measure $2R$ units from a wall, and then inflates the ball according to the instructions.
If Rich's ruler and Rich's measurement skills create an error of 3\% in excess of what he thinks he is measuring
(that is, instead of inflating to a diameter of $2R$ it is inflated to a diameter of $1.03(2R)$),
should we expect the ball to survive this initial inflation?

Use the technique of estimating change to investigate.

\begin{proof}[Solution]
\end{proof}


\question \begin{enumerate}[(a)]
  \item For $a\in\R$, $a > 0$, write down the recursive sequence that Newton's Method would generate for $f(x)=x^2-a$.
        That is, write the formula for $x_{n+1}$ in terms of $x_n$.
        \begin{proof}[Solution]
        \end{proof}
  \item Use Newton's Method to approximate each of the following, correct to 4 decimal places.
        \begin{enumerate}[(i)]
          \item $\sqrt{7}$ (use initial guess $x_1=3$)
                \begin{proof}[Solution]
                \end{proof}
          \item $\sqrt{\pi}$ (use initial guess $x_1=2$)
                \begin{proof}[Solution]
                \end{proof}
        \end{enumerate}
\end{enumerate}

\question Consider the function $f(x)=\dfrac{6x+1}{3x+5}$.
\begin{enumerate}[(a)]
  \item What is the domain of $f$?
        \begin{proof}[Solution]
          Apply rules for rational functions: $x\in\R\setminus\{-\frac53\}$.
        \end{proof}
  \item Find $f'(x)$.
        \begin{proof}[Solution]
          Apply the quotient rule:
          \begin{align*}
            f'(x) & = \frac{(3x+5)(6x+1)'-(6x+1)(3x+5)'}{(3x+5)^2} \\
                  & = \frac{6(3x+5) - 3(6x+1)}{(3x+5)^2}           \\
                  & = \frac{27}{(3x+5)^2} \qedhere
          \end{align*}
        \end{proof}
  \item There is only one point $c\in\R$ where $f(c) = 0$, find it directly.
        \begin{proof}[Solution]
          Notice that the numerator is zero when $x=-\frac16$, and the denominator is not.
          Therefore, $f(-\frac16)=0$.
        \end{proof}
  \item Now, starting with $x_1=5$ (a particularly foolish choice),
        perform 3 iterations of Newton's Method. Use 5 decimal places.
        \begin{proof}[Solution]
        \end{proof}
  \item It is clear that starting with $x_1=5$ will not lead us to the root.
        In fact, the sequence generated by Newton's Method in this case diverges.
        Prove that using $x_1=5$ as a starting value,
        Newton's Method will not converge to the root of this function.
        (Hint: show the recursive sequence you get from Newton's Method is strictly decreasing).
        \begin{proof}
        \end{proof}
\end{enumerate}


\question Using the fact that $(\ln x)' = \dfrac{1}{x}$,
prove that $(\log_a x)'=\dfrac{1}{x\ln a}$ for $a > 0$, $a \neq 1$.
\begin{proof}
  Let $0 < a \neq 1$.
  Recall the change of base formula: $\log_a x = \frac{\ln x}{\ln a}$.

  Since $\frac{1}{\ln a}$ is a constant with respect to $x$, we may pull it out of the derivative,
  and it immediately follows that $(\log_a x)' = \frac{1}{\ln a}(\ln x)' = \frac{1}{x\ln a}$.
\end{proof}


\question Find $f'(x)$ if
\begin{enumerate}[(a)]
  \item $f(x)=\sqrt{\log_{10}(7+\sin x)}$
        \begin{proof}[Solution]
          Apply the chain rule a bunch of times:
          \begin{align*}
            f'(x) & = (\log_{10}(7+\sin x)^{1/2})'                                                         \\
                  & = \frac12(\log_{10}(7+\sin x)^{-1/2})(\log_{10}(7+\sin x))'                            \\
                  & = \frac{1}{2\sqrt{\log_{10}(7+\sin x)}}\cdot\frac{1}{(7+\sin x)\ln 10}\cdot(7+\sin x)' \\
                  & = \frac{\cos x}{2\sqrt{\log_{10}(7+\sin x)}(7+\sin x)\ln 10} \qedhere
          \end{align*}
        \end{proof}
  \item $f(x)=\arcsin(\tan x+x^3)e^x$
        \begin{proof}[Solution]
          Recall that $(\arcsin x)' = \frac{1}{\sqrt{1-x^2}}$, and apply the product and chain rule:
          \begin{align*}
            f'(x) & = (\arcsin(\tan x+x^3))'e^x + \arcsin(\tan x+x^3)(e^x)'                         \\
                  & = \frac{(\tan x+x^3)'}{1-(\tan x+x^3)^2}e^x + \arcsin(\tan x+x^3)e^x            \\
                  & = \frac{\sec^2 x + 3x^2}{1-(\tan x+x^3)^2}e^x + \arcsin(\tan x+x^3)e^x \qedhere
          \end{align*}
        \end{proof}
  \item $f(x)=\frac{x}{\arctan(x^2+1)}$
        \begin{proof}[Solution]
          Recall that $(\arctan x)' = \frac{1}{1+x^2}$, and apply the quotient and chain rule:
          \begin{align*}
            f'(x) & = \frac{\arctan(x^2+1)(x)' - (x)(\arctan(x^2+1))'}{\arctan^2(x^2+1)}            \\
                  & = \frac{\arctan(x^2+1) - x\frac{(x^2+1)'}{1+(x^2+1)^2}}{\arctan^2(x^2+1)}       \\
                  & = \frac{1}{\arctan(x^2+1)} - \frac{\frac{x(2x)}{x^4+2x^2+2}}{\arctan^2(x^2+1)}  \\
                  & = \frac{1}{\arctan(x^2+1)} - \frac{2x^2}{(x^4+2x^2+2)\arctan^2(x^2+1)} \qedhere
          \end{align*}
        \end{proof}
\end{enumerate}


\question Let $f(x)=\frac13x^3+x-1$. This function is invertible (you do not need to prove this).
\begin{enumerate}
  \item Find $f^{-1}(-1)$.
        \begin{proof}[Solution]
          We must solve $-1=\frac13x^3+x-1$, or $0=x(\frac13x^2+1)$.

          Clearly, $x=0$, so $f^{-1}(-1)=0$.
        \end{proof}
  \item Find $(f^{-1})'(-1)$.
        \begin{proof}[Solution]
          Recall the rule for derivatives of inverses: $(f^{-1}(f(a)))' = \frac{1}{f'(a)}$.
          We know from above that $a=0$, so we have $\frac{1}{f'(0)}=\frac{1}{(0)^2+1}=1$.
        \end{proof}
  \item Find $L^{f^{-1}}_{-1}$.
        \begin{proof}[Solution]
          Apply the formula:
          \begin{align*}
            L^{f^{-1}}_{-1} & = f^{-1}(-1)+(f^{-1})'(-1)(x+1) \\
                            & = 0 + 1(x+1)                    \\
                            & = x+1 \qedhere
          \end{align*}
        \end{proof}
\end{enumerate}

\end{document}