% chktex-file 29
\usepackage{physics}
\usepackage{amsfonts,amsmath,amssymb,amsthm}
\usepackage{enumerate}
\usepackage{titlesec}
\usepackage{fancyhdr}
\usepackage{multicol}

\headheight 13.6pt
\setlength{\headsep}{10pt}
\textwidth 15cm 
\textheight 24.3cm
\evensidemargin 6mm
\oddsidemargin 6mm
\topmargin -1.1cm
\setlength{\parskip}{1.5ex}

\author{James Ah Yong}

\pagestyle{fancy}
\fancyhf{}
\fancyfoot[c]{\thepage}
\makeatletter
\lhead{\@title}
\rhead{\@author}

\fancypagestyle{firstpage}{
    \fancyhf{}
    \rhead{\@author}
    \fancyfoot[c]{\thepage}
}

% Sets
\newcommand{\N}{\mathbb{N}}
\newcommand{\Z}{\mathbb{Z}}
\newcommand{\Q}{\mathbb{Q}}
\newcommand{\R}{\mathbb{R}}
\newcommand{\C}{\mathbb{C}}

% Functions
\DeclareMathOperator{\sgn}{sgn}
\DeclareMathOperator{\im}{im}

% Operators
\newcommand{\Rarr}{\Rightarrow}
\newcommand{\Larr}{\Leftarrow}

% Typesetting
\usepackage{array}   % for \newcolumntype macro
\newcolumntype{C}{>{$}c<{$}} % math version of "C" column type
\newcommand{\dlim}{\displaystyle\lim} % totally not \dfrac ripoff
\newcommand{\dilim}[1]{\dlim_{#1\to\infty}} % infinite limits
\newcommand{\ilim}[1]{\lim_{#1\to\infty}}
\usepackage{cancel}

% Auto-number questions
\newcommand{\QType}{Q}
\renewcommand{\theparagraph}{\textbf{\QType\ifnum\value{paragraph}<10 0\fi\arabic{paragraph}.}}
\setcounter{secnumdepth}{4}
\newcommand{\question}{\par\refstepcounter{paragraph}\theparagraph\space}

% Question sections
\titleformat{\section}{\normalsize\bfseries}{\thesection}{1em}{}
\newcommand{\qsection}[2]{%
    \renewcommand{\QType}{#2}
    \section*{#1}
    \refstepcounter{section}
}

\title{MATH 137 Fall 2020: Practice Midterm 1}

\begin{document}
\thispagestyle{firstpage}

\textbf{\@title}

\qsection{Multiple Choice}{MC}

\question For the sequence $\{a_n\}$ where $a_1=1$, $a_n=\sqrt{6+a_{n-1}}$ for $n \geq 2$.
The value of $\lim_{n\to\infty} a_n$ is
\begin{choices}
  \item $-3$
  \item $-2$
  \item $2$
  \item \fbox{None of the above.}
  \emph{$a_2=\sqrt{7}>2$ and $\{a_n\}$ is increasing}
\end{choices}

\question $\dlim{x}{1} \frac{\sin(x-1)}{x^2-1}$
\begin{choices}
  \item $=1$.
  \item \fbox{$=\frac12$.}
  \emph{Translate to $\dlim{y}{0}\frac{\sin y}{y^2+2y}=\dlim{y}{0}\frac{\sin y}{y}\cdot\frac{1}{y+2}$}
  \item Does not exist.
  \item None of the above.
\end{choices}

\question If $f$ is not continuous at $x=2$ then it must be the case that
\begin{choices}
  \item $f(2) \geq 0$.
  \item $f(2)$ is undefined.
  \item $f(2)$ is defined.
  \item \fbox{None of the above.}
  \emph{For examples, $f(x)=\dfrac{1}{x-2}$ and $f(x)=\begin{cases}
        0 & x \neq 2 \\ -1 & x = 2
      \end{cases}$}
\end{choices}

\question The sequence defined by $a_n = \dfrac{n^2+1}{n+3}$
\begin{choices}
  \item converges.
  \item is non-increasing.
  \item \fbox{is bounded below.}
  \emph{Notice that $a_n\to\infty$}
  \item None of the above.
\end{choices}

\question If $f(x)=7$ for all $x \in \R$ then $f'(x)$
\begin{choices}
  \item \fbox{exists for all $x \in \R$.} \emph{And is equal to 0}
  \item is not continuous for all $x \in \R$.
  \item $= 1$.
  \item None of the above.
\end{choices}


\qsection{True/False}{TF}
\setcounter{paragraph}{5}

\question Three functions, $f$, $g$ and $h$, are defined on an open interval $I$ containing $x=a$.
If for each $x \in I$, $g(x)< f(x)< h(x)$ and $\dlim{x}{a^+} g(x) = L = \dlim{x}{a^+} h(x)$, then $\dlim{x}{a} f(x) = L$.

False. If $\dlim{x}{a^-} g(x) \neq \dlim{x}{a^-} h(x)$, then $\dlim{x}{a^-} f(x)$ can be any value between (or undefined).

\question The Fundamental Trigonometric Limit tells us that if $\theta$ is small, then $\cos\theta \approx \theta$.

False. This is true for $\sin\theta$.

\question If $f$ is continuous on $\R$ and $f(0) > 0$ then there exists $\delta > 0$ so that $f(x) > 0$ for all $x\in(0,\delta)$.

True. This follows from the \epsdel{} definition and that $\dlim{x}{0}f(x)=f(0)>0$.

\question $\dilim{x} \frac{x^p}{e^x}=0$ for all $p \in \R$.

True. For positive $p$, see course notes.
For negative $p$, we have $\frac{1}{x^q e^x}$ for positive $q$, which converges to 0.
For zero $p$, $\frac{1}{e^x}$ converges to 0.

\question If $f(a)$ exists, then $f'(a)$ exists too.

False. Consider for example $f(x)=\floor{x}$, which is defined but is not continuous at $x=2$.


\qsection{Short Answer}{SA}

\question For the function \begin{equation*}
  f(x) = \begin{cases}
    1+\sin x & x < 0             \\
    \cos x   & 0 \leq x \leq \pi \\
    \sin x   & \pi < x
  \end{cases}
\end{equation*} determine
\begin{enumerate}[(a)]
  \item $\dlim{x}{0}f(x)$, or write DNE if it does not exist.
        \begin{proof}[Solution]
          From below, $\lim_{x\to0^-}(1+\sin x)=1+\sin0=1$.
          From above, $\lim_{x\to0^+}\cos x=\cos 0=1$.
          Since the one-sided limits agree, the limit exists and is \fbox{1}.
        \end{proof}
  \item $\dlim{x}{\pi}f(x)$, or write DNE if it does not exist.
        \begin{proof}[Solution]
          From below, $\lim_{x\to\pi^-}\cos x=\cos \pi=1$.
          From above, $\lim_{x\to\pi^+}\sin x=\sin \pi=0$.
          Since the one-sided limits do agree, the limit \fbox{DNE}.
        \end{proof}
\end{enumerate}

\question Write all solutions to $\abs{x-1} = \abs{2x}$
\begin{proof}[Solution]
  For $x > 1$: $x-1=2x \implies x=-1$. \\
  For $0 < x < 1$: $-(x-1)=2x \implies x=\frac13$. \\
  For $x < 0$: $-(x-1)=-(2x) \implies x=-1$.

  Therefore, $\boxed{x\in\left\{-1,\frac13\right\}}$.
\end{proof}

\question Give an example of a function such that the Extreme Value Theorem does not apply to it on the interval $[0,5]$.
\begin{proof}[Solution]
  Let $\boxed{f(x) = \dfrac{1}{x-2}}$.

  There is a vertical asymptote at $x=2$, which breaks the Extreme Value Theorem.
\end{proof}

\question State the formal \epsdel{} definition of what it means for $\dlim{x}{a}f(x)=L$.
\begin{proof}[Solution]
  For all $\epsilon > 0$, there exists a $\delta > 0$, such that for every $x$ in the domain of $f$, $0 < |x-a| < \delta$ implies $|f(x)-L| < \epsilon$. That is,
  \[\boxed{\forall\epsilon>0,\exists\delta>0,\forall x,0<|x-a|<\delta\implies|f(x)-L|<\epsilon} \qedhere \]
\end{proof}


\qsection{Long Answer}{LA}

\question Use the \epsdel[N] definition of the limit of a sequence to show $\dilim{n} \frac{n+1}{3n+2}=\frac13$.
\begin{proof}
  Let $a_n=\frac{n+1}{3n+2}$ and $\epsilon > 0$.
  We must find $N$ so $n \geq N$ implies $\abs{a_n-\frac13}<\epsilon$.

  Because $a_n-\frac13=\frac{3n+3-3n-2}{9n+6}=\frac{1}{9n+6}$, it suffices to show $\abs{\frac1{9n+6}}<\epsilon$.
  Since $9n+6$ is positive for all positive $n$, we may drop the absolute value bars.

  Let $N = \frac{1}{9\epsilon}$. Then,
  \begin{alignat*}{3}
    n \geq N & \implies n \geq \frac{1}{9\epsilon} \\
             & \implies 9n \geq \frac{1}{\epsilon} \\
             & \implies 9n+6 > \frac{1}{\epsilon}  \\
             & \implies \frac{1}{9n+6} < \epsilon
  \end{alignat*}
  exactly as desired.

  Therefore, by the \epsdel[N] definition of a limit of a sequence, $\dilim{n}a_n=\frac13$.
\end{proof}


\question Compute the following sequence limits, or show that they do not exist.
\begin{enumerate}[(a)]
  \item $\dilim{n} \frac{\cos n}{n}$
        \begin{proof}
          We propose that the limit is 0 and prove it.
          Recall that $-1 \leq \cos n \leq 1$ for all $n$.
          Then, for positive $n$, $-\frac1n < \frac{\cos n}{n} < \frac1n$.

          Trivially, $-\frac1n \to 0$ and $\frac1n \to 0$.
          The limits agree and $\frac{\cos n}{n}$ is bounded by them above and below.

          Therefore, by the squeeze theorem, $\frac{\cos n}{n}$ also converges to 0.
        \end{proof}
  \item $\dilim{n} \frac{2n^2-n-1}{5n^2+n-3}$
        \begin{proof}
          Recall that for any rational function $\frac{f(x)}{g(x)}$, if $\deg f=\deg g$, the limit at infinity is the ratio of the leading coefficients.

          Therefore, the limit is $\frac{2}{5}$.
        \end{proof}
\end{enumerate}


\question Consider the recursive sequence $a_1=5$ and $a_{n+1}=\dfrac{a_n+1}{3}$ for $n \geq 1$.
\begin{enumerate}[(a)]
  \item Prove that the sequence is decreasing and is bounded below by 0.
        \begin{proof}
          We prove by induction of the sentence $0<a_{n+1}<a_n$ on $n$.

          For the base case, notice that $a_1=5$ and $a_2=\frac{5+1}{3}=2$.
          We have $0 < a_2 < a_1$.

          Now, suppose that $0<a_{k+1}<a_k$ for some $k$. Then,
          \begin{alignat*}{3}
            1           & < a_{k+1}+1           & < a_k+1           \\
            \frac{1}{3} & < \frac{a_{k+1}+1}{3} & < \frac{a_k+1}{3} \\
            0           & < a_{k+2}             & < a_{k+1}
          \end{alignat*}
          as desired.
          Therefore, by induction, $0<a_{n+1}<a_n$ for all $n$, that is,
          $a_n$ is decreasing and bounded below by 0.
        \end{proof}
  \item Prove that the sequence converges and find its limit.
        \begin{proof}
          Because $a_n$ is non-increasing and bounded below, the limit exists and is equal to $L$ by the monotone convergence theorem.

          Recall that if $a_n \to L$, then $a_{n+1} \to L$. Then,
          \begin{align*}
            \ilim{n} a_n & = \ilim{n} a_{n+1}         \\
            L            & = \ilim{n} \frac{a_n+1}{3} \\
            L            & = \frac{\ilim{n}a_n+1}{3}  \\
            L            & = \frac{L+1}{3}            \\
            L            & = \frac{1}{2} \qedhere
          \end{align*}
        \end{proof}
\end{enumerate}


\question Use the \epsdel{} definition of the limit of a function to show that
$\dlim{x}{2}\left(x^2+2x-3\right)=5$.

\begin{proof}
  Let $\epsilon > 0$.
  We must find $\delta$ so that $0<|x-2|<\delta$ implies $\abs{(x^2+2x-3)-5}=|x^2+2x-8|<\epsilon$.

  We can limit $\delta$ by having it equal $\min(\{\frac\epsilon7,1\})$.
  Then, when $|x-2| < \delta$ we have $|x+4| < 7$.

  We now have $|x-2| < \delta \leq \frac\epsilon7$ and $|x+4| < 7$. Multiplying,
  \begin{align*}
    |x-2|\cdot|x+4| & < \frac\epsilon7\cdot7 \\
    |(x-2)(x+4)|    & < \epsilon             \\
    |x^2+2x-8|      & < \epsilon
  \end{align*}

  Therefore, by the \epsdel{} definition of the limit of a function, $\dlim{x}{2}\left(x^2+2x-3\right)=5$.
\end{proof}


\question Compute the following function limits, if possible.
If the limit does not exist, prove it.
\begin{enumerate}[(a)]
  \item $\dlim{x}{0} \frac{5x^2-3x}{2x^3-x^2}$
        \begin{proof}[Solution]
          Recall the continuity of polynomials and quotients.
          It follows that all rational functions $\frac{p(x)}{q(x)}$ are continuous at any $x=a$ so long as $q(a)\neq 0$.

          Let $f(x) = \frac{5x^2-3}{2x^3-x^2}$.
          At $x=0$, we have $2x^3-x^2=0$.
          Therefore, $f$ is not continuous at $x=0$ and we analyze the one-sided limits to determine the type of discontinuity.

          First, notice that we may factor as $f(x)=\frac{1}{x^2} \cdot \frac{5x-3}{2x-1}$.

          Consider the sequence $a_n=\frac1n$, a sequence which converges to 0. We have
          \[
            \ilim{n} f(a_n)
            = \ilim{n} n^2\left(\frac{\frac{5}{n}-3}{\frac{2}{n}-1}\right)
            = \infty
          \]
          By the sequential characterization of limits,
          if the limit of $f(a_n)$ does not exist, so too does the limit $f$ at $x=0$.

          Therefore, $\dlim{x}{0} \frac{5x^2-3x}{2x^3-x^2}$ does not exist.
        \end{proof}
  \item $\dlim{x}{0} \frac{3x^2-1}{x^2-x+1}$
        \begin{proof}
          Recall again the continuity of rational functions.

          Here, the denominator is $(0)^2-(0)+1 = 1 \neq 0$,
          therefore we may simply evaluate the function at $x=0$.
          This is $\frac{0-1}{0-0+1}=-1$.
        \end{proof}
  \item $\dilim{x} \frac{\ln x^2 - \ln x}{x^2-x}$
        \begin{proof}[Solution]
          Simplify:
          \begin{align*}
            \ilim{x} \frac{\ln x^2 - \ln x}{x^2-x}
             & = \ilim{x} \frac{2\ln x - \ln x}{x(x-1)} \by{logarithm laws}            \\
             & = \ilim{x} \frac{\ln x}{x}\cdot\ilim{x} \frac{1}{x-1} \by{limit laws}   \\
             & = 0\cdot0                                             \by{FLL \qedhere}
          \end{align*}
        \end{proof}
\end{enumerate}

\question Given the function \begin{equation*}
  f(x) = \begin{cases}
    k^2+5 & x \leq -2 \\
    x^2+k & x > -2
  \end{cases}
\end{equation*}
If $f(x)$ is continuous at $x=-2$, determine all value(s) for $k$. % chktex 36
\begin{proof}[Solution]
  Recall that for $f$ to be continuous at $x=-2$, $\dlim{x}{-2}f(x)=f(-2)$.

  This limit exists if and only if the one-sided limits,
  \[ \lim_{x\to-2^-}f(x) = k^2+5 \qquad \text{and} \qquad \lim_{x\to-2^+}f(x) = 4+k \]
  agree as $x$ approaches $-2$ from above and below. That is,
  \begin{align*}
    k^2 + 5 & = 4 + k       \\
    0       & = k^2 - k + 1
  \end{align*}
  which is a quadratic in $k$.
  However, the discriminant, $(-1)^2-4(1)(1)=-3$, is negative.
  This means there are no real solutions for $k$.
\end{proof}

\question Determine, with justification, all vertical asymptotes of the function
\[ f(x) = \frac{x+3}{\abs{x^2-2x-15}}. \]
\begin{proof}[Solution]
  Recall the continuity of polynomials and quotients.
  It follows that all rational functions $\frac{p(x)}{q(x)}$ are continuous at any $x=a$ so long as $q(a)\neq 0$.

  Factoring, $\abs{x^2-2x-15}=\abs{(x-5)(x+3)}=\abs{x-5}\cdot\abs{x+3}$.
  This is zero at $x=-3,5$.
  We consider these two options:
  \begin{itemize}
    \item Consider $x=-3$. The limit from below is:
          \begin{equation*}
            \dlim{x}{-3^-} \frac{x+3}{\abs{x-5}\cdot\abs{x+3}}
            = \dlim{x}{-3^-} \frac{x+3}{-(x-5)\cdot-(x+3)}
            = \dlim{x}{-3^-} \frac{1}{x-5}
            = \frac{1}{8}
          \end{equation*}
          and the limit from above is:
          \begin{equation*}
            \dlim{x}{-3^+} \frac{x+3}{\abs{x-5}\cdot\abs{x+3}}
            = \dlim{x}{-3^+} \frac{x+3}{-(x-5)(x+3)}
            = \dlim{x}{-3^+} -\frac{1}{x-5}
            = -\frac{1}{8}
          \end{equation*}

          These limits do not agree but they exist.
          Therefore, there is a jump discontinuity.
    \item Consider $x=5$. The limit from below is:
          \begin{equation*}
            \dlim{x}{5^-} \frac{x+3}{\abs{x-5}\cdot\abs{x+3}}
            = \dlim{x}{5^-} \frac{x+3}{-(x-5)(x+3)}
            = \dlim{x}{5^-} -\frac{1}{x-5}
            = \dlim{x_0}{0^-} -\frac{1}{x_0}
            = \infty
          \end{equation*}
          This is enough to say that there exists a vertical asymptote at $x=5$.
  \end{itemize}

  Therefore, discontinuities exist only at $x=-3,5$, where $x=5$ is a vertical asymptote.
\end{proof}

\question Suppose $A,B \in \R$, $A>0$, $B>0$ and $f:\R\to\R$ is a function such that
if $|x-y| < A$ then $|f(x)-f(y)| < B|x-y|$ for all $x,y\in\R$.

Prove that $f$ is continuous on $\R$.
\begin{proof}
  Let $A$ and $B$ be positive reals, and $f$ be a function on the reals such that
  $|x-y| < A$ implies $|f(x)-f(y)| < B|x-y|$ for any $x$ and $y$.

  We must show that $\dlim{n}{a}f(n)=f(a)$ for all $a$.
  That is, for any tolerance $\epsilon > 0$,
  we may find a $\delta$ such that $0 < |n-a| < \delta$ implies $|f(n)-f(a)| < \epsilon$.

  Let $\epsilon > 0$ and $a$ be a real. Select $\delta = \min(\{A,\frac{\epsilon}{B}\})$.

  Suppose that $0 < |n-a| < \delta$.
  That is, $|n-a| < A$ and $|n-a| < \frac{\epsilon}{B}$.

  It also follows that $|f(n)-f(a)| < B|n-a|$.
  But we supposed that $|n-a| < \frac{\epsilon}{B}$, so
  \[ \abs{f(n)-f(a)} < B\frac{\epsilon}{B} = \epsilon \]
  This is exactly what was needed to show that $f$ is continuous for any $a$.
\end{proof}

\question Prove that $x^2+x\cos x = 1$ has at least two real solutions.
\begin{proof}
  Let $f(x) = x^2+x\cos x$.
  Recall that polynomials and cosine are both continuous on $\R$.
  Therefore, their sum/product, $f$, is also continuous.

  At $x=0$, we have $f(x)=0+0=0$.

  At $x=-\pi$, we have $f(x)=\pi^2-\pi(-1)=\pi^2+\pi > 1$.
  We then have that $f(-\pi) < 1 < f(0)$.
  So, by the intermediate value theorem, there exists some $a\in(-\pi,0)$ where $f(x)=1$.

  Likewise at $x=\pi$, we have $f(x)=\pi^2+\pi(-1)=\pi^2-\pi > 1$.
  We then have that $f(0) < 1 < f(\pi)$.
  So, by the intermediate value theorem, there exists some $b\in(0,\pi)$ where $f(x)=1$.

  Therefore, there must exist at least two real solutions, $a$ and $b$, to $x^2+x\cos x=1$.
\end{proof}

\end{document}