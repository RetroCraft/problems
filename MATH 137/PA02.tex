\documentclass[11pt]{article}
\usepackage{physics}
\usepackage{amsfonts,amsmath,amssymb,amsthm}
\usepackage{enumerate}
\usepackage{titlesec}
\usepackage{fancyhdr}
\usepackage{multicol}

\headheight 13.6pt
\setlength{\headsep}{10pt}
\textwidth 15cm 
\textheight 24.3cm
\evensidemargin 6mm
\oddsidemargin 6mm
\topmargin -1.1cm
\setlength{\parskip}{1.5ex}

\author{James Ah Yong}

\pagestyle{fancy}
\fancyhf{}
\fancyfoot[c]{\thepage}
\makeatletter
\lhead{\@title}
\rhead{\@author}

\fancypagestyle{firstpage}{
    \fancyhf{}
    \rhead{\@author}
    \fancyfoot[c]{\thepage}
}

% Sets
\newcommand{\N}{\mathbb{N}}
\newcommand{\Z}{\mathbb{Z}}
\newcommand{\Q}{\mathbb{Q}}
\newcommand{\R}{\mathbb{R}}
\newcommand{\C}{\mathbb{C}}

% Functions
\DeclareMathOperator{\sgn}{sgn}
\DeclareMathOperator{\im}{im}

% Operators
\newcommand{\Rarr}{\Rightarrow}
\newcommand{\Larr}{\Leftarrow}

% Typesetting
\usepackage{array}   % for \newcolumntype macro
\newcolumntype{C}{>{$}c<{$}} % math version of "C" column type
\newcommand{\dlim}{\displaystyle\lim} % totally not \dfrac ripoff
\newcommand{\dilim}[1]{\dlim_{#1\to\infty}} % infinite limits
\newcommand{\ilim}[1]{\lim_{#1\to\infty}}
\usepackage{cancel}

% Auto-number questions
\newcommand{\QType}{Q}
\renewcommand{\theparagraph}{\textbf{\QType\ifnum\value{paragraph}<10 0\fi\arabic{paragraph}.}}
\setcounter{secnumdepth}{4}
\newcommand{\question}{\par\refstepcounter{paragraph}\theparagraph\space}

% Question sections
\titleformat{\section}{\normalsize\bfseries}{\thesection}{1em}{}
\newcommand{\qsection}[2]{%
    \renewcommand{\QType}{#2}
    \section*{#1}
    \refstepcounter{section}
}


\title{MATH 137 Fall 2020: Practice Assignment 2}

\begin{document}
\parindent=0pt
\thispagestyle{firstpage}

\textbf{\@title}

\textbf{Q01.}
Use the formal definition of limits to prove each statement below:

\begin{enumerate}[(a)]

  \item $\dilim{n}\frac{2n}{n+1}=2$
        \begin{proof}
          Let $\epsilon>0$.
          We have to find $N$ such that $n\geq N$ implies $\frac{2n}{n+1}\in(2-\epsilon,2+\epsilon)$, or $\abs{\frac{2n}{n+1}-2}<\epsilon$.
          Simplifying:
          \[ \abs{\frac{2n}{n+1}-2} = \abs{\frac{2n-2(n+1)}{n+1}} = \abs{\frac{2}{n+1}} = \frac{2}{n+1} < \epsilon \]

          Now, take $N=\frac2\epsilon$.
          Then, $n\geq N\implies n\geq\frac2\epsilon\implies n+1>\frac2\epsilon\implies\frac2{n+1}<\epsilon$
        \end{proof}

  \item $\dilim{n}\frac{6n-3n^2-2}{(n-1)^2}=-3$
        \begin{proof}
          Let $\epsilon>0$. We must find $N$ such that $n\geq N\implies \abs{\frac{6n-3n^2-2}{(n-1)^2}-(-3)}<\epsilon$. Again, simplifying:
          \[ \abs{\frac{-3n^2+6n-2}{(n-1)^2}+3} = \abs{\frac{3n^2-6n+2}{(n-1)^2}-3} = \abs{\frac{-1}{(n-1)^2}}= \frac{1}{(n-1)^2} < \epsilon \]

          Let $N=\sqrt{\frac1\epsilon}+2$. Then, $n\geq N$ implies that $n\geq\sqrt{\frac1\epsilon}+2 \implies (n-1)^2>\frac1\epsilon\implies\frac{1}{(n-1)^2}<\epsilon$
        \end{proof}

  \item $\dilim{n}1-2^n=-\infty$
        \begin{proof}
          Let $M<0$. We have to find $N$ such that $n\geq N$ implies $1-2^n<M$. Notice that since $2^n>0$ for all $n$, this can be rewritten as $2^n>1-M$. Let $N=\log_2(1-M)+1$. This is valid since $M$ is defined to be negative, so $1-M$ is always positive. Now, $n\geq N\implies n>\log_2(1-M)\implies 2^n>1-M\implies1-2^n<M$
        \end{proof}

\end{enumerate}



\textbf{Q02.}
Determine if the following statements are true or false. If true, argue your case mathematically, if false, provide a counterexample.

\begin{enumerate}[(a)]

  \item If $\dilim{n}a_n=\dilim{n}b_n=\infty$, then $\dilim{n}(a_n+b_n)=\infty$
        \begin{proof}
          Let $M>0$. If $a_n\to\infty$, then there exists an $N_1$ such that $n\geq N_1$ implies $a_n>\frac{M}{2}$. Likewise, if $b_n\to\infty$, then there exists an $N_2$ such that $n\geq N_2$ implies $b_n>\frac{M}{2}$.

          Let $N=\max \{N_1,N_2\}$. If $n\geq N$, then $a_n+b_n>\frac{M}{2}+\frac{M}{2}=M$.
          Therefore, $a_n+b_n$ diverges to infinity.
        \end{proof}

  \item If $\dilim{n}a_n=\dilim{n}b_n=\infty$, then $\dilim{n}(a_n-b_n)=0$
        \begin{proof}
          Let $a_n=n$ and $b_n=2n$. Both diverge to positive infinity. However, $a_n-b_n=-n$, which diverges to negative infinity. Therefore, by counterexample, the statement is false.
        \end{proof}

  \item If $a_n \leq b_n \leq c_n$ for all $n$, $\dilim{n}a_n=L$ and $\dilim{n}c_n=M$, then $\dilim{n}b_n=K$ with $L \leq K \leq M$
        \begin{proof}
          Consider the similar proof presented below as Q05.
          Since $a_n \leq b_n \leq c_n \equiv a_n \leq b_n \land b_n \leq c_n$, we can apply that proof twice to show that $L \leq K$ and $K \leq M$, which is just $L \leq K \leq M$.
        \end{proof}

\end{enumerate}



\textbf{Q03.}
Consider the sequence
\[
  a_n=\begin{cases}
    1           & \textrm{when $n$ is a perfect square} \\
    \frac{1}{n} & \textrm{otherwise}
  \end{cases}
\]
Use the definition of convergence to show this sequence does not have a limit of 0. Hint: consider building a contradiction.
\begin{proof}
  Note that $a_n>0$ for all $n$.
  If the statement is false and $a_n\to 0$, then, for any $\epsilon>0$, we can find a $N>0$ such that $n\geq N$ implies $a_n<\epsilon$.
  Alternatively stated, there is a tail of $\{a_n\}$ consisting only of numbers less than $\epsilon$.
  Select $\epsilon<1$. Since $a_n=1$ if $\sqrt{n}\in\N$, the corresponding tail given by $N$ must contain no square numbers.
  However, $N^2\in\N$ and $N^2>N$. Therefore, no such $N$ can exist, and the limit of $a_n$ cannot be zero.
\end{proof}



\textbf{Q04.}
Show that if a sequence $\{a_n\}$ converges to $L$ then there are infinitely many terms of the sequence that can be made arbitrarily close to one another.
Specifically, show that eventually $|a_n-a_m|$ can be made arbitrarily small (i.e.\ for $n,m$ past a certain point)
Note that $m$ and $n$ are not necessarily consecutive integers.
Such sequences are called \emph{Cauchy Sequences}.
\begin{proof}
  For any $a_i$, by the definition of the limit, we can write it as $L+\epsilon_i$ for some $|\epsilon_i|>0$.
  Now, rewrite $|a_n-a_m|$ as $|L+\epsilon_n-L-\epsilon_m|=|\epsilon_n-\epsilon_m|$.
  Since it is guaranteed by the definition of the limit that $\epsilon$ can be made arbitrarily small, the quantity $|\epsilon_n-\epsilon_m|$ can also be made arbitrarily small.
\end{proof}



\textbf{Q05.}
In this question we will prove the following:
\begin{center}
  If $\dilim{n}a_n=L$, $\dilim{n}b_n=M$ and $a_n \leq b_n$ for all $n$, then it must be the case that $L \leq M$
\end{center}

\begin{enumerate}[(a)]
  \item Use the definition of limits to show that for all $\epsilon$, eventually
        \[ L-\epsilon < a_n \leq b_n < M+\epsilon \]
        (the word ``eventually'' is meant to take the place of the statement ``for all $n$ greater than some $N$'')
        \begin{proof}
          Let $\epsilon>0$. Then, by the definition of the limit, there is an $N_1$ such that for all $n\geq N_1$, $|a_n-L|<\epsilon \implies a_n > L-\epsilon$.
          Likewise, there is an $N_2$ such that for all $n\geq N_2$, $|b_n-M|<\epsilon \implies b_n < M+\epsilon$.
          Given that $a_n\leq b_n$ for all $n$, we can combine these inequalities by taking $n\geq\max \{N_1,N_2\} \implies L-\epsilon < a_n \leq b_n < M+\epsilon$
        \end{proof}

  \item Since $\epsilon$ is arbitrary, the inequality above might help you ``feel'' that $L \leq M$.
        That is, we can make $\epsilon$ so small that ``basically $L \leq M$''.
        One way of showing this mathematically is to assume that $L > M$ and come up with a contradiction.
        That is, let $L=M+d$ for some positive number $d$.
        Use this to arrive at a contradiction and thus deduce $L \leq M$.
        \begin{proof}
          If $L>M$, we can let $L=M+d$ for some $d>0$.
          Repeat the conclusion from part (a), $M+d-\epsilon < M+\epsilon$,
          and add $\epsilon$ to both sides: $M+d < M+2\epsilon$.
          Since $d$ is defined, we can let $\epsilon<\frac{d}{2}$.
          Because $\frac{d}{2}>0$, this is a valid choice of $\epsilon$.
          However, the inequality now reads $M+d < M+d$, which is clearly false.
          We can conclude that the inequality is false, so its negation $L \leq M$, is true.
        \end{proof}
\end{enumerate}



\textbf{Q06.}
Prove (using the definition) that if $a_n>0$ for all $n$ and $\dilim{n}a_n=L$, then  $\dilim{n}\sqrt{a_n}=\sqrt{L}$.
  [Hint: Consider the cases $L=0$ and $L\neq0$ separately.]
\begin{proof}
  Consider the case where $L=0$.
  Let $\epsilon > 0$, which implies $\epsilon^2>0$.
  Given the known limit $\ilim{n}a_n=0$, we can find an $N$ such that $n \geq N$ implies $|a_n-0|=a_n<\epsilon^2$.
  Taking the square roots of both sides, $\sqrt{a_n}<\epsilon$ as required.

  Consider when $L\neq0$.
  Because $\sqrt{L}$ exists, $L > 0$.
  Let $\epsilon > 0$.
  Given the known limit, we can find an $N$ such that $n \geq N$ implies $|a_n-L| < \epsilon$.
  Notice that we can use $a_n-L$ to create an expression for $\sqrt{a_n}-\sqrt{L}$:
  \begin{align*}
    |a_n-L|                   & =\abs{(\sqrt{a_n}-\sqrt{L})(\sqrt{a_n}+\sqrt{L})}                                       \\
                              & =\abs{\sqrt{a_n}-\sqrt{L}}\abs{\sqrt{a_n}+\sqrt{L}}                                     \\
    \abs{\sqrt{a_n}-\sqrt{L}} & = \frac{|a_n-L|}{\sqrt{a_n}+\sqrt{L}} < \frac{\epsilon}{\sqrt{a_n}+\sqrt{L}} < \epsilon
  \end{align*}
  (since the denominator is a sum of two positive numbers) as required.
\end{proof}

\end{document}