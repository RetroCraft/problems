\documentclass[class=math239,notes]{agony}
\titleformat{\section}{\normalfont\large\bfseries}{Lecture~\thesection}{1em}{}
\settowidth{\cftsecnumwidth}{Lecture~0000~}
\renewcommand\cftsecpresnum{Lecture~}

\title{MATH 239 Fall 2022: Lecture Notes}
\begin{document}
\thispagestyle{firstpage}
\tableofcontents

\pagebreak

\chapter*{Course Information}

Lecture notes taken, unless otherwise specified,
in section 002 of the Fall 2022 offering of MATH 239,
taught by Luke Postle.

\part{Enumeration}

\chapter{Introduction}

\section{(09/07; Section 004)}

By convention, define the natural numbers $\N = \{0,1,2,\dotsc\}$.
If we want the non-negative integers, we can write $\N_{\geq1}$.

Recall from set theory, for sets $A$ and $B$,
the union $A \cup B = \{c : c \in A \lor c \in B\}$
and intersection $A \cap B = \{c : c \in A \land c \in B\}$.
We call a union $A \cup B$ disjoint if $A \cap B = \varnothing$.
When considering the sizes of finite sets,
the disjoint union corresponds to addition:
$\abs{A \cup B} = \abs{A} + \abs{B}$.

Recall also the Cartesian product $A \times B = \{(a,b) : a \in A, b \in B\}$.
Again for finite sets, $\abs{A \times B} = \abs{A} \cdot \abs{B}$.

\begin{defn}[bijection]
  A function $f : A \to B$ such that $f$ is injective (i.e., $f(a) = f(a') \implies a = a'$)
  and surjective (i.e., $\forall b, \exists a, f(a) = b$).
  When $f$ exists, we write $A \bijects B$.
\end{defn}

\begin{theorem}[Bijective Proofs]
  If $A \bijects B$, then $\abs{A} = \abs{B}$.
\end{theorem}
\begin{prf}
  Let $f : A \to B$ be a bijection.
  For each element of $B$, it is the image of
  at least one element of $A$ under $f$ (by surjectivity)
  and at most one element (by injectivity).
  Therefore, $A$ contains the same number of elements as $B$.
\end{prf}

\section{(09/09)}

By convention, let $[n] := \{1,\dotsc,n\}$.

\begin{theorem}
  Let $A$ be the set of subsets of $[n]$.
  Let $B$ be the set of binary strings of length $n$.
  There exists a bijection from $A$ to $B$.
\end{theorem}
\begin{prf}
  Let $f : A \to B$ be defined as $f(S) = a_1\dots a_n$
  where $a_i = 1$ if $i \in S$ and $0$ otherwise.
  Then, $f^{-1}(a_1\dots a_n) = \{i \in [n] : a_i = 1\}$.
  Since $f^{-1} \circ f = \id$ and $f \circ f^{-1} = \id$, we have a bijection.
\end{prf}

\begin{theorem}
  Let $A$ be the set of subsets of $[n]$ of size exactly $k$.
  Let $B$ be the set of binary strings of length $n$ with exactly $k$ ones.
  There exists a bijection from $A$ to $B$.
\end{theorem}
\begin{prf}
  Restrict the domain of $f$ from above.
\end{prf}

\begin{defn}[permutation]
  A list of $[n]$ for some positive integer $n$.
  That is, a bijection from $[n]$ to $[n]$.
  Notate the set of permutations of $[n]$ by $P_n$.
\end{defn}

\begin{theorem}[1.3]
  The number of subsets of $[n]$ is $2^n$.
\end{theorem}
\begin{prf}
  $\mathcal{P}([n]) \bijects \{0,1\}^n$ by the above theorem.
  But we know $\abs{\{0,1\}^n} = \abs{\{0,1\}}^n = 2^n$.
  Then, $\abs{\mathcal{P}([n])} = 2^n$.
\end{prf}

\begin{theorem}[1.5]
  The number of subsets of $[n]$ of size $k$ is $\binom{n}{k}$.
\end{theorem}
\begin{prf}
  Let $S_{n,k}$ be the subsets of $[n]$ of size $k$.

  For each $A \in S_{n,k}$ define $P_A := \{\sigma_1\dots\sigma_n \in P_n : \{\sigma_1,\dotsc,\sigma_n\} = A\}$.
  Then, $P_n$ is the disjoint union of the sets $P_A$.
  Also, $P_A \bijects P_k \times P_{n-k}$ because the first $k$ entries are a list of $A$
  and the last $n-k$ entries are a list of $[n] \setminus A$.

  Therefore, $\abs{P_A} = k! \cdot (n-k)!$ and
  $\abs{S_{n,k}} = \frac{\abs{P_n}}{k!\cdot(n-k)!} = \binom{n}{k}$
  as the sets are of equal size.
\end{prf}

In general, to prove that $A$ has some size, we can either
\begin{enumerate}[(1),nosep]
  \item Prove $A$ is a disjoint union of smaller sets of known size
  \item Prove $A$ is a Cartesian product of smaller sets of known size
  \item Give a bijection $A \bijects B$ to a set of known size
  \item Show a family of sets $A_i$ satisfies a recurrence relation and use induction
\end{enumerate}

\begin{defn}[composition]
  A finite sequence $n=(m_1,\dotsc,m_k)$ of positive integers called parts.
  The size of the composition $\abs{n} = \sum m_i$.
\end{defn}

\begin{theorem}
  For all $n \geq 1$, there are $2^{n-1}$ compositions of size $n$
\end{theorem}
\begin{prf}
  Proceed by induction on $n$.

  When $n = 1$, there is exactly one composition (1). Assume $n \geq 2$.

  Let $A_n$ be compositions of size $n$.
  Also let $B_{1,n} := \{(a_1,\dotsc) \in A_n : a_1 = 1\}$
  and $B_{2,n} := \{(a_1,\dotsc) \in A_n : a_1 \geq 2\}$.
  Notice that since $a_1$ is a positive integer so either $a_1 = 1$ or $a_1 \geq 2$,
  so $A_n$ is the disjoint union of $B_{1,n}$ and $B_{2,n}$.

  Let $f : B_{1,n} \to A_{n-1}$ be given by $f((b_1,\dotsc,b_k)) = (b_2,\dotsc,b_k)$
  and we can find the inverse $f^{-1}((a_1,\dotsc,a_k)) = (1,a_1,\dotsc,a_k)$.

  Let $g : B_{2,n} \to A_{n-1}$ be given by $g((b_1,\dotsc,b_k)) = (b_1-1,b_2\dotsc,b_k)$
  whose inverse we can likewise find $g^{-1}((a_1,\dotsc,a_k)) = (a_1+1,a_2,\dotsc,a_k)$
  assuming $A_{n-1}$ is nonempty, i.e., $n \geq 2$.

  As these are bijections, $\abs{B_{1,n}} = \abs{n-1} = \abs{B_{2,n}}$.
  By induction $\abs{A_{n-1}} = 2^{n-2}$ and hence
  $\abs{A_n} = \abs{B_{1,n}} + \abs{B_{2,n}} = 2\abs{A_{n-1}} = 2^{n-1}$ as desired.
\end{prf}

\section{(09/12)}

\begin{theorem}[Binomial Theorem]
  For all $n \geq 1$, $(1+x)^n = \sum_{k=0}^n \binom{n}{k}x^k$
\end{theorem}
\begin{prf}
  Must choose either 1 or $x$ from each monomial,
  so we have $(1+x)^n = \sum_{S \subseteq [n]} x^{\abs{S}} = \sum_{k=0}^n\binom{n}{k}x^k$
  by grouping choices by the number of $x$ chosen and applying Theorem 1.5.
\end{prf}

\begin{corollary}
  If $x=1$, then $2^n = \sum_{k=0}^n \binom{n}{k}$, i.e., Theorem 1.3.
\end{corollary}
\begin{prf}
  From Theorem 1.2, $\abs{\powerset([n])} = 2^n$.
  But $\abs{\powerset([n])} = \abs{\bigcup S_{n,k}}
    = \sum \abs{S_{n,k}} = \sum \binom{n}{k}$ by Theorem 1.5.
\end{prf}

\begin{corollary}
  If $x=-1$, then $0 = \sum_{k=0}^n \binom{n}{k}(-1)^k$.
\end{corollary}
\begin{prf}
  See Tutorial 1-2.
\end{prf}

\begin{defn}[multiset of $t$ types]
  Sequence $a_1,\dotsc,a_t$ where each $a_i$ is a non-negative integer.
  The $a_i$ are the \emph{parts} and the sum $\sum a_i$ is the \emph{size}.
\end{defn}

Like compositions but permitting zero and restricting $i = 1,\dotsc,t$.
For example, consider the multiset $(3,2,0,1)$ as being like a ``set'' $\{a_1,a_1,a_1,a_2,a_2,a_4\}$.

\begin{theorem}[1.9]
  The number of multisets of $t$ types and size $n$ is $\binom{n+t-1}{t-1}$.
\end{theorem}
\begin{prf}
  Consider the creation of a multiset like dividing up the line of elements
  (e.g. $(3,2,0,1)$ as $***\mid**\mid\;\mid*$).

  Encode this as a binary string where 0 means take an element
  and 1 means switch to the next set (in this case, $000100110$).
  This is a string of length $n+t-1$ with exactly $t-1$ ones.
  That is, a subset of $[n+t-1]$ with size $t-1$, of which there are $\binom{n+t-1}{t-1}$.

  To be formal, write out and prove that the set of multisets $A$ bijects with $B=S_{n+t-1,t-1}$.

  Under $f : A \to B : (a_1,\dotsc,a_t) \mapsto \{a_1+1,\dotsc,a_{t-1}+1\}$
  with inverse $f^{-1} : B \to A : \{b_1,\dotsc,b_{n-1}\} \mapsto (b_1-1,b_2-b_1-1,b_3-b_2-1,\dotsc,n-b_{t-1}-1)$,
  $A \bijects B$ and we have $\abs{A} = \binom{n+t-1}{t-1}$, as desired.
\end{prf}

\begin{theorem}[Negative Binomial Series]
  $\frac{1}{(1-x)^t} = \sum_{t=0}^\infty \binom{n+t-1}{t-1}x^n$
\end{theorem}
\begin{prf}
  Notice that $\frac{1}{(1-x)^t} = \underbrace{(1+x+x^2+\dotsb)\dotsb(1+x+x^2+\dotsb)}_{t\text{ times}}$.
  The coefficient of $x^k$ is the number of multisets
  $(\alpha_1,\dotsc,\alpha_t)$ of $k$ size so apply Theorem 1.9.
\end{prf}

\section{(09/14)}

\begin{example}[Pascal's Identity]
  $\binom{n}{k} = \binom{n-1}{k} + \binom{n-1}{k-1}$
\end{example}
\begin{prf}[informal]
  We are counting $S \subseteq [n]$.
  There are two kinds of subsets: $n \in S$ and $n \not\in S$.
  If $n \in S$, then $S \setminus\{n\} \subseteq [n-1]$ with size $n-1$,
  i.e., there are $\binom{n-1}{k-1}$ of these.
  If $n \not\in S$, then $S \subseteq [n-1]$, i.e., there are $\binom{n-1}{k}$ of these.
\end{prf}
\begin{prf}
  Let $S_{n,k} := \{ S \subseteq [n] : \abs{S} = k\}$ be the set of subsets of $[n]$ with size exactly $k$.
  By Theorem 1.5, $\abs{S_{n,k}} = \binom{n}{k}$.

  Let $A_1 = \{S \in S_{n,k} : n \in S\}$ and $A_2 = \{S \in S_{n,k} : n \not\in S\}$.
  Then, $S_{n,k}$ is the disjoint union of $A_1$ and $A_2$ since either $n \in S$ or $n \not\in S$.
  Thus, $\abs{S_{n,k}} = \abs{A_1} + \abs{A_2}$.

  Claim $A_1 \bijects S_{n-1,k-1}$  under $f : A_1 \to S_{n-1,k-1} : S \mapsto S \setminus \{n\}$
  with inverse $f^{-1} : S_{n-1,k-1} \to A_1 : T \mapsto T \cup \{n\}$.
  Then, $\abs{A_1} = \abs{S_{n-1,k-1}} = \binom{n-1}{k-1}$.

  Claim $A_2 \bijects S_{n-1,k}$ under $g = \id_{A_2}$ with $g^{-1} = \id_{S_{n-1,k}}$.
  Then, $\abs{A_2} = \abs{S_{n-1,k}} = \binom{n-1}{k}$.

  Finally, $\binom{n}{k} = \abs{S_{n-1,k}} = \binom{n-1}{k} + \binom{n-1}{k-1}$.
\end{prf}

\begin{example}
  $\binom{n}{k} = \sum_{i=k}^{n} \binom{i-1}{k-1}$ for all $n \geq k \geq 1$
\end{example}
\begin{prf}[informal]
  Subsets of $[n]$ of size exactly $k$ come in $n-k+1$ types, namely, classify by their maxima.
  Then, if $i$ is largest, we have to chose $k-1$ remaining elements from $[i-1]$.
  This goes down to $k$ being largest by the pigeonhole principle where we have $k-1$ elements from $[k-1]$.
\end{prf}
\begin{prf}
  By Theorem 1.5, $\abs{S_{n,k}} = \binom{n}{k}$.

  Then, since the maximum of a set is unique, we can partition $S_{n,k}$
  into the disjoint union $A_k \cup \dotsb \cup A_n$ where for each $i$,
  $A_i = \{S \in S_{n,k} : \max S = i\}$.
  Thus, $\abs{S_{n,k}} = \sum_{i=k}^n \abs{A_i}$

  Claim $A_i \bijects S_{i-1,k-1}$ under the bijection
  $f : A_i \to S_{i-1,k-1} : S \mapsto S \setminus \{i\}$ with inverse
  $f^{-1} : S_{i-1,k-1} \to A_i : T \mapsto T \cup \{i\}$ from above.
  Then, $\abs{A_i} = \binom{i-1}{k-1}$.

  Finally, $\binom{n}{k} = \abs{S_{n,k}} = \sum_{i=k}^n \binom{i-1}{k-1}$, as desired.
\end{prf}

\chapter{Generating Series}

\section{(09/16)}

\newcommand{\preim}{\operatorname{preim}}
\begin{defn}[weight function]
  $w : A \to \N$ such that $\abs{\preim_w(n)}$ is finite for all $n \in \N$.
  Call $w(a)$ the weight of $a$.
\end{defn}

In general, answer questions of the form "how many elements of $A$ have weight $n$?"

\begin{defn}[generating series]
  $\Phi_A^w(x) = \sum_{a\in A} x^{w(a)}$ for a set $A$ and weight function $w$.
\end{defn}

This is not a polynomial but a formal power series with infinite terms.
We do not ever evaluate $\Phi_A^w(x)$ so convergence is irrelevant.

Basically, take the sequence $(\abs{\preim_w(0)},\abs{\preim_w(1)},\dotsc)$
and make it into coefficients.

Notate $[x^n]f(x)$ for the coefficient of $x^n$ in $f(x)$.

\begin{example}
  For $A = \powerset([n])$, $w = \abs{\cdot}$,
  we have $\Phi_A^w(x) = \sum_{k=0}^n\binom{n}{k}x^k = (1+x)^n$
\end{example}

\begin{prop}
  The number of elements of weight $i$ is the $i$th coefficient in $\Phi_A^w(x)$.
  Equivalently, $\Phi_A^w(x) = \sum_{n\geq0}a_n x^n$ where $a_n = \abs{\preim_w(n)}$.
\end{prop}

\begin{example}
  For set of multisets with $t$ types $A$ and $w$ size of the multiset,
  $\Phi_A^w(x) = \sum_{n \geq 0}\binom{n+t-1}{t-1}x^n = \frac{1}{(1-x)^t}$.
\end{example}
\begin{example}
  For set of binary strings $A$ and $w$ length,
  $\Phi_A^w(x) = \sum_{n\geq0} 2^n x^n = \sum_{n\geq0}(2x)^n = \frac{1}{1-2x}$.
\end{example}
\begin{example}
  For set of compositions $A$ and $w$ size,
  $\Phi_A^w(x) = 1 + x + 2x^2 + 4x^3 + \dotsb = 1+x(1+2x+4x^2+\dotsb)$.
  This is $1+x(\frac{1}{1-2x}) = \frac{1-x}{1-2x}=\frac{1}{1-\frac{x}{1-x}}$.
\end{example}

\begin{defn}[formal power series]
  $A(x) = \sum_{n\geq0}a_n x^n$ where $a_n$ is finite for all $n \in \N$
\end{defn}
We define addition, subtraction, and equality of formal power series akin
to polynomials: $[x^n](A(x) + B(x)) = a_n + b_n$, $[x^n](A(x) - B(x)) = a_n - b_n$,
$A(x) = B(x) \iff \forall n(a_n = b_n)$.

\section{(09/19)}

Define multiplication of formal power series:
$A(x) B(x) = \sum_{n\geq0}(\sum_{k=0}^n a_k b_{n+k})x^n$.

Define division using inverses:
$\frac{1}{A(x)}$ is the power series $C(x)$ such that $A(x)C(x) = 1$
where we consider the unit power series $1+0x+0x^2+\dotsb$.

If $A(x)C(x) = 1$, then each coefficient $\sum_{k=0}^n a_k c_{n-k} = 0$
for $n \geq 1$ and $a_0 c_0 = 1$.

Solving, $c_0 = \frac{1}{a_0}$, $c_1 = -\frac{a_1c_0}{a_0} = -\frac{a_1}{a_0^2}$,
$c_2 = -\frac{1}{a_0}(a_1c_1-a_2c_0)$, etc.

In general, $c_0 = \frac{1}{a_0}$ and
$c_n = -\frac{1}{a_0}\sum_{k=0}^{n-1}a_k c_{n-k}$ for $n \geq 1$.

\begin{theorem}
  $A \in \R[[x]]$ has an inverse $C = A^{-1}$ if and only if $a_0 \neq 0$.
  If $C(x)$ exists, $c_0 = a_0^{-1}$ and $c_n = -a_0^{-1}\sum_{k=1}^n a_k c_{n-k}$.
\end{theorem}

Given these four operations, we have the ring $\R[[x]]$ of power series.
This is a superring of the polynomials $\R[x]$.

\begin{theorem}
  $A \circ B \in \R[[x]]$ exists if and only if $A \in \R[x]$ or $b_0 = 0$
\end{theorem}

\section{(09/21)}

\begin{lemma}[Sum Lemma]
  Let $C$ be the disjoint union $A \cup B$.
  Let $w$ be a weight function of $C$.
  Then, $\Phi_C^w(x) = \Phi_A^w(x)+\Phi_B^w(x)$.

  Note: Since $w : C \to \N$, we implicitly let $\Phi_A^w = \Phi_A^{w\vert_A}$ and $\Phi_B^w = \Phi^{w\vert_B}_B$
\end{lemma}
\begin{prf}
  We will show equality by the coefficient definition, i.e.,
  $[x^n]\Phi_C^w(x) = [x^n](\Phi_A^w(x)+\Phi_B^w(x)) = [x^n]\Phi_A^w(x) + [x^n]\Phi_B^w(x)$.
  The LHS is just the number of elements in $C=A \cup B$ of weight $n$
  and the RHS is the sum of the elements in $A$ and $B$ of weight $n$.
  These must be equal because $A \cap B = \varnothing$.
\end{prf}
\begin{prf}
  Expand:
  \[
    \Phi_C^w(x) = \sum_{c \in C} x^{w(c)} = \sum_{a \in A} x^{w(a)} + \sum_{b \in B}x^{w(b)} = \Phi_A^w(x) + \Phi_B^w(x)
  \]
  where we can divide the sum because every $c \in C$ is in exactly one of $A$ or $B$.
\end{prf}

\begin{lemma}[Infinite Sum Lemma]
  Let $C$ be the disjoint union $\bigcup A_i$ of countably infinite sets.
  Let $w$ be a weight function of $C$.
  Then, $\Phi_C^w(x) = \sum \Phi_{A_i}^w(x)$.
\end{lemma}
\begin{prf}
  Since $w$ is a weight function, the preimage is finite.
  That is, $[x^n]\Phi_C^w(x)$ is finite and we can decompose
  the finite set $\{c \in C : w(c) = n\}$ into finitely many
  $\{c \in A_i : w(c) = n\}$, which gives us what we want as above.
\end{prf}

Note: The proof must go in this direction. For weight functions $w_i : A_i \to \N$, $\operatorname{preim}_{w_i}(n)$ is finite but $\bigcup\operatorname{preim}_{w_i}(n)$ is not guaranteed to be finite.

\begin{lemma}[Product Lemma]
  Let $A$ and $B$ be sets and $w_A$ and $w_B$ be weight functions.
  Define $w_{A\times B} : A \times B \to \N : (a,b) \mapsto w_A(a) + w_B(b)$.
  Then, $\Phi_{A\times B}^{w_{A\times B}} = \Phi_A^{w_A}\cdot\Phi_B^{w_B}$.
\end{lemma}
\begin{prf}
  Recall by definition $\Phi_{A\times B}^{w_{A\times B}} = \smashoperator{\sum\limits_{(a,b)\in A\times B}}x^{w_{A\times B}((a,b))}$.

  But this is $\smashoperator{\sum\limits_{(a,b)\in A\times B}}x^{w_A(a) + w_B(b)}=\smashoperator{\sum\limits_{(a,b)\in A\times B}}x^{w_A(a)}x^{w_B(b)}=\sum\limits_{a\in A}\sum\limits_{b\in B}x^{w_A(a)}x^{w_B(b)}$.

  We split the sum to get $\left(\sum\limits_{a\in A}x^{w_A(a)}\right)\left(\sum\limits_{b\in B}x^{w_B(b)}\right)=\Phi_A^{w_A}(x)\cdot\Phi_B^{w_B}(x)$.
\end{prf}
\begin{corollary}[Finite Products]
  Define $w(a_1,\dotsc,a_k)$ on $A_1\times\dotsb\times A_k$ by $\sum\limits_{i=1}^k w_{A_i}(a_i)$.
  Then, \[ \Phi_{A_1\dotsb\times A_k}^w(x) = \prod\limits_{i=1}^k \Phi_{A_i}^{w_{A_i}}(x) \]
\end{corollary}
\begin{corollary}[Cartesian Product Lemma]
  For set $A$ with weight function $w$, $\Phi_{A^k}^{w_k}(x) = (\Phi_A^w(x))^k$ where $w_k((a_i)) = \sum w(a_i)$.
\end{corollary}
\begin{defn}[strings]
  $A^* = \bigcup_{i=0}^\infty A^i$ where $A^0 = \{\varepsilon\}$ with the empty string $\varepsilon = \varnothing$.
  That is, $A^*$ is the set of all strings (sequences) with entries from $A$.
\end{defn}
\begin{example}
  For $A = \N \setminus \{0\}$, then $A^*$ is the set of all compositions.
  For $A = \{2,4,6,\dotsc\}$, then $A^*$ is the set of all compositions with even parts.
\end{example}

\section{(09/23)}

\begin{lemma}[String Lemma; 2.14]
  Given $A$ and $w$ on $A$, define $w^* : A^* \to \N : (a_i) = \sum w(a_i)$ with $w^*(\varepsilon)=0$.
  Also, there are no elements of $A$ with weight 0.
  Then, $\Phi_{A^*}^{w^*}(x) = \frac{1}{1-\Phi_A^w(x)}$.
\end{lemma}
\begin{prf}
  By definition, $A^* = A^0 \cup A^1 \cup \dotsb$ and this is a disjoint union.

  Then, since $w^*$ is a weight function of $A^*$ becaues $A$ has no weight 0 elements,
  we apply the Infinite Sum Lemma.
  This gives $\Phi_{A^*}^{w^*}(x) = \sum \Phi_{A^i}^{w^*}(x)$.

  By the Product Lemma, $\Phi_{A^i}^{w*}(x) = (\Phi_A^w(x))^i$.
  That is, $\Phi_{A^*}^{w^*}(x) = \sum (\Phi_A^w(x))^i$.
  Now, we can compose the geometric series $\frac{1}{1-x}$ with $\Phi_A^w(x)$
  so long as $[x^0]\Phi_A^w(x)=0$ which is true by supposition.

  Therefore, we get $\Phi_{A^*}{w^*}(x) = \sum(\Phi_A^w(x))^i = \frac{1}{1-\Phi_A^w(x)}$.
\end{prf}

\begin{example}
  Let $A = \N_{\geq 1}$. Then, $A^*$ is the set of all compositions.
  Define $w = \id$, so $A$ has no elements of weight 0 and we have $w^*$ is a weight function.
  Namely, $w^*$ gives the size of a composition.

  By the String Lemma, $\Phi_{A^*}^{w*}(x) = \frac{1}{1-\Phi_A^w(x)}$.
  By inspection, $\Phi_A^w(x) = x + x^2 + \dotsb$.
  By the geometric series, we have $x(1+x+\dotsb) = \frac{x}{1-x}$
  so $\Phi_{A^*}^{w^*}(x)=\frac{1}{1-\frac{x}{1-x}} = \frac{1-x}{1-2x} = 1+\frac{x}{1-2x}$
  and we can write this as $1+x+2x^2+4x^3+\dotsb$
  or \[ [x^n] \Phi_{A^*}^{w^*}(x) = \begin{cases}1 & n = 0\\2^{n-1} & n \geq 1\end{cases} \]
\end{example}

\begin{example}
  Let $A = 2\N_{\geq1}$. Then, $A^*$ is the set of all compositions with even parts
  and as above we can define $w = \id$ and $w^*$ is a weight funciton giving the size.

  By inspection, $\Phi_A^w(x) = x^2 + x^4 + x^6 + \dotsb = x^2(1+(x^2)+(x^2)^2+\dotsb) = \frac{x^2}{1-x^2}$.

  Then, $\Phi_{A^*}^{w^*}(x)=\frac{1}{1-\frac{x^2}{1-x^2}} = \frac{1-x^2}{1-2x^2} = 1 + \frac{x^2}{1-2x^2}$
  and we can write \[ [x^n]\Phi_{A^*}^{w^*}(x)=\begin{cases*}1 & $n=0$ \\ 2^{\frac{n}{2}-1} & $n$ even \\ 0 & $n$ odd\end{cases*} \]
\end{example}

\begin{example}
  $A = \{1,2,3,4\}$, $w = \id$, $A^*$ is the set of all compositions with parts 1 to 4.
  Again, $\Phi_A^w(x) = x + x^2 + x^3 + x^4$ and $\Phi_{A^*}^{w^*} = \frac{1}{1-x-x^2-x^3-x^4}$.
\end{example}

\begin{example}
  $A = \{1,2\}$, $w = \id$, $\Phi_A^w(x) = x + x^2$,
  $\Phi_{A^*}^{w^*}(x)=\frac{1}{1-x-x^2}$ which is the Fibonacci series.
\end{example}

\begin{example}
  $A = \{6,7,8,\dotsc,100\}$, $\Phi_A^w(x) = x^6 + x^7 + \dotsb + x^{100}$.
  By the Finite Geometric Series, this is $\frac{x^6-x^{101}}{1-x}$.
  Then, $\Phi_{A^*}^{w^*}(x) = \frac{1}{1-\frac{x^6-x^{101}}{1-x}} = \frac{1-x}{1-x^6+x^{101}}$.
\end{example}

\chapter{Binary Strings}

\section{(09/26)}

\begin{defn}[binary string]
  A sequence of 0's and 1's.
  The \emph{length} of a binary string is the total number of 0's and 1's.
  We notate the \emph{empty string} by $\varepsilon = \varnothing$.
  Formally, it is an element of $\{0,1\}^*$ written without parentheses and commas.
\end{defn}

\begin{remark}
  Define a weight function for $\{0,1\}^*$ to get the length.
  Let $A = \{0,1\}$. Define $w(a) = 1$.
  Then if we apply Lemma 2.13, $w^*$ is a weight function
  and by the String Lemma, $\Phi_{A^*}^{w^*}(x) = \frac{1}{1-\Phi_A^w(x)} = \frac{1}{1-2x}$.
\end{remark}

\begin{defn}[regular expression]
  A string of finite length that is, up to parentheses, any one of:
  \begin{itemize}[nosep]
    \item $\varepsilon$, 0, and 1;
    \item $\rx R \smile \rx S$ for regular expressions $\rx R$ and $\rx S$;
    \item $\rx{RS}$ for regular expressions $\rx R$ and $\rx S$; or
    \item $\rx R*$ for regular expression $\rx R$
  \end{itemize}
\end{defn}

\begin{defn}[concatenation product]
  For binary strings $\alpha = \alpha_1\dots\alpha_n \in \{0,1\}^*$
  and $\beta = \beta_1\dots\beta_n \in \{0,1\}^*$, define
  $\alpha\beta = \alpha_1\dots\alpha_n\beta_1\dots\beta_n$

  For sets of binary strings $A,B \subseteq \{0,1\}^*$,
  define $AB = \{ \alpha\beta : \alpha \in A, \beta \in B\}$.
\end{defn}

\begin{remark}
  The concatenation product acts like a flatten over the Cartesian product.
  That is, $((1,0),1) \neq (1,(0,1))$ but $(1,0)(1) = (1)(0,1)$.
  It follows that $\abs{AB} \leq \abs{A}\abs{B}$.
\end{remark}

\begin{defn}[rational language]
  $\rl R \subseteq \{0,1\}^*$ produced by a regular expression $\rx R$:
  \begin{itemize}[nosep]
    \item $\varepsilon$, 0, and 1 produce themselves;
    \item $\rx R\smile \rx S$ produces $\rl R \cup \rl S$;
    \item $\rx{RS}$ produces the concatenation product $\rl{RS}$; and
    \item $\rx R^*$ produces $\smashoperator{\bigcup_{k\geq 0}} \rl R^k$ where $\rl R^k$ is the concatenation power.
  \end{itemize}
\end{defn}

\begin{example}
  What languages do $(1)^*$, $(1 \smile 11)^*$, $(0 \smile 1)^*$, $1(11)^*$, $(01)^*$ produce?
\end{example}
\begin{sol}
  $(1)^*$ produces the set of binary strings of only ones, i.e., $\{1\}^*$.

  $(1 \smile 11)^*$ produces the same set.

  $(0 \smile 1)^*$ produces $\{0,1\}^*$, the set of all binary strings.

  $1(11)^*$ produces the set of all odd numbers of ones.

  $(01)^*$ produces $\{\varepsilon, 01, 0101, 010101, \dotsc\}$.
\end{sol}

\begin{example}
  Not every set is a rational language.
  The set $\{e,01,0011,000111,0^i1^i\}$ is not a rational language
  because it cannot be produced by a regular expression.
\end{example}

\section{(09/28; from Bradley)}

\begin{defn}[unambiguity]
  Regular expression $\rx R$ that produces every element in the corresponding
  rational language $\rl R$ exactly once.
\end{defn}

\begin{lemma}
  The following regular expressions are unambiguous:
  \begin{itemize}[nosep]
    \item $\varepsilon$, 0, and 1
    \item $\rx R \smile \rx S$ given $\rl R \cap \rl S = \varnothing$
    \item $\rx{RS}$ given either (1) $\abs{\rl{RS}}=\abs{\rl R}\abs{\rl S}$,
          (2) for all $t \in \rl{RS}$, there is a unique
          $r \in \rl R$, $s \in \rl S$ such that $rs = t$, or
          (3) there does not exist $r_1,r_2\in \rl R$, $s_1,s_2 \in \rl S$
          such that $r_1 s_1 = r_2 s_2$.
    \item $\rx R^*$ given $\rx R$ unambiguous and either
          (1) all of $\varepsilon$, $\rl R$, and $\rl R^2$ are disjoint or
          (2) all of the $\rl R^i$ are unambiguous as products
  \end{itemize}
\end{lemma}

\begin{example}
  Consider the following:
  \begin{itemize}[nosep]
    \item $1^*$ is unambiguous because $\varepsilon$, $1$, $1^2$, etc.
          are all disjoint meaning that $1^k$ is unambiguous for all $k$.
    \item $(1 \smile 11)^*$ is ambiguous because we can produce 11 as either $(11)^1$ or $1^2$.
    \item $(0 \smile 1)^*$ produces all binary strings and is unambiguous.
          The union is clearly disjoint.
          The binary strings of lengths $k$ are clearly disjoint.
    \item $(101)^*$ produces $\{\varepsilon,101,1011101,\dotsc\}$ and is unambiguous.
    \item $(1 \smile 10 \smile 01)^*$ is ambiguous because 101 is produced by $1^1(01)^1$ and $(10)^11^1$.
  \end{itemize}
\end{example}

\begin{defn}
  $\rx R$ leads to a rational function $R(x)$ according to its structure:
  \begin{itemize}[nosep]
    \item $\varepsilon$ leads to 1
    \item 0 leads to $x$
    \item 1 leads to 1
    \item $R \smile S$ leads to $R(x) + S(x)$
    \item $RS$ leads to $R(x)\cdot S(x)$
    \item $R^*$ leads to $\frac{1}{1-R(x)}$
  \end{itemize}
\end{defn}

\begin{example}
  Consider the same regular expressions:
  \begin{itemize}[nosep]
    \item $1^*$ leads to $\frac{1}{1-x}$
    \item $(1 \smile 11)^*$ leads to $\frac{1}{1-(x+x^2)} = \frac{1}{1-x-x^2}$
    \item $(0 \smile 1)^*$ leads to $\frac{1}{1-2x}$.
    \item $(101)^*$ leads to $\frac{1}{1-x^3}$
    \item $(1 \smile 10 \smile 01)^*$ leads to $\frac{1}{1-x-2x^2}$
    \item $0^*(11^*00^*)^*1^*$ leads to $\frac{1}{1-x}\frac{1}{1-\frac{x^2}{(1-x)^2}}\frac{1}{1-x}=\frac{1}{1-2x}$
  \end{itemize}
\end{example}

\begin{lemma}
  Let $\rx R$ be a regular expression, $\rl R$ its rational language,
  $R(x)$ the rational function it leads to, and $w$ the length weight function on binary strings.

  If $\rx R$ is unambiguous, then $\Phi_{\rl R}^w(x) = R(x)$.
\end{lemma}
\begin{prf}
  Proceed by structural induction on the above unambiguity lemma:

  Notice that $\Phi_{\{e\}}^w(x) = 1$, $\Phi_{\{0\}}^w(x) = x$, and $\Phi_{\{1\}}^w(x) = x$.

  If $\rx R = \rx S \smile \rx T$ and is unambiguous, then $\rx R$ produces $\rl S \cup \rl T$ and
  \begin{equation*}
    \Phi_{\rl S \cup \rl T}^w(x) = \Phi_{\rl S}^w(x) + \Phi_{\rl T}^w(x) = S(x) + T(x) = R(x)
  \end{equation*}
  by the Sum Lemma.

  Likewise by the Product Lemma, if $\rx R = \rx{ST}$ we can write
  \begin{equation*}
    \Phi_{\rl S\rl T}^w(x) = \Phi_{\rl S \times \rl T}^w(x) = \Phi_{\rl S}^w(x) \cdot \Phi_{\rl T}^w(x) = S(x) \cdot T(x) = R(x)
  \end{equation*}
  because $\rl{ST} = \rl{S}\times\rl{T}$ by unambiguity.

  The case for $\rx R = \rx S^*$ goes similarly by the Infinite Sum Lemma and Product Lemma.
\end{prf}

Be careful to distinguish between:
\begin{itemize}[nosep]
  \item A regular expression $\rx R$
  \item A rational language $\rl R \subseteq \{0,1\}^*$ it \emph{produces}
  \item A rational function $R(x) \in \Z[[x]]$ it \emph{leads to},
        equal to $\Phi_{\rl R}(x)$ when $\rx R$ is unambiguous
\end{itemize}

\section{(09/30; from Bradley)}

\begin{defn}[block]
  A maximal subsequence of a binary string with the same digit.
\end{defn}

\begin{lemma}[Block Decompositions]
  The set of all binary strings is unambiguously produced by
  $0^*(11^*00^*)^*1^*$ and $1^*(00^*11^*)^*0^*$.
\end{lemma}
\begin{prf}
  \WLOG{} consider the second regular expression.
  We decompose every binary string after each block of 0s.

  This means each string is of the form
  (a (possibly empty) initial block of 0s,
  first pair of blocks of 1s, second pair of 0s, ..., last pair,
  (possibly empty) terminal block of 1s).
  Moreover, first/last pair may not exist.

  This decomposition is unique and we can express it
  as a regular expression of the form $\rx{I(M)^*T}$ where
  \begin{itemize}[nosep]
    \item $\rx I$ is a regular expression for the initial (possibly empty) block of 0s
    \item $\rx M$ is a regular expression for a middle pair of blocks,
          non-empty block of 1s followed by non-empty block of 0s
    \item $\rx T$ is a regular expression for the terminal (possibly empty) block of 1s
  \end{itemize}
  Then, we can unambiguously write $\rx I = 0^*$, $\rx M = (11^* 00^*)^*$,
  and $\rx T = 1^*$.
  Using the extra 1/0 ensures each block is non-empty.
\end{prf}

\begin{example}
  Write an unambiguous regular expression for ``the set of binary strings where
  each block of 0s has length at least 2 and each block of 1s has even length''.
\end{example}
\begin{sol}
  Follow the $\rx{I(M)^*T}$ pattern beginning with a block of 0s.

  To get either no 0s or at least two 0s, set $\rx I = \varepsilon \smile (000^*)$.

  To get either no 1s or an even number of 1s, set $\rx T = (11)^*$.

  Combine these ideas to get $\rx M = 11(11)^*000^*$.

  Altogether, write $(\varepsilon \smile (000^*))(11(11)^*000^*)^*(11)^*$.
\end{sol}

\begin{example}\label{exa:11-2}
  Write an unambiguous regular expression for ``the set of all binary strings
  where each block of 0s has length at least 5 and congruent to 2 (mod 3)
  and each block of 1s has length at least 2 and at most 8''.
\end{example}
\begin{sol}
  Follow the $\rx{I(M)^*T}$ pattern beginning with a block of 0s.

  To get either no 0s or $5 + 3k$ 0s, set $\rx I = \varepsilon \smile (00000(000)^*)$.

  To get either no 1s or between two and eight 1s, set $\rx T = (11 \smile 111 \smile \dotsb \smile 1^8 \smile \varepsilon)$.

  Combine to get $\rx M = (11 \smile 111 \smile \dotsb \smile 1^8)(00000(000)^*)$.
  Altogether,
  \[
    (\varepsilon \smile (00000(000)^*))((11 \smile 111 \smile \dotsb \smile 1^8)(00000(000)^*))^*(11 \smile 111 \smile \dotsb \smile 1^8 \smile \varepsilon)
  \]
\end{sol}

\begin{example}
  Same as \cref{exa:11-2} with the additional restriction
  that the string is non-empty and starts with 0.
\end{example}
\begin{sol}
  $(0^5(000)^*)((1^2 \smile 1^3 \smile \dotsb \smile 1^8)(0^5(000)^*))^*(1^2 \smile 1^3 \smile \dotsb \smile 1^8 \smile \varepsilon)$.

  If we wanted it to start with 1 instead, decompose the blocks beginning with 1s
  to get a similar answer:
  $(1^2 \smile 1^3 \smile \dotsb \smile 1^8)((0^5(000)^*)(1^2 \smile 1^3 \smile \dotsb \smile 1^8))^*(0^5(000)^* \smile \varepsilon)$.
\end{sol}

\begin{example}
  Starts with 0 and at least 2 blocks
\end{example}
\begin{sol}
  $(0^5(000)^*)(1^2 \smile \dotsb \smile 1^8)((0^5(000)^*)(1^2 \smile \dotsb \smile 1^8))^*(0^5(000)^* \smile \varepsilon)$.
\end{sol}

\begin{example}
  Starts with 0 and ends with 0
\end{example}
\begin{sol}
  $(0^5(000)^*)((1^2 \smile 1^3 \smile \dotsb \smile 1^8)(0^5(000)^*))^*$.
\end{sol}

\end{document}