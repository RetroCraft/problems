\section{Binary Strings}

\begin{xca}
  Prove Lemma 3.9 (Unambiguous Expressions)
\end{xca}
\begin{prf}
  Let $\rx R$ and $\rx S$ be regular expressions producing $\rl R$ and $\rl S$.
  Proceed by cases.

  If $\rx R = \varepsilon$, 0, or 1, then $\rl R = \{\varepsilon\}$, $\{0\}$, or $\{1\}$.
  Notice each is produced exactly once, so $\rx R$ is unambiguous.

  Now, suppose $\rx R$ and $\rx S$ are unambiguous.
  We proceed by contrapositives.

  Suppose $\rx R \smile \rx S$ which produces $\rl R \cup \rl S$ is ambiguous.
  Then, since $\rx R$ and $\rx S$ are unambiguous,
  the union operation must produce a duplicate.
  That is, $\rl R \cap \rl S \neq \varnothing$.
  Conversely, if $\rl R \cap \rl S$ is non-empty,
  then whatever expressions are in the intersection are produced twice by $\rx R \smile \rx S$.
  Therefore, $\rx R \smile \rx S$ is ambiguous if and only if $\rl R \cap \rl S \neq \varnothing$.

  Suppose $\rx R\rx S$ is ambiguous.
  Then, $\rl R \rl S = \{\rho\sigma : \rho \in \rl R, \sigma \in \rl S\}$
  produces some element $\rho\sigma = \rho'\sigma'$ twice.
  Notice that for all $(\rho,\sigma) \in \rl R \times \rl S$, $f(\rho,\sigma) = \rho\sigma \in \rl R \rl S$.
  That is, $\abs{\rl R \times \rl S} \geq \abs{\rl R \rl S}$.
  However, since $\rho\sigma = \rho'\sigma'$, $f$ is not injective.
  Therefore, $\abs{\rl R \times \rl S} \neq \abs{\rl R \rl S}$.
  This means $\rl R\times\rl S \mathrel{\not\bijects} \rl R\rl S$.

  Conversely, suppose $\rl R\times\rl S \mathrel{\not\bijects} \rl R\rl S$.
  Then, $\abs{\rl R \times \rl S} > \abs{\rl R \rl S}$.
  This means that under $f$, multiple pairs must be sent to one string,
  which is exactly what it means for $\rl R\rl S$ to be ambiguous.

  Finally, consider $\rx R^*$ is ambiguous.
  Then, since $\rx R$ is unambiguous, the ambiguity must be introduced
  by either a single $\rx R^k$ being ambiguous or the union.
  By induction on the second point, the union is unambiguous if and only if
  $\bigcup_{k=0}^\infty \rl R^k$ is a disjoint union.
  Therefore, $\rx R^*$ is ambiguous if and only if all the $\rx R^k$ are unambiguous
  and the union of the $\rl R^k$ is disjoint.
\end{prf}

\begin{xca}
  Let $\rx A = (10 \smile 101)$ and $\rx B = (001 \smile 100 \smile 1001)$.
  For each of $\rx{AB}$ and $\rx{BA}$, is the expression unambiguous?
  What is the generating series (by length) of the set it produces?
\end{xca}
\begin{sol}
  Write out $\rl{AB} = \{10001, 10100, 101001, 101001, 101100, 1011001\}$.
  Notice $101001$ appears twice, so $\rx{AB}$ is ambiguous.
  The (meaningless) generating series is $2x^5 + 2x^6 + x^7$.

  Write out $\rl{BA} = \{00110, 001101, 10010, 100101, 100110, 1001101\}$.
  No element appeared twice, so $\rx{BA}$ is unambiguous.
  The generating series is $2x^5 + 3x^6 + x^7$.
\end{sol}

\begin{xca}
  Let $\rx A = (00 \smile 101 \smile 11)$ and $\rx B = (00 \smile 001 \smile 10 \smile 110)$.
  Prove that $\rx A^*$ is unambiguous and $\rx B^*$ is ambiguous.
  Find the generating series (by length) for the set $\rl A^*$ produced by $\rx A^*$.
\end{xca}
\begin{sol}
  Notice that $\rx A$ is unambiguous.
  There is no way to combine any two of the three strings to create the other one:
  creating 00 from 101 and 11 is clearly impossible;
  11 from 00 and 101 can only be made by 101101 but 10 and 01 cannot be made by 00;
  101 cannot be made from 00 and 11 since there is no single 0.
  Therefore, $\rx A^*$ is unambiguous.

  For $\rx B$, notice that $00110 = (001)(10) = (00)(110)$, so it is ambiguous.

  By Theorem 3.13, $\Phi_{\rl A^*} = (A^*)(x) = \frac{1}{1-A(x)} = \frac{1}{1-2x^2-x^3}$.
\end{sol}

\begin{xca}
  For each of the following sets of binary strings,
  write an unambiguous expression which produces that set.
\end{xca}
\begin{enumerate}
  \item Binary strings that have no block of 0's of size 3,
        and no block of 1's of size 2.
        \begin{sol}
          A valid block of 0's is matched by $0 \smile 00 \smile 00000^*$.
          Likewise, a valid block of 1's is $1 \smile 1111^*$.
          Then, the block decomposition is
          \[
            \rx R = (\varepsilon \smile 0 \smile 0^2 \smile 0^40^*)
            ((1\smile 1^31^*)(0 \smile 0^2 \smile 0^40^*))^*
            (\varepsilon \smile 1 \smile 1^31^*)
          \]
          which is, as a block decomposition, unambiguous.
        \end{sol}
  \item Binary strings that have no substring of 0's of length 3,
        and no substring of 1's of length 2.
        \begin{sol}
          This means blocks of 0's have length 1 or 2, i.e., $(0 \smile 00)$
          and blocks of 1's have length 1, i.e., $(1)$.
          Then, the block decomposition is
          \[
            \rx R = (\varepsilon \smile 0 \smile 00)(1(0 \smile 00))^*(\varepsilon \smile 1)
          \]
          which an unambiguous block decomposition.
        \end{sol}
  \item Binary strings in which the substring 011 does not occur.
        \begin{sol}
          There are no ways that 011 overlaps itself.
          Therefore, we need only force blocks of 1's after a 0 to have length 1.
          Using a block decomposition depending if we start with 1 or 0,
          \[
            \rx R = \varepsilon \smile (1 1^*)(0 0^* 1)^*0^* \smile (00^*)(1 00^*)^*(\varepsilon \smile 1)
          \]
          which is unambiguous.
        \end{sol}
  \item Binary strings in which the blocks of 0's have even length and
        the blocks of 1's have odd length.
        \begin{sol}
          Blocks of 0's are matched by $00(00)^*$ and 1's by $1(11)^*$. Then,
          \[
            (00)^*(1(11)^*00(00)^*)^*(\varepsilon \smile 1(11)^*)
          \]
          is an unambiguous block decomposition.
        \end{sol}
\end{enumerate}

\begin{xca}
  Let $\rx G = 0^*((11)^*1(00)^*00 \smile (11)^*11(00)^*0)^*$,
  and let $\rl G$ be the set of binary strings produced by $\rx G$.
\end{xca}
\begin{enumerate}
  \item Give a verbal description of the strings in the set $\rl G$.
        \begin{sol}
          The set of binary strings where blocks of 0's
          have the opposite parity of the preceding block of 1's.
        \end{sol}
  \item Find the generating series (by length) of $\rl G$.
        \begin{sol}
          We know by Theorem 3.13 $\Phi_{\rl G}(x) = G(x)$, so
          \begin{align*}
            \Phi_{\rl G}(x)
             & = \frac{1}{1-x}\cdot\frac{1}{1-\qty(\frac{x}{1-x^2}\cdot\frac{x^2}{1-x^2} + \frac{x^2}{1-x^2}\cdot\frac{x}{1-x^2})} \\
             & = \frac{1}{1-x}\cdot\frac{1}{1-\frac{2x^3}{(1-x^2)^2}}                                                              \\
             & = \frac{1}{1-x}\cdot\frac{(1-x^2)^2}{(1-x^2)^2-2x^3}                                                                \\
             & = \frac{(1+x)(1-x^2)}{1-2x^2-2x^3+x^4}                                                                              \\
             & = \frac{1+x-x^2-x^3}{1-2x^2-2x^3+x^4}
          \end{align*}
          as desired.
        \end{sol}
  \item For $n \in \N$, let $g_n$ be the number of strings in $\rl G$ of length $n$.
        Give a recurrence relation and enough initial conditions
        to uniquely determine $g_n$ for all $n \in \N$.
        \begin{sol}
          Apply Theorem 4.8 and read off the linear recurrence relation:
          \[
            g_n - 2g_{n-2} - 2g_{n-3} + g_{n-4} = \begin{cases}
              1  & n = 0    \\
              1  & n = 1    \\
              -1 & n = 2    \\
              -1 & n = 3    \\
              0  & n \geq 4
            \end{cases}
          \]
          from which we calculate initial conditions $g_0 = 1$,
          $g_1 = 1$, $g_2 = 2g_0 - 1 = 1$, $g_3 = 2g_1 + 2g_0 - 1 = 3$.
        \end{sol}
\end{enumerate}

\begin{xca}\end{xca}
\begin{enumerate}
  \item Show that the generating series (by length) for binary strings
        in which every block of 0's has length at least 2
        and every block of 1's has length at least 3 is $\frac{(1-x+x^3)(1-x+x^2)}{1-2x+x^2-x^5}$.
        \begin{sol}
          This set of strings is produced by
          $\rx R = (\varepsilon \smile 000^*)(1111^*000^*)^*(\varepsilon \smile 1111^*)$.
          This leads to the rational function
          \begin{align*}
            R(x)
             & = \qty(1 + \frac{x^2}{1-x})\frac{1}{1-(\frac{x^3}{1-x}\cdot\frac{x^2}{1-x})}\qty(1 + \frac{x^3}{1-x}) \\
             & = \frac{1-x+x^2}{1-x}\cdot\frac{1}{1-\frac{x^5}{(1-x)^2}}\cdot\frac{1-x+x^3}{1-x}                     \\
             & = \frac{(1-x+x^2)(1-x+x^3)}{(1-x)^2}\cdot\frac{(1-x)^2}{(1-x)^2-x^5}                                  \\
             & = \frac{(1-x+x^2)(1-x+x^3)}{1-2x+x^2-x^5}
          \end{align*}
          which is equal to the generating series by Theorem 3.13.
        \end{sol}
  \item Give a recurrence relation and enough initial conditions to
        determine the coefficients of this power series.
        \begin{sol}
          Expand the numerator to get $\Phi_{\rl R}(x) = \frac{1-2x+2x^2-x^4+x^5}{1-2x+x^2-x^5}$.
          Then, apply Theorem 4.8 to read off the linear recurrence relation:
          \[
            r_n - 2r_{n-1} + r_{n-2} - r_{n-5} = \begin{cases}
              1  & n = 0    \\
              -2 & n = 1    \\
              2  & n = 2    \\
              0  & n = 3    \\
              -1 & n = 4    \\
              1  & n = 5    \\
              0  & n \geq 6
            \end{cases}
          \]
          and calculate $r_0 = 1$, $r_1 = -2 + 2r_0 = 0$,
          $r_2 = 2 + 2r_1 - r_0 = 1$, $r_3 = 0 + 2r_2 - r_1 = 2$,
          $r_4 = -1 + 2r_3 - r_2 = 2$, and $r_5 = 1 + 2r_4 - r_3 + r_0 = 4$.
        \end{sol}
\end{enumerate}

\begin{xca}\end{xca}
\begin{enumerate}
  \item For $n \in \N$, let $h_n$ be the number of binary strings of length $n$
        such that each even-length block of 0's is followed by a block of exactly one 1
        and each odd-length block of 0's is followed by a block of exactly two 1's.
        Show that $h_n = [x^n]\frac{1+x}{1-x^2-2x^3}$.
        \begin{sol}
          Let $\rl H$ be the relevant set and notice it is produced by
          the block decomposition $\rx H = 1^*(00(00)^*1 \smile 0(00)^*11)$.
          Note that since a block of 0's is followed by a block of 1's, we must end on a block of 1's.
          By Theorem 3.11, $\Phi_{\rl H}(x) = H(x)$ which is
          \begin{align*}
            H(x)
             & = \frac{1}{1-x}\cdot\frac{1}{1-(\frac{x^3}{1-x^2} + \frac{x^3}{1-x^2})} \\
             & = \frac{1}{1-x}\cdot\frac{1-x^2}{1-x^2-2x^3}                            \\
             & = \frac{1+x}{1-x^2-2x^3}
          \end{align*}
          and $h_n = [x^n]H(x)$ by Proposition 2.7, as desired.
        \end{sol}
  \item Give a recurrence relation and enough initial conditions to
        determine $h_n$ for all $n \in \N$.
        \begin{sol}
          Again, read off a recurrence relation from $H(x)$ by Theorem 4.8:
          \[
            h_n - h_{n-2} - 2h_{n-3} = \begin{cases}
              1 & n = 0    \\
              1 & n = 1    \\
              0 & n \geq 2
            \end{cases}
          \]
          and calculate $h_0 = 1$, $h_1 = 1$, $h_2 = 0 + h_0 = 1$,
          and $h_3 = 0 + h_1 + 2h_0 = 3$.
        \end{sol}
\end{enumerate}

\begin{xca}
  Let $\rl K$ be the set of binary strings in which any block of 1's
  which immediately follows a block of 0's must have length at least
  as great as the length of that block of 0's.
\end{xca}
\begin{enumerate}
  \item Derive a formula for $K(x) = \sum_{\kappa \in \rl K} x^{\ell(\kappa)}$.
        \begin{sol}
          First, we recursively define $\rx L = 11^* \smile 0\rx L 1$
          so that $\rl L = 0^i 1^j$ where $j \geq i \geq 1$ which is unambiguous.
          Now, define $\rx K = 1^*\rx L^* 0^*$ as a block decomposition.

          We can now calculate $L(x) = \frac{x}{1-x} + x^2L(x)$
          so $L(x) = \frac{x}{(1-x)(1-x^2)}$. Then,
          \begin{align*}
            K(x)
             & = \frac{1}{1-x}\cdot\frac{1}{1-\frac{x}{(1-x)(1-x^2)}}\cdot\frac{1}{1-x} \\
             & = \frac{1}{(1-x)^2}\cdot\frac{(1-x)(1-x^2)}{(1-x)(1-x^2)-x}              \\
             & = \frac{1+x}{1-2x-x^2+x^3}
          \end{align*}
          which is the generating series $\Phi_{\rl K}(x)$ by Theorem 3.11.
        \end{sol}
  \item Give a recurrence relation and enough initial conditions to
        determine the coefficients $[x^n]K(x)$ for all $n \in \N$.
        \begin{sol}
          Let $k_n = [x^n]K(x)$. By Theorem 4.8, we have the recurrence relation
          \[
            k_n - 2k_{n-1} - k_{n-2} + k_{n-3} = \begin{cases}
              1 & n = 0    \\
              1 & n = 1    \\
              0 & n \geq 2
            \end{cases}
          \]
          and calculate $k_0 = 1$, $k_1 = 1 + 2k_0 = 3$,
          and $k_2 = 0 + 2k_1 + k_0 = 7$.
        \end{sol}
\end{enumerate}

\begin{xca}\end{xca}
\begin{enumerate}
  \item Fix an integer $m \geq 1$.
        Find the generating series (by length) of the set of binary strings
        in which no block has length greater than $m$.
        \begin{sol}
          Blocks of 0's and 1's become $\smilesum_{i=1}^m 0^i$
          and $\smilesum_{i=1}^m 1^i$.
          Then, the block decomposition
          \[
            \rx R = (\smilesum_{i=0}^m 0^i)
            (\smilesum_{i=1}^m 1^i\smilesum_{i=1}^m 0^i)^*
            (\smilesum_{i=0}^m 1^i)
          \]
          produces our desired set.
          By Theorem 3.11, the generating series $\Phi_{\rl R}(x) = R(x)$ which is
          \begin{align*}
            R(x)
             & = \qty(\sum_{i=0}^m x^i)\qty(\frac{1}{1-(\sum_{i=1}^m x^i)^2})\qty(\sum_{i=0}^m x^i) \\
             & = \qty(\frac{1-x^{m+1}}{1-x})^2\qty(\frac{1}{1-(\frac{x-x^{m+1}}{1-x})^2})           \\
             & = \frac{(1-x^{m+1})^2}{(1-x)^2}\cdot\frac{(1-x)^2}{(1-x)^2-(x-x^{m+1})^2}            \\
             & = \frac{(1-x^{m+1})^2}{(1-x)^2-(x-x^{m+1})^2}
          \end{align*}
          which I'm sure simplifies further.
        \end{sol}
  \item Fix integers $m, k \geq 1$.
        Find the generating series (by length) of the set of binary strings
        in which no block of 0's has length greater than $m$
        and no block of 1's has length greater than $k$.
        \begin{sol}
          Proceed as above: we instead get the block decomposition
          \[
            \rx R = (\smilesum_{i=0}^m 0^i)
            (\smilesum_{i=1}^k 1^i\smilesum_{i=1}^m 0^i)^*
            (\smilesum_{i=0}^k 1^i)
          \]
          which produces the set we want. By Theorem 3.11,
          \begin{align*}
            \Phi_{\rl R}(x)
             & = R(x)                                                                                                                   \\
             & = \qty(\sum_{i=0}^m x^i)\qty(\frac{1}{1-(\sum_{i=1}^k x^i)(\sum_{i=1}^m x^i)})\qty(\sum_{i=0}^k x^i)                     \\
             & = \qty(\frac{1-x^{m+1}}{1-x})\qty(\frac{1}{1-(\frac{x-x^{k+1}}{1-x})(\frac{x-x^{m+1}}{1-x})})\qty(\frac{1-x^{k+1}}{1-x}) \\
             & = \frac{(1-x^{m+1})(1-x^{k+1})}{(1-x)^2}\cdot\frac{(1-x)^2}{(1-x)^2 - (x-x^{k+1})(x-x^{m+1})}                            \\
             & = \frac{(1-x^{m+1})(1-x^{k+1})}{(1-x)^2 - (x-x^{k+1})(x-x^{m+1})}                                                        \\
          \end{align*}
          as desired.
        \end{sol}
\end{enumerate}

\begin{xca}
  Let $\rl L$ be the set of binary strings in which
  each block of 1's has odd length, and which do not contain the substring 0001.
  Let $\rl L_n$ be the set of strings in $\rl L$ of length $n$
  and let $L(x) = \sum_{n=0}^\infty \abs{\rl L_n}x^n$.
\end{xca}
\begin{enumerate}
  \item Give an expression that produces the set $\rl L$ unambiguously,
        and explain briefly why it is unambiguous and produces $\rl L$.
        \begin{sol}
          Notice that 0001 does not overlap itself.
          Then, we need only prevent a non-terminal block of 3 or more 0's.
          Also, blocks of 1's must be odd.
          Write $\rx L = (\varepsilon \smile 1(11)^*)((0 \smile 00)1(11)^*)^*0^*$
          and notice it is unambiguous as a block decomposition.
        \end{sol}
  \item Use your expression from part (a) to show that
        $L(x) = \frac{1+x-x^2}{1-x-x^2+x^3+x^4}$.
        \begin{sol}
          By Theorem 3.11, the generating series $L(x)$ is
          \begin{align*}
            L(x)
             & = (1+\frac{x}{1-x^2})\frac{1}{1-\frac{(x+x^2)x}{1-x^2}}\frac{1}{1-x} \\
             & = \frac{1+x-x^2}{(1-x^2)(1-x)}\cdot\frac{1-x^2}{1-x^2-(x+x^2)x}      \\
             & = \frac{1+x-x^2}{(1-x)(1-2x^2-x^3)}                                  \\
             & = \frac{1+x-x^2}{1-x-x^2+x^3+x^4}
          \end{align*}
          as desired.
        \end{sol}
\end{enumerate}

\begin{xca}
  Let $\rl M$ be the set of binary strings in which
  each block of 0's has length at most two
  and which do not contain 00111 as a substring.
  Let $\rl M_n$ be the set of strings in $\rl M$ of length $n$
  and let $M(x) = \sum_{n=0}^\infty \abs{\rl M_n}x^n$.
\end{xca}
\begin{enumerate}
  \item Give an expression that produces the set $\rl M$ unambiguously,
        and explain briefly why it is unambiguous and produces $\rl M$.
        \begin{sol}
          Write $\rx M = 1^*(011^* \smile (001 \smile 0011))^*(\varepsilon \smile 0 \smile 00)$.
          We split the middle blocks by whether there are one or two zeroes.
          If there are two zeroes, then we can only have two ones to avoid 00111.
          This is unambiguous as it is an (albeit weird-looking) block decomposition.
        \end{sol}
  \item Use your expression from part (a) to show that $M(x) = \frac{1+x+x^2}{1-x-x^2-x^3+x^5}$.
        \begin{sol}
          Write by Theorem 3.11 that
          \begin{align*}
            M(x)
             & = \frac{1}{1-x}\cdot\frac{1}{1-(\frac{x^2}{1-x}+x^3+x^4)}\cdot(1+x+x^2) \\
             & = \frac{1+x+x^2}{1-x}\cdot\frac{1-x}{(1-x^3-x^4)(1-x)-x^2}              \\
             & = \frac{1+x+x^2}{1-x-x^2-x^3+x^5}
          \end{align*}
          as desired.
        \end{sol}
\end{enumerate}

\begin{xca}
  Let $\rl N$ be the set of binary strings in which
  each block of 0's has odd length
  and each block of 1's has length 1 or 2.
  Let $\rl N_n$ be the set of strings in $\rl N$ of length $n$
  and let $N(x) = \sum_{n=0}^\infty \abs{\rl N_n}x^n$.
\end{xca}
\begin{enumerate}
  \item Show that $N(x) = \frac{1+2x+x^2-x^4}{1-2x^2-x^3} = -2+x+\frac{3+x-3x^2}{1-2x^2-x^3}$.
        \begin{sol}
          Write $\rx N = (\varepsilon \smile 0(00)^*)((1\smile 11)0(00)^*)^*(\varepsilon\smile 1\smile 11)$.
          By Theorem 3.11,
          \begin{align*}
            N(x)
             & = \qty(1+\frac{x}{1-x^2})\cdot\frac{1}{1-\frac{(x+x^2)x}{1-x^2}}\cdot(1+x+x^2) \\
             & = \frac{(1+x-x^2)(1+x+x^2)}{1-x^2}\cdot\frac{1-x^2}{(1-x^2)-(x^2+x^3)}         \\
             & = \frac{(1+x-x^2)(1+x+x^2)}{1-2x^2-x^3}                                        \\
             & = \frac{1+2x+x^2-x^4}{1-2x^2-x^3}
          \end{align*}
          as desired.
        \end{sol}
  \item Derive an exact formula for $\abs{\rl N_n}$ as a function of $n$.
        \begin{sol}
          Apply partial fractions on $\frac{3+x-3x^2}{1-2x^2-x^3} = \frac{A}{1+x} + \frac{B+Cx}{1-x-x^2}$.
          Equate numerators to get $3+x-3x^2 = A(1-x-x^2) + (B+Cx)(1+x) = (A+B) + (-A+B+C)x + (-C)x^2$.

          This gives the system
          \[ \systeme[ABC]{A+B=3,-A+B+C=1,-A+C=-3} \]
          which solves to $A=-1$, $B=-4$, $C=4$.
          Finally, we have that
          \begin{align*}
            [x^n]\frac{3+x-3x^2}{1-2x^2-x^3}
             & = -[x^n]\frac{1}{1+x}-4[x^n]\frac{1}{1-x-x^2}+4[x^n]\frac{x}{1-x-x^2} \\
             & = -(-1)^n - 4f_n + 4f_{n-1}                                           \\
             & = (-1)^{n+1} - 4(f_n - f_{n-1})                                       \\
             & = 4f_{n-2} - (-1)^n
          \end{align*}
          where $f_n$ is the $n$th Fibonacci number.
        \end{sol}
\end{enumerate}

\begin{xca}
  For $n \in N$, let $p_n$ be the number of binary strings of length $n$
  in which every block of 0's is followed by a block of 1's with the same parity of length.
\end{xca}
\begin{enumerate}
  \item Determine the generating series $P(x) = \sum_{n=0}^\infty p_n x^n$
        \begin{sol}
          Write $\rx P = 1^*(00(00)^*11(11)^* \smile 0(00)^*1(11)^*)0^*$.
          Then, by Theorem 3.11,
          \begin{align*}
            P(x)
             & = \frac{1}{1-x}\cdot\frac{1}{1-(\frac{x^4}{(1-x^2)^2} + \frac{x^2}{(1-x^2)^2})}\cdot\frac{1}{1-x} \\
             & = \frac{1}{(1-x)^2}\cdot\frac{1}{1-\frac{x^4+x^2}{(1-x^2)^2}}                                     \\
             & = \frac{1}{(1-x)^2}\cdot\frac{(1-x^2)^2}{(1-x^2)^2-x^2-x^4}                                       \\
             & = \frac{1+2x+x^2}{1-3x^2}
          \end{align*}
          as desired.
        \end{sol}
  \item Show that if $n \geq 2$, then $p_n = 2\cdot 3^{\floor{n/2}-1}$.
        \begin{sol}
          By part (a), we have that
          \begin{align*}
            p_n
             & = [x^n] \frac{1+2x+x^2}{1-3x^2}                                              \\
             & = [x^n] \frac{1}{1-3x^2} + 2[x^n] \frac{x}{1-3x^2} + [x^n]\frac{x^2}{1-3x^2} \\
             & = a_n + 2a_{n-1} + a_{n-2}
          \end{align*}
          where $\sum a_n x^n = \frac{1}{1-3x^2}$ which is valid because $n \geq 2$.
          Notice that $\frac{1}{1-3x^2} = \sum 3^n x^{2n}$.
          If $n$ is even, then we have $3^{n/2}+ 2(0) + 3^{n/2-1} = 4\cdot 3^{\floor{n/2}-1}$.
          If $n$ is odd, then we have $0 + 2\cdot 3^{(n-1)/2} + 0 = 2\cdot 3^{\floor{n/2}-1}$.

          Which isn't right???
        \end{sol}
\end{enumerate}

\begin{xca}\end{xca}
\begin{enumerate}
  \item Let $\rl Q$ be the set of binary strings that do not contain 11000 as a substring.
        For $n \in \N$, let $\rl Q_n$ be the set of strings in $\rl Q$ with length $n$.
        Obtain a formula for the generating series $Q(x) = \sum \abs{\rl Q_n}x^n$.
        \begin{sol}
          Notice that 11000 cannot overlap itself.
          Therefore, the generating series $Q(x) = \frac{1+0}{(1-2x)(1+0)+x^5} = \frac{1}{1-2x+x^5}$
          by Theorem 3.26.
        \end{sol}
  \item Let $\rl R$ be the set of compositions, of any length,
        in which each part is at most $4$.
        For $n \in \N$, let $\rl R_n$
        be the set of compositions in $\rl R$ of size $n$.
        Obtain a formula for the generating series $R(x) = \sum \abs{\rl R_n}x^n$.
        \begin{sol}
          The parts are $\rl P = \{1,2,3,4\}$ with generating series
          $P(x) = x + x^2 + x^3 + x^4 = \frac{x-x^5}{1-x}$.
          Then, by the String Lemma, we have that the generating series for $\rl R = \rl P^*$
          is $R(x) = \frac{1}{1-\frac{x-x^5}{1-x}} = \frac{1-x}{1-2x+x^5}$
        \end{sol}
  \item Deduce that for all integers $n \geq 1$, $\abs{\rl R_n} = \abs{\rl Q_n} - \abs{\rl Q_{n-1}}$.
        \begin{prf}
          By (a) and (b), we have that:
          \begin{align*}
            \abs{\rl Q_n} - \abs{\rl Q_{n-1}}
             & = [x^n]\frac{1}{1-2x+x^5} - [x^{n-1}]\frac{1}{1-2x+x^5} \\
             & = [x^n]\frac{1}{1-2x+x^5} - [x^n]\frac{x}{1-2x+x^5}     \\
             & = [x^n]\frac{1-x}{1-2x+x^5}                             \\
             & = \abs{\rl R_n}
          \end{align*}
          as desired.
        \end{prf}
  \item Part (c) implies that for every integer $n \geq 1$,
        there exists a bijection $\rl R_n \bijects \rl Q_n \cup \rl Q_{n-1}$.
        Can you determine such a bijection precisely?
\end{enumerate}

\begin{xca}
  Let $\rl V$ be the set of binary strings that do not contain 0110 as a substring.
  Find the generating series for $\rl V$.
\end{xca}
\begin{sol}
  Notice that 0110 can only overlap itself as $(0110)(110)$,
  so we consider the set of suffixes $\{110\}$
  Therefore, if we let $C(x) = x^3$, we have by Theorem 3.26 that
  $\Phi_{\rl V}(x) = \frac{1+x^3}{(1-2x)(1+x^3)+x^4} = \frac{1+x^3}{1-2x+x^3-x^4}$.
\end{sol}

\begin{xca}\end{xca}
\begin{enumerate}
  \item Let $\rl W$ be the set of binary strings that do not contain 0101 as a substring.
        Obtain a formula for the generating series (by length) of $\rl W$.
        \begin{sol}
          Observe that 0101 only overlaps itself as $(0101)(01)$.
          The set of suffixes to consider is $\{01\}$
          with generating series $C(x) = x^2$.
          Therefore, by Theorem 3.26,
          $\Phi_{\rl W}(x) = \frac{1+x^2}{(1-2x)(1+x^2)+x^4} = \frac{1+x^2}{1-2x+x^2-2x^3+x^4}$.
        \end{sol}
  \item Fix a positive integer $r \geq 1$ and consider the binary string $(01)^r$.
        Obtain a formula for the generating series
        of the set of binary strings that do not contain $(01)^r$.
        \begin{sol}
          The set of suffixes to consider will be $\{(01)^1,(01)^2,\dotsc,(01)^{r-1}\}$.
          This has generating series $C(x) = x^2 + x^4 + \dotsb + x^{2(r-1)}
            = x^2(1+x^2+\dotsb+x^{2(r-2)}) = \frac{x^2-x^{2r}}{1-x^2}$.
          Notice that $1+C(x) = \frac{1-x^{2r}}{1-x^2}$.
          Now, by Theorem 3.26,
          \begin{align*}
            \Phi(x) & = \frac{\frac{1-x^{2r}}{1-x^2}}{(1-2x)(\frac{1-x^{2r}}{1-x^2})+x^{2r}} \\
                    & = \frac{1-x^{2r}}{(1-2x)(1-x^{2r})+x^{2r}(1-x^2)}                      \\
                    & = \frac{1-x^{2r}}{1-2x+2x^{2r+1}-x^{2r+2}}
          \end{align*}
          as desired.
        \end{sol}
\end{enumerate}

\begin{xca}
  Let $\rx S = \rx A^* \rx B$ be an unambiguous prefix decomposition
  producing some set of strings $\rl S \subseteq \{0,1\}^*$.
  Show that the recursion $\rx{R = B \smile A R}$ defines an expression $\rx R$
  that produces the same set of strings $\rl S \subseteq \{0,1\}^*$.
  Also check that both $\rx S$ and $\rx R$ lead to the rational function $B(x)/(1 - A(x))$
\end{xca}
\begin{prf}
  Let $\rl S_n = \rl A^n \rl B$.
  By definition, $\rl S = \bigcup_{n\geq 0} \rl S_n$
  and because of unambiguity, this is a disjoint union.

  Now, consider the set produced by $\rx R$.
  Produce a disjoint union based on the number of times $m$ we choose to recurse
  (i.e., choose $\rx{AR}$) before terminating (i.e., choosing $\rx{B}$).
  Then, $\rl R_m = \rl A^m \rl B = \rl S_m$.
  Therefore, the set produced is $\bigcup_{n\geq 0} \rl R_n = \bigcup_{n\geq 0} \rl S_n = \rl S$.

  For the generating series, notice that we can write $S(x) = \frac{1}{1-A(x)}\cdot B(x)
    = \frac{B(x)}{1-A(x)}$ and $R(x) = B(x) + A(x)R(x)
    \implies R(x)(1-A(x)) = B(x) \implies R(x) = \frac{B(x)}{1-A(x)}$
  as desired.
\end{prf}
