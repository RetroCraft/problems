\section{Basic Principles}

\begin{xca}
  Fix integers $n \geq 0$ and $t \geq 1$.
  Consider a randomly chosen multiset of size $n$ with elements of $t$ types.
  For each part below, calculate the probability that the multiset has the stated property,
  and give a brief explanation.
\end{xca}
\begin{enumerate}
  \item Every type of element occurs at most once.
        \begin{sol}
          Every element either appears or not.
          That is, we have $2^t/\binom{n+t-1}{t-1}$
        \end{sol}
  \item Every type of element occurs at least once.
        \begin{sol}
          This is equivalent to including every element,
          then creating a multiset of the remaining $n-t$ spots.
          That gives $\binom{n-1}{t-1}/\binom{n+t-1}{t-1}$.
        \end{sol}
  \item Every type of element occurs an even number of times.
        \begin{sol}
          Make a multiset of size $\frac{n}{2}$ and then double every item:
          $\binom{\frac{n}{2}+t-1}{t-1}/\binom{n+t-1}{t-1}$.
          Note that 0 is even, so we don't require all types appearing.
        \end{sol}
  \item Every type of element occurs an odd number of times.
        \begin{sol}
          Include every element, then create an even multiset
          with the remaining $n-t$ spots:
          $\binom{\frac{n-t}{2}+t-1}{t-1}/\binom{n+t-1}{t-1}$.
        \end{sol}
  \item For $k \in \N$, exactly $k$ types of element occur
        with multiplicity at least one.
        \begin{sol}
          Pick $k$ types, then a multiset of size $n-k$:
          $\binom{t}{k}\binom{n-k+t-1}{t-1}/\binom{n+t-1}{t-1}$.
        \end{sol}
  \item For $k \in \N$, exactly $k$ types of element occur
        with multiplicity at least two.
        \begin{sol}
          Pick $k$ types, then a multiset of size $n-2k$:
          $\binom{t}{k}\binom{n-2k+t-1}{t-1}/\binom{n+t-1}{t-1}$.
        \end{sol}
\end{enumerate}

\begin{xca}
  Consider rolling six fair 6-sided dice, which are distinguishable,
  so that there are $6^6 = 46656$ equally likely outcomes.
  Count how many outcomes are of each of the following types:
\end{xca}
\begin{enumerate}
  \item Six-of-a-kind.
        \boxed{6}
  \item Five-of-a-kind and a single.
        \boxed{(6 \cdot 5) \cdot \binom{6}{1} = 180}
  \item Four-of-a-kind and a pair.
        \boxed{(6 \cdot 5) \cdot \binom{6}{4} = 450}
  \item Four-of-a-kind and two singles.
        \boxed{(6 \cdot 5 \cdot 4) \cdot \frac{\binom{6}{4}\binom{2}{1}}{2} = 1800}
  \item Two triples.
        \boxed{(6 \cdot 5) \cdot \frac{\binom{6}{3}}{2} = 300}
  \item A triple, a pair, and a single.
        \boxed{(6 \cdot 5 \cdot 4) \cdot \binom{6}{3}\binom{3}{2} = 7200}
  \item A triple and three singles.
        \boxed{(6 \cdot 5 \cdot 4 \cdot 3) \cdot \frac{\binom{6}{3}\binom{3}{1}\binom{2}{1}}{3\cdot2} = 7200}
  \item Three pairs.
        \boxed{(6 \cdot 5 \cdot 4) \cdot \frac{\binom{6}{2}\binom{4}{2}}{3\cdot2} = 1800}
  \item Two pairs and two singles.
        \boxed{(6 \cdot 5 \cdot 4 \cdot 3) \cdot \frac{\binom{6}{2}\binom{4}{2}}{2\cdot2} = 16200}
  \item One pair and four singles.
        \boxed{(6 \cdot 5 \cdot 4 \cdot 3 \cdot 2)\cdot\binom{6}{2} = 10800}
  \item Six singles. \boxed{6! = 720}
\end{enumerate}

\begin{xca}
  Let $m \geq 1$, $d \geq 2$, and $k \geq 0$ be integers.
  When rolling $m$ fair dice, each of which has $d$ sides,
  what is the probability of rolling exactly $k$ pairs and $m - 2k$ singles
\end{xca}
\begin{sol}
  There are $k+(m-2k) = m-k$ distinct sides in the roll.
  There are $\frac{d!}{(d-(m-k))!}$ ways to pick those sides.

  Then, there are $\binom{m}{2}\binom{m-2}{2}\cdots\binom{m-2(k-1)}{2} = \prod_{i=0}^{k-1} \binom{m-2i}{2}$
  ways to choose the locations of the pairs.
  Since some of these are duplicates, we divide out $k!$ permutations of pairs.

  Finally, this gives us $\frac{d!}{(d-m+k)!k!}\prod_{i=0}^{k-1}\binom{m-2i}{2}$.
\end{sol}

\begin{xca}\end{xca}
\begin{enumerate}
  \item Prove that $\bijects$ is an equivalence relation.
        \begin{prf}
          We must show reflexivity, symmetry, and transitivity.
          Let $\rl A$, $\rl B$, and $\rl C$ be sets.

          Notice that $\id : \rl A \to \rl A : a \mapsto a$
          is both surjective (there always exists $a$ such that $\id(a) = a$, namely $a$)
          and injective ($\id(a) = \id(b) \implies a = b$).
          Therefore, $\rl A \bijects \rl A$.

          Suppose $\rl A \bijects \rl B$.
          Then, there exists a bijection $f : \rl A \to \rl B$.
          Because $f$ is surjective, a preimage under $f$ exists for all $b \in \rl B$.
          Since $f$ is injective, the preimage of $b$ under $f$ is a single element $a$.

          Define $g : \rl B \to \rl A$ using the unique preimages of $f$.
          This is surjective (for all $a$, there exists $b = f(a)$ such that $g(b) = a$)
          and injective ($g(b) = g(b') \implies f(g(b)) = f(g(b')) \implies b = b'$).
          Then, $g$ is a bijection and $\rl B \bijects \rl A$.

          Suppose $\rl A \bijects \rl B \bijects \rl C$.
          Then, there exist bijections $f : \rl A \to \rl B$ and $g : \rl B \to \rl C$.
          Define $h = g \circ f : \rl A \to \rl C$.
          Then, for all $c \in \rl C$, there exists $a \in \rl A$
          such that $h(a) = g(f(a)) = c$ because of the surjectivity of $g$ and $f$.
          Also, $h(a) = h(a') \implies g(f(a)) = g(f(a')) \implies f(a) = f(a') \implies a = a'$
          by the injectivity of $g$ and $f$, so $h$ is injective.
          Therefore, $h$ is bijective and $\rl A \bijects \rl C$.

          It follows that $\bijects$ is an equivalence relation.
        \end{prf}
  \item Prove Proposition 1.11.
        \begin{prf}
          Suppose $g(f(a)) = a$ and $f(g(b)) = b$ for all $a \in \rl A$ and $b \in \rl B$.

          \WLOG, consider $f : \rl A \to \rl B$.

          Surjectivity: Let $b \in \rl B$.
          Since $f(g(b)) = b$ and $g(b)$ exists, $f$ is surjective.

          Injectivity: Let $a, a' \in \rl A$ and suppose $f(a) = f(a')$.
          Then, $g(f(a)) = g(f(a'))$ which means $a = a'$ by supposition.

          Therefore, $f$ is bijective and likewise for $g$.

          Now, suppose $f(a) = b$. Then, $g(f(a)) = a = g(b)$.
          Likewise if $g(b) = a$, then $f(g(b)) = b = f(a)$.
          Therefore, $f(a) = b \iff g(b) = a$, as desired.
        \end{prf}
\end{enumerate}

\begin{xca}
  Define $f : \Z \to \N$ as follows: for $a \in \Z$,
  $f(a) = \begin{cases}
      2a    & a \geq 0 \\
      -1-2a & a < 0
    \end{cases}$

  Show that $f$ is a bijection by Proposition 1.11.
\end{xca}
\begin{prf}
  We define the function $g : \N \to \Z : b \mapsto \begin{cases}
      \frac{b}{2}    & b \bmod 2 = 0 \\
      -\frac{b+1}{2} & b \bmod 2 = 1
    \end{cases}$

  Then, consider $g(f(a))$.
  If $a \geq 0$, then $g(f(a)) = g(2a) = \frac{2a}{2} = a$ since $2a$ is even.
  Otherwise, $g(f(a)) = g(-(1+2a)) = -\frac{-(2a+1)+1}{2} = a$ since $-(2a+1)$ is odd.

  Now, consider $f(g(b))$.
  If $b = 2k$ is even, then $f(g(2k)) = f(k) = 2k = b$ since $k \geq 0$ (because $b \in \N$).
  Likewise, if $b = 2k+1$ is odd, then $f(g(2k+1)) = f(-(k+1)) = -1-2(-(k+1)) = 2k+2-1 = 2k+1 = b$.

  Therefore, by Proposition 1.11, $f$ is a bijection.
\end{prf}

\begin{xca}
  Complete Example 1.13.
\end{xca}
\begin{prf}
  We must show that $f$ is a bijection from
  the set of subsets of $[n+t-1]$ of size $t-1$
  to set of all multisets of size $n$ with $t$ types $\rl M(n,t)$:
  \begin{quote}
    Given a subset $S \subseteq [n+t-1]$,
    sort it in increasing order $s_1 < \dotsb < s_{t-1}$.
    Then, define $m_i = s_i - s_{i-1} - 1$.
    For convenience, define $s_0 = 0$ and $s_t = n+t$.

    Define $f(S) = (m_1,\dotsc,m_t)$.
  \end{quote}
  The inverse is also given:
  \begin{quote}
    Given a multiset $\mu = (m_1,\dotsc,m_t)$ of size $n$,
    let $s_i = m_1 + \dotsb + m_i + i$.

    Define $f^{-1}(\mu) = \{s_1,\dotsc,s_{t-1}\}$.
  \end{quote}
  Consider $f(f^{-1}(\mu)) = f(f^{-1}(m_1,\dotsc,m_t)) = f(\{s_1,\dotsc,s_{t-1}\})$.
  Notice that the $s_i$ are already in increasing order
  because we add on the $m_i$ terms. Then,
  \begin{align*}
    m'_i & = s_i - s_{i-1} - 1                                                 \\
         & = \qty(\sum_{j=1}^i m_j + i) - \qty(\sum_{j=1}^{i-1} m_j + i-1) - 1 \\
         & = m_i + \sum_{j=1}^{i-1}(m_j - m_j) + (i - i + 1 - 1)               \\
         & = m_i
  \end{align*}
  and we have that $f(f^{-1}(\mu)) = (m'_1,\dotsc,m'_t) = (m_1,\dotsc,m_t) = \mu$.

  Conversely, consider $f^{-1}(f(S)) = f^{-1}(f(\{s_1,\dotsc,s_{t-1}\})) = f^{-1}(m_1,\dotsc,m_t)$.
  \WLOG, suppose that the $s_i$ are already in increasing order. Then,
  \begin{align*}
    s'_i & = i + \sum_{j=1}^i m_j                            \\
         & = \sum_{j=1}^i (m_j + 1)                          \\
         & = \sum_{j=1}^i ((s_j - s_{j-1} - 1) + 1)          \\
         & = (s_1 + \dotsb + s_i) - (s_0 + \dotsb + s_{i-1}) \\
         & = s_i
  \end{align*}
  and it follows that $f^{-1}(f(S)) = \{s'_1,\dotsc,s'_{t-1}\} = \{s_1,\dotsc,s_{t-1}\} = S$.

  Therefore, by Proposition 1.11, $f$ is a bijection, completing Example 1.13.
\end{prf}

\begin{xca}\label{xca:1-7}
  Give bijective proofs of the following identities:
\end{xca}
\begin{enumerate}
  \item For all $n \in \N$, $\sum_{k=0}^n \binom{n}{k}k = n2^{n-1}$
        \begin{prf}
          Consider the set $\rl S$ of subsets of $[n]$ with one ``highlighted'' element.
          For example, $\{1,2,3,\underline{4},10,12\} \subseteq [12]$.

          We can construct $\rl S = \bigcup \rl S_k$
          where $\rl S_k := \{S \in \rl S : \abs{S} = k\}$.
          To construct an element $S$ of $\rl S_k$,
          create a subset of size $k$, of which there are $\binom{n}{k}$,
          then select one of those $k$ elements to highlight.
          This gives $\abs{\rl S_k} = \binom{n}{k}k$.
          As a disjoint union, $\abs{\rl S} = \sum \binom{n}{k}k$.

          Alternatively, construct $S \in \rl S$ directly.
          Pick a single element from $[n]$ to highlight,
          of which there are $n$.
          Then, fill out the rest of the subset using the remaining $n-1$ items,
          of which there are $2^{n-1}$.
          That is, $\abs{\rl S} = n2^{n-1}$.

          Therefore, under the identity bijection,
          $\sum \binom{n}{k}k = \abs{\rl S} = n2^{n-1}$, as desired.
        \end{prf}
  \item For all $n \in \N$, $\sum_{k=0}^n \binom{n}{k}k(k-1) = n(n-1)2^{n-2}$
        \begin{prf}
          Proceed analogously to part (a), but with two highlighted entries in the subset,
          e.g., $\{1,\underline{2},\underline{\overline{4}},6\} \subseteq [10]$.
          Let this set of subsets be $\rl T$.

          As in (a), consider elements of $\rl T_k$.
          We select \emph{two} elements to underline from the $k$ elements in the subset,
          giving us $\binom{n}{k}k(k-1)$.
          Then, as a disjoint union, $\abs{\rl T} = \sum \abs{\rl T_k} = \sum \binom{n}{k}k(k-1)$.

          Again, considering an element of $\rl T$ directly,
          pick two elements to highlight from $n$ and the $n-1$ elements remaining,
          then of the remaining $n-2$ elements construct a subset.
          This gives $\abs{\rl T} = n(n-1)2^{n-2}$.

          Therefore, $\sum \binom{n}{k}k(k-1) = n(n-1)2^{n-2}$, as desired.
        \end{prf}
\end{enumerate}

\begin{xca}
  For an integer $n \geq 1$, give a bijective proof that
  $\sum_{2 \mid n} \binom{n}{k} = \sum_{2 \nmid n} \binom{n}{k}$.
\end{xca}
\begin{prf}
  We must establish a bijection between
  the set of even subsets $\rl E = \{ S \subseteq [n] : 2 \mid \abs{S} \}$
  and the set of odd subsets $\rl O = \{ S \subseteq [n] : 2 \nmid \abs{S} \}$.

  Define $f : S_n \to S_n : f(S) = \begin{cases}
      S \cup \{1\}      & 1 \not\in S \\
      S \setminus \{1\} & 1 \in S
    \end{cases}$

  Let $f_{\rl E}$ and $f_{\rl O}$ be $f$ restricted to the respective set.

  Notice that $f(S)$ always either increases or decreases the size of a set by 1,
  meaning that it will send sets in $\rl E$ to $\rl O$ and vice versa.

  Also, it is obvious that $f(f(S)) = S \cup \{1\} \setminus \{1\}$ or $S \setminus \{1\} \cup \{1\} = S$,
  so $f$ is its own inverse.

  It follows by Proposition 1.11 that $\rl E \bijects \rl O$, as desired.
\end{prf}

\begin{xca}
  Let $n$ be a positive integer.
  Let $\rl S_n$ be the set of ordered pairs of subsets $(A,B)$
  in which $A \subseteq B \subseteq [n]$.
  Let $\rl T_n$ be the set of all functions $f : [n] \to [3]$.
\end{xca}
\begin{enumerate}
  \item What is $\abs{\rl T_n}$?
        \begin{sol}
          Set-theoretically, a function $f : [n] \to [3]$
          is a set of ordered pairs for each value
          $\{(1,f(1)),(2,f(2)),\dotsc,(n,f(n))\}$.
          We pick $n$ values here for $f(1),\dotsc,f(n) \in [3]$.
          That is, $3^n$ choices.
          Therefore, $\abs{\rl T_n} = 3^n$.
        \end{sol}
  \item Define a bijection $g : \rl S_n \to \rl T_n$.
        Explain why $g((A,B)) \in \rl T_n$ for any $(A,B) \in \rl S_n$.
        \begin{sol}
          Given $(A,B) \in \rl S_n$, every element $i \in [n]$ is either:
          (1) not in $A$ or $B$,
          (2) in $B$ but not in $A$, or
          (3) in $A$ (and $B$ since $A \subseteq B$).

          Let $f(i)$ be the number of the case listed above.
          This is a function $[n] \to [3]$, so $f \in \rl T_n$.
        \end{sol}
  \item Define the inverse function $g^{-1} : \rl T_n \to \rl S_n$
        of the bijection $g$ from part (b).
        \begin{sol}
          Construct $A$ and $B$ from $f \in \rl T_n$:

          Read the case list in (b) in reverse. For all $i \in [n]$:
          if $f(i) = 2$, place $i \in B$; if $f(i) = 3$, place $i \in A$.

          Finally, place all elements of $A$ in $B$.
          Then, we have $(A,B) \in \rl S_n$.
        \end{sol}
\end{enumerate}

\begin{xca}
  Fix integers $n \geq 0$ and $k \geq 1$.
  Let $\rl A(n,k)$ be the set of sequences $(a_i) \in \N^k$
  such that $\sum a_i = n$ and $j \mid a_j$ for all $j$.

  Let $\rl B(n,k)$ be the set of sequences $(b_i) \in \N^k$
  such that $\sum b_i = n$ and $b_1 \geq b_2 \geq \dotsb \geq b_k$.

  Construct a pair of mutually inverse bijections
  between the sets $\rl A(n,k)$ and $\rl B(n,k)$.
\end{xca}
\begin{sol}
  \newcommand{\one}{\mathbb{1}}
  Fix $n$ and $k$ and imply the parameters on $\rl A$ and $\rl B$.
  We will treat the sequences as vectors, i.e., $\vb a = (a_1,\dotsc,a_k)$.

  Let $f : \rl A \to \rl B$.
  We will take the sum of vectors of ones.
  Let $\one_i = (\underbrace{1,\dotsc,1}_{\text{$i$ ones}}, \underbrace{0,\dotsc,0}_{\text{$k-i$ zeroes}})$.

  Then, define $f(\vb a) = \sum_{i=1}^k \frac{a_i}{i}\one_i$.
  For example, if $n=7$ and $k=3$,
  \[ f((2,2,3)) = 2(1,0,0) + \frac{2}{2}(1,1,0) + \frac{3}{3}(1,1,1) = (4,2,1) \]
  Notice that since the $\one_i$ are non-increasing for all $i$,
  their linear combination with positive coefficients is also non-increasing.
  Also, we are ``distributing'' the multiples of $i$ into $i$ ones,
  meaning that the sum $\sum a_i = n$ does not change.

  That is, for all $\vb a \in \rl A$, $f(\vb a) = \vb b$ for some $\vb b \in \rl B$.

  We can define an inverse $f^{-1}(\vb b)$ by starting at $i=k$
  and recursively taking out the largest muliple of $i$ from all $k$ entries.

  For the above example, start with $(4,2,1)$ and take out 1 from all 3 entries.
  This sets $a_3 = 1$ and gives $(3,1,0)$.
  Then, take out 1 from the first 2 entries, setting $a_2 = 1$ and giving $(2,0,0)$.
  Finally, take out 2 from the first 1 entry, setting $a_1 = 2$.
  This gives $\vb a = (2,1,1)$, as expected.

  Formally, we define $\vb a = f^{-1}(\vb b)$ as follows:
  \begin{align*}
    a_i'  & = \min\{b_1 - \sum_{j > i} a_i', \dotsc, b_i - \sum_{j > i} a_i'\} \\
    \vb a & = (a_1', 2a_2', \dotsc, ka_k')
  \end{align*}
  which follows the process described above.

  Then, since the processes are inverses, $f$ is a bijection and $\rl A \bijects \rl B$.
\end{sol}

\begin{xca}
  For $n \geq 0$ and $t \geq 2$, prove bijectively that
  $\binom{n+t-1}{t-1} = \sum_{k=0}^n\binom{n-k+t-2}{t-2}$.
\end{xca}
\begin{prf}
  The left-hand side counts the set $\rl S$ of multisets of size $n$ and $t$ types.

  Since there are at least 2 types, partition $\rl S$
  according to the number of times that 1 appears in the multiset.
  Let $\rl S_k = \{ S \in \rl S : \abs{\{1 \in S\}} = k \}$.

  Then, we can ignore the 1's.
  This means to create an element of $\rl S_k$,
  we must create a multiset of size $n-k$ with $t-1$ types
  and then add $k$ 1's.
  This gives us $\abs{\rl S_k} = \binom{(n-k)+(t-1)-1}{(t-1)-1} = \binom{n-k+t-2}{t-2}$.

  Finally, since the number of 1's in a multiset is unique,
  this is a disjoint union and $\abs{\rl S} = \sum \abs{\rl S_k} = \sum \binom{n-k+t-2}{t-2}$.

  Therefore, $\binom{n+t-1}{t-1} = \abs{\rl S} = \sum\binom{n-k+t-2}{t-2}$, as desired.
\end{prf}

\begin{xca}
  For $n \geq 1$ and $t \geq 1$, prove bijectively that
  $\binom{n+t-1}{t-1} = \sum_{k=0}^t\binom{t}{k}\binom{n-1}{k-1}$.
\end{xca}
\begin{prf}
  Again, the LHS counts the set $\rl S$ of multisets of size $n$ and $t$ types.

  Notice that a multiset need not use all $t$ types.
  Consider the set $\rl S_k$ of multisets of size $n$ which use $k \leq t$ types.
  This set will have at least one of each of the $k$ types
  and the remainder is a multiset of $k$ types and size $n-k$.
  That is, there are $\binom{n-k+k-1}{k-1} = \binom{n-1}{k-1}$ of these.
  We also had to pick the $\binom{t}{k}$ types.
  Therefore, $\abs{\rl S_k} = \binom{t}{k}\binom{n-1}{k-1}$.

  Since the number of types used by a multiset is unique,
  this is a disjoint union and $\abs{\rl S} = \sum \abs{\rl S_k} = \sum \binom{t}{k}\binom{n-1}{k-1}$.

  Therefore, $\binom{n+t-1}{t-1} = \abs{\rl S} = \sum\binom{t}{k}\binom{n-1}{k-1}$.
\end{prf}

\begin{xca}
  Choose a permutation $\sigma$ of $\{1, 2, \dotsc, 7\}$ at random,
  so that each of the $7! = 5040$ permutations are equally likely.
  What are the probabilities of the following events?
\end{xca}
\begin{enumerate}[1.]
  \item Numbers 1 and 2 are consecutive
        \begin{sol}
          Let $\sigma_i$ be the index of $i$ in $\sigma$.
          That is, $\sigma_1 = 2$ means 1 is in position 2.

          Consider when 12 appears in the permutation.
          There are 6 choices to place $\sigma_1 = 1,\dotsc,6$
          so that $\sigma_2 = 2,\dotsc,7$.
          There are also 6 choices to place 21.
          Fill the remaining spots with $5!$.
          Therefore, the probability is $\frac{(6+6)5!}{5040} = \frac{2}{7}$.
        \end{sol}
  \item Number 1 is to the left of 2
        \begin{sol}
          There are 7 choices for $\sigma_2$.
          Then, there are $\sigma_2-1$ choices for $\sigma_1$.
          That is, there are $\sum_{\sigma_2=1}^7 (\sigma_2 - 1) = \frac{7\cdot 8}{2} - 7 = 21$
          of these permutations.
          Fill the remaining 5 spots with $5!$.
          Therefore, the probability is $\frac{21\cdot 5!}{5040} = \frac{1}{2}$.
        \end{sol}
  \item No two odd numbers are consecutive
        \begin{sol}
          There are four odd numbers and three evens.
          This means the only way to separate them is to write OEOEOEO.
          We can permute the odd numbers in $4!$ ways and evens in $3!$ ways.
          This gives a probability $\frac{4!\cdot 3!}{5040} = \frac{2}{70}$.
        \end{sol}
\end{enumerate}

\begin{xca}
  Let $r \geq 2$ and $s \geq 2$ be integers.
  Consider a (non-standard) deck of $rs$ cards,
  divided into $s$ suits each with cards of $r$ different values.
  The cards in each suit are numbered $A, 2, 3, \dotsc, r$,
  and $A$ can be either below $2$ or above $r$.
  Choose five cards from such a deck in one of $\binom{rs}{5}$ ways.
  How many ways are there to produce each kind of hand for this
  ``poker in an alternate universe''?
\end{xca}
\begin{enumerate}
  \item Count ``quints'' (five-of-a-kinds).
        \boxed{r}
  \item Count straight flushes.
        \boxed{(r-4)s}
  \item Count quads.
        \boxed{r\binom{s}{4}\cdot\binom{(r-1)s}{1}}
  \item Count full houses.
        \boxed{r\binom{s}{3}\cdot(r-1)\binom{s}{2}}
  \item Count flushes.
        \boxed{s(r! - (r-4))}
  \item Count straights.
        \boxed{(r-4)(5^s - 5)}
  \item Count trips.
        \boxed{r\binom{s}{3}\cdot\qty(s^2\binom{r-1}{2})}
  \item Count two-pairs.
        \boxed{\binom{r}{2}\binom{s}{2}\binom{s}{2}\cdot(s(r-2))}
  \item Count one-pairs.
        \boxed{r\binom{s}{2}\cdot\binom{(r-1)s}{3}}
  \item Count busted hands.
        \boxed{\qty(\binom{r}{5}-(r-4))\cdot(s^5-s)}
\end{enumerate}

\begin{xca}
  The game called ``Crowns and Anchors'' or ``Birdcage''
  was popular on circus midways early in the 20th century.
  It is a game between a Player and the House, played as follows.
  First, the Player wagers $w$ dollars on an integer $p$ from one to six.
  Next, the House rolls three six-sided dice.
  For every die that shows $p$ dots on top, the House pays the Player $w$ dollars,
  but if no dice show $p$ dots on top then the Player's wager is forfeited,
  and goes to the House. (Assume that the dice are fair, so that every outcome is equally likely.)

  For example, if I wager two dollars on the number five,
  and the dice show five, five, and three dots, respectively,
  then the House pays me four dollars for a total of six (a profit of four dollars).
  However, if in this case the dice show four, three, and two dots, respectively,
  then the House takes my wager for a total of zero (a loss of two dollars).
\end{xca}
\begin{enumerate}
  \item For every dollar that the Player wagers,
        how much money should the Player expect to win back in the long run?
        Would you play this game?
        \begin{sol}
          Consider the expected value for each dollar the Player wagers on $k$:

          There are $6^3 = 216$ total outcomes.
          Of these, there is 1 where $k$ appears 3 times,
          5 where $k$ appears twice,
          and $5^2 = 25$ where $k$ appears once.

          This gives an expected value of $\frac{1}{216}(3k + 5(2k) + 25k) = \frac{38}{216}k$.

          The Player will want to maximize payout and always pick $k=6$.

          This gives an expected payout of $\frac{33\cdot6}{216} \approx \$1.06$.
          This is more than the \$1 wager, so the game is worth playing.
        \end{sol}
  \item In a parallel universe there is a game of Crowns and Anchors
        being played with $m \geq 1$ dice, each of which has $d \geq 2$ sides.
        (Assume that the dice are fair, so that every outcome is equally likely.)
        In which universes does the Player win in the long run?
        In which universes does the House win in the long run?
        In which universes is the game completely fair?
        \begin{sol}
          The Player, as above, will always place a dollar on the highest number $d$.

          Then, there are $d^m$ total outcomes.
          For each possible payout $1 \leq i\cdot d \leq m\cdot d$,
          there are $i$ occurrences of $d$
          and there are $(d-1)^{m-i}$ ways to pick the remaining $m-i$ dice.
          As above, we sum to calculate the expected value.

          This gives $\frac{1}{d^m}\sum_{i=1}^m i(d-1)^{m-i}$.
        \end{sol}
\end{enumerate}
