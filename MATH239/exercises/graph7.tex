\section{Planar Graphs}

\setcounter{subsection}{2}
\subsection{Planarity, Euler's Formula, Stereographic Projection}

\begin{xca}
  Prove that every planar embedding has either a vertex of degree at most 3
  or a face of degree 3.
\end{xca}
\begin{prf}
  Let $G$ be a component of a planar graph.

  If $G$ has no faces (besides the outer face),
  then $G$ is a tree with at least two leaves (by Theorem 5.1.8)
  and those leaves have degree $1 \leq 3$.

  Otherwise, let $V = \abs{V(G)}$,
  $E = \abs{E(G)}$, and $F = \abs{F(G)}$.
  Then, by Euler's Formula, $V - E + F = 2$.

  Suppose that all vertices have degrees at least 4
  and all faces have degrees at least 4.
  Then, by the Handshaking Lemma, $2E = \sum \deg(v) \geq 4V$ so $E \geq 2V$.
  Likewise, by the Faceshaking Lemma, $2E = \sum \deg(f) \geq 4F$ so $E \geq 2F$.
  That is, $2E \geq 2(V + F)$, i.e., $E \geq V + F$.
  But then $2 = (V + F) - E < 0$, a contradiction.

  Therefore, there either exists a vertex with degree at most 3
  or a face of degree 3.\footnote{Note that inner faces have degree at least 3
    and smaller degree implies a tree.}
\end{prf}

\begin{xca}
  Prove that each of the graphs shown in Figure 7.6 is planar,
  by exhibiting a planar embedding.
\end{xca}
\begin{sol}
  Draw planar embeddings:
  \begin{tikzpicture}
    \graph{subgraph C_n[V={b,c,d,i,e,f,g}, clockwise] -!- h, {b,c,i,e,f,g}--h, a[y=1.2,x=2]--{c,d}, a--[bend right]b, a--[bend left]i, f--[out=130,in=90,looseness=1.2]a};
  \end{tikzpicture}
  \quad
  \begin{tikzpicture}
    \graph{subgraph C_n[V={c,g,b,f,a,i,e,h}, clockwise], f--{e,c,g}, d[x=-2], d--{h,i}, a--[bend left]d, c--[bend right]d, f--[out=-100,in=-90,looseness=1.3]d};
  \end{tikzpicture}
\end{sol}

\begin{xca}
  Let $n \geq 3$ be an integer.
  Suppose that a convex $n$-gon is drawn in the plane
  and then each pair of nonadjacent corner points
  is joined by a straight line through the interior.
  Suppose that no three of these lines through the interior
  meet at a common point in the interior.

  Let $f_n$ be the number of regions into which
  the interior of the $n$-gon is divided by this process.
  Use Euler's Formula to find $f_n$.
\end{xca}
\begin{sol}
  Let $v_n$, $e_n$, and $f_n$ be the number of vertices, edges, and faces as described.
  Picking four corners of the $n$-gon uniquely determines
  a pair of intersecting lines (i.e., an added vertex),
  so we have in total $v_n = n + \binom{n}{4}$ vertices.

  Each vertex from the $n$-gon has degree $n-1$
  and each new vertex has degree 4 (as the intersection of two lines),
  so by the Handshaking Lemma, $2e = n(n-1) + 4\binom{n}{4}$
  or equivalently $e = \binom{n}{2} + 2\binom{n}{4}$.

  Therefore, by Euler's Formula, discounting the outer face because
  we are counting only the interior of the $n$-gon, $f_n = 1 - v_n + e_n
    = 1 - n + \binom{n}{2} + \binom{n}{4}$.\footnote{This is a \href{https://www.youtube.com/watch?v=K8P8uFahAgc}{3Blue1Brown} video.}
\end{sol}

\subsection{Platonic Solids}

Not covered in Fall 2022 offering.

\setcounter{subsection}{5}
\subsection{Nonplanar Graphs, Kuratowski's Theorem}

\setcounter{subsection}{7}
\subsection{Colouring and Planar Graphs, Dual Planar Maps}
