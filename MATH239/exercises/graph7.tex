\section{Planar Graphs}

\setcounter{subsection}{2}
\subsection{Planarity, Euler's Formula, Stereographic Projection}

\begin{xca}
  Prove that every planar embedding has either a vertex of degree at most 3
  or a face of degree 3.
\end{xca}
\begin{prf}
  Let $G$ be a component of a planar graph.

  If $G$ has no faces (besides the outer face),
  then $G$ is a tree with at least two leaves (by Theorem 5.1.8)
  and those leaves have degree $1 \leq 3$.

  Otherwise, let $V = \abs{V(G)}$,
  $E = \abs{E(G)}$, and $F = \abs{F(G)}$.
  Then, by Euler's Formula, $V - E + F = 2$.

  Suppose that all vertices have degrees at least 4
  and all faces have degrees at least 4.
  Then, by the Handshaking Lemma, $2E = \sum \deg(v) \geq 4V$ so $E \geq 2V$.
  Likewise, by the Faceshaking Lemma, $2E = \sum \deg(f) \geq 4F$ so $E \geq 2F$.
  That is, $2E \geq 2(V + F)$, i.e., $E \geq V + F$.
  But then $2 = (V + F) - E < 0$, a contradiction.

  Therefore, there either exists a vertex with degree at most 3
  or a face of degree 3.\footnote{Note that inner faces have degree at least 3
    and smaller degree implies a tree.}
\end{prf}

\begin{xca}
  Prove that each of the graphs shown in Figure 7.6 is planar,
  by exhibiting a planar embedding.
\end{xca}
\begin{sol}
  Draw planar embeddings:
  \begin{tikzpicture}
    \graph{subgraph C_n[V={b,c,d,i,e,f,g}, clockwise] -!- h, {b,c,i,e,f,g}--h, a[y=1.2,x=2]--{c,d}, a--[bend right]b, a--[bend left]i, f--[out=130,in=90,looseness=1.2]a};
  \end{tikzpicture}
  \quad
  \begin{tikzpicture}
    \graph{subgraph C_n[V={c,g,b,f,a,i,e,h}, clockwise], f--{e,c,g}, d[x=-2], d--{h,i}, a--[bend left]d, c--[bend right]d, f--[out=-100,in=-90,looseness=1.3]d};
  \end{tikzpicture}
\end{sol}

\begin{xca}
  Let $n \geq 3$ be an integer.
  Suppose that a convex $n$-gon is drawn in the plane
  and then each pair of nonadjacent corner points
  is joined by a straight line through the interior.
  Suppose that no three of these lines through the interior
  meet at a common point in the interior.

  Let $f_n$ be the number of regions into which
  the interior of the $n$-gon is divided by this process.
  Use Euler's Formula to find $f_n$.
\end{xca}
\begin{sol}
  Let $v_n$, $e_n$, and $f_n$ be the number of vertices, edges, and faces as described.
  Picking four corners of the $n$-gon uniquely determines
  a pair of intersecting lines (i.e., an added vertex),
  so we have in total $v_n = n + \binom{n}{4}$ vertices.

  Each vertex from the $n$-gon has degree $n-1$
  and each new vertex has degree 4 (as the intersection of two lines),
  so by the Handshaking Lemma, $2e = n(n-1) + 4\binom{n}{4}$
  or equivalently $e = \binom{n}{2} + 2\binom{n}{4}$.

  Therefore, by Euler's Formula, discounting the outer face because
  we are counting only the interior of the $n$-gon, $f_n = 1 - v_n + e_n
    = 1 - n + \binom{n}{2} + \binom{n}{4}$.\footnote{This is a \href{https://www.youtube.com/watch?v=K8P8uFahAgc}{3Blue1Brown} video.}
\end{sol}

\subsection{Platonic Solids}

Not covered in Fall 2022 offering.

\setcounter{subsection}{5}
\subsection{Nonplanar Graphs, Kuratowski's Theorem}

\begin{xca}
  For each of the graphs in Figure 7.13, determine if it is planar.
  Prove your conclusion in each case.
\end{xca}

\begin{xca}
  Let $G$ be a connected planar graph with $p$ vertices and $q$ edges and girth $k$.
  Show that $q \leq \frac{k(p-2)}{k-2}$ and if equality holds, all faces of $G$ have degree $k$.
\end{xca}
\begin{prf}[sooshi]
  Assume that $k < \infty$.
  By Lemma 7.5.1, the boundary walk of every face in $G$ contains a cycle.
  Therefore, the degree of each face is at least $k$.
  It follows that by Lemma 7.5.2, $q \leq \frac{k}{k-2}(p-2)$.

  Suppose $q = \frac{k(p-2)}{k-2}$.
  Then $kq - 2q = kp - k2$ and we have $2 = p - q + \frac{2}{k}q$.
  But by Euler's Formula, since $G$ is connected, $2 = p - q + \abs{F(G)}$,
  so we must have $2q = k\abs{F(G)}$.
  By the Faceshaking Lemma, $2q = \sum \deg(f) \geq k\abs{F(G)}$
  with equality if and only if $\deg(f) = k$ for all $f \in F(G)$.
  Therefore, if equality holds on the original inequality,
  every face has degree $k$.
\end{prf}

\begin{xca}
  Prove that the Peterson graph is nonplanar,
  without using any form of Kuratowski's Theorem.
\end{xca}
\begin{prf}
  Recall the Peterson graph $P$:
  \begin{center}
    \tikz\graph[clockwise] {subgraph K_n[name=inner,V={6,7,8,9,10}] -- subgraph C_n[name=outer,n=5]};
  \end{center}
  Suppose it is planar.
  Notice that $\abs{E(P)} = 25$ and $\abs{V(P)} = 10$.
  By Theorem 7.5.3, since $\abs{V(P)} \geq 3$, we have
  $25 = \abs{E(P)} \leq 3\abs{V(P)} - 6 = 24$, a contradiction.
  Therefore, $P$ is nonplanar.
\end{prf}

\begin{xca}\end{xca}
\begin{enumerate}
  \item Prove that the Peterson graph is nonplanar by Kuratowski's Theorem,
        finding a subgraph that is an edge subdivision of $K_{3,3}$.
        \begin{prf}
          Highlight the edge subdivision of $K_{3,3} = (\textcolor{red}{A},\textcolor{blue}{B})$.
          \begin{center}
            \tikz\graph[clockwise=5] {
            {[clique] 6[red], 7[blue], 8, 9, 10[blue]} -- {1[blue], 2[red], 3, 4, 5[red]};
            1--2--3--4--5--1;
            1--[ultra thick]{5,6,2};
            10--[ultra thick]{5,6,8};
            7--[ultra thick]{6,2,9};
            4--[ultra thick]{9,5};
            3--[ultra thick]{8,2};
            };
          \end{center}
          Therefore, since such an edge subdivision exists, by Kuratowski's Theorem,
          the Peterson graph is nonplanar.
        \end{prf}
  \item Show that there exist two edges of the Peterson graph whose deletion leaves a planar graph.
        \begin{prf}
          Recall from Figure 4.8 that this is an isomorphic representation of the Peterson graph:
          \begin{center}
            \tikz\graph[clockwise=9] {j[y=-1] -!- {a,b,c,d,e,f,g,h,i};
            a--b--c--d--e--f--g--h--i--a; j--{a,d,g}; b--[red]f; c--[bend right,looseness=1.5,out=-100,in=-80,dashed]h; c--h; e--[red]i};
          \end{center}
          If we delete the edges in \textcolor{red}{red},
          then we have a planar graph (bend the edge $ch$ above the rest of the graph).
        \end{prf}
\end{enumerate}

\begin{xca}
  Prove that the $n$-cube is not planar when $n \geq 4$,
  without using any form of Kuratowski's Theorem.
\end{xca}
\begin{prf}
  Recall from Problem 4.4.2 that there are $2^n$ vertices and $n2^{n-1}$
  edges in the $n$-cube for $n \geq 0$.
  Also recall from Problem 4.4.3 that the $n$-cube is bipartite.

  Suppose that the $n$-cube is planar for some $n \geq 4$.
  Then, by Theorem 7.5.6, since $2^n \geq 3$,
  we have $n2^{n-1} \leq 2(2^n) - 4 = 4(2^{n-1}) - 4$
  which implies $4 \leq 2^{n-1}(4-n)$.
  But $4-n \leq 0$ and $2^{n-1} > 0$, so we have $4 \leq 0$, a contradiction.

  Therefore, the $n$-cube is nonplanar for $n \geq 4$.
\end{prf}

\setcounter{subsection}{7}
\subsection{Colouring and Planar Graphs, Dual Planar Maps}
