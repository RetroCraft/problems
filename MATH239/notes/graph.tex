
\part{Graph Theory}

\setcounter{chapter}{3}
\chapter{Introduction to Graph Theory}

Note: In the notes, there are two chapter 4's
(Recurrence Relations and Introduction to Graph Theory).
This numbering is maintained.

\section{(10/21; from Bradley)}

\begin{defn}[graph]
  A graph $G = (V, E)$ is a finite set of \term{vertices} $V$
  and a set\tablefootnote{if we allow multisets, they are \term{multiple} or \term{parallel} edges}
  of unordered\tablefootnote{if ordered, the graph is \term{directed}}
  pairs\tablefootnote{if larger tuples, it is a \term{hypergraph}}
  of distinct\tablefootnote{if not, the edge is a \term{loop}} vertices $E$ called \term{edges}.
  Denote the set of vertices $V(G)$ and edges $E(G)$.
\end{defn}
\spewnotes

Note: if the graph includes loops and/or parallel edges, it is a \term{multigraph}.

\begin{defn}[complete graph on $n$ vertices]
  The graph $K_n$ given by $V(K_n) = [n]$ and $E(K_n) = \{S \subseteq [n] : \abs{S} = 2\}$.
\end{defn}
\begin{example}
  Draw $K_1$ through $K_6$.
\end{example}
\begin{sol}
  Connect every node in a $n$-gon:
  \begin{center}
    \tikz\graph{ subgraph K_n [n=1, clockwise] };\quad
    \tikz\graph{ subgraph K_n [n=2, clockwise] };\quad
    \tikz\graph{ subgraph K_n [n=3, clockwise] };\quad
    \tikz\graph{ subgraph K_n [n=4, clockwise] };\quad
    \tikz\graph{ subgraph K_n [n=5, clockwise] };\quad
    \tikz\graph{ subgraph K_n [n=6, clockwise] };
  \end{center}
  as desired.
\end{sol}

\begin{defn}[path on $n$ vertices]
  The graph $P_n$ given by $V(P_n) = [n]$ and $E(P_n) = \{ \{i,i+1\} : i \in [n-1] \}$.
\end{defn}

\begin{defn}[cycle on $n$ vertices]
  The graph $C_n$ for $n \geq 3$ given by $V(C_n) = [n]$ and $E(C_n) = E(P_n) \cup \{1,n\}$.
\end{defn}

\begin{defn}[bipartite]
  Given a graph $G$, there exists a partition $(A,B)$ of $V(G)$
  such that every edge has exactly one element in $A$ and one element in $B$.
\end{defn}
\begin{example}\label{exa:17-2}
  Which of the above graphs are bipartite?
\end{example}
\begin{sol}
  $K_1$ and $K_2$ are bipartite, but $K_n$ for $n \geq 3$ is note.

  In general, $P_n$ is bipartite with bipartition $A = \{i \in [n] : \text{$i$ odd}\}$
  and $B = \{i \in [n] : \text{$i$ even}\}$.

  In general, $C_n$ is bipartite when $n$ is even
  (under the same bipartition as $P_n$) and not when $n$ is odd.
\end{sol}

\begin{defn}[complete bipartite graph]
  The graph $K_{m,n}$ described by
  $V(K_{m,n}) = \{s_i : i \in [m]\} \cup \{t_j : j \in [n]\}$
  and $E(K_{m,n}) = \{\{s_i, t_j\} : i \in [n], j \in [m]\}$.
\end{defn}
\begin{example}
  Draw $K_{1,0}$, $K_{2,1}$, $K_{2,3}$, $K_{3,2}$, and $K_{3,3}$.
\end{example}
\begin{sol}
  Connect $m$ nodes with $n$ nodes:
  \begin{center}
    \tikz\graph{ s_1 };\quad
    \tikz\graph{ subgraph K_nm [V={s_1,s_2}, W={t_1}] };\quad
    \tikz\graph{ subgraph K_nm [V={s_1,s_2}, W={t_1,t_2,t_3}] };\quad
    \tikz\graph{ subgraph K_nm [V={s_1,s_2,s_3}, W={t_1,t_2}] };\quad
    \tikz\graph{ subgraph K_nm [V={s_1,s_2,s_3}, W={t_1,t_2,t_3}] };
  \end{center}
  Notice that $K_{2,3}$ and $K_{3,2}$ have the same structure.
\end{sol}

\begin{defn}[planar]
  A graph that can be drawn on the plane such that no two edges cross
\end{defn}
\begin{example}
  Which of $K_n$, $P_n$, $C_n$, and $K_{m,n}$ are planar?
\end{example}
\begin{sol}
  Obviously, $K_1$, $K_2$, and $K_3$ are planar.

  We can redraw $K_4$ like \tikz[baseline=-2pt]\graph{ subgraph C_n [n=3, clockwise] -- 4 };
  to make it planar.

  However, $K_5$ can't do that.

  $P_n$ and $C_n$ are planar for all $n$.

  For $K_{m,n}$, when $m=0$, we just have $n$ unconnected vertices, which is planar.

  When $m=1$, we have a star
  (e.g. \tikz[baseline=-1pt]\graph{ subgraph C_n [V={t_1,t_2,t_3,t_4,t_5,t_6}, clockwise, -!-] -- s_1};).

  When $m=2$, put $s_1$ on the left and $s_2$ on the right and connect
  (e.g. \tikz[baseline=-1pt]\graph{ s_1 -- {t_1[y=1], t_2[y=1], t_3[y=1]} -- s_2 };)

  When $m \geq 3$, this doesn't work. So $K_{m,n}$ is planar if $m \leq 2$ or $n \leq 2$.
\end{sol}

\section{(10/24; from Bradley)}

\begin{defn}
  Given vertices $u$, $v$ and an edge $e$ in $G$:
  \begin{itemize}[nosep]
    \item If $uv \in E(G)$, then $u$ and $v$ are \term{adjacent}
    \item If $e = uv$, then $e$ \term{joins} the \term{endpoints} $u$ and $v$
    \item If $v \in e$, then $e$ is \term{incident} with $v$
    \item If $u$ and $v$ are adjacent, $u$ is a \term{neighbour} of $v$
    \item The \term{neighbourhood of $v$} $N_G(v)$ is the set of neighbours of $v$
  \end{itemize}
\end{defn}

\begin{defn}[degree]
  The number of neighbours $\deg_G(v) = \abs{N_G(v)}$ of a vertex $v$ in a graph $G$.
\end{defn}

\begin{defn}[$k$-regularity]
  A graph where every vertex has degree exactly $k$.
\end{defn}

\begin{example}
  What are the degrees for $K_n$, $P_n$, $C_n$, and $K_{m,n}$?
\end{example}
\begin{sol}
  Every vertex in $K_n$ connects to the other $n-1$,
  so they all have degree $n-1$ and $K_n$ is $(n-1)$-regular.

  Every vertex $v \in V(C_n)$ has $\deg(v) = 2$, so $C_n$ is 2-regular.

  For $P_n$, the endpoints $\deg(v_1) = \deg(v_n) = 1$
  and the middle vertices $\deg(v_i) = 2$ for $2 \leq i \leq n-1$.
  Edge case: in $P_0$, $\deg(v_1) = 0$.

  In $K_{m,n}$, $\deg(s_i) = \abs{N(s_i)} = \abs{\{t_i\}} = n$
  and $\deg(t_i) = \abs{N(t_i)} = \abs{\{s_i\}} = m$.
\end{sol}

\begin{defn}[degree sequence]
  An ordering $d_1 \geq d_2 \geq \dotsb \geq d_n$ of degrees of vertices.
\end{defn}

\begin{theorem}[Handshaking Lemma]\label{thm:hand}
  If $G$ is a graph, then $2\abs{E(G)} = \smashoperator{\sum\limits_{v \in V(G)}} \deg_G(v)$.
\end{theorem}
\begin{prf}[Informal]
  Every edge is a ``handshake'' betwen two ``people''.
  The number of hands is twice the number of handshakes,
  since each handshake takes two hands.
\end{prf}
\begin{prf}[Formal]
  Let $S = \{(e,v) : e \in E(G), v \in V(G), \text{$v$ incident with $e$}\}$.
  Counting edges,
  \[
    \abs{S} = \smashoperator{\sum_{e \in E(G)}} \abs{\{(e,v) \in S\}}
    = \smashoperator{\sum_{v \in V(G)}} 2
    = 2 \abs{E(G)}
  \]
  since every edge has exactly two endpoints.
  Counting vertcies,
  \[
    \abs{S} = \smashoperator{\sum_{v \in V(G)}} \abs{\{(e,v) \in S\}}
    = \smashoperator{\sum_{v \in V(G)}} \deg_G(v)
  \]
  by definition of $N_G(v)$.
  Therefore, $2\abs{E(G)} = \abs{S} = \smashoperator{\sum\limits_{v \in V(G)}} \deg_G(v)$.
\end{prf}

\begin{corollary}[4.3.2]
  The number of vertices of odd degree in a graph is even.
\end{corollary}
\begin{prf}
  Let $\mathcal O(G)$ be the set of odd degree vertices.
  Let $\mathcal E(G)$ be the set of even degree vertices.
  By the \nameref{thm:hand},
  \[
    2\abs{E(G)} = \smashoperator{\sum_{v \in V(G)}} \deg_G(v)
    = \smashoperator{\sum_{v \in \mathcal O(G)}} \deg_G(v) + \smashoperator{\sum_{v \in \mathcal E(G)}} \deg_G(v)
  \]
  Take the congruence modulo 2:
  \[
    0 \equiv \smashoperator{\sum_{v \in \mathcal O(G)}} 1 + \smashoperator{\sum_{v \in \mathcal E(G)}} 0
    \equiv \abs{\mathcal O(G)} + 0 \pmod{2}
  \]
  which means that $\abs{\mathcal O(G)}$ is even, as desired.
\end{prf}

\begin{corollary}[4.3.3]
  The average degree for a graph $G$ is $\frac{2\abs{E(G)}}{\abs{V(G)}}$.
\end{corollary}
\begin{prf}
  Average degree is $\frac{\sum_{v\in V(G)} \deg_G(v)}{\abs{V(G)}} = \frac{2\abs{E(G)}}{\abs{V(G)}}$.
\end{prf}

\begin{corollary}
  If $G$ is a $k$-regular graph, then $\abs{E(G)} = \frac{k}{2}\abs{V(G)}$.
\end{corollary}
\begin{prf}
  By the \nameref{thm:hand}, $2\abs{E(G)} = \smashoperator{\sum\limits_{v \in V(G)}} \deg_G(v)
    = \smashoperator{\sum\limits_{v \in V(G)}} k = k\abs{V(G)}$.
\end{prf}

\begin{defn}[isomorphism]
  Graphs $G$ and $H$ are isomorphic if there exists a \term{(graph) isomorphism}
  between them, i.e., a bijection $f : V(G) \to V(H)$ such that
  $uv \in E(G)$ if and only if $f(u)f(v) \in E(H)$.
\end{defn}

\begin{example}
  $K_1$ is isomorphic to $P_1$, $K_2$ to $P_2$, and $K_3$ to $C_3$.
\end{example}

\begin{example}
  The following are isomorphic representations of the Peterson graph:
  \begin{center}
    \tikz\graph[clockwise] {subgraph K_n[name=inner,V={6,7,8,9,10}] -- subgraph C_n[name=outer,n=5]};
    \qquad
    \tikz\graph[clockwise] {j[y=-1] -!- subgraph C_n[V={a,b,c,d,e,f,g,h,i}]; i--e, b--f, h--c, j--{a,g,d}};
  \end{center}
  under the isomorphism $(a,b,c,d,e,f,g,h,i,j) \mapsto (1,6,9,7,10,8,3,4,5,2)$.
\end{example}

\section{(10/26)}

\begin{defn}[subgraph]
  Graph $H$ of a graph $G$ where $V(H) \subseteq V(G)$
  and $E(H) \subseteq \{ uv \in E(G), u,v \in V(H) \}$.
\end{defn}
Equivalently, a subgraph is obtained from $G$ by deleting a subset of $V(G)$
and the edges incident with those vertices and then deleting a subset of remaining edges.

\begin{example}{}
  \tikz[baseline=-30pt]\graph[simple necklace layout]{ 1--2--3--4--5--1, 2--5 };
  has a subgraph
  \tikz[baseline=-30pt]\graph[simple necklace layout]{ "", 2--3, 4--5, 2--5 };
\end{example}

\begin{defn}[types of subgraphs]
  A subgraph $H$ of $G$ is \term{spanning} if $V(H) = V(G)$ (only edge deletions),
  \term{proper} if $H \neq G$,
  and \term{induced} if $E(H) = \{uv \in E(G) : u,v \in V(H)\}$ (only vertex deletions).
\end{defn}

\begin{defn}[walk]
  A sequence $W = (v_0,e_1,v_1,\dotsc,e_k,v_k)$ where $v_i \in V(G)$
  and $e_i = v_{i-1}v_i \in E(G)$ is a \term{walk from $v_0$ to $v_k$}.
  If $v_0 = v_k$, then $W$ is \term{closed}.
  Call $k$ the \term{length} of the walk.
\end{defn}

\begin{example}\label{exa:bowtie}
  In the graph $G$
  \tikz[baseline=-16pt]\graph[spring layout]{
    u_1--u_2--u_3--u_1,
    u_4--u_5--u_6--u_4,
    u_3--u_4
  };,
  the sequence\\
  $(u_5,u_5u_6,u_6,u_6u_5,u_5,u_5u_4,u_4,u_4u_3,u_3,u_3u_1)$ is a walk.
\end{example}

\begin{defn}[path]
  A walk $W = (v_0,\dotsc,v_k)$ where all the $v_i$ are distinct.
  Then, the distinct $v_0$ and $v_k$ are the \term{ends} of the path.
\end{defn}

We call this a path because the subgraph $H$ formed by $V(H) = \{v_0,\dotsc,v_k\}$
and $E(H) = \{e_1,\dotsc,e_k\}$ is isomorphic to $P_{k+1}$.

\begin{example}
  In the graph from \cref{exa:bowtie}, $(u_5,u_5u_4,u_4,u_4u_3,u_3,u_3u_1,u_1)$ is a path.
\end{example}

We can abbreviate paths (and walks) as $v_0v_1v_2\dots v_k$ since it is unambiguous.

\begin{theorem}[4.6.2]\label{thm:pathwalk}
  Given a graph $G$ and there exists a walk from $u$ to $v$,
  then there exists a path from $u$ to $v$.
\end{theorem}
\begin{prf}[informal]
  If $W$ has no repeat vertices, then $W$ is already a path.

  Otherwise, there exists a vertex that repeats twice.
  Remove the subsequence between these two instances (and repeat).
\end{prf}
\begin{prf}
  Let $W = v_0,\dotsc,v_k$ be the shortest walk from $u$ to $v$.
  If $W$ has no repeated vertices, then we are done.

  Otherwise, there exists a subsequence starting and ending with a repeated
  node $v_i = v_j$, so we delete $e_i$ through $v_j$.
  This is a strictly shorter path $W'$, so $W$ is not the shortest. Contradiction.
\end{prf}

\section{(10/28)}

\begin{corollary}[4.6.3]\label{cor:463}
  Let $G$ be a graph. If there exists a path from $u$ to $v$ in $G$
  and there exists a path from $v$ to $w$ in $G$,
  then there exists a path from $u$ to $w$ in $G$.
\end{corollary}
\begin{prf}
  Append the two paths together to get a walk from $u$ to $w$ in $G$.
  Then, by \nameref{thm:pathwalk} there exists a path from $u$ to $w$ in $G$.
\end{prf}

\begin{defn}[cycle]
  A subgraph isomorphic to $C_k$ for some $k \geq 3$.
  The \term{length} of a cycle $k$ is the number of edges in it
  and the \term{girth} of a graph is the length of its largest cycle.
  A \term{Hamilton cycle} is a spanning cycle
  and a \term{Hamiltonian graph} contains one.
\end{defn}

\begin{theorem}[4.6.4]\label{thm:deg2cycle}
  If $G$ is a graph and every vertex of $G$ has degree at least 2,
  then $G$ contains a cycle.
\end{theorem}
\begin{prf}[algorithmic sketch]
  We give an algorithm that always finds such a cycle.

  Let $P = v_0\cdots v_k$ be a path in $G$ with $k \geq 2$.

  The node $v_k$ has at least two neighbours, so there is one other than $v_{k-1}$.
  If the other neighbour $v_i \in V(P)$,
  then $v_i \cdots v_k$ is a cycle.
  Otherwise, the neighbour $v_{k+1} \not\in V(P)$.
  Set $P \gets v_0\cdots v_k v_{k+1}$ and repeat.

  We have to prove correctness (that it generates a path and finds cycles)
  and that it terminates (since length increases and $V(G)$ finite).
\end{prf}
\begin{prf}
  Let $P = v_0 \cdots v_k$ be a longest path in $G$.
  Since a vertex has degree at least 2, $k \geq 2$.

  Suppose $v_k$ has a neighbour $v_i \in V(P)$ where $0 \leq i \leq k-2$.
  Then, $C = v_i \cdots v_k$ is a cycle, as desired.

  Otherwise, $N(v_k) \cap V(P) = \{v_{k-1}\}$.
  Since $v_k$ has degree at least two, $\abs{N(v_k)} \geq 2$.
  Therefore, $N(v_k) \setminus V(P) \neq \varnothing$.
  Select $v_{k+1} \in N(v_k) \setminus V(P) \neq \varnothing$.
  But then $P' = v_0 \cdots v_k v_{k+1}$ is a path in $G$ with length longer than $P$.
  Contradiction.
\end{prf}
\begin{corollary}[cycle corollary]\label{cor:464-pos}
  Let $G$ be a graph. \TFAE:
  (1) $G$ contains a cycle; and
  (2) $G$ contains a subgraph where every vertex has degree at least 2.
\end{corollary}
\begin{prf}
  If $G$ contains a cycle, then the cycle is indeed such a subgraph.
  If $G$ contains such a subgraph, then by \nameref{thm:deg2cycle} it contains a cycle.
\end{prf}
\begin{corollary}[no cycle corollary]\label{cor:464-neg}
  Let $G$ be a graph. Then, \Tfae:
  (1) $G$ does not contain a cycle;
  (2) all subgraphs of $G$ contain a vertex with degree at most 1; and
  (3) there exists an ordering $v_1,\dotsc,v_n$ of $V(G)$ such that
  $\abs{N(v_i) \cap \{v_1,\dotsc,v_{i-1}\}} \leq 1$ for all $i \in [n]$.
\end{corollary}
\begin{prf}
  (1) and (2) are equivalent by the last corollary.

  (2) implies (3) by induction on $V(G)$.
  By assumption, $G$ has a vertex $\deg v_n \leq 1$.
  Let $G' = G$ without $v_n$.
  By assumption, every subgraph of $G'$ has a vertex with degree at most 1 since $G$ does.
  By induction, there exists the desired ordering $v_1,\dotsc,v_{n-1}$ of $V(G')$
  but then just let $v_1,\dotsc,v_n$ as desired.

  (3) implies (2): let $H$ be a subgraph of $G$
  and $i$ be the largest index such that $v_i \in V(H)$.
  Then, $\abs{N_H(v_i)} \leq \abs{N_G(v_i) \cap \{v_1,\dotsc,v_{i-1}\}} \leq 1$
  by assumption.
\end{prf}

Consider now the question: can we decide efficiently whether a graph $G$ contains a cycle?

\begin{defn}[polynomial time]
  A decision problem is in polynomial time if it can be solved in
  $O(\abs{V(G)}^k)$ for some $k \in \N$.
  Denote the set of such problems by $\P$.
\end{defn}

Then, ask if the decision problem ``does $G$ contain a cycle?'' lies in $\P$.

\section{(10/31)}

\begin{defn}[nondeterministic polynomial time]
  A decision problem is in $\NP$ if the problem of verifying
  a certificate of correctness for ``yes'' results is in $\P$.

  Equivalently, a nondeterministic algorithm can solve ``yes'' instances in $\P$.
\end{defn}
\begin{defn}[$\coNP$ time]
  A decision problem is in $\coNP$ if verification of ``no'' instances is in $\P$.
\end{defn}

\begin{remark}
  Clearly, $\P \subseteq \NP$ and $\P \subseteq \coNP$.
\end{remark}

It is simple to verify a cycle exists after finding one, so this is in $\NP$.

Point (3) of the \nameref{cor:464-neg} is a sufficient ``no'' certificate,
so this is also in $\coNP$.

It is not obvious that this problem is in $\P$ or not in $\P$.

\begin{conjecture}
  $\P \neq \NP$
\end{conjecture}

\begin{conjecture}
  $\P = \NP \cap \coNP$
\end{conjecture}

\begin{defn}[$\NP$-complete]
  A problem in $\NP$ and, if it is in $\P$, implies that $\P = \NP$.
  Informally, a problem in $\NP$ that is as hard as possible.
\end{defn}

We can use the \nameref{cor:464-neg} to find whether a cycle exists in $G$:
\begin{enumerate}[1.,nosep]
  \item $i \gets \abs{V(G)}$
  \item $G_n \gets G$
  \item \textbf{while} $\exists v_i \in V(G)$, $\deg v_i \leq 1$ \textbf{do:} \\
        \hspace*{2em} $G_{i-1} \gets G_i \setminus \{v_i\}$ \\
        \hspace*{2em} $i \gets i - 1$
  \item \textbf{if} $i = 0$ \textbf{then} NO \textbf{else} YES
\end{enumerate}

\begin{defn}[connectedness]
  For every $u,v\in V(G)$, there exists a path from $u$ to $v$.
\end{defn}
\begin{example}
  $K_n$, $C_n$, and $P_n$ are all connected for every $n$ (trivial).
  $K_{m,n}$ is connected as long as either $m \neq 0$ or $n \neq 0$.
\end{example}

\begin{defn}[component]
  A non-empty subset $X$ of $V(G)$ such that the graph induced by $X$ is connected
  and $X$ is maximal with respect to this property.
\end{defn}

\begin{example}
  The graph \tikz[baseline=-15pt]\graph{ 1[y=-0.5]--2--3--4[y=-0.5], 2--5[x=1.5]--3 };
  is connected and has one component.
\end{example}
\begin{example}
  The graph \tikz[baseline=-15pt]\graph[grow=right]{
  2--{1,3} -!- 4--{5,6}; 5--6;
  }; has two unconnected components.
\end{example}

\begin{theorem}
  The components of a graph are pairwise disjoint and partition $V(G)$.
\end{theorem}

\begin{defn}[equivalence relation]
  A binary relation $\sim$ if it is (1) reflexive, i.e., $x \sim x$,
  (2) symmetric, i.e., $x \sim y \implies y \sim x$,
  and (3) transitive, i.e., $x \sim y \sim z \implies x \sim z$.
\end{defn}

\begin{defn}[equivalence class]
  The set $[x]_\sim := \{y : x \sim y\}$.
\end{defn}

\begin{prop}
  Given equivalence relation $\sim : S \to S$,
  the equivalence classes of $\sim$ partition $S$.
\end{prop}

\section{(11/02)}

\begin{theorem}[path equivalence]\label{thm:patheq}
  Let $\sim_G$ be a binary relation on $V(G)$
  such that $x \sim_G y$ if a path exists from $x$ to $y$ in $G$.
  Then, $\sim_G$ is an equivalence relation.
\end{theorem}
\begin{prf}
  Show reflexivity, symmetry, and transitivity.
  Let $x,y,z \in V(G)$.

  Notice that a path from $x$ to $x$ exists, namely, the path $\vb p = (x)$.
  Then, $x \sim_G x$.

  Suppose $x \sim_G y$.
  Then, there exists a path $(x,xv_1,\dotsc,v_ky,y)$ from $x$ to $y$.
  But then $(y,yv_k,\dotsc,v_1x,x)$ is a path from $y$ to $x$.
  Therefore, $y \sim_G x$.

  Suppose $x \sim_G y \sim_G z$.
  Then, there exist paths from $x$ to $y$ and $y$ to $z$.
  By Corollary \nameref{cor:463}, there exists a path from $x$ to $z$.
  Therefore, $x \sim_G z$.

  Thus, $\sim_G$ is an equivalence relation.
\end{prf}
\begin{corollary}
  A subset of $V(G)$ is an equivalence class of $\sim_G$
  if and only if it is a component.
\end{corollary}
\begin{prf}
  Exercise.
\end{prf}

The immediate consequence is the theorem

\begin{theorem}
  The components of a graph $G$ partition $V(G)$.
\end{theorem}

Consider the decision problem if a graph $G$ is connected.
Is this in \P? \NP? \coNP?

Given a set of paths from every pair of vertices $(x,y)$,
it will take $O(\abs{V(G)}^3)$ time to verify that all the paths exist.
This is polynomial, so we are in \NP.

\begin{defn}[cut]
  For graph $G$ and $X \subseteq V(G)$,
  the \term{cut induced by $X$} is
  $\delta(X) = \{xy \in E(G) : x \in X, y \not\in X\}$.
\end{defn}

\begin{example}
  In \tikz[baseline=-15pt]\graph{{1,2}--3--4--{5,6}};,
  $\delta(\{1,2,4\}) = \{13,23,43,45,46\}$.
\end{example}

\begin{example}
  In \tikz[baseline=-15pt]\graph{ {1,2}--3-!-{4,5}-!-{6,7}; 1--2, 4--5--6--7;};,
  $\delta(\{1,3,7\}) = \{12,23,67\}$
  and $\delta(\{1,2,3\}) = \varnothing$.
\end{example}

\begin{theorem}[4.8.5]
  A graph $G$ is connected if and only if there does not exist
  non-empty $X \subsetneq V(G)$ with $\delta(X) = \varnothing$.
\end{theorem}
\begin{prf}
  Suppose $G$ is not connected.
  Then, $G$ has at least two components.
  Let $X$ be one of them, so that $X \subseteq V(G)$ and $X \neq V(G)$.
  Then, $\delta(X) = \varnothing$ because if $u \in X$ and $uv \in \delta(X)$,
  then $u \sim_G v$, so $v \in X$ and $uv \not\in \delta(X)$.

  Conversely, there exists non-empty $X \subsetneq V(G)$
  with $\delta(X) = \varnothing$.
  Then, pick $x \in \delta(X)$ and $y \in V(G) \setminus \delta(X)$.

  Notice that for there to be a path $xv_1\cdots v_ky$ from $x$ to $y$,
  we must cross between $X$ and $V(G) \setminus X$.
  This is because either (1) $v_1,\dotsc,v_k$ are all in $X$,
  meaning $v_ky \in \delta(X)$
  or (2) some $v_i$ is the first that is not in $X$, meaning $v_{i-1}v_i \in \delta(X)$.

  Since $\delta(X)$ is empty, there cannot be a path from $x$ to $y$.
  Therefore, $G$ is not connected.
\end{prf}

With this theorem, connectivity is in \coNP.
Given a component $\varnothing \neq X \subsetneq V(N)$,
we need only examine $\delta(X)$ and find it is empty.
This runs in $O(\abs{V(G)} + \abs{E(G)})$.

\section{(11/04)}

Recall the Seven Bridges of K\"onigsberg across the Pregel:
there is a bridge between the islands of Kneiphof and Lomse,
two bridges from each bank of the Pregel to Kneiphof
and one bridge from each bank to Lomse.
Is it possible to start and end in the same place and traverse each bridge once?

Recast as a graph theory problem:
given the graph \tikz[baseline=-5pt]\graph{1 --[bend left] {2[y=1],3} -- 4; 1 --[bend right] {2,3}, 1--4;};
is there a closed walk that uses every edge exactly once?
Generalize.

\begin{defn}[Eulerian circuit]
  A closed walk in $G$ that uses every edge exactly once.
\end{defn}

Finding an Eulerian circuit is in $\NP$:
given an Eulerian circuit, traverse it to verify correctness.

\begin{theorem}[4.9.2]
  Let $G$ be a connected graph.
  Then, $G$ has an Eulerian circuit if and only if every vertex of $G$ has even degree.
\end{theorem}
\begin{prf}
  Suppose an Eulerian circuit exists.
  Let $v \in V(G)$.
  Every appearance of $v$ in the circuit has an incoming and outgoing edge.
  Then, since the circuit uses all edges, $\abs{\delta(v)} = \deg v$ is even.

  Conversely, suppose that $G$ is connected and every vertex has even degree.

  Proceed by induction on $\abs{E(G)}$.

  If there exists a vertex with degree 0,
  then since $G$ is connected, it is the only vertex and $G$ has no edges.
  Then, the empty walk is an Eulerian circuit.

  If every vertex has degree at least 2,
  then by Theorem \nameref{thm:deg2cycle}, $G$ contains a cycle $C$.

  Let $G'$ be the subgraph of $G$ without $E(C)$.
  Note that every vertex of $G'$ still has even degree,
  since we either leave it alone or remove two edges.
  Let $C_i$ be the components of $G'$.
  Inductively, each component $C_i$ has an Eulerian circuit $W_i$
  since $\abs{E(C_i)} < \abs{E(G)}$.

  Finally, for each $C_i$, there is a $v_i \in V(C_i) \cap V(C)$
  since $G$ is connected.
  Insert at $v_i$ the circuit $W_i$ to create an Eulerian circuit of $G$.
\end{prf}

\begin{corollary}
  Let $G$ be a graph. Then $G$ has an Eulerian circuit if and only if
  every vertex has even degree and all edges are in the same component.
\end{corollary}

Now, it is obvious that finding an Eulerian circuit is in \coNP:
either an odd degree vertex or an empty cut with an edge on both sides.
We can also show it is in \P, analyzing the algorithm.

\begin{defn}[bridge]
  An edge $e \in E(G)$ such that $G - e$ has more components than $G$.
\end{defn}

\begin{example}
  In the graph \tikz[baseline=-30pt]\graph{1[y=-1]--{2,3,4} -!- 5[y=-1] -- {6,7[y=-1]}; 2--3--4, 3--5, 6--7;};,
  the edge 35 is a bridge.
\end{example}

\begin{example}
  In the graph
  \begin{center}
    \tikz\graph[spring layout]{1--2--3--1, 3--4--5, 6--7, 8--9--10--8, 10--11--12, 9--13};
  \end{center}
  the edges 34, 45, 67, 913, 1011, and 1012 are bridges.
\end{example}

\begin{lemma}[4.10.2]\label{lem:bridgecomp}
  Let $G$ be a connected graph.

  If $e = xy$ is a bridge of $G$, then $G-e$ has exactly two components
  and $x$ and $y$ are in different components of $G-e$.
\end{lemma}

\begin{prf}
  Let $A = \{v \in V(G) : v \sim_{G-e} x\}$
  and $B = \{v \in V(G) : v \sim_{G-e} y\}$.

  Claim that $V(G) = A \cup B$.
  Since $G$ is connected, there exists a path $P$ from $x$ to $v$.
  If $y \not\in V(P)$, then $P$ is a path from $x$ to $v$ in $G-e$, hence $v \in A$.

  Assume $y \in V(P)$. Then, the subpath $P'$ of $P$ from $y$ to $v$
  does not use $e$ and hence $v \in B$. This completes the proof of the claim.

  If $A \cap B = \varnothing$, then all of $G-e$ is in one equivalence class of $\sim_{G-e}$,
  contradicting that $G-e$ has at least 2 components.

  Therefore, $A$ and $B$ partition $V(G)$ and indeed are the components of $G-e$.
\end{prf}

\section{(11/07)}

\begin{prop}[converse of \nameref{lem:bridgecomp}]\label{prop:compbridge}
  Let $e = xy \in E(G)$.
  If $x$ and $y$ are in different components of $G-e$, $e$ is a bridge.
\end{prop}
\begin{prf}
  First, note that if $u \not\sim_G v$, then $u \not\sim_{G-e} v$.
  This implies that $G/{\sim_G} \subseteq G_{G-e}/{\sim_{G-e}}$.
  Then, since $x \sim_G y$ (i.e., $[x] = [y]$)
  but $x \not\sim_{G-e} y$ (i.e., $[x] \neq [y]$),
  $G/{\sim_G}$ must be a strict subset.
  Since $G-e$ has more components than $G$, $e$ is a bridge.
\end{prf}

\begin{theorem}[4.10.3]\label{thm:bridge}
  Let $e = xy \in E(G)$. Then, \Tfae,
  \begin{enumerate}[(1),nosep]
    \item $e$ is a bridge of $G$;
    \item $x$ and $y$ are in different components of $G-e$;
    \item there does not exist a path from $x$ to $y$ in $G-e$;
    \item there does not exist a cycle of $G$ containing $e$
  \end{enumerate}
\end{theorem}
\begin{prf}
  (1) and (2) are equivalent by Lemma \nameref{lem:bridgecomp}
  and the \nameref{prop:compbridge}.

  (2) and (3) are equivalent since a component of $G-e$
  is exactly an equivalence class of $\sim_{G-e}$,
  so being in different components means that there is no path.

  (3) and (4) are equivalent since if there is a path $P$ from $x$ to $y$ in $G-e$,
  then $P + e$ is a cycle in $G$.
  Conversely, if $C$ is a cycle in $G$ containing $e$, then $C-e$
  is a path from $x$ to $y$ in $G-e$.

  Note: $(1) \iff (4)$ is Theorem 4.10.3.
\end{prf}


\begin{corollary}[4.10.4]\label{cor:pathcycle}
  Let $G$ be a graph and $u,v \in V(G)$.
  If there exist two distinct paths from $u$ to $v$ in $G$,
  then $G$ contains a cycle.
\end{corollary}
\begin{prf}
  Let $P_1 = u x_1 \cdots x_k v$ and $P_2 = u y_1 \cdots y_\ell v$.

  Since $P_1 \neq P_2$, there exists $i$ such that
  $x_i \neq y_i$ and for all $j < i$, $x_j = y_j$.

  Notice that $P_2' = y_{i-1}y_i\cdots y_\ell v$
  is a path from $y_{i-1}$ to $v$ in $G-x_{i-1}x_i$.
  Also, $P_1' = x_i x_{i+1} \cdots x_k v$ is a path from $x_{i+1}$ to $v$
  in $G - x_{i-1}x_i$.
  That is, $x_i \sim_{G-e} v \sim_{G-e} y_{i-1} = x_{i-1}$.
  Therefore, $x_i$ and $x_{i-1}$ are in the same component of $G-x_ix_{i-1}$.

  By Theorem \nameref{thm:bridge}, there exists a cycle of $G$ containing $x_ix_{i-1}$.
\end{prf}

\chapter{Trees}

\begin{defn}[tree]
  A connected graph that has no cycles.
\end{defn}

\begin{defn}[forest]
  A graph that has no cycles.
\end{defn}

Notice that every tree is a forest.
Also, the components of a forest are trees.
Taking a subgraph cannot induce a cycle,
so subgraphs of forests are forests
and subgraphs of trees are forests.
Finally, every forest is the disjoint union of trees (namely, its components).

\begin{lemma}[5.1.3]\label{lem:uniquepath}
  If $u$ and $v$ are vertices of a tree $T$,
  then there is a unique $u$,$v$-path in $T$.
\end{lemma}
\begin{prf}
  There exists at least one such path since $T$ is connected.
  But there is at most one path by Corollary \nameref{cor:pathcycle}
  since $T$ has no cycle by definition.
\end{prf}

\begin{lemma}[5.1.4]\label{lem:treebridge}
  Every edge of a tree $T$ is a bridge.
\end{lemma}
\begin{prf}
  There are no cycles in $T$, so by Theorem \nameref{thm:bridge}, every edge is a bridge.
\end{prf}

\begin{defn}[leaf]
  A vertex in a tree of degree exactly 1.
\end{defn}

\begin{theorem}[leaf existence]\label{thm:leafexist}
  Every tree $T$ with at least two vertices has a leaf.
\end{theorem}
\begin{sol}
  Since $T$ has no cycle, then $T$ has a vertex $\deg(v) \leq 1$ by
  the \nameref{cor:464-neg}.

  If $\deg(v) = 0$, then $v$ is a component by itself.
  But since $\abs{V(T)} \geq 2$, there is another component,
  contradicting that $T$ is a tree.

  Therefore, $v$ is a leaf.
\end{sol}

\section{(11/09)}

\begin{theorem}
  \TFAE:
  \begin{enumerate}[(1),nosep]
    \item $G$ is a tree
    \item $G$ is connected and every connected subgraph of $G$
          with at least two vertices has a leaf
    \item There exists an ordering $(v_i,\dotsc,v_n)$ such that
          $\abs{N(v_i) \cap \{v_1,\dotsc,v_{i-1}\}} = 1$ for all $2 \leq i \leq n$
  \end{enumerate}
\end{theorem}

\begin{theorem}[very useful theorem]\label{thm:leafremove}
  Let $T$ be a tree and $v$ a leaf of $T$.
  Then, $T-v$ is a tree.
\end{theorem}
\begin{prf}
  Clearly, $T-v$ is a forest (removal cannot create a cycle).

  By Lemma~\nameref{lem:treebridge}, every edge of $T$ is a bridge.

  Let $u$ be the unique neighbour of $v$ and $e=uv$.
  Then, $e$ is a bridge.

  Since $T$ is connected, $T-e$ has exactly two components
  and $u$ and $v$ are in different components.
  Since $v$ now has degree 0 in a component on its lonesome,
  $V(T) \setminus \{v\}$ is a component of $T-e$ and in fact of $T-v$.

  Therefore, $T-v$ is connected and $T-v$ is a tree
\end{prf}

Notice that induction on this theorem proves equivalence of (1) and (3) above.

\begin{theorem}[5.1.8]\label{thm:2leaves}
  Every tree with at least two vertices has at least two leaves.
\end{theorem}
\begin{prf}[Postle]
  Proceed by induction on $\abs{V(T)}$.

  If $\abs{V(T)} = 2$, then $T = K_2$ with both vertices being leaves.

  Otherwise, if $\abs{V(T)} \geq 3$, by \nameref{thm:leafexist},
  $T$ has at least one leaf $v$.
  Then, by the \nameref{thm:leafremove}, $T-v$ is a tree
  where $\abs{V(T-v)} = \abs{V(T)} - 1$.
  By induction, $T-v$ has at least two leaves $v_1$ and $v_2$.

  Let $u$ be the neighbour of $v$.
  Then, $u$ cannot be both $v_1$ and $v_2$, so either $v_1$ or $v_2$ is another
  leaf in $T-v$ and in $T$, as desired.
\end{prf}
\begin{prf}[Book]
  Let $P = v_1 \cdots v_k$ be a longest path in $T$.
  Then, $v_1$ and $v_k$ are leaves of $T$
  (if not, then there is either a longer path or a cycle).
\end{prf}

\begin{theorem}[5.1.5]\label{thm:edgenum}
  If $T$ is a tree, then $\abs{E(T)} = \abs{V(T)} - 1$.
\end{theorem}
\begin{prf}
  Proceed by induction on $\abs{V(T)}$.
  If $\abs{V(T)} = 1$, then $E(T) = 0 = \abs{V(T)} - 1$.

  Assume $\abs{V(T)} \geq 2$.
  Then, $T$ has a leaf by Theorem~\nameref{thm:2leaves}.
  But $T-v$ is a tree by the \nameref{thm:leafremove}.
  Since $T-v$ removed a leaf, it removed only one edge.
  That is, by induction,
  $\abs{E(T)} = \abs{E(T-v)} + 1 = (\abs{V(T-v)} - 1) + 1 = \abs{V(T)} - 1$
  as desired.
\end{prf}

\begin{corollary}[5.1.6]\label{cor:forestedgenum}
  If $G$ is a forest with exactly $k$ components,
  then $\abs{E(G)} = \abs{V(G)} - k$.
\end{corollary}
\begin{prf}
  Let $T_1,\dotsc,T_k$ be the components of $G$.
  Then, $T_i$ are trees.
  By Theorem~\nameref{thm:edgenum}, $\abs{E(T_i)} = \abs{V(T_i)} - 1$.

  Then, $\sum \abs{E(T_i)} = \sum(\abs{V(T_i)} - 1) = \abs{V(T)} - k$.
\end{prf}

\begin{prop}
  If $T$ is a tree with at least two vertices
  and $n_r = \{ v \in V(T) : \deg(v) = r \}$, then we can write
  $n_1 = 2 + \sum_{r \geq 3} (r-2) n_r$.
\end{prop}
\begin{prf}
  By Theorem~\nameref{thm:edgenum}, $\abs{E(T)} = \abs{V(T)} - 1$.
  But by the Handshaking Lemma, $2\abs{E(T)} = \sum \deg(r) = \sum n_r r$.
  Combining, $2(\abs{V(T)}-1) = 2\abs{E(T)} = \sum n_r r$.

  This gives $2(\sum n_r r) - 2 = \sum r n_r$,
  so that $2n_1 + 2n_2 + \dotsb - 2 = n_1 + 2n_2 + \dotsb$
  and we can write $n_1 = 2 + \sum_{r \geq 3}(r-2)n_r$.
\end{prf}

This proposition gives another altnernate proof for Theorem~\nameref{thm:2leaves}.

\begin{defn}[spanning tree]
  A spanning subgraph of $G$ that is a tree.
\end{defn}

\begin{theorem}[5.2.1]\label{thm:conspan}
  A graph $G$ is connected if and only if $G$ has a spanning tree.
\end{theorem}
\begin{prf}
  If $G$ has a spanning tree, then $G$ is clearly connected
  since paths exist by walking along the connected spanning tree.

  If $G$ is connected, then we proceed by induction on $\abs{E(G)}$.
  If $G$ is a tree, then we are done.

  Since $G$ is connected, it contains a cycle $C$ by Theorem~\nameref{thm:deg2cycle}.
  Let $e \in C$.
  Then, by Theorem~\nameref{thm:bridge}, $e$ is not a bridge.
  Hence, $G-e$ has the same number of components.

  But $\abs{E(G-e)} = \abs{E(G)} - 1 < \abs{E(G)}$.
  By induction, $G-e$ has a spanning tree $T$.
  Since we deleted only an edge, $T$ is a spanning tree for $G$.
\end{prf}

\section{(11/11)}

Consider again the forwards (hard) direction of Theorem~\nameref{thm:conspan}.

\begin{prf}[``grow a tree'']
  Let $T$ be a tree in $G$ such that $\abs{V(T)}$ is maximized.
  Such a $T$ exists as $G$ has a single vertex.
  If $V(T) = V(G)$, then $T$ is spanning and we are done.

  Assume $V(T) \neq V(G)$.
  Then, since $G$ is connected, there exists a non-empty cut $\delta(V(T))$
  because $\varnothing \neq V(T) \neq V(G)$.
  Therefore, there exists an edge $e \in \delta(V(T))$.
  Then, $T' = T + e$ is a tree with $\abs{V(T')} = \abs{V(T)} + 1 > \abs{V(T)}$
  contradicting the choice of $T$.
\end{prf}

\begin{prf}[``grow a forest'']
  Let $F$ be a spanning forest such that $\abs{E(F)}$ is maximized.
  This exists since $F$ with all vertices, no edges is such a graph.

  If $\abs{E(F)} = \abs{V(G)} - 1$,
  then $F$ is a spanning tree.

  Assume $\abs{E(F)} \leq \abs{V(G)} - 1$
  so that $F$ is not a tree.

  Let $F_1$ be a component of $F$.
  Since $G$ is connected, $\delta(F_1) \neq \varnothing$
  so $\exists e \in \delta(F_1)$.
  Then, $F' = F + e$ is a spanning forest with more edges.
\end{prf}

\begin{prf}[cute]
  Induct on $\abs{V(G)}$.
  Let $v \in V(G)$. Then $G-v$ has components $H_1,\dotsc,H_k$ for $k \geq 1$.
  These components are connected with fewer vertices,
  so by induction, spanning trees $T_1,\dotsc,T_k$ exist.
  Then, since $G$ is connected, there exist $u_i \in V(H_i)$ where $vu_i \in E(G)$.
  Finally, $\bigcup (T_i + vu_i)$ is a spanning tree of $G$.
\end{prf}

\begin{corollary}[5.2.2]\label{cor:treenum}
  If $G$ is connected and $\abs{E(G)} = \abs{V(G)} - 1$, then $G$ is a tree.
\end{corollary}
\begin{prf}
  By Theorem~\nameref{thm:conspan}, $G$ has a spanning tree $T$ where $V(G) = V(T)$.
  By Theorem~\nameref{thm:edgenum}, $\abs{E(T)} = \abs{V(T)} - 1 = \abs{V(G)} - 1$.
  Since $E(T) \subseteq E(G)$, we have that $E(G) = E(T)$ and $G$ is the tree $T$.
\end{prf}

\begin{theorem}[5.2.3]
  If $T$ is a spanning tree of $G$ and $e$ is an edge of $G$ not in $T$,
  then $T + e$ contains exactly one cycle $C$.\tablefootnote{This is the \term{fundamental cycle of $e$ with respect to $T$}}
  Also, if $e' \in E(C)$, then $T + e - e'$ is a spanning tree of $G$.
\end{theorem}\spewnotes
\begin{prf}
  Since $T$ is a tree, every cycle of $T+e$ must contain $e$.

  Let $C$ be a cycle containing $e = uv$.
  Then, $C-e$ is a path in $T$ from $u$ to $v$.
  By Lemma~\nameref{lem:uniquepath}, $C-e$, and indeed $C$, is unique.

  Suppose $e' \in E(C)$.
  Then, by Lemma~\nameref{lem:bridgecomp}, $e'$ is not a bridge.
  That is, $T+e-e'$ has the same number of components as $T+e$, i.e., it is connected.
  But $\abs{E(T+e-e')} = \abs{E(T)}$ so by Corollary~\nameref{cor:treenum}, it is a tree.
\end{prf}

\begin{theorem}[5.2.4]
  If $T$ is a spanning tree of $G$ and $e \in E(T)$,
  then $T-e$ has two components.
  If $e'$ is in the cut induced by one of the components,
  $T-e+e'$ is a spanning tree.
\end{theorem}
\begin{prf}
  The first part is obvious by Lemma~\nameref{lem:bridgecomp}.

  Let $X$ and $Y$ be the components of $T-e$.
  Since $e' \in \delta(X) = \delta(Y)$,\footnote{The \term{fundamental cut of $e$ with respect to $T$}}
  we have that $X$ and $Y$ are the same component
  in $T-e+e'$, meaning that it is connected.
  As above, by Corollary~\nameref{cor:treenum}, it is a tree.
\end{prf}

Equivalently, if we \emph{add} an edge $e$ to $T$,
then \emph{deleting} an edge in its fundamental cycle yields a spanning tree;
and if we instead \emph{delete} the edge,
\emph{adding} from its fundamental cut gives a spanning tree.

\begin{lemma}[5.3.1]\label{lem:oddbi}
  An odd cycle is not bipartite.
\end{lemma}
\begin{prf}
  See \Cref{exa:17-2}.

  Let $V(C_n) = \{v_1,\dotsc,v_n\}$.
  Suppose for a contradiction there exists a bipartition
  $(A,B)$ of $V(C_n)$ with $E(C_n) \subseteq \delta(A)$.

  We assume \Wlog that $v_1 \in A$.
  Given $v_1v_2 \in E(G)$, this means $v_2 \in B$.
  Likewise up to $v_n \in A$.
  But then $v_1v_n \not\in \delta(A)$.
  Contradiction.
\end{prf}

\begin{prop}
  If $G$ is bipartite, then every subgraph of $G$ is bipartite.
\end{prop}
\begin{prf}
  There exists a bipartition of $G$.
  If we restrict it to a subgraph, it is stil valid.
\end{prf}

\section{(11/14)}

\begin{theorem}[5.3.2]\label{thm:oddbi}
  A graph is not bipartite if and only if it has no odd cycles.
\end{theorem}
\begin{prf}
  The forwards direction follows from the above proposition
  and by Lemma~\nameref{lem:oddbi}.

  Suppose $G$ has no odd cycles.
  By the proposition, $G$ is bipartite if and only if its components are bipartite.

  Then, we may assume $G$ is connected.
  By Theorem~\nameref{thm:conspan}, there exists a spanning tree $T$.
  Let $v \in V(T)$.
  Partition $(A,B)$ based on the parity of the distance from $v$.

  If this is not the desired bipartition, then there exists $u_1u_2 \in E(G)$
  with $u_1$ and $u_2$ both in either $A$ or $B$.
  Let $w$ be the vertex of the unique $(u_1, u_2)$-path $P$
  whose $(w,v)$-path has minimum length.

  Then, let $P_1$ be the $(u_1,v)$-path in $T$, $P_2$ be the $(u_2,v)$-path,
  $P_1'$ be the $(u_1,w)$-path, $P_2'$ be the $(u_2,w)$-path,
  and $Q$ be the $(v,w)$-path.
  By path uniqueness, $\abs{E(P_1)} = \abs{E(P_1')} + \abs{E(Q)}$
  and $\abs{E(P_2)} = \abs{E(P_2')} + \abs{E(Q)}$.

  But since $u_1$ and $u_2$ are in the same partition,
  $\abs{E(P_1)} \equiv \abs{E(P_2)} \pmod{2}$.
  It follows that $\abs{E(P_1')} \equiv \abs{E(P_2')} \pmod{2}$.
  Therefore, $\abs{E(P)} = \abs{E(P_1')} + \abs{E(P_2')} \equiv 0 \pmod{2}$.

  Therefore, $P$ has an even number of edges, so $P + u_1u_2$ is an odd cycle in $G$.
\end{prf}

Consider the decision problem of whether $G$ is bipartite.

It is in \NP, since we can just give the bipartition.
By Theorem~\nameref{thm:oddbi}, it is also in \coNP, since we can give an odd cycle.

We can give a polynomial-time algorithm to find the bipartition/odd cycle:

\begin{enumerate}[1.,nosep]
  \item \textbf{for} $H_i$ component of $G$ \textbf{do:}
  \item \hspace*{0.5cm} $T_i \gets$ spanning tree of $H_i$
  \item \hspace*{0.5cm} pick $v_i \in V(T_i)$
  \item \hspace*{0.5cm} construct $A_i$ and $B_i$ as in the proof
  \item \hspace*{0.5cm} \textbf{if} there are no edges with both ends in $A_i$ or $B_i$ \textbf{then:}
  \item \hspace*{1cm} \textbf{continue}
  \item \hspace*{0.5cm} \textbf{else:}
  \item \hspace*{1cm} \textbf{return} NO
  \item \textbf{return} YES
\end{enumerate}

\setcounter{chapter}{6}
\chapter{Planar Graphs}

\begin{defn}[planar graph]
  A graph that can be \term{drawn in the plane} without edges crossing,
  i.e., vertices are distinct points and
  edges are continuous, non-interescting curves connecting those points.
  This drawing is a \term{planar embedding}.
  A \term{plane graph} is a planar graph with a planar embedding.
\end{defn}

\begin{example}
  These two distinct planar embeddings embed the same graph:
  \begin{center}
    \tikz\graph[dots nodes]{ subgraph C_n[n=3,clockwise] -- 4 -!- 5[x=-1,y=0.5] };
    \qquad
    \tikz\graph[dots nodes]{ subgraph C_n[n=3,clockwise] -- 4 -!- 5[x=-1.4,y=0.3] };
  \end{center}
\end{example}

\begin{defn}[face]
  A connected region of $\R^2 - (E(G) \cup V(G))$.
  Let $F(G)$ be the set of faces of $G$.
\end{defn}

\begin{defn}[boundary walk]
  A sequence $v_0,e_1,v_1,\dotsc,v_0$
  which traverses the edges on the boundary of a face.\tablefootnote{i.e., ``incident'' to the face}
\end{defn}\spewnotes

Note: a vertex may appear multiple times in a boundary walk.

\begin{example}
  In the graph
  \begin{center}
    \tikz\graph[spring layout, dots nodes]{
      1--2--3--4--1, 1--5--6--1, 1--7, 1--8
    };
  \end{center}
  the central vertex appears 4 times in the boundary walk of the ``outside''.
\end{example}

An edge appears twice if and only if that edge is a bridge.

\begin{defn}[degree (length) of a face]
  The length of the boundary walk, counting multiplicity.
\end{defn}

If the boundary is not connected, then the boundary walk is the union
of the boundary walks with respect to each component.

\begin{lemma}[Faceshaking Lemma]\label{lem:face}
  Let $G$ be a plane graph. Then,
  \[ \smashoperator{\sum_{f \in F(G)}} \deg(f) = 2\abs{E(G)} \]
  (by analogy to the \nameref{thm:hand})
\end{lemma}
\begin{prf}
  Each edge contributes exactly 2 to this sum.
  If it is a bridge, 2 to the one face it is incident with;
  otherwise, 1 to each of the 2 faces it is incident with.
\end{prf}

\begin{corollary}
  There are an even number of odd-degree faces.
\end{corollary}
\begin{corollary}
  The average degree of a face is $\frac{2\abs{E(G)}}{\abs{F(G)}}$.
\end{corollary}

\section{(11/16; from Bradley)}

\begin{theorem}[Euler's Formula]\label{thm:euler}
  Let $G$ be a connected planar graph.
  Then, $\abs{V(G)} - \abs{E(G)} + \abs{F(G)} = 2$.
\end{theorem}
\begin{prf}
  Proceed by induction on $\abs{E(G)}$.

  If $G$ is a tree, the number of edges is minimized.
  Then, $\abs{V(G)} = \abs{E(G)} - 1$ by Corollary~\nameref{cor:treenum}
  and there is one face, so $\abs{V(G)} - \abs{E(G)} + \abs{F(G)} = 1 + 1 = 2$.

  Otherwise, $G$ is not a tree.
  Then, $G$ contains a cycle $C$.
  Let $e \in C$ and $G' = G - e$.

  Then, $\abs{V(G')} = \abs{V(G)}$ and $\abs{E(G')} = \abs{E(G)} - 1$.
  Since $e$ was in a cycle, it was incident to two faces that are now joined,
  so $\abs{F(G')} = \abs{F(G)} - 1$.
  Finally, since $e$ is a bridge, $G'$ is connected.
  Therefore, by induction, since $\abs{E(G')} < \abs{E(G)}$,
  we have $\abs{V(G')} - \abs{E(G')} + \abs{F(G')} = 2$.
  Substituting, $\abs{V(G)} - \abs{E(G)} + \abs{F(G)} = 2 + 1 - 1 = 2$ as desired.
\end{prf}

\begin{corollary}[Euler's Formula for unconnected graphs]\label{cor:euler}
  Let $G$ be a graph with $k$ components.
  Then, $\abs{V(G)} - \abs{E(G)} + \abs{F(G)} = 1 + k$.
\end{corollary}
\begin{prf}
  Do the same thing but with the base case of a forest instead of a tree.
  Alternatively, add the minimum $k-1$ edges to connect the components
  and apply the ordinary formula.
\end{prf}

\begin{lemma}[7.5.1]\label{lem:751}
  Let $G$ be a planar graph. If $G$ contains a cycle,
  the boundary of any face contains a cycle.
\end{lemma}
\begin{prf}
  Since $G$ contains a cycle, there are at least two faces.
  Then, a face $f$ is incident to at least one other face $g$.
  Pick an edge $e$ on the boundary of $f$ and $g$.
  Consider the component of the boundary of $f$ that $e$ lies in
  and consider the boundary walk.
  Since $e$ is incident to two faces, $e$ appears once in the walk.
  Therefore, there is a cycle in the component since $e$ is not a bridge.
\end{prf}

\begin{lemma}[7.5.2]\label{lem:752}
  Let $G$ be a planar graph. If each face of $G$ has degree at least $g$,
  then $\abs{E(G)} \leq \frac{g}{g-2}(\abs{V(G)}-2)$.
\end{lemma}
\begin{prf}
  By the~\nameref{lem:face}, $2\abs{E(G)} = \sum \deg(f) \geq g \abs{F(G)}$.
  Then, by \nameref{cor:euler},
  \begin{align*}
    \abs{V(G)} - \abs{E(G)} + \abs{F(G)} = 1 + k & \leq 2               \\
    \abs{V(G)} - 2 + \abs{F(G)}                  & \leq \abs{E(G)}      \\
    g(\abs{V(G)} - 2) + g\abs{F(G)}              & \leq g\abs{E(G)}     \\
    g(\abs{V(G)} - 2) + 2\abs{E(G)}              & \leq g\abs{E(G)}     \\
    g(\abs{V(G)} - 2)                            & \leq (g-2)\abs{E(G)} \\
    \frac{g}{g-2}(\abs{V(G)} - 2)                & \leq \abs{E(G)}
  \end{align*}
  as desired.
\end{prf}

\begin{theorem}[7.5.3]\label{thm:753}
  Let $G$ be a planar graph with $\abs{V(G)} \geq 3$.
  Then, $\abs{E(G)} \leq 3\abs{V(G)} - 6$.
\end{theorem}
\begin{prf}
  Suppose $G$ does not contain a cycle.
  Then, it is a forest and by Corollary \nameref{cor:forestedgenum},
  $\abs{E(G)} = \abs{V(G)} - k \leq \abs{V(G)} - 1 \leq \abs{V(G)} - 6$
  since there is at least 1 component and 3 vertices.

  Otherwise, $G$ contains a cycle.
  By Lemma~\nameref{lem:751}, every face contains a cycle in its boundary.
  Since a cycle has at least 3 edges, the degree of every face is at least 3.
  So by Lemma~\nameref{lem:752}, $\abs{E(G)} \leq \frac{3}{3-2}(\abs{V(G)-2}) = 3\abs{V(G)} - 6$,
  as desired.
\end{prf}
\begin{corollary}[7.5.4]
  $K_5$ is not planar.
\end{corollary}
\begin{prf}
  Notice that $\abs{E(K_5)} = \binom{5}{2} = 10$ but $3\abs{V(G)} - 6 = 9$.
\end{prf}

\begin{theorem}[7.5.6]
  Let $G$ be a planar graph with $\abs{V(G)} \geq 3$.
  If $G$ is triangle-free, then $\abs{E(G)} \leq 2\abs{V(G)}-4$.
\end{theorem}
\begin{prf}
  Follow the proof of Theorem~\nameref{thm:753}
  but notice that cycles will have length at least 4,
  so $\abs{E(G)} \leq \frac{4}{4-2}(\abs{V(G)} - 2) = 2\abs{V(G)} - 4$.
\end{prf}

\begin{lemma}[7.5.7]
  $K_{3,3}$ is not planar.
\end{lemma}
\begin{prf}
  First, notice that since $K_{3,3}$ is bipartite, it is triangle-free.
  But $\abs{E(G)} = 9$ and $2\abs{V(G)} - 4 = 8$.
\end{prf}

\section{(11/18)}

Recall Euler's formula: $V - E + F = 2$ if connected
or in general $V - E + F = k+1$ for $k$ components.

If $V \geq 3$ and $G$ planar, then $E \leq 3V - 6$.
Thus, $K_5$ and $K_{3,3}$ are non-planar.

Consider the decision problem of whether $G$ is planar.
We cannot certify planarity by giving equations of curves,
since showing no intersections is not efficient and the equations can be arbitrarily complex.
However, a planar embedding is equivalently a list of boundary walks of faces.
To certify, assert that every edge appears twice, Euler's formula holds,
and edges cut a vertex cone in a clockwise order.
Therefore, planarity is in \NP.

\begin{prop}
  If $G$ is a planar graph and $H$ is a subgraph of $G$,
  then $H$ is planar.
\end{prop}
\begin{prf}
  Obvious (delete from the planar embedding of $G$ until you get $H$).
\end{prf}

\begin{corollary}
  If $G$ contains $K_5$ or $K_{3,3}$ as a subgraph, then $G$ is not planar.
\end{corollary}

\begin{defn}[edge subdivision]
  Let $e = uv \in E(G)$ and $z \not\in V(G)$.
  Obtain $G'$ as $V(G') = V(G) \cup \{z\}$
  and $E(G') = E(G) \setminus \{e\} \cup \{uz, zv\}$.
\end{defn}

\begin{example}
  If $G$ is \tikz[baseline=-18pt]\graph[dots nodes]{1[y=-0.5]--{2,3}--4[y=-0.5],2--3};,
  we can draw $G'$ as \tikz[baseline=-18pt]\graph[dots nodes]{1[y=-0.5]--{2,3}-!-5[x=-0.5,y=-0.25]--4[x=-1,y=-0.5],2--3,3--4,2--5};
\end{example}

\begin{defn}[subdivision]
  A graph obtained from $G$ by any number (including none)
  of repeated edge subdivisions.
\end{defn}

\begin{example}
  If $G = K_4$, then $G'$ could be
  \tikz[baseline=-3pt]\graph[dots nodes, no placement]{ subgraph C_n[n=3, clockwise] -- 4, 5[x=0.4,y=-0.5], 6[x=-0.4,y=-0.5], 7[x=0,y=-0.5], 8[x=0.4,y=-0.23], 9[x=0,y=0.5], 10[x=0.4,y=0.38]};
\end{example}

\begin{prop}
  If $G$ is a planar graph and $H$ is a subdivision of $G$, then $H$ is planar.
\end{prop}
\begin{prf}
  Subdivide the planar embedding of $G$ and notice it is still planar.
\end{prf}
\begin{corollary}
  If $G$ is a subdivision of $K_{3,3}$ or a subdivision of $K_5$,
  then $G$ is not planar.
\end{corollary}

\begin{theorem}[Kuratowski's Theorem]\label{thm:kur}
  A graph $G$ is planar if and only if
  it does not contain a subdivision of $K_{3,3}$ or a subdivision of $K_5$.
\end{theorem}
\begin{prf}
  Very hard. See CO 342.
\end{prf}

Now, we can say that planarity is in \coNP because we can give by Kuratowski
a subgraph that is a subdivision of $K_{3,3}$ or a subdivision of $K_5$.

\begin{example}
  Is the graph \tikz[baseline=-3pt]\graph[dots nodes]{ subgraph C_n[n=8,clockwise], 1--5,2--6,3--7,4--8 }; planar?
\end{example}
\begin{sol}
  No. We can highlight $K_{3,3}$ as a subgraph:
  \tikz[baseline=-40pt]\graph[dots nodes, simple necklace layout, simple, radius=0.25cm]{ 1[red]--2[blue]--3[red]--4--5[blue]--6[red]--7[blue]--8--1, 1--5,2--6,3--7 };
\end{sol}

We can show that planarity is in \P with an $O(\abs{V(G)}^3)$ algorithm.

\begin{defn}[contraction]
  Let $e = uv \in E(G)$ and $z \not \in V(G)$.
  Obtain $G' = G/e$ as $V(G') = V(G) \setminus \{u,v\} \cup \{z\}$
  and $E(G') = E(G \setminus \{u,v\}) \cup \{wz : w \in N_G(u) \cup N_G(v)\}$.
\end{defn}

\begin{example}
  If $G$ is \tikz[baseline=-18pt]\graph[dots nodes]{{1,2}--3[y=-0.5]--[red, "$e$"]4[y=-0.5]--{5,6}, 1--2, 5--6};,
  then $G/e$ is \tikz[baseline=-18pt]\graph[dots nodes]{{1,2}--3[y=-0.5]--{5,6}, 1--2, 5--6};.
\end{example}

\begin{defn}[minor]
  A graph $H$ is a minor of a graph $G$ if it can be
  obtained from a subgraph of $G$ by contracting edges.
\end{defn}

\begin{prop}
  If $G$ is planar and $H$ is a minor of $G$, then $H$ is planar.
\end{prop}
\begin{corollary}
  If $G$ contains $K_5$ or $K_{3,3}$ as a minor, then $G$ is not planar.
\end{corollary}

\section{(11/21)}

We can equivalently state \nameref{thm:kur} in terms of minors due to Wagner:

\begin{theorem}[Kuratowski's Theorem for minors]
  A graph $G$ is planar if and only if
  $G$ does not contain $K_{3,3}$ or $K_5$ as a minor.
\end{theorem}
\begin{prf}[sketch]
  We must show that containing $K_{3,3}$ or $K_5$ as a subdivision
  is equivalent to containing $K_{3,3}$ or $K_5$ as a minor.

  In fact, claim that $K_5$ subdivisions give $K_5$ minors
  and $K_{3,3}$ subdivisions give $K_{3,3}$ minors.
  If we contract the paths in subdivisions to single edges,
  we get the respective minors.

  Conversely, a $K_{3,3}$ minor will give a $K_{3,3}$ subdivision.
  But a $K_5$ minor can give either a $K_5$ or $K_{3,3}$ subdivision
  (depending on if a vertex in the minor is the result of contracting).
\end{prf}

\begin{theorem}[Graph Minors Theorem]
  Every minor-closed property is characterized by a finite list of forbidden minors.
\end{theorem}

\begin{defn}[colouring]
  Assignment of one of $k$ colours to the vertices of $G$
  such that no two adjacent vertices have the same colour.
  If $G$ has a $k$-colouring, it is \term{$k$-colourable}.
\end{defn}
Equivalently, partition $V(G)$ into $k$ sets such that for all edges $uv \in E(G)$,
$u$ and $v$ are in different sets.
When $k=2$, this is bipartiteness.

Clearly, every graph is $\abs{V(G)}$-colourable, so we are interested
in the minimum $k$ such that $G$ is $k$-colourable.

\begin{defn}[chromatic number]
  The minimum $\chi(G)$ such that $G$ is $\chi(G)$-colourable.
\end{defn}

\begin{example}
  \tikz[spring layout, horizontal=5 to 6]\graph[dots nodes]{1[red]--2[blue]--3[red]--4[blue]--5[green]--1, 6[red]--7[blue]--8[green]--6, 5--6};
  has chromatic number 3.
\end{example}

\begin{example}
  \tikz\graph[layered layout, dots nodes, grow down, simple]{1[red]--{[same layer] 2[green],3[blue],4[green],5[blue]}--{[same layer] 6[red],7}; 2--3, 4--5, 3--6, 3-!-7, 6--7};
  is clearly not bipartite (has triangle) and trying to 3-colour gives a contradiction,
  so $\chi(G) \geq 4$ and in fact we can colour it to get $\chi(G) = 4$.
\end{example}

If we fix $k$, consider the decision problem of $k$-colourability.
This is in \NP, since we can just show the colouring.
It is probably not in \coNP because if it were, we could show $\P \neq \NP$
because it is \NP-complete.

\begin{theorem}[Four-Colour Theorem]
  Every planar graph is 4-colourable.
\end{theorem}
\begin{prf}
  Pain.
\end{prf}

We settle for the next best things: the Six- and Five-Colour Theorems.

\begin{lemma}[7.5.5]\label{lem:pla.deg.5}
  Every planar graph has a vertex of degree at most 5.
\end{lemma}
\begin{prf}
  Suppose a planar graph $G$ exists where every vertex has degree 6 or greater.
  Then, $\abs{V(G)} \geq 7$.
  By Theorem 7.5.3, since $\abs{V(G)} \geq 3$,
  we have $\abs{E(G)} \leq 3\abs{V(G)} - 6$, i.e., $2\abs{E(G)} \leq 6\abs{V(G)} - 12$.

  But by the \nameref{thm:hand}, $2\abs{E(G)}
    = \smashoperator{\sum\limits_{v \in V(G)}} \deg(v) \geq 6\abs{V(G)}$,
  a contradiction.
\end{prf}

\begin{theorem}[Six-Colour Theorem]
  Every planar graph is 6-colourable.
\end{theorem}
\begin{prf}
  Proceed by induction on $\abs{V(G)}$.

  If $\abs{V(G)} = 1$, then $G$ is 1-colourable.

  Suppose $\abs{V(G)} \geq 2$.
  Then, by Lemma~\nameref{lem:pla.deg.5},
  there exists a vertex $v$ of $G$ with $\deg(v) \leq 5$.

  Let $G' = G-v$. Note that $\abs{V(G')} < \abs{V(G)}$ and $G'$ is planar.

  Then, by induction, there exists a 6-colouring $\phi : V(G') \to [6]$ of $G'$.
  But since $\deg(v) < 6$, there exists
  $c \in [6] \setminus \{\phi(u) : u \in N_G(v)\}$.
  Define $\phi(v) = c$ and $\phi$ is a 6-colouring for $G$.

  Therefore, by induction, every planar graph is 6-colourable.
\end{prf}

\section{(11/23)}

\begin{theorem}[Five-Colour Theorem]
  Every planar graph is 5-colourable.
\end{theorem}
\begin{prf}
  Proceed by induction on $\abs{V(G)}$.

  If $\abs{V(G)} = 1$, then $G$ is 1-colourable.

  Suppose $\abs{V(G)} \geq 2$. Then, by Lemma~\nameref{lem:pla.deg.5},
  $G$ has a vertex $v$ with $\deg(v) \leq 5$.

  If $\deg_G(v) \leq 4$, let $G' = G-v$.
  Since $\abs{V(G')} = \abs{V(G)} - 1 < \abs{V(G)}$,
  by induction, $G'$ is 5-colourable.
  Let $\phi$ be a 5-colouring of $G'$
  and extend it so that $\phi(v) \in [5] \setminus \im_\phi(N_G(v))$
  which exists since $\abs{N_G(v)} \leq 4$.
  Then, $\phi$ is a 5-colouring of $G$.

  Otherwise, $\deg_G(v) = 5$.
  Since $G$ is planar, $G$ does not contain $K_5$ as a subgraph.
  Therefore, of the five neighbours of $v$,
  there exist two, $a$ and $b$, that are not adjacent in $G$.

  Let $G' = G / \{av,bv\}$, i.e., merge $a$, $b$, and $v$ into a new vertex $z$.
  Then, $\abs{V(G')} = \abs{V(G)} - 2 < \abs{V(G)}$
  and since planarity is closed under contraction,
  we have by induction that $G'$ is 5-colourable.

  Let $\phi$ be a 5-colouring of $G'$.
  Construct $\phi' : G \to [5]$:
  \[
    \phi'(x) = \begin{cases}
      \phi(x)                                 & x \in V(G) \setminus \{a,b,v\} \\
      \phi(z)                                 & x \in \{a,b\}                  \\
      c \in [5] \setminus \im_{\phi'}(N_G(v)) & x = v
    \end{cases}
  \]
  and verify this is a colouring.

  Let $e = xy \in E(G)$.

  If $x,y \in V(G) \setminus \{a,b,v\}$,
  then $\phi'(x) = \phi(x) \neq \phi(y) = \phi'(y)$
  since $\phi$ is a colouring on $G'$.

  If $x \in V(G) \setminus\{a,b,v\}$ and $y \in \{a,b\}$
  then $\phi'(x) = \phi(z) \neq \phi(y) = \phi'(y)$.

  If $x = v$ and $y \in N_G(v)$, then $\phi'(x) \neq \phi'(y)$ by construction.

  Note that $e \neq ab$ because we chose $a$ and $b$ not adjacent.
\end{prf}

\begin{quote}
  \textbf{Generalizing the Four-Colour Theorem} (NOT COURSE CONTENT)

  Is there a generalization of the Four-Colour Theorem to all graphs?

  Notice that $\chi(K_n) \leq n$ since every vertex can get at most one colour
  and $\chi(K_n) \geq n$ since any vertex is adjacent to all others.
  Then, $\chi(K_n) = n$.

  \textbf{Haj\'os' Conjecture.} If $G$ does not contain a subdivision of $K_t$
  as a subgraph, then $G$ is $(t-1)$-colourable.

  This was proven false for all $t \geq 7$
  and almost every graph is a counterexample.
  In fact, $\frac{t^2}{\sqrt{\log t}} \leq \max \chi(G) \leq t^2$
  so not even on the right order of magnitude.

  \textbf{Hadwiger's Conjecture.} If $G$ does not contain $K_t$ as a minor,
  then $G$ is $(t-1)$-colourable.

  This is proved for $t \leq 6$ but open for $t \geq 7$.
  For general $t$, $\max \chi(G) \leq t(\log \log t)$ due to Delcourt--Postle.
\end{quote}

\begin{defn}[dual]
  Given a planar graph $G$, $G^*$ is a planar multigraph where $V(G^*) = F(G)$
  and then for every $e \in E(G)$, draw an edge $e^*$ in $G^*$ crossing $e$
  and connecting the faces of $G$ to the left and right of $e$ in $G$.
\end{defn}

\begin{example}
  \tikz\graph[dots nodes]{{1,2}--3[y=-0.5]--4[y=-0.5]--5[x=-0.4,y=-0.5]--{6[x=-1],7[x=-1]}, 1--2, 6--7, 4--6, 4--7};
  has a dual \tikz\graph[dots nodes]{left[y=-0.5]--outer[y=-0.5]--{1,2}--3[y=-0.5], left--[bend right]outer, left--[bend left]outer, 1--2, 3--[bend left=-90]outer, outer--[loop below]outer};
\end{example}

Note: this is a multigraph, so loops can be created by crossing bridges
and parallel edges if two faces share more than one edge.

It is planar, so it satisfies \nameref{thm:euler}.

Observe that handshaking for $G^*$ is equivalent to faceshaking for $G$.

If $G$ is connected, then $(G^*)^* = G$.

\setcounter{chapter}{7}
\chapter{Matchings}

\section{(11/25)}

\begin{defn}[matching]
  On a graph $G$, a set of edges $M \subseteq E(G)$ such that no two edges
  share a common endpoint.
\end{defn}

\begin{example}
  \tikz\graph[dots nodes, simple]{{1,2}--3[y=-0.5]--{4,5}--6[y=-0.5]--[red, very thick]7; 1--[red, very thick]2, 4--[red, very thick]5, 5-!-6};
  is a matching.
\end{example}

\begin{defn}
  $M$ \term{saturates} $v$ if $v$ is an end of an edge in $M$.

  $M$ is \term{perfect} if every vertex of $G$ is saturated by $M$.

  The \term{size} of $M$ is the number of edges it contains, i.e., $\abs{M}$.

  $M$ is a \term{maximum matching} if it has maximum size over all matchings of $G$.
\end{defn}

Notice that a perfect matching obviously cannot exist
for components an odd number of vertices.

Finding a perfect matching is in \NP (just give the matching),
\coNP (since CO 250 expresses this as a linear program),
and \P (for proof, take CO 342).

\begin{defn}[alternating]
  Given a matching $M$ of $G$,
  a path $P$ in $G$ is \term{$M$-alternating}
  if every other edge of $P$ is in $M$.
  Equivalently, all $v \in V(P)$ are saturated by $M$.
\end{defn}

\begin{example}
  The path \tikz\graph[dots nodes]{0--1--[red, very thick]2--3--[red, very thick]4--5};
  is \textcolor{red}{$M$}-alternating.
\end{example}

\begin{defn}[augmenting]
  A path $P$ which is $M$-alternating and whose ands are not saturated by $M$.
\end{defn}

\begin{lemma}[8.1.1]\label{lem:811}
  Let $G$ be a graph with matching $M$.
  If there exists an $M$-augmenting path, then $M$ is not a maximum matching.
\end{lemma}
\begin{prf}
  Let $P$ be an $M$-augmenting path.
  Then, $M' = M \sym E(P)$ is a larger matching of $G$.
  This is because, starting from the unsaturated ends,
  flip the exactly one incident edges to the other incident edge
  which exists because the degree of vertices on a path are exactly two.
  Finally, $\abs{M'} > \abs{M}$ because $\abs{E(P) \setminus M} > \abs{E(P) \cap M}$
  by definition of $M$-augmenting.
\end{prf}

\begin{lemma}[converse of Lemma 8.1.1]
  Let $G$ be a graph with matching $M$.
  If $M$ is not a maximum matching,
  then there exists an $M$-augmenting path of $G$.
\end{lemma}
\begin{prf}
  Let $M'$ be a maximum matching of $G$.
  Consider $M \sym M'$.
  For example, the matchings \textcolor{red}{$M$} and \textcolor{blue}{$M'$}
  \begin{center}
    \begin{tikzpicture}
      \graph[dots nodes]{1--[very thick,blue]2--[decoration=snake,decorate,red]3--[very thick,blue]4 -!- 5--[decoration=snake,decorate,red]6, 5--[very thick,blue]6};
    \end{tikzpicture}
  \end{center}
  have symmetric difference \textcolor{violet}{$M \sym M'$}
  \begin{center}
    \begin{tikzpicture}
      \graph[dots nodes]{1--[very thick,violet]2--[very thick,violet]3--[very thick,violet]4 -!- 5--6};
    \end{tikzpicture}
  \end{center}
  Observe that since $M$ and $M'$ are matchings,
  then $M \sym M'$ is a disjoint union of $M$-alternating paths (or $M'$-alternating paths).

  Let $P_1,\dotsc,P_k$ be the components of $M \sym M'$.
  If the ends of $P_i$ are saturated by $M'$
  (i.e., $\abs{E(P_i) \cap M'} > \abs{E(P_i) \cap M}$),
  then $P_i$ is an $M$-augmenting path as desired.

  Otherwise, $\abs{M'} = \abs{M' \cap M} + \sum\abs{E(P_i) \cap M'}
    \leq \abs{M' \cap M} + \sum\abs{E(P_i) \cap M} = \abs{M}$,
  a contradiction.
\end{prf}

\section{(11/28)}

\begin{defn}[cover]
  A set $X \subseteq V(G)$ such that every edge of $G$ has at least one end in $X$.
\end{defn}

\begin{example}\label{exa:331}
  \tikz\graph[dots nodes,spring layout,horizontal=1 to 4]{1[red]--2--3[red]--1--4};
  \quad
  \tikz\graph[dots nodes,simple necklace layout]{1[red]--2[red]--3[red]--4--1, 1--3, 2--4};
  \quad
  \tikz\graph[dots nodes,simple necklace layout]{1[red]--2--3[red]--4[red]--5[red]--1};
  have covers in red.
\end{example}

Every graph has the obvious cover $X = V(G)$.

\begin{defn}[minimum cover]
  A cover of minimum size (i.e., $\abs{X}$),
  which exists because $V(G)$ is a cover.
\end{defn}

Notice that in \Cref{exa:331}, the first and second covers are minimal.

Determining if a graph has a cover of size $k$ is obviously in \NP.
It is in fact \NP-complete, so it is not in \coNP or \P (assuming $\P \neq \NP$).

\begin{lemma}[8.2.1]\label{lem:821}
  Let $M$ be a matching and $C$ be a cover of the graph $G$.
  Then, $\abs{M} \leq \abs{C}$.
\end{lemma}
\begin{prf}
  Let $e_i$ be the edges of $M$.
  For each edge, at least one of the two incident vertices must be covered by $C$.
  Therefore, $\abs{C} \geq \sum\abs{C \cap e_i} \geq \sum 1 = k = \abs{M}$.
\end{prf}

\begin{corollary}
  Let $G$ be a graph, $M$ be a maximum matching of $G$,
  and $C$ be a minimum cover of $G$.
  Then, $\abs{M} \leq \abs{C}$.
\end{corollary}

\begin{example}
  \tikz\graph[dots nodes,spring layout,horizontal=1 to 4]{1[red]--2--[blue,very thick]3[red]--1--[blue,very thick]4};
  \quad
  \tikz\graph[dots nodes,simple necklace layout]{1[red]--2--3[red]--4--1, 1--[blue,very thick]3, 2--[blue,very thick]4};
  \quad
  \tikz\graph[dots nodes,simple necklace layout]{1[red]--[blue,very thick]2--3[red]--4--[blue,very thick]5[red]--1};
  have minimum covers in red and maximum matchings in blue.
\end{example}

How many graphs have the property that $\abs{M} = \abs{C}$?
We must exclude subgraphs of $C_{2n+1}$ since the minimum cover is $n+1$
but the maximum matching is $n$.

\begin{theorem}[K\"onig's Theorem]\label{thm:konig}
  Let $G$ be a bipartite graph, $M$ be a maximum matching of $G$,
  and $C$ be a minimum cover of $G$.
  Then, $\abs{M} = \abs{C}$.
\end{theorem}

\begin{lemma}[8.2.2]\label{lem:822}
  Let $G$ be a graph, $M$ be a matching of $G$, and $C$ be a cover of $G$.
  If $\abs{M} = \abs{C}$, then $M$ is a maximum matching and $C$ is a minimum cover.
\end{lemma}
\begin{prf}
  Let $M'$ be a matching of $G$.
  Since $C$ is a cover, then by Lemma~\nameref{lem:821}, $\abs{M'} < \abs{C}$.
  Since $\abs{C} = \abs{M}$, then $\abs{M'} < \abs{M}$
  and $M$ is maximum, as desired.

  Likewise, let $C'$ be a covery of $G$.
  Then, by Lemma~\nameref{lem:821}, $\abs{M} \leq \abs{C'}$.
  Since $\abs{M} = \abs{C}$,
  then $\abs{C} \leq \abs{C'}$ and $\abs{C} \leq \abs{C'}$
  for all covers, meaning $C$ is minimum, as desired.
\end{prf}

\section{(11/30)}

\begin{defn}[reachability]
  Given a graph $G$ with matching $M$ and $v \in V(G)$,
  a vertex $u$ is $M$-reachable from $v$
  if there exists an $M$-alternating $u,v$-path of $G$.
  For a set $S \subseteq V(G)$, $u$ is $M$-reachable from $S$
  if there exists a $v \in S$ such that $u$ is $M$-reachable from $v$.
\end{defn}

\begin{example}
  \tikz\graph[dots nodes]{{1[Green],2[Green]}-!-{3[Green],4[Green]}-!-{5[red],6[Green]}-!-{7[orange],8[red]}; 2--[blue,very thick]1--4, 2--3--[blue,very thick]6--7, 6--5--8};
  has a \textcolor{Green}{set of vertices} that are \textcolor{blue}{$M$}-reachable from \textcolor{orange}{$v$}.
  and a \textcolor{red}{set of \textcolor{blue}{$M$}-unreachable vertices}.
\end{example}

Recall \nameref{thm:konig}.

\begin{prf}
  Let $G$ be a graph with bipartition $(A, B)$.

  Let $M$ be a maximum matching of $G$ and
  $U$ be the set of $M$-unsaturated vertices of $G$.

  Let $X_0 = U \cap A$.

  First, suppose $X_0 = \varnothing$.
  Let $C := A$. Then, $A$ is a cover of $G$
  because every edge has exactly one end in $A$.
  Also, since $X_0 = \varnothing$, we have that $\abs{C} = \abs{A} = \abs{M}$
  because every vertex of $A$ is $M$-saturated.
  
  By Lemma~\nameref{lem:822}, it follows that $M$ is maximum, $C$ is minimum,
  and $\abs{M} = \abs{C}$ as desired.

  Assume $X_0 \neq \varnothing$.
  Let $X \subseteq A$ and $Y \subseteq B$ be the $M$-reachable vertices from $X_0$.
  Define $C := (A \setminus X) \cup Y$.
  We claim that (1) $C$ is a cover and then that (2) $\abs{C} = \abs{M}$.

  Suppose for a contradiction $C$ is not a cover, i.e.,
  there exists $e = uv \in E(G)$ where $u,v \not\in C$.
  \WLOG assume $u \in A$, $v \in B$.
  Then, since $u,v \not\in C$,
  we have $u \not\in A \setminus X$, so $u \in X$.
  That is, $u$ is $M$-reachable from $X_0$.
  We also have $v \not\in Y$, so $v$ is not $M$-reachable from $X_0$.

  Let $P$ be an $M$-alternating $u,x$-path for some $x \in X_0$.
  Then, since $v$ is not $M$-reachable, $v \not\in V(P)$.

  Let $P' = P + uv$.
  The edge of $P$ incident with $x$ is not in $M$ since $x \in X_0$ is $M$-unsaturated.
  Then, $\abs{E(P)}$ is even because $u,x \in A$,
  so we know that the edge of $P$ incident with $u$ is in $M$.
  Therefore, $P'$ is an $M$-alternating $u,v$-path, so $v \in Y$.
  This is a contradiction, giving us (1).

  Claim now (2a) that $Y \cap U = \varnothing$.
  Suppose that $Y \cap U \neq \varnothing$.
  Let $w \in Y \cap U$.
  Since $w \in Y$, there is an $M$-alternating $w,x$-path $P$ for some $x \in X_0$.
  But then $w \in U$ and $x \in U$ and hence $P$ is $M$-augmenting,
  contradicting Lemma~\nameref{lem:811} since $M$ is a maximum matching.
  Therefore, every vertex of $Y$ is $M$-saturated, i.e., (2a).

  Finally, suppose there exists $e = ab \in M$
  where $b \in Y$ and $a \in A \setminus X$.
  Then, there exists an $M$-alternating $b,x$-path $P$ for some $x \in X_0$.
  This path has odd length because it goes from $A$ to $B$.
  It starts on $x \in U$, so the last edge is not in $M$.
  Again, we can now take $P' = P + ba$
  to get a contradicting $M$-alternating path.
  Therefore, (2b) no edge from $A \setminus X$ to $Y$ exists.

  Now, recall that $(A \setminus X) \cap U = \varnothing$
  since $U \cap A = X_0 \subseteq X$.
  That is, every vertex in $A \setminus X$ is $M$-saturated.
  Finally (actually this time), by (2b),
  notice that all edges from $A \setminus X$, i.e.,
  the saturated edges of the matching, must then go to $B \setminus Y$.

  Let $M_1 \subseteq E(M)$ have ends in $Y$ and $M_2 \subseteq E(M)$ have ends in $A \setminus X$.
  From before, there does not exist edges from $A \setminus X$ to $Y$,
  so $M_1 \cap M_2 = \varnothing$.
  Yet $C = (A \setminus X) \cup Y$ is a cover.
  So $M \subseteq M_1 \cup M_2$ and in fact $M = M_1 \cup M_2$.
  But by (2a), $Y \cap U = \varnothing$, so $\abs{M_1} = \abs{Y}$.
  Similarly $\abs{M_2} = \abs{A \setminus X}$.
  
  Finally (for the third time), $\abs{M} = \abs{M_1} + \abs{M_2}
    = \abs{Y} + \abs{A \setminus X} = \abs{C}$, i.e., (2).

  Then, since (1) and (2) hold, by Lemma~\nameref{lem:822},
  it follows that $M$ is maximum and $C$ is minimum, as desired.
\end{prf}

\section{(12/02; missing)}

XY Algorithm

\section{(12/05)}

\begin{theorem}[Hall's Theorem]\label{thm:hall}
  Let $G = (A,B)$ be a bipartite graph.
  Then, $G$ has a matching saturating every vertex of $A$
  if and only if $\forall D \subseteq A, \abs{D} \leq \abs{N(D)}$
  where $N(D) = \bigcup\limits_{v \in D} N(v)$.
\end{theorem}
\begin{prf}
  Suppose $G$ has a matching $M$ saturating every vertex of $A$.
  Let $D \subseteq A$.
  Then, let $X := \{u \in B : \exists v \in A, uv \in M\}$.
  Since $M$ is a matching, $\abs{X} = \abs{D}$.
  But $X \subseteq N(D)$, so $\abs{D} \leq \abs{N(D)}$, as desired.

  Suppose that there exists a matching saturating $A$.
  Recall that by \nameref{thm:konig}, the size of a maximum matching
  is equal to the size of a minimum cover.

  Let $C$ be a (possibly minimum) cover of $G$.
  Then, because $G$ is bipartite, there does not exist an edge $e = ab$
  where $a \in A \setminus C$ and $b \in B \setminus C$.
  It follows that $N(A \setminus C) \subseteq B \cap C$.

  Then, $\abs{A \setminus C} \leq N(A \setminus C) \leq \abs{B \cap C}$
  because $A \setminus C$ is a subset of $A$.

  Finally, $\abs{A} = \abs{A \cap C} + \abs{A \setminus C}
    \leq \abs{A \cap C} + \abs{B \cap C} = \abs{C}$.
  Therefore, a minimum cover has size at least $\abs{A}$.
  By K\"onig's, a maximum matching has size at least $\abs{A}$
  and the matching saturates all of $A$ by bipartiteness.
\end{prf}

Note: K\"onig's and Hall's theorems show that for a bipartite graph $G = (A,B)$,
deciding if there exists a matching saturating all of $A$ is in \coNP.

\begin{corollary}[8.6.1]\label{cor:bip.pm}
  A bipartite graph $G = (A,B)$ has a perfect matching
  if and only if $\abs{A} = \abs{B}$ and $\forall D \subseteq A, \abs{D} \leq \abs{N(D)}$.
\end{corollary}
\begin{prf}
  Suppose $G$ has a perfect matching $M$.
  Then, $\abs{A} = \abs{B}$ by bipartiteness
  and $\forall D \subseteq A, \abs{D} \leq \abs{N(D)}$ as in \nameref{thm:hall}.

  Suppose $\abs{A} = \abs{B}$ and $\forall D \subseteq A, \abs{D} \leq \abs{N(D)}$.
  Then, by \nameref{thm:hall}, there exists a matching saturating $A$.
  But since $\abs{A} = \abs{B}$, $M$ must be perfect by bipartiteness.
\end{prf}

\begin{theorem}[8.6.2]
  All $k$-regular bipartite graphs with $k \geq 1$ have perfect matchings.
\end{theorem}
\begin{prf}
  By Corollary~\nameref{cor:bip.pm}, it suffices to show that $\abs{A}=\abs{B}$
  and $\forall D \subseteq A, \abs{D} \leq \abs{N(D)}$.
  Since $G$ is bipartite,
  \[ \sum_{v \in A}\deg(v) = \abs{E(G)} = \sum_{v \in B}\deg(v) \]
  But by $k$-regularity,
  \[ \sum_{v \in A}\deg(v) = k\abs{A} = \abs{E(G)} = k\abs{B} = \sum_{v \in B}\deg(v) \]
  and $\abs{A} = \abs{B}$ since $k \geq 1$.

  Let $D \subseteq A$.
  Then, $\sum_{v \in D} \deg(v)$ is the number of edges with one end in $D$,
  which is at least the number of edges with one end in $N(D)$, i.e.,
  $\sum_{v \in N(D)} \deg(v)$.
  As above, this gives us $k\abs{D} \leq k\abs{N(D)} \implies \abs{D} \leq \abs{N(D)}$, as desired.
\end{prf}

\begin{theorem}
  Given a bipartite graph $G = (A,B)$
  and integer $k \geq 1$ such that every vertex in $A$ has degree at least $k$
  and every vertex in $B$ has degree at most $k$,
  then $G$ has a matching saturating $A$.
\end{theorem}
